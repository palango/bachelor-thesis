\documentclass[11pt, ngerman,colorback,accentcolor=tud2d]{tudreport}
\usepackage[utf8]{inputenc}
\usepackage[ngerman]{babel}
\usepackage{amsmath}

\usepackage{microtype}

\usepackage{pgfplots}
\pgfplotsset{compat=1.3}

\usepackage{listings}
\begin{document}

\newcommand{\pder}[2][]{\frac{\partial#1}{\partial#2}}
\newcommand{\pderf}[1]{\frac{\partial f}{\partial#1}}
\newcommand{\pderfs}[1]{\frac{\partial^2 f}{\partial#1}}



\chapter{Truncation Error}
\label{cha:Truncation_Error}
\section{Diffusionsterme}
\label{sec:Diffusionsterme}


Um den Truncation Error der Diffusionsterme zu bestimmen, leiten wir die Differenzenquotienten
für die Ableitungen erster Ordnung aus den Taylorreihendarstellungen her. Beispielhaft
soll das anhand des östlichen Randes des Kontrollvolumens geschehen.

Zuerst entwickeln wir die Taylordarstellungen vom Punkt $x_e$ aus in Richtung der anliegenden
Kontrollvolumenmittelpunkte.

\begin{align}
  \phi_E &= \phi_e + \phi'_e(x_E-x_e)+\frac{1}{2}\phi''_e(x_E-x_e)^2
  +\frac{1}{6}\phi'''_e(x_E-x_e)^3+HOT
  \label{eq:taylor_eE}\\
  \phi_P &= \phi_e + \phi'_e(x_P-x_e)+\frac{1}{2}\phi''_e(x_P-x_e)^2
  +\frac{1}{6}\phi'''_e(x_P-x_e)^3+HOT
  \label{eq:taylor_eP}
\end{align}

Wird nun Gleichung~\eqref{eq:taylor_eP} von Gleichung~\eqref{eq:taylor_eE} subtrahiert, 
so ergibt sich:

\begin{equation*}
  \phi_E-\phi_P=\phi'_e(x_E-x_P)+
  \frac{1}{2}\phi''_e\left[{{(x_E-x_e)}^2-{(x_P-x_e)}^2}\right]+
  \frac{1}{6}\phi'''_e\left[{{(x_E-x_e)}^3-{(x_P-x_e)}^3}\right]+HOT
\end{equation*}

Nach Umstellen ergibt sich daraus für die Ableitung $\phi'_e$ der folgende Term:

\begin{equation}
  \phi'_e = \frac{\phi_E-\phi_P}{x_E-x_P}+\frac{1}{2}\phi''_e
\left({\frac{{(x_P-x_e)}^2-{(x_E-x_e)}^2}{x_E-x_P}}\right)+
\frac{1}{6} \phi'''_e \left({\frac{{(x_P-x_e)}^3-{(x_E-x_e)}^3}{(x_E-x_P)}}\right)+HOT
\end{equation}


Die hier auftretenden Ableitungen $\phi''_e$ und $\phi'''_e$ sind nicht bekannt und 
müssen diskretisiert werden. Mit der Nutzung von
möglichst lokale Differenzenquotienten ergeben sich die folgenden Ausdrücke.

\begin{align*}
  \phi''_e &= \frac{1}{(x_E-x_P)}\left({
\frac{\phi_{EE}-\phi_P}{x_{EE}-x_P}-\frac{\phi_E-\phi_W}{x_E-x_W}}\right)\\
\phi'''_e &= \frac{1}{(x_E-x_P)}\left({
\frac{1}{(x_{ee}-x_e)}
\left({\frac{\phi_{EE}-\phi_E}{x_{EE}-x_E}-\frac{\phi_E-\phi_P}{x_E-x_P} }\right)
-\frac{1}{(x_e-x_w)}
\left({\frac{\phi_E-\phi_P}{x_E-x_P} - \frac{\phi_P-\phi_W}{x_P-x_W}  }\right)
}\right)
\end{align*}


Der Truncation Error des Diffusionsterms an der östlichen Grenze des Kontrollvolumens
lässt sich damit über folgende Gleichung beschreiben.

\begin{align*}
  {TE}_e &= \frac{1}{2 (x_E-x_P)}\left({
\frac{\phi_{EE}-\phi_P}{x_{EE}-x_P}-\frac{\phi_E-\phi_W}{x_E-x_W}}\right) \left({\frac{{(x_P-x_e)}^2-{(x_E-x_e)}^2}{x_E-x_P}}\right)\\
&+
\left({
\frac{1}{(x_{ee}-x_e)}
\left({\frac{\phi_{EE}-\phi_E}{x_{EE}-x_E}-\frac{\phi_E-\phi_P}{x_E-x_P} }\right)
-\frac{1}{(x_e-x_w)}
\left({\frac{\phi_E-\phi_P}{x_E-x_P} - \frac{\phi_P-\phi_W}{x_P-x_W}  }\right)
}\right)\\
&\frac{1}{6(x_E-x_P)}\left({\frac{{(x_P-x_e)}^3-{(x_E-x_e)}^3}{(x_E-x_P)}}\right)
+HOT
\end{align*}

Der Truncation Error im Westen ergibt sich äquivalent zu:

\begin{align*}
  \phi_P &= \phi_w + \phi'_w(x_P-x_w)+\frac{1}{2}\phi''_w(x_P-x_w)^2
  +\frac{1}{6}\phi'''_w(x_P-x_w)^3+HOT\\
  \phi_W &= \phi_w + \phi'_w(x_W-x_w)+\frac{1}{2}\phi''_w(x_W-x_w)^2
  +\frac{1}{6}\phi'''_w(x_W-x_w)^3+HOT
\end{align*}

\begin{equation}
  \phi'_w = \frac{\phi_P-\phi_W}{x_P-x_W}+\frac{1}{2}\phi''_w
\left({\frac{{(x_W-x_w)}^2-{(x_P-x_w)}^2}{x_P-x_W}}\right)+
\frac{1}{6} \phi'''_w \left({\frac{{(x_W-x_w)}^3-{(x_P-x_w)}^3}{(x_P-x_W)}}\right)+HOT
\end{equation}

\begin{align*}
  \phi''_w &= \frac{1}{(x_P-x_W)}\left({
\frac{\phi_{E}-\phi_W}{x_{E}-x_W}-\frac{\phi_P-\phi_{WW}}{x_P-x_{WW}}}\right)\\
 \phi'''_w &= \frac{1}{(x_P-x_W)}\left({
\frac{1}{(x_e-x_w)}
\left({\frac{\phi_E-\phi_P}{x_E-x_P}-\frac{\phi_P-\phi_W}{x_P-x_W} }\right)
-\frac{1}{(x_w-x_{ww})}
\left({\frac{\phi_P-\phi_W}{x_P-x_W} - \frac{\phi_W-\phi_{WW}}{x_W-x_{WW}}  }\right)
}\right)
\end{align*}

\begin{align*}
  TE_w &= \frac{1}{2 (x_P-x_W)} \left({
\frac{\phi_{E}-\phi_W}{x_{E}-x_W}-\frac{\phi_P-\phi_{WW}}{x_P-x_{WW}}}\right)
  \left({\frac{{(x_W-x_w)}^2-{(x_P-x_w)}^2}{x_P-x_W}}\right)\\
&+
\left({
\frac{1}{(x_e-x_w)}
\left({\frac{\phi_E-\phi_P}{x_E-x_P}-\frac{\phi_P-\phi_W}{x_P-x_W} }\right)
-\frac{1}{(x_w-x_{ww})}
\left({\frac{\phi_P-\phi_W}{x_P-x_W} - \frac{\phi_W-\phi_{WW}}{x_W-x_{WW}}  }\right)
}\right)\\
&\frac{1}{6(x_P-x_W)}\left({\frac{{(x_W-x_w)}^3-{(x_P-x_w)}^3}{(x_P-x_W)}}\right)
  +HOT
\end{align*}

\subsection{Äquidistante Gitter}

Für äquidistante Gitter vereinfache sich die Terme des Truncation Error. So löschen
sich beispielsweise die quadratischen Terme gegenseitig aus. Es ergeben sich die
folgenden Fehler:

\begin{align*}
  {TE}_e &= \left({
\frac{1}{6\Delta x^2}
\left({\frac{\phi_{EE}-\phi_E}{\Delta x}-\frac{\phi_E-\phi_P}{\Delta x} }\right)
-\frac{1}{6\Delta x^2}
\left({\frac{\phi_E-\phi_P}{\Delta x} - \frac{\phi_P-\phi_W}{\Delta x}  }\right)
}\right)\left({-\frac{\Delta x^2}{4} }\right)+HOT\\
&= -\frac{1}{24\Delta x}\left({
\phi_{EE}-3\phi_E+3\phi_P-\phi_W}\right)+HOT
\end{align*}

\begin{align*}
  TE_w &=\left({
\frac{1}{6 \Delta x^2}
\left({\frac{\phi_E-\phi_P}{\Delta x}-\frac{\phi_P-\phi_W}{\Delta x} }\right)
-\frac{1}{6\Delta x^2}
\left({\frac{\phi_P-\phi_W}{\Delta x} - \frac{\phi_W-\phi_{WW}}{\Delta x}  }\right)
}\right)
\left({-\frac{\Delta x^2}{4} }\right)+HOT\\
&= -\frac{1}{24 \Delta x}\left({
\phi_E-3\phi_P+3\phi_W-\phi_{WW}}\right)+HOT
\end{align*}


\subsection{Kontrollvolumen am Rand}


Wie man sieht, werden für die Differenzenquotienten die Funktionswerte benachbarter
Kontrollvolumen benötigt. Deshalb müssen die Approximationen der Ableitung für
Kontrollvolumen am Rand angepasst werden.

\paragraph{Westlicher Rand}

Es ergeben sich für gegebenes $\phi_w$:


\begin{equation*}
  \phi''_{w, W-Rand} = \frac{1}{(x_P-x_w)}\left({
\frac{\phi_{e}-\phi_w}{x_{e}-x_w}-\frac{\phi_P-\phi_w}{x_P-x_w}}\right)
\end{equation*}

Da bei der dritten Ableitung zwei linke Kontrollvolumen genutzt werden, müssen hier
vom Kontrollvolumen am Rand sowohl der westliche also auch der östliche Rand
betrachtet werden.

\begin{align*}
  \phi'''_{w, W-Rand} &= \frac{1}{(x_p-x_w)} \left({
  \frac{1}{(x_e-x_w)} \left({
    \frac{\phi_E-\phi_P}{x_E-x_P} - \frac{\phi_P-\phi_w}{x_P-x_w}
    }\right) -
  \frac{1}{x_P-x_w} \left({
    \frac{\phi_e-\phi_w}{x_e-x_w} - \frac{\phi_P-\phi_w}{x_P-x_w}
    }\right)
  }\right)
  \\
  \phi'''_{e, W-Rand} &= \frac{1}{(x_E-x_P)} \left({
  \frac{1}{(x_{ee}-x_e)} \left({
      \frac{\phi_{EE}-\phi_E}{x_{EE}-x_E} - \frac{\phi_E-\phi_P}{x_E-x_P}
    }\right) -
  \frac{1}{x_e-x_w} \left({
    \frac{\phi_E-\phi_P}{x_E-x_P} - \frac{\phi_P-\phi_w}{x_P-x_w}
    }\right)
  }\right)
\end{align*}

Hier wird noch der Wert $\phi_e$ benutzt, der aber nicht bekannt ist. Er wird deshalb
durch eine lineare Interploation von $\phi_E$ und $\phi_P$ bestimmt.

\begin{equation}
  \phi_e = \phi_P \frac{x_E-x_e}{x_E-x_P} + \phi_E \frac{x_e-x_P}{x_E-x_P}
\end{equation}

Damit ergibt sich für $\phi''_{w,Rand}$und $\phi'''_{w, Rand}$:

\begin{align}
  \phi''_{w, W-Rand} &= \frac{1}{(x_P-x_w)}\left({
\frac{
  \left({\phi_P \frac{x_E-x_e}{x_E-x_P} + \phi_E \frac{x_e-x_P}{x_E-x_P}
}\right)
-\phi_w}{x_{e}-x_w}-\frac{\phi_P-\phi_w}{x_P-x_w}}\right)\\
  \phi'''_{w, W-Rand} &= \frac{1}{(x_p-x_w)} \left({
  \frac{1}{(x_e-x_w)} \left({
    \frac{\phi_E-\phi_P}{x_E-x_P} - \frac{\phi_P-\phi_w}{x_P-x_w}
    }\right) -
  \frac{1}{x_P-x_w} \left({
    \frac{ \phi_P \frac{x_E-x_e}{x_E-x_P} + \phi_E \frac{x_e-x_P}{x_E-x_P}
-\phi_w}{x_e-x_w} - \frac{\phi_P-\phi_w}{x_P-x_w}
    }\right)
  }\right)
\end{align}


\paragraph{Östlicher Rand}
Äquivalent ergibt sich $\phi''_{e, E-Rand}$ bei gegebenem $\phi_e$:

\begin{align*}
  \phi''_{e, E-Rand} &= \frac{1}{(x_e-x_P)}\left({
\frac{\phi_{e}-\phi_P}{x_{e}-x_P}-\frac{\phi_e-\phi_w}{x_e-x_w}}\right)
\\
  \phi'''_{e, E-Rand} &= \frac{1}{(x_e-x_P)} \left({
  \frac{1}{(x_e-x_P)} \left({
    \frac{\phi_e-\phi_P}{x_e-x_P} - \frac{\phi_e-\phi_w}{x_e-x_w}
    }\right) -
  \frac{1}{x_e-x_w} \left({
    \frac{\phi_e-\phi_P}{x_e-x_P} - \frac{\phi_P-\phi_W}{x_P-x_W}
    }\right)
  }\right)
  \\
  \phi'''_{w, E-Rand} &= \frac{1}{(x_P-x_W)} \left({
  \frac{1}{(x_e-x_w)} \left({
      \frac{\phi_e-\phi_P}{x_e-x_P} - \frac{\phi_P-\phi_W}{x_P-x_W}
    }\right) -
    \frac{1}{x_w-x_{ww}} \left({
        \frac{\phi_P-\phi_W}{x_P-x_W} - \frac{\phi_W-\phi_{WW}}{x_W-x_{WW}}
    }\right)
  }\right)
\end{align*}

$\phi_w$ wird wie folgt linear interpoliert:

\begin{equation*}
  \phi_w = \phi_W \frac{x_P-x_w}{x_P-x_W} + \phi_P \frac{x_w-x_W}{x_P-x_W}
\end{equation*}

Damit ergibt sich für $\phi''_{e,Rand}$:

\begin{align}
  \phi''_{e,Rand} &= \frac{1}{(x_e-x_P)}\left({
\frac{\phi_{e}-\phi_P}{x_{e}-x_P}-\frac{\phi_e-
  \left({
   \phi_W \frac{x_P-x_w}{x_P-x_W} + \phi_P \frac{x_w-x_W}{x_P-x_W}
  }\right)
}{x_e-x_w}}\right)
\\
  \phi'''_{e, E-Rand} &= \frac{1}{(x_e-x_P)} \left({
  \frac{1}{(x_e-x_P)} \left({
    \frac{\phi_e-\phi_P}{x_e-x_P} - \frac{\phi_e- \phi_W \frac{x_P-x_w}{x_P-x_W} + \phi_P \frac{x_w-x_W}{x_P-x_W}
}{x_e-x_w}
    }\right) -
  \frac{1}{x_e-x_w} \left({
    \frac{\phi_e-\phi_P}{x_e-x_P} - \frac{\phi_P-\phi_W}{x_P-x_W}
    }\right)
  }\right)
\end{align}


\subsubsection{Äquidistante Gitter}

\paragraph{Westlicher Rand}

\begin{align}
  \phi''_{w, W-Rand} &= \frac{1}{\Delta x^2} (\phi_E-3\phi_P+2\phi_w)\\
  \phi'''_{w, W-Rand} &= 0\\
  \phi'''_{e, W-Rand} &= \frac{1}{\Delta x^3} (\phi_{EE} -3\phi_E + 4\phi_P -2\phi_w)
\end{align}

\paragraph{Östlicher Rand}

\begin{align}
  \phi''_{e, E-Rand} &= \frac{1}{\Delta x^2} (\phi_W-3\phi_P+2\phi_e)\\
  \phi'''_{e, E-Rand} &= 0\\
  \phi'''_{w, E-Rand} &= \frac{1}{\Delta x^3} (-\phi_{WW} +3\phi_W - 4\phi_P +2\phi_e)
\end{align}

\section{Diffusion 2D}

Äquivalent zu den Herleitungen des Truncation Error im Osten und Westen ergeben sich
sich für die Ableitungen $\phi'_n$ und $\phi'_s$ die folgenden Terme:

\begin{equation}
  \phi'_n = \frac{\phi_N-\phi_P}{y_N-y_P}+\frac{1}{2}\phi''_n
\left({\frac{{(y_P-y_n)}^2-{(y_N-y_n)}^2}{y_N-y_P}}\right)+
\frac{1}{6} \phi'''_n \left({\frac{{(y_P-y_n)}^3-{(y_N-y_n)}^3}{y_N-y_P}}\right)+HOT
\end{equation}


\begin{equation}
  \phi'_s = \frac{\phi_P-\phi_S}{y_P-y_S}+\frac{1}{2}\phi''_s
\left({\frac{{(y_S-y_s)}^2-{(y_P-y_s)}^2}{y_P-y_S}}\right)+
\frac{1}{6} \phi'''_s \left({\frac{{(y_S-y_s)}^3-{(y_P-y_s)}^3}{y_P-y_S}}\right)+HOT
\end{equation}

Nach Diskretisierung der auftretenden, unbekannten Ableitungen ergeben sich die folgenden
Terme für den Truncation Error:

\begin{align*}
  {TE}_n &= \frac{1}{2 (y_N-y_P)}\left({
\frac{\phi_{NN}-\phi_P}{y_{NN}-y_P}-\frac{\phi_N-\phi_S}{y_N-y_S}}\right) \left({\frac{{(y_P-y_n)}^2-{(y_N-y_n)}^2}{y_N-y_P}}\right)\\
&+
\left({
\frac{1}{(y_{nn}-y_n)}
\left({\frac{\phi_{NN}-\phi_N}{y_{NN}-y_N}-\frac{\phi_N-\phi_P}{y_N-y_P} }\right)
-\frac{1}{(y_n-y_s)}
\left({\frac{\phi_N-\phi_P}{y_N-y_P} - \frac{\phi_P-\phi_S}{y_P-y_S}  }\right)
}\right)\\
&\frac{1}{6(y_N-y_P)}\left({\frac{{(y_P-y_n)}^3-{(y_N-y_n)}^3}{y_N-y_P}}\right)
+HOT
\end{align*}


\begin{align*}
  TE_s &= \frac{1}{2 (y_P-y_S)} \left({
\frac{\phi_{N}-\phi_S}{y_{N}-y_S}-\frac{\phi_P-\phi_{SS}}{y_P-y_{SS}}}\right)
  \left({\frac{{(y_S-y_s)}^2-{(y_P-y_s)}^2}{y_P-y_S}}\right)\\
&+
\left({
\frac{1}{(y_n-y_s)}
\left({\frac{\phi_N-\phi_P}{y_N-y_P}-\frac{\phi_P-\phi_S}{y_P-y_S} }\right)
-\frac{1}{(y_s-y_{ss})}
\left({\frac{\phi_P-\phi_S}{y_P-y_S} - \frac{\phi_S-\phi_{SS}}{y_S-y_{SS}}  }\right)
}\right)\\
&\frac{1}{6(y_P-y_S)}\left({\frac{{(y_S-y_s)}^3-{(y_P-y_s)}^3}{y_P-y_S}}\right)
  +HOT
\end{align*}

\subsection{Äquidistante Gitter}

\begin{align*}
  {TE}_n &= \left({
\frac{1}{6\Delta y^2}
\left({\frac{\phi_{NN}-\phi_N}{\Delta y}-\frac{\phi_N-\phi_P}{\Delta y} }\right)
-\frac{1}{6\Delta y^2}
\left({\frac{\phi_N-\phi_P}{\Delta y} - \frac{\phi_P-\phi_S}{\Delta y}  }\right)
}\right)\left({-\frac{\Delta y^2}{4} }\right)+HOT\\
&= -\frac{1}{24\Delta y}\left({
\phi_{NN}-3\phi_N+3\phi_P-\phi_S}\right)+HOT
\end{align*}

\begin{align*}
  TE_s &=\left({
\frac{1}{6 \Delta y^2}
\left({\frac{\phi_N-\phi_P}{\Delta y}-\frac{\phi_P-\phi_S}{\Delta y} }\right)
-\frac{1}{6\Delta y^2}
\left({\frac{\phi_P-\phi_S}{\Delta y} - \frac{\phi_S-\phi_{SS}}{\Delta y}  }\right)
}\right)
\left({-\frac{\Delta y^2}{4} }\right)+HOT\\
&= -\frac{1}{24 \Delta y}\left({
\phi_N-3\phi_P+3\phi_S-\phi_{SS}}\right)+HOT
\end{align*}


\end{document}
