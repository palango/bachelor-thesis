\documentclass[bigchapter,twoside,report,11pt,type=bsc,colorback,accentcolor=tud2c]{tudthesis}
\usepackage{ngerman}
\usepackage[utf8]{inputenc}
\usepackage[ngerman]{babel}
\usepackage{amsmath}

\usepackage{microtype}
\usepackage[ngerman,pdfview=FitH,pdfstartview=FitV]{hyperref}

\usepackage{pgfplots}
\pgfplotsset{compat=newest}

%\KOMAoptions{cleardoublepage=realempty}

\usepackage{bibgerm}
\usepackage{listings}
\usepackage{inconsolata}
%\usepackage[T1]{fontenc}
\lstset{framextopmargin=50pt,frame=bottomline}
\lstset{language=Matlab,
   %keywords={break,case,catch,continue,else,elseif,end,for,function,
      %global,if,otherwise,persistent,return,switch,try,while},
   basicstyle=\ttfamily\footnotesize,
   breaklines=true,
   keywordstyle=\color{tud1c},
   commentstyle=\color{tud9c},
   stringstyle=\color{tud4c},
   %numbers=left,
   %numberstyle=\tiny\color{tud0c},
   stepnumber=1,
   numbersep=5pt,
   backgroundcolor=\color{white},
   tabsize=4,
   showspaces=false,
   showstringspaces=false,
   literate=%
{Ö}{{\"O}}1
{Ä}{{\"A}}1
{Ü}{{\"U}}1
{ß}{{\ss}}2
{ü}{{\"u}}1
{ä}{{\"a}}1
{ö}{{\"o}}1}

\usepackage{tikz}


\newcommand{\getmydate}{%
  \ifcase\month%
    \or Januar\or Februar\or M\"arz%
    \or April\or Mai\or Juni\or Juli%
    \or August\or September\or Oktober%
    \or November\or Dezember%
  \fi\ \number\year%
}
\newcommand{\pder}[2][]{\frac{\partial#1}{\partial#2}}
\newcommand{\pderf}[1]{\frac{\partial f}{\partial#1}}
\newcommand{\pderfs}[1]{\frac{\partial^2 f}{\partial#1}}


\begin{document}
  %\thesistitle{Ableitung und Untersuchung des Abbruchfehlersch"atzers f"ur mittels
  %Finite Volumen diskretisierten Navier-Stokes Gleichungen\linebreak[1]}%
    %{Derivation and examination of the truncation error for finite volume discretised Navier-Stokes equations}
  %\author{Paul Lange}
  %\birthplace{Darmstadt}
  %\referee{Clemens v. Loewenich}{Ulrich Falk}
  %\department{Fachbereich Maschinenbau}
  %\group{Institut f"ur Numerische Berechnungsverfahren}
  %\dateofexam{\today}{\today}
  %\tuprints{12345}{1234}
  %\makethesistitle
  %\affidavit{P. Lange}

  %\newpage
  \tableofcontents
  %\cleardoublepage

  \chapter{Einleitung}
  \section{Motivation}
Durch steigende Computerleistung sowie Verbesserungen in den numerischen Methoden konnten
sich die simulationsbasierten Produktentwicklungsprozesse einen festen Platz neben
Experimenten erarbeiten. Neben Kostenvorteilen ist es vor allem die Möglichkeit,
schnell Aussagen über Anpassungen und Modifikationen an Produkten treffen zu können, die
simulationsbasierten Methoden zu einem stetigen Wachstum verhelfen.

Für die ingenieurmäßige Berechnung von Strömungsproblemen hat sich die Methode der
Finiten Volumen auf breitem Gebiet durchgesetzt. Der Fokus liegt hierbei auf der Simulation
der Navier-Stokes Gleichungen, oft auch bei der Ausnutzung von zulässigen Vereinfachungen
wie der Vernachlässigung kompressibler Strömungseffekte.

Um die Genauigkeit der Ergebnisse zu verbessern werden immer feinere Gebiete aufgelöst,
bei denen auch heutige Computersysteme an ihre Grenzen ihrer Leistungsfähigkeit stoßen. Es ist deshalb notwendig
auf adaptive Methoden zurückzugreifen, die die Auflösung an interessanten Stellen wie
zum Beispiel Grenzschichten oder besonders interessanten Strömungsgebieten selektiv erhöhen können.
Weiterhin sollen Fehler, die durch die numerische Behandlung der kontinuierlichen Strömungsgebiete
entstehen, minimiert werden. Dafür ist es notwendig, Indikatoren für den lokalen Fehler
der berechneten Lösung zu definieren.

Heutige Methoden schätzen diesen Fehler ab, negieren aber seine Ursachen bei der Diskretisierung
und damit dem Übergang von der kontinuierlichen Differentialgleichung zur diskreten Gleichung.
Hier liegt der Ansatzpunkt der Fehlerbetrachtung über den Abbruchfehler. Er schätzt
den durch Diskretisierungen entstandenen Fehler auf Basis von Taylorreihenentwicklungen ab.
Der Abbruchfehler wird in der
vorliegenden Arbeit für skalare Transportprobleme hergeleitet und seine Genauigkeit an Beispielen
verifiziert. Weiterhin wird er zur Gitteradaption eingesetzt.


\section{Stand der Forschung}

Der Abbruchfehler als Indikator für die Gitteradaption ist in der Literatur bereits bekannt.
Roy~\cite{roy2} leitet ihn für die eindimensionale Burger-Gleichung mittels
Finiten-Differenzen her und vergleicht anschließend den abbruchfehlerbasierten Fehlerindikator
mit lösungsbasierten Indikatoren wie dem Lösungsgradient oder der Lösungskrümmung. Weiterhin wird gezeigt,
dass für lineare Probleme der Abbruchfehler
als Quelle des Diskretisierungsfehlers interpretiert werden kann.
Dadurch bewirkt eine Gitterverfeinerung basierend auf dem Abbruchfehler
eine Erhöhung der Auflösung an der Quelle des Diskretisierungsfehlers und
verringert damit den gesamten Diskretisierungsfehler.

Vor der Gitteradaption muss eine Lösung auf dem Problemgebiet berechnet werden. Diese
Lösung und das zu ihrer Berechnung verwendete Lösungsprogramm müssen
auf ihre Korrektheit überprüft und verifiziert werden. Veluri~\cite{veluri} demonstriert
Methoden zur Verifikation von Lösungsprogrammen. Diese basieren auf dem Vergleich von formaler und beobachteter
Konvergenzordnung und stellt außerdem die Methode der konstruierten Lösung vor, die genutzt werden kann
um analytische Lösungen zum Zweck der Verifikation von Computerprogrammen zu erzeugen.
Auch Roache~\cite{roache} stellt diese vor und demonstriert ihre Nützlichkeit bei der Verifikation von
Lösungsprogrammen.

In praktisch relevanten Problemfällen kann nur in den wenigsten Fällen mit
kartesischen oder orthogonalen Gittern gerechnet werden. Die Nichtorthogonalität
von Gittern bewirkt neue Fehlerterme, die bereits untersucht wurden. So gehen
Huang und Prosperetti~\cite{grid_ortho} auf den Einfluss nicht-orthogonaler Gitter
auf die Lösungsgenauigkeit gehen 
ein. Lee und Tsuei~\cite{lee} demonstrieren die Ableitung des Abbruchfehlers für Finite-Differenzen Verfahren
auf nicht-orthogonalen Gittern.

Einen Überblick über Methoden zur Gitteradaption gibt Fath~\cite{fath}.
Neben den Grundlagen geht sie auch auf die Ableitung des Abbruchfehlers für
orthogonale Gitter sowie die Transformation von physikalischen zu logischen Gebiet ein.



\section{Zielsetzung}

Im ersten Teil der vorliegenden Arbeit werden grundlegende Konzepte und Begriffe vorgestellt,
die für das Verständnis der Thesis notwendig sind. Grundlegende Formeln wie
die skalare Transportgleichung oder die Navier-Stokes Gleichungen werden vorgestellt
und ein Überblick über die verwendeten numerischen Methoden gegeben. Außerdem
wird der Abbruchfehler eingeführt und die Grundlagen der Verifikation von
Lösungsprogrammen erläutert.

Anschließend wird der Abbruchfehler formal
abgeleitet. Dabei werden ein- und zweidimensionale Probleme betrachtet auf kartesischen, orthogonalen
und nicht-orthogonalen Gittern betrachtet. Der Abbruchfehler wird für Konvektions-
und Diffusionsterme sowie für Quellterme hergeleitet. Die Herleitung beschränkt sich damit auf
skalare Transportgleichungen. Diese decken jedoch bereits die wesentlichen Terme der Navier-Stokes Gleichungen
ab und lassen sich in Zukunft mit geringem Aufwand auf die Navier-Stokes Gleichungen übertragen.
Zudem lassen sich die erzielten Ergebnisse an Transportproblemen leichter untersuchen und verifizieren.

Anschließend wird der abgeleitete Abbruchfehlerindikator an konkreten Testfällen untersucht.
Dazu werden Lösungsprogramme für ein- und zweidimensionale Problemgebiete implementiert
und ihre Korrektheit nachgewiesen. Darauf folgt die Implementation
der Abbruchfehlers sowie die Untersuchung auf den verschiedenen Gittertypen.

Abschließend wird dann der Abbruchfehlerindikator zur lokalen Gitteradaption genutzt.
Zum Vergleich werden weitere gängige Indikatoren wie der Lösungsgradient oder
die Fehlerabschätzung mit der Richardson-Extrapolation implementiert und mit dem Abbruchfehler verglichen.

\cleardoublepage


\chapter{Grundlagen}
\section{Skalare Probleme}

Eine Vielzahl praktisch relevanter Probleme lässt sich durch skalare (partielle)
Differentialgleichungen beschreiben. Anwendungsgebiete reichen hier von Wärmetransportproblemen
über strukturmechanische Probleme (z.B. Auslenkung von Stäben und Membranen bei elastischem
Werkstoffverhalten) bis hin zu strömungsmechanischen Problemen (z.B. Geschwindigkeitspotential
wirbelfreier Strömungsfelder).

\subsection{Einfache Feldprobleme}

Einige kontinuumsmechanische Problemstellungen können durch eine Differentialgleichung
der Form

\begin{equation}
  -\frac{\partial}{\partial x_i}\left({\alpha \frac{\partial \phi}{\partial x_i} }\right)=g
  \label{eq:feldproblem_stat}
\end{equation}
 beschrieben werden. Diese muss auf dem gesamten betrachteten Gebiet $V$ gelten, das
vom Rand $S$ begrenzt wird. Bei $\phi=\phi(\mathbf{x})$ handelt es sich um die gesuchte,
skalare Feldgröße. $g=g(\mathbf{x})$ und $\alpha=\alpha(\mathbf{x})$ sind vorgegeben.


Weiterhin müssen auf dem gesamten Rand $S$ Randbedingungen
gegeben sein. Üblich sind hierbei folgende Typen von Randbedingungen:

\begin{itemize}
  \item Dirichletsche Randbedingung: $\quad \phi=\phi_S$
  \item Neumannsche Randbedingung: $\quad \alpha \frac{\partial \phi}{\partial x_i} n_i = b_s $
  \item Cauchysche Randbedingung: $\quad c_S \phi + \alpha \frac{\partial \phi}{\partial x_i} n_i = b_s$
\end{itemize}

Hierbei sind $\phi_S$, $b_S$ und $c_S$ vorgegebene Funktionen auf dem Rand $S$. Die
Komponenten des nach außen gerichteten Normaleneinheitsvektors an $S$ werden mit $n_i$ bezeichnet.

Treten mehrere Randbedingungstypen in einem Problem auf spricht man von gemischten Randwertproblemen.

Die mit Gleichung~\eqref{eq:feldproblem_stat} beschriebenen Probleme beinhalten keine Zeitabhängigkeit
und werden deshalb als stationär bezeichnet.
Im instationären, also zeitabhängigem Fall, erhalten alle Größen neben der Ortsabhängigkeit
eine Zeitabhängigkeit. Die entsprechende Differentialgleichung lautet damit
\begin{equation}
  \frac{\partial \phi}{\partial t}
  -\frac{\partial}{\partial x_i}\left({\alpha \frac{\partial \phi}{\partial x_i} }\right)=g
  \label{eq:feldproblem_instat}
\end{equation}
für die unbekannte Größe $\phi=\phi(\mathbf{x}, t)$.
Für instationäre Probleme muss neben den Randbedingungen auch eine Anfangsbedingung
$\phi(\mathbf{x}, t_0) = \phi_0(\mathbf{x})$ gegeben werden.

Im weiteren Verlauf der Arbeit werden nur stationäre Probleme betrachtet. Alle Ergebnisse
lassen sich aber auf instationäre Probleme übertragen.

\subsection{Allgemeine Transportgleichung}
\label{sec:transportgl}

Eine wichtige Problemklasse innerhalb des Maschinenbaus stellen Transportprobleme in
Festkörpern oder Fluiden dar. Bei der transportierten Größe kann es sich dabei beispielsweise
um joulsche Wärme (Wärmetransportprobleme) oder Stoffmengen (Stofftransportprobleme) handeln.

Stationäre Transportprobleme können in Differentialform durch die Gleichung
\begin{equation}
  \frac{\partial}{\partial x_i} \left({\rho v_i \phi
- \alpha \frac{\partial \phi}{\partial x_i} }\right) = f
\label{eq:transportgl}
\end{equation}
beschrieben werden. Hierbei stellt $f$ einen allgemeinen Quellterm,
$\frac{\partial}{\partial x_i} \rho v_i \phi$ den Konvektionsterm sowie
$\alpha \frac{\partial^2 \phi}{\partial x_i \partial x_i}$ den Diffusionsterm dar.
Für spezielle Probleme müssen die entsprechenden Parameter angepasst werden (s. z.B. \cite{num_maschbau}).

Alle Betrachtungen in der vorliegenden Arbeit bauen auf Problemen auf, die durch die
Transportgleichung beschrieben werden können.

  \section{Numerische Methoden}

Zum Erkenntnisgewinn in den Ingenieurwissenschaften stehen drei grundsätzlichen
Wege zur Verfügung \cite{num_maschbau}. Diese lauten:
\begin{itemize}
  \item Theoretische Methoden
  \item Experimentelle Versuche
  \item Numerische Simulationen
\end{itemize}

Theoretische Methoden, also insbesondere analytische Betrachtungen der das Problem
beschreibenden Gleichungen sind im allgemeinen nur sehr beschränkt möglich. Das heißt das
die Gleichungen, die zur Beschreibung realer Prozesse genutzt werden meist nur für
bestimmte Randbedingungen sowie gezielten Vereinfachungen überhaupt analytisch lösbar
sind. Da jedoch diese Vereinfachungen in realen Prozessen nicht vernachlässigbar klein sind
sowie die geforderten Randbedingungen nicht erfüllbar oder präzise einhaltbar sind, ist es
für komplexere Problemstellungen unmöglich analytische Lösungen zu finden.

Die Intention bei experimentellen Untersuchungen ist über Versuche an Modellen oder
realen Bauteilen an die benötigten Systemgrößen heranzukommen. Diese Vorgehensweise
bereitet jedoch in vielen Fällen Probleme:
\begin{itemize}
  \item Messungen bestimmter Systemgrößen sind oft schwierig bis unmöglich. Gründe
    dafür können die Dimensionen der Objekte (z.B. Molekulare Prozesse, Weltmeere),
    die Geschwindigkeit der Zustandsänderung (z.B. Explosionen) oder moralischen Gründen
    (z.B. Versuche an Mensch und Tieren, Versuche mit Gefahrenstoffen).
  \item Experimente an Modellen lassen nur begrenzte Rückschlüsse auf das reale Objekt
    zu. So lassen sich beispielsweise Erkenntnisse aus dem Windkanal nur teilweise auf
    das reale Automobil oder Flugzeug übernehmen.
  \item Experimente sind oft teuer und zeitraubend. So muss beispielsweise um die Auswirkungen
    einer Änderung zu testen ein neues Modell bzw. Objekt hergestellt werden. In anderen Fällen
    ist das Modell nach dem Versuch zerstört (z.B. Crashtest). Generell ist auch der Betrieb
    von Messeinrichtungen teuer.
\end{itemize}

Aufgrund der Nachteile der oben genannten Verfahren setzt sich im Maschinenbau und der
Naturwissenschaft im allgemeinen die numerische Simulation immer weiter durch. Vorteile
gegenüber den anderen Verfahren liegen auf der Hand:

\begin{itemize}
  \item Simulationen sind meist schneller und kosteneffizienter zu erstellen als experimentelle
    Versuche.
  \item Änderungen am Objekt sowie Parameterstudien zur Optimierung lassen sich leicht
    erstellen.
  \item Die Ergebnisse der Simulation enthalten meist alle problemrelevanten Größen,
    deren Messung im Versuch mit viel Aufwand verbunden wäre.
\end{itemize}

Grundlegend für die Nutzung dieser Vorteile ist jedoch die Zuverlässigkeit der
berechneten Ergebnisse. Deshalb gehen mit der Verbesserung der numerischen Simulationsmethoden
immer auch experimentelle Validierung von Ergebnissen und verwendeten Modellen einher.

  \subsection{Numerische Gitter}
\label{sec:num_gitter}

Sind das mathematische Modell mit seinen Differentialgleichungen sowie den nötigen
Rand- sowie gegebenenfalls Anfangsbedingungen festgelegt, besteht der nächste Schritt
bei der Anwendung eines numerischen Berechnungsverfahren darin, das kontinuierliche
Gebiet von Raum und ggf. Zeit durch eine endliche Menge von Teilgebieten zu approximieren,
in denen dann der Wert des gesuchten Größe berechnet wird.

Diese Teilgebiete werden im Allgemeinen in Form eines Gitters über den zu untersuchenden
Bereich $V$ gelegt, weshalb dieser Arbeitsschritt auch oft Gittergenerierung genannt wird.

Ein wichtiges Unterscheidungskriterium ist die Anordnung der Gitterpunkte.
Man unterscheidet grundsätzlich zwischen strukturierten und unstrukturierten Gittern.

Bei strukturierten Gittern ist die Lage der Gitterpunkte zueinander festgelegt. Diese
Lagebeziehungen müssen deshalb nicht aufwendig gespeichert werden und die Gittergenerierung
ist schnell und ohne großen Rechenaufwand nötig. Zwar ist es möglich bestimmte Gebiete
aus der Gittergenerierung auszuschließen, eine Anpassung an die Problemgeometrie ist
jedoch nur innerhalb der vorgegebenen Lagebeziehungen möglich.

Bei unstrukturierten Gittern hingegen gibt es keine Vorschriften zu den Lagebeziehungen
der einzelnen Gitterpunkte. Diese müssen deshalb aufwendig gespeichert werden, was
Gittergeneration sowie Lösung komplizierter macht. Andererseits ist es Problemlos möglich
das Gitter an der Problemgeometrie auszurichten oder in bestimmten Gebieten selektiv zu verfeinern.

\subsubsection{Physikalisches und logisches Gebiet}

Oftmals lässt die Geometrie des zu untersuchenden Problems keine Anwendung eines
strukturierten Gitters zu. Nicht in jeden Fall ist es jedoch nötig daraufhin unstrukturierte
Gitter zu verwenden. Vielmehr ist es möglich das physikalische Gebiet des Problems
mittels einer geeigneten mathematischen Transformation auf ein sogenanntes
logisches Gebiet abzubilden, auf welchem dann die Anwendung eines strukturierten Gitters
möglich ist.

Die Abbildung von physikalischem auf logisches Gebiet muss dabei umkehrbar sein, das heißt
die Abbildungsfunktion muss EINEINDEUTIG??? sein. Die Achsen des physikalischen Gebiets
werden im Allgemeinen mit lateinischen Buchstaben bezeichnet ($x$, $y$, $z$), während
für das logische Gebiet griechische Buchstaben verwendet werden ($\xi$, $\eta$, $\theta$).
Beispielhaft soll hier die formale, mathematische Beschreibung anhand eines ebenen
Beispiels gezeigt werden:
\begin{equation}
  f=f(x, y)\qquad \text{mit} \qquad x=x(\xi, \eta),\quad y=y(\xi, \eta)
\end{equation}



  \section{Finite Volumen Methode}

Die Finite-Volumen Methoden sind ein numerisches Verfahren zur Lösung von (partiellen)
Differentialgleichungen. Sie sind heute das Standardverfahren zu Lösung von Strömungsproblemen
wie zum Beispiel den Euler- oder Navier-Stokes-Gleichungen aber keineswegs nur darauf beschränkt.
Die charakteristische Eigenschaft der Finite-Volumen Methoden ist ihre Konservativität, also
die Erfüllung der den mathematischen Modellen zugrunde liegenden Erhaltungsprinzipien in den
diskreten Gleichungen.

\subsection{Ableitung der Finite Volumen diskretisierten Transportgleichung}

Ausgangspukt ist die in Abschnitt~\ref{sec:num_gitter} beschriebene Zerlegung des Problemgebiets
in diskrete Teilgebiete. Diese werden bei der Finite-Volumen Methode Kontrollvolumen
genannt (im zweidimensionalen Fall steht Volumen stellvertretend für Fläche).
Für jedes dieser Kontrollvolumen werden die Erhaltungsgleichungen in Integralform formuliert
beziehungsweise durch Integration aus den entsprchenden Diffenrentialgleichungen gewonnen.

Beispielhaft soll das Vorgehen hier anhand der in Abschnitt~\ref{sec:transportgl}
vorgestellten stationären Transportgleichung im zweidimensionalen Fall
($i=1$, $2$) vorgestellt werden.
\begin{equation}
  \frac{\partial}{\partial x_i} \left({\rho v_i \phi
- \alpha \frac{\partial \phi}{\partial x_i} }\right) = f
\label{eq:transp_fvm}
\end{equation}
Durch Integration von Gleichung~\eqref{eq:transp_fvm} über ein Kontrollvolumen $V$ (Gl.~\ref{eq:fvm1})
erhält man bei anschließender Anwendung des Gaußschen Integralsatzes die in Gleichung~\eqref{eq:fvm2} beschriebene Beziehung.
\begin{align}
  \int_V \frac{\partial}{\partial x_i} \left({\rho v_i \phi
- \alpha \frac{\partial \phi}{\partial x_i} }\right) dV &= \int_V f dV \label{eq:fvm1}\\
  \int_S  \left({\rho v_i \phi
- \alpha \frac{\partial \phi}{\partial x_i} }\right) n_i dS&= \int_V f dV \label{eq:fvm2}
\end{align}
Hierbei beschreibt $S$ die Oberfläche des Kontrollvolumens $V$, $dS$ ein Oberflächenelement
und $n_i$ die Komponenten des Einheitsnormalenvektors auf der Oberfläche.

Geht man nun von beliebig geformten Kontrollvolumen zu viereckigen Kontrollvolumen über,
so lässt sich das Integral über die Oberfläche $S$ in Gleichung~\eqref{eq:fvm2} durch eine
Summation über die Integrale der vier Seitenflächen umformen. Hierbei beschreibt $c$ die
einzelnen Seitenflächen und ist bei viereckigen Kontrollvolumen als $c=(e,w,n,s)$ definiert.
\begin{equation}
  \sum_c \int_{S_c} \left(\rho v_i \phi - \alpha \frac{\partial \phi}{\partial x_i}
\right) n_{ci} dS_c = \int_V f dV\label{eq:fvm3}
\end{equation}
Gleichung~\eqref{eq:fvm3} stellt nun die Bilanzgleichung für die konvektiven Flüsse
$F_c^C$ sowie die diffusiven Flüsse $F_c^D$ durch die Seiten des Kontrollvolumens $V$ dar.
\begin{align}
  F_c^C &=  \int_{S_c} \left(\rho v_i \phi \right) n_{ci} dS_c \\
  F_c^D &=  -\int_{S_c} \left(\alpha \frac{\partial \phi}{\partial x_i}\right) n_{ci} dS_c 
\end{align}

\subsection{Integralapproximation}

In dieser Arbeit wird von einer zellenorientierten Variablenanordnung (s. \cite{num_maschbau})
ausgegangen. Damit liegen die gesuchte diskrete Größe im Mittelpunkt jedes Kontrollvolumens.
Die Integrale der konvektiven und diffusiven Flüsse müssen durch diese Variablen
approximiert werden. Dabei ist es zweckmäßig zuerst die Oberflächenintegrale durch Variablen
auf der jeweilen Seite des Kontrollvolumens zu beschrieben und anschließend diese durch die Werte
im Zentrum des Kontrollvolumens zu beschreiben.

Beispielhaft soll hier das Oberflächenintegral
\begin{equation*}
  \int_{S_w} p_i n_{wi}\,dS_w
\end{equation*}
der Westseite $S_w$ betrachtet werden. Hierbei bezeichnet $p_i$ die Komponenten einer allgemeinen
Funktion $\mathbf{p} = (p_1(\mathbf{x}), p_2(\mathbf{x}))$. Die einfachste
Approximationsmöglichkeit besteht darin, den Funktionswert von $\mathbf{p}$ im Mittelpunkt
der Seite zu nutzen.
\begin{equation}
  \int_{S_w} p_i n_{wi}\,dS_w \approx q_w \delta S_w
\end{equation}
Die Normalkomponenten von $\mathbf{p}$ an der Stelle $w$ werden hierbei mit $q_w = p_{wi} n_{wi}$ bezeichnet.
Aufgrund der Nutzung des Funktionswertes im Seitenmittelpunkt wird diese Approximation auch Mittelpunktsregel genannt.
Sie hat eine Ordnung von zwei bezüglich der Seitenlänge $\delta S_w$.

Weitere gängige Approximationen umfassen die Trapezregel oder die Simpsonregel, auf die jedoch
in der vorliegenden Arbeit nicht näher eingegangen wird.

Für die Flüsse $F_c^C$ und $F_c^D$ ergeben sich mit der Mittelpunktsregel folgende Approximationen:
\begin{align*}
  F_c^C &\approx \underbrace{\rho v_i n_{ci} \delta S_c}_{\dot{m_c}}  \phi_c\\
  F_c^D &\approx  -\alpha  n_{ci} \delta S_c \left(\frac{\partial \phi}{\partial x_i}\right)_c
\end{align*}
wobei $\dot{m_c}$ den Massenstrom durch die Seite $S_c$ bezeichnet.

Im nächsten Schritt müssen nun die Variablenwert $\phi_c$ durch die Variablenwerte in den
Mittelpunkten der Kontrollvolumen ausgedrückt werden.
Zunächst soll jedoch die rechte Seite von Gleichung~\eqref{eq:fvm3} betrachtet werden.
Da der Quellterm $f$ oft nicht analytisch integrierbar ist erfolgt im Allgemeinen eine Approximation
mittels numerischer Integration. Oftmals ist hierbei die Nutzung der Mittelpunktsregel mit der Ordnung zwei
ausreichend. Sie geht von der Annahme aus, das der Wert $f_P$ im Mittelpunkt des Kontrollvolumens
einen Mittelwert über die Werte von $f$ im gesamten Kontrollvolumen darstellt.

Die Auswertung des Volumenintegrals ergibt sich damit im zweidimensionalen Fall zu
\begin{equation}
  \int_V f\,dV \approx f_p\,\delta V
\end{equation}
wobei $\delta V$ das Volumen des Kontrollvolumens beschreibt und im Falle eines orthogonalen Gitters
mittels
\begin{equation}
  \delta V = (x_e - x_w)(y_n-y_s) = \Delta x \Delta y
\end{equation}
berechnet werden kann. Auf den bei dieser Appoximation entstehenden Abbruchfehler wird in
Abschnitt~\ref{sec:Quellterm} genauer eingegegangen.
\subsection{Ableitung diffusiver Fluss}
\subsection{Ableitung konvektiver Fluss}

  \documentclass[11pt, ngerman,colorback,accentcolor=tud2d]{tudreport}
\usepackage[utf8]{inputenc}
\usepackage[ngerman]{babel}
\usepackage{amsmath}

\usepackage{microtype}

\usepackage{pgfplots}
\pgfplotsset{compat=1.3}

\usepackage{listings}
\begin{document}

\newcommand{\pder}[2][]{\frac{\partial#1}{\partial#2}}
\newcommand{\pderf}[1]{\frac{\partial f}{\partial#1}}
\newcommand{\pderfs}[1]{\frac{\partial^2 f}{\partial#1}}



\chapter{Truncation Error}
\label{cha:Truncation_Error}


\section{Konvektionsterme}

\subsection{CDS-Verfahren}
Entwickelt man die Taylorreihe von $\phi$ um den Punkt $x_P$ und wertet sie anschließend
an den Punkten $x_e$ und $x_E$ aus, so erhält man folgende Gleichungen.

\begin{align}
  \phi_e &= \phi_P + \phi'_P(x_e-x_P)+\frac{1}{2}\phi''_P(x_e-x_P)^2
  +\frac{1}{6}\phi'''_P(x_e-x_P)^3+HOT
  \label{eq:taylor_konv_eP}\\
  \phi_E &= \phi_P + \phi'_P(x_E-x_P)+\frac{1}{2}\phi''_P(x_E-x_P)^2
  +\frac{1}{6}\phi'''_P(x_E-x_P)^3+HOT
  \label{eq:taylor_konv_eE}
\end{align}

Werden die Gleichungen~\eqref{eq:taylor_konv_eE} und \eqref{eq:taylor_konv_eP} nun
voneinander subtrahiert und nach $\phi_e$ umgestellt, so ergibt sich:

\begin{align*}
  \frac{\phi_e}{x_e-x_P} &= \frac{\phi_E}{x_E-x_P} + \frac{\phi_P}{x_e-x_P} -
  \frac{\phi_P}{x_E-x_P} + \frac{1}{2} \phi''_P \left({(x_e-x_P)-(x_E-x_P)}\right)
  + \frac{1}{6} \phi'''_P \left({(x_e-x_P)^2-(x_E-x_P)^2}\right)\\
  \phi_e &= \phi_E \frac{x_e-x_P}{x_E-x_P} + \phi_P \left({1-\frac{x_e-x_P}{x_E-x_P} }\right)
  + \frac{1}{2} \phi''_P (x_e-x_E)(x_e-x_P)
  + \frac{1}{6} \phi'''_P \left({(x_e-x_P)^2-(x_E-x_P)^2}\right)(x_e-x_P)\\
  \phi_e &= \phi_E \gamma_e + \phi_P (1-\gamma_e)+ \frac{1}{2} \phi''_P (x_e-x_E)(x_e-x_P)
  + \frac{1}{6} \phi'''_P \left({(x_e-x_P)^2-(x_E-x_P)^2}\right)(x_e-x_P)
\end{align*}

Der Truncation Error lässt sich damit ablesen zu:

\begin{equation*}
  TE_{e, CDS} =  \frac{1}{2} \phi''_P (x_e-x_E)(x_e-x_P)+ \frac{1}{6} \phi'''_P \left({(x_e-x_P)^2-(x_E-x_P)^2}\right)(x_e-x_P)
\end{equation*}


Hier müssen nun wiederum die auftretenden Ableitungen diskretisiert werden.

\begin{align}
  \phi'_P &= \frac{\phi_E-\phi_W}{x_E-x_W}\\
  \phi''_P &= \frac{1}{(x_e-x_w)} \left({\frac{\phi_E-\phi_P}{x_E-x_P}
  - \frac{\phi_P-\phi_W}{x_P-x_W} }\right)\\
  \phi'''_P &= \frac{1}{(x_e-x_w)} \left({
  \frac{1}{(x_E-x_P)} \left({\frac{\phi_{EE}-\phi_P}{x_{EE}-x_P}- \frac{\phi_E-\phi_W}{x_E-x_W} }\right)-
  \frac{1}{(x_P-x_W)} \left({\frac{\phi_E-\phi_W}{x_E-x_W} - \frac{\phi_P-\phi_{WW}}{x_P-x_{WW}} }\right)
  }\right)
\end{align}

Der diskretisierte Truncation Error ergibt sich damit zu:

\begin{align}
  TE_{e, CDS} &=  \frac{1}{2} \frac{1}{(x_e-x_w)} \left({\frac{\phi_E-\phi_P}{x_E-x_P}
  - \frac{\phi_P-\phi_W}{x_P-x_W} }\right) (x_e-x_E) \nonumber \\
  &+
 \frac{1}{(x_e-x_w)} \left({
  \frac{1}{(x_E-x_P)} \left({\frac{\phi_{EE}-\phi_P}{x_{EE}-x_P}- \frac{\phi_E-\phi_W}{x_E-x_W} }\right)-
  \frac{1}{(x_P-x_W)} \left({\frac{\phi_E-\phi_W}{x_E-x_W} - \frac{\phi_P-\phi_{WW}}{x_P-x_{WW}} }\right)
  }\right) \nonumber \\
  &\frac{1}{6} \left({(x_e-x_P)^2-(x_E-x_P)^2}\right)(x_e-x_P)
\end{align}




\subsection{UDS-Verfahren}

\paragraph{Osten}

Wertet man die Taylorreihe von $\phi$ mit dem Entwicklungspunkt $x_P$ im Punkte $\phi_e$
aus, so ergibt sich:

\begin{equation*}
  \phi_e = \phi_P +(x_e-x_P) \phi'_P + \frac{1}{2} (x_e-x_P)^2 \phi''_P+HOT
\end{equation*}

Da im Falle einer positiven Geschwindigkeit das UDS-Verfahren $\phi_e$ mit $\phi_P$
gleichsetzt, lässt sich der Truncation Error hier direkt ablesen.

\begin{equation*}
  TE_{e, UDS} = (x_e-x_P) \phi'_P + \frac{1}{2} (x_e-x_P)^2 \phi''_P
\end{equation*}

Mit den diskretisierten Ableitungen von oben ergibt sich damit:

\begin{equation}
  TE_{e, UDS} = (x_e-x_P) \frac{\phi_E-\phi_W}{x_E-x_W}+
  \frac{(x_e-x_P)^2}{2(x_e-x_w)} \left({\frac{\phi_E-\phi_P}{x_E-x_P}
  - \frac{\phi_P-\phi_W}{x_P-x_W} }\right)
\end{equation}

Für äquidistante Gitter ergibt sich damit:

\begin{equation}
  TE_{e, UDS} = \frac{3}{8} \phi_E-\frac{1}{4} \phi_P - \frac{1}{8} \phi_W
\end{equation}


\paragraph{Westen}

Wertet man die Taylorreihe von $\phi$ mit dem Entwicklungspunkt $x_P$ im Punkte $\phi_w$
aus, so ergibt sich:

\begin{equation*}
  \phi_w = \phi_W +(x_w-x_W) \phi'_W + \frac{1}{2} (x_w-x_W)^2 \phi''_W+HOT
\end{equation*}

Da im Falle einer positiven Geschwindigkeit das UDS-Verfahren $\phi_w$ mit $\phi_W$
gleichsetzt, lässt sich der Truncation Error hier direkt ablesen.

\begin{equation*}
  TE_{w, UDS} = (x_w-x_W) \phi'_W + \frac{1}{2} (x_w-x_W)^2 \phi''_W
\end{equation*}

%Mit den diskretisierten Ableitungen von oben ergibt sich damit:

%\begin{equation}
  %TE_{w, UDS} = (x_w-x_W) \frac{\phi_E-\phi_W}{x_E-x_W}+
  %\frac{(x_e-x_P)^2}{2(x_e-x_w)} \left({\frac{\phi_E-\phi_P}{x_E-x_P}
  %- \frac{\phi_P-\phi_W}{x_P-x_W} }\right)
%\end{equation}

%Für äquidistante Gitter ergibt sich damit:

%\begin{equation}
  %TE_{e, UDS} = \frac{3}{8} \phi_E-\frac{1}{4} \phi_P - \frac{1}{8} \phi_W
%\end{equation}



\subsection{``Flux-Blending''-Verfahren}

Das Flux-Blending-Verfahren setzt sich aus UDS- und CDS-Verfahren zusammen. Beide
Verfahren werden dabei über den Faktor $\beta$ gewichtet.

\begin{equation*}
\phi_e \approx (1-\beta)\phi_e^{UDS} + \beta \phi_e^{CDS} 
\end{equation*}

Aufgrund dessen ist es möglich den Truncation Error des Flux-Blending-Verfahrens aus
den vorangegenagen Betrachtungen zu UDS- und CDS-Verfahren herzuleiten.
Dabei werden Die Abbruchfehler der einzelnen Verfahren ebenso über $\beta$ gewichtet.

\begin{equation}
  TE_{e, Flux} = (1-\beta) TE_{e, UDS} + \beta TE_{e, CDS}
\end{equation}



\section{Truncation Error eines Kontrollvolumens}
\label{sec:Truncation Error eines Kontrollvolumens}

Der Truncation Error für ein Kontrollvolumen setzt sich nun aus den Fehlern von Quell-
und Diffusionstermen zusammen:

\begin{equation*}
  TE = TE_{source} - TE_e - TE_w
\end{equation*}

Für den äquidistanten Fall ergibt sich damit für zentrale Kontrollvolumen der folgende
Truncation Error. Wichtig ist es, hier die durch den Gauß'schen Integralsatz 
entstehenden Vorzeichen mit zu beachten.

\begin{align}
  TE &= \frac{\Delta x}{24} \left({f_E-2f_P+f_W}\right)
   +\frac{1}{24\Delta x}\left({
\phi_{EE}-3\phi_E+3\phi_P-\phi_W}\right)
  -\frac{1}{24 \Delta x}\left({
\phi_E-3\phi_P+3\phi_W-\phi_{WW}}\right)
\end{align}

\begin{figure}[h]
\begin{tikzpicture}
    \begin{axis}[
        height=7cm,
        width=0.9*\textwidth,
        grid=major,
        xlabel=$x$,
        ylabel=TE,
%scaled y ticks = false,
    ]
    \addplot[tud2d, mark=*] file {data/te_20_aequi.txt};
    \end{axis}
\end{tikzpicture}
\caption{Truncation Error}
\end{figure}

\section{Randwerte}
\label{sec:Randwerte}

Für die Auswertung der Ableitungen an Randpunkten ist es nicht möglich zentrale Differenzenquotienten
zu finden, da man dazu Punkte außerhalb des betrachteten Gebietes einbeziehen müsste.

Stattdessen erstellt man für die Randpunkte Interpolationspolynome und leitet diese ab.
Sie können dann einfach an den gewünschten Punkten ausgewertet werden.

Will man eine zweite Ableitung diskretisieren, so muss das Interpolationspolynom
ebensfalls die Ordnung 2 haben, um genügend Ableitungen zur Verfügung zu stellen.
Verallgemeinert bedeutet dies, dass für eine n-te Ableitung ein Polynom vom Grad n
aufgestellt werden muss.

Das Aufstellen der Polynome kann zum Beispiel über die Lagrange-Formel erfolgen.
Für Polynomgrade größer zwei ist es sinnvoll ein CAS-System zur Berechnung zu verwenden.
Ein möglicher Weg soll hier für das Programmpaket Maxima aufgezeigt werden.

\begin{lstlisting}
  load(interpol);
  data: [[a, fa], [b, fb], [c, fc]];
  poly: lagrange(data);
  d_poly: diff(poly, x);
  
\end{lstlisting}

\end{document}

  \section{Manufactured Solution}



Wir wählen die folgende Lösung für $\phi$ und berechnen die Ableitungen.

\begin{align*}
  \phi(x) &= A + \sin(Bx)\\
  \phi'(x) &= B \cos(Bx)\\
  \phi''(x) &= -B^2\sin(Bx)
\end{align*}

Nun führen wir den zusätzlichen Quellterm $f_{ad}$ ein und setzen $\phi(x)$ ein.

\begin{equation*}
  \alpha\frac{\partial^2 \phi}{\partial x^2}-f=f_{ad}
\end{equation*}

so erhält man:

\begin{equation*}
  f_{ad} = -\alpha B^2\sin(BX)-f
\end{equation*}

Damit ergibt sich als Manufactured Solution:
\begin{align}
  \alpha\frac{\partial^2 \phi}{\partial x^2} &= f -\alpha B^2\sin(Bx)-f\\
                                             &= -\alpha B^2\sin(Bx)
\end{align}

\subsection{Randbedingungen}
\label{sec:Randbedingungen}

Um die Konstanten $A$ und $B$ in der Manufactured Solution bestimmen zu können,
müssen konkrete Randbedingungen für das Problem festgelegt werden. Beispielhaft
soll das hier mit den Randwerten $\phi(0) = 1$ und $\phi(1) = 1$ geschehen.
Damit ergeben sich die Konstanten zu $A = 1$ sowie $B = \pi$. Die 
konstruierte Lösung $\phi(x)$ hat damit die folgende Form und der Quellterm $f$
ergibt sich zu:
\begin{equation*}
  \phi(x) = 1 + \sin(\pi x)
\end{equation*}
\begin{equation}
  f=-\pi^2 \sin(\pi x)
  \label{eq:quellterm}
\end{equation}

\subsection{Konvergenzuntersuchung}
\label{sec:Konvergenzuntersuchung}

Der Fehler $E$ wird definiert als Differenz zwischen diskreter Lösung $f(\Delta)$ sowie
der exakten Lösung $f_{exakt}$.
\begin{equation*}
  E=f(\Delta) - f_{exakt}
\end{equation*}

Für finite Methoden der Ordnung $p$ sollte sich der Fehler $DE$ proportional zu
$h^p$ verhalten. Daraus folgt mit dem Proportionalitätskoeffizienten $C$:

\begin{equation*}
  DE=f(\Delta) - f_{exakt}=C h^p + HOT
\end{equation*}

Wird nun die Gitterweite $h$ systematisch verkleinert, so lässst sich der beobachtete
Genauigkeitsgrad $p$ wie folgt berechnen.

\begin{equation}
  p=\frac{\ln \left(\frac{DE_2}{DE_1}\right)}{\ln \left(\frac{h_2}{h_1}\right)}
\end{equation}


  %\cleardoublepage




  \chapter{Ableitung des Abbruchfehlers}
  \documentclass[11pt, ngerman,colorback,accentcolor=tud2d]{tudreport}
\usepackage[utf8]{inputenc}
\usepackage[ngerman]{babel}
\usepackage{amsmath}

\usepackage{microtype}

\usepackage{pgfplots}
\pgfplotsset{compat=1.3}

\usepackage{listings}
\begin{document}

\newcommand{\pder}[2][]{\frac{\partial#1}{\partial#2}}
\newcommand{\pderf}[1]{\frac{\partial f}{\partial#1}}
\newcommand{\pderfs}[1]{\frac{\partial^2 f}{\partial#1}}



\chapter{Truncation Error}
\label{cha:Truncation_Error}

\section{Quellterm 1D}
\label{sec:Quellterm}

Die Taylorreihenentwicklung einer Funktion $f(x)$ um den Punkt $x_0$
liefert:

\begin{equation*}
  f(x) = f(x_0) + f'(x_0)(x-x_0) + \frac{1}{2} f''(x_0)(x-x_0)^2 + HOT
\end{equation*}

Nach Integration im Intervall $a$ bis $b$ ergibt sich:

\begin{align*}
  \int_a^b f(x) dx &= \int_a^b f(x_0) dx + \int_a^b f'(x_0)(x-x_0) dx
+ \int_a^b \frac{1}{2} f''(x_0)(x-x_0)^2 dx + HOT\\
&= f(x_0) (b-a) + f'(x_0) \frac{(b-x_0)^2-(a-x_0)^2}{2}
+ \frac{1}{2} f''(x_0) \frac{(b-x_0)^3-(a-x_0)^3}{3} +HOT
\end{align*}

Die Anwendung auf ein Kontrollvolumen mit den Randpunkten $x_e$ und
$x_w$ sowie dem Mittelpunkt $x_P$ ergibt damit die folgende Formel:
3.7118037236e-04
\begin{equation*}
  \int_{x_w}^{x_e} f(x)dx = f_P(x_e-x_w)
+ \frac{1}{2} f'_P \left[{(x_e-x_P)^2-(x_w-x_P)^2}\right]
+ \frac{1}{6} f''_P \left[{{(x_e-x_P)}^3-{(x_w-x_P)}^3}\right] + HOT
\end{equation*}

Hier müssen nun die auftretenden Ableitungen durch einen Differenzenquotienten
beschrieben werden.

\begin{align*}
  f'_P &=\frac{f_E-f_W}{x_E-x_W} \\
  f''_P &= \frac{1}{x_e-x_w}\left(\frac{f_E-f_P}{x_E-x_P}-\frac{f_P-f_W}{x_P-x_W}\right)
\end{align*}

Der Truncation Error der Diskretisierung des Quelltermes lässt sich damit über die
folgende Gleichung beschrieben.

\begin{align}
  TE_{source} &=
 \frac{f_E-f_W}{2(x_E-x_W)}  \left[{(x_e-x_P)^2-(x_w-x_P)^2}\right]\nonumber\\
&+\frac{1}{6(x_e-x_w)}\left(\frac{f_E-f_P}{x_E-x_P}-\frac{f_P-f_W}{x_P-x_W}\right)
  \left[{{(x_e-x_P)}^3-{(x_w-x_P)}^3}\right] + HOT
\end{align}

\subsection{Äquidistante Gitter}

Für äquidistante Gitter vereinfacht sich der Term und es ergibt sich:

\begin{equation*}
  TE_{source} = \frac{\Delta x}{24} \left({f_E-2f_P+f_W}\right)
\end{equation*}

\subsection{Randkontrollvolumen}

\paragraph{Westrand}

%Mit der Interpolation:

%\begin{equation}
  %\phi_e = \phi_P \frac{x_E-x_e}{x_E-x_P} + \phi_E \frac{x_e-x_P}{x_E-x_P}
%\end{equation}

\begin{align*}
  f'_P &=\frac{f_e-f_w}{x_e-x_w}\\
  f''_P &= \frac{1}{x_e-x_w}\left(\frac{f_E-f_P}{x_E-x_P}-\frac{f_P-f_w}{x_P-x_w}\right)
\end{align*}

\begin{align}
  TE_{source, W-Rand} &=
 \frac{f_e-f_w}{2(x_e-x_w)}  \left[{(x_e-x_P)^2-(x_w-x_P)^2}\right]\nonumber\\
&+\frac{1}{6(x_e-x_w)}\left(\frac{f_E-f_P}{x_E-x_P}-\frac{f_P-f_w}{x_P-x_w}\right)
  \left[{{(x_e-x_P)}^3-{(x_w-x_P)}^3}\right] + HOT
\end{align}

\begin{equation*}
  TE_{source, W-Rand, äquidistant} = \frac{\Delta x}{24} \left({f_E-3f_P+2f_w}\right)
\end{equation*}


\paragraph{Ostrand}

\begin{align*}
  f'_P &=\frac{f_e-f_w}{x_e-x_w}\\
  f''_P &= \frac{1}{x_e-x_w}\left(\frac{f_e-f_P}{x_e-x_P}-\frac{f_P-f_W}{x_P-x_W}\right)
\end{align*}

\begin{align}
  TE_{source, E-Rand} &=
 \frac{f_e-f_w}{2(x_e-x_w)}  \left[{(x_e-x_P)^2-(x_w-x_P)^2}\right]\nonumber\\
&+\frac{1}{6(x_e-x_w)}\left(\frac{f_e-f_P}{x_e-x_P}-\frac{f_P-f_W}{x_P-x_W}\right)
  \left[{{(x_e-x_P)}^3-{(x_w-x_P)}^3}\right] + HOT
\end{align}

\begin{equation*}
  TE_{source, E-Rand, äquidistant} = \frac{\Delta x}{24} \left({2f_e-3f_P+f_W}\right)
\end{equation*}

\section{Quellterm 2D}

Bei gleichem Vorgehen wie in \ref{sec:Quellterm} beschrieben, ergibt sich für
eine zweidimensionale Funktion $f(x, y)$ der folgende Ausdruck:

\begin{align*}
  \int_{y_s}^{y_n}\int_{x_w}^{x_e} f dx dy &= f_P (x_e-x_w)(y_n-y_s)\\
                                           &+ \pderf{x} \frac{(x_e-x_P)^2 - (x_w-x_P)^2}{2} (y_n-y_s)
  + \pderf{y} \frac{(y_n-y_P)^2-(y_s-y_P)^2}{2} (x_e-x_w) \\
  &+ \frac{1}{2} \pderfs{x^2}\frac{(x_e-x_P)^3 - (x_w-x_P)^3}{3} (y_n-y_s)
  + \frac{1}{2} \pderfs{y^2} \frac{(y_n-y_P)^3-(y_s-y_P)^3}{3} (x_e-x_w) \\
  &+ \pderfs{x\partial y} \frac{(x_e-x_P)^2 - (x_w-x_P)^2}{2} \cdot  \frac{(y_n-y_P)^2-(y_s-y_P)^2}{2} + HOT
\end{align*}

Der Wert von $f$ ist im Mittelpunkt des Kontrollvolumens bekannt, nicht aber die auftretenden
Ableitungen von $f$. Diese müssen deshalb diskretisiert werden. Der Fehler beträgt damit:

\begin{align*}
  TE_{source} &= \frac{f_E-f_W}{x_E-x_W} \frac{(x_e-x_P)^2 - (x_w-x_P)^2}{2} (y_n-y_s)\\
              &+ \frac{f_N-f_S}{y_N-y_S}  \frac{(y_n-y_P)^2-(y_s-y_P)^2}{2} (x_e-x_w) \\
              &+ \frac{1}{2}
  \left[{\frac{f_E-f_P}{(x_E-x_P)(x_e-x_w)}-\frac{f_P-f_W}{(x_P-x_W)(x_e-x_w)}  }\right]
  \frac{(x_e-x_P)^3 - (x_w-x_P)^3}{3} (y_n-y_s)\\
  &+ \frac{1}{2}
  \left[{\frac{f_N-f_P}{(y_N-y_P)(y_n-y_s)}-\frac{f_P-f_S}{(y_P-y_S)(y_n-y_s)}  }\right]
  \frac{(y_n-y_P)^3-(y_s-y_P)^3}{3} (x_e-x_w) \\
  &+ \left[{\frac{f_{NE}-f_{SE}}{(y_{NE}-y_{SE})(x_E-x_W) }- \frac{f_{NW}-f_{SW}}{(y_{NW}-y_{SW})(x_E-x_W)} }\right]\\
  &\cdot \frac{(x_e-x_P)^2 - (x_w-x_P)^2}{2} \cdot \frac{(y_n-y_P)^2-(y_s-y_P)^2}{2} + HOT
\end{align*}

\subsection{Äquidistante Gitter}

Für äquidistante Gitter vereinfacht sich der Term zu:

\begin{equation}
  \frac{1}{24} \Delta x \Delta y \left[{f_E+f_W+f_N+f_S - 4f_P}\right]
\end{equation}

\end{document}

  \documentclass[11pt, ngerman,colorback,accentcolor=tud2d]{tudreport}
\usepackage[utf8]{inputenc}
\usepackage[ngerman]{babel}
\usepackage{amsmath}

\usepackage{microtype}
\usepackage[ngerman,pdfview=FitH,pdfstartview=FitV]{hyperref}

\usepackage{pgfplots}
\pgfplotsset{compat=newest}

\begin{document}

\newcommand{\pder}[2][]{\frac{\partial#1}{\partial#2}}
\newcommand{\pderf}[1]{\frac{\partial f}{\partial#1}}
\newcommand{\pderfs}[1]{\frac{\partial^2 f}{\partial#1}}

\tableofcontents


\chapter{Truncation Error}
\label{cha:Truncation_Error}
\section{Diffusionsterme}
\label{sec:Diffusionsterme}


Um den Truncation Error der Diffusionsterme zu bestimmen, leiten wir die Differenzenquotienten
für die Ableitungen erster Ordnung aus den Taylorreihendarstellungen her. Beispielhaft
soll das anhand des östlichen Randes des Kontrollvolumens geschehen.

Zuerst entwickeln wir die Taylordarstellungen vom Punkt $x_e$ aus in Richtung der anliegenden
Kontrollvolumenmittelpunkte.

\begin{align}
  \phi_E &= \phi_e + \phi'_e(x_E-x_e)+\frac{1}{2}\phi''_e(x_E-x_e)^2
  +\frac{1}{6}\phi'''_e(x_E-x_e)^3+HOT
  \label{eq:taylor_eE}\\
  \phi_P &= \phi_e + \phi'_e(x_P-x_e)+\frac{1}{2}\phi''_e(x_P-x_e)^2
  +\frac{1}{6}\phi'''_e(x_P-x_e)^3+HOT
  \label{eq:taylor_eP}
\end{align}

Wird nun Gleichung~\eqref{eq:taylor_eP} von Gleichung~\eqref{eq:taylor_eE} subtrahiert, 
so ergibt sich:

\begin{equation*}
  \phi_E-\phi_P=\phi'_e(x_E-x_P)+
  \frac{1}{2}\phi''_e\left[{{(x_E-x_e)}^2-{(x_P-x_e)}^2}\right]+
  \frac{1}{6}\phi'''_e\left[{{(x_E-x_e)}^3-{(x_P-x_e)}^3}\right]+HOT
\end{equation*}

Nach Umstellen ergibt sich daraus für die Ableitung $\phi'_e$ der folgende Term:

\begin{equation}
  \phi'_e = \frac{\phi_E-\phi_P}{x_E-x_P}+\frac{1}{2}\phi''_e
\left({\frac{{(x_P-x_e)}^2-{(x_E-x_e)}^2}{x_E-x_P}}\right)+
\frac{1}{6} \phi'''_e \left({\frac{{(x_P-x_e)}^3-{(x_E-x_e)}^3}{(x_E-x_P)}}\right)+HOT
\end{equation}


Die hier auftretenden Ableitungen $\phi''_e$ und $\phi'''_e$ sind nicht bekannt und 
müssen diskretisiert werden. Mit der Nutzung von
möglichst lokale Differenzenquotienten ergeben sich die folgenden Ausdrücke.

\begin{align*}
  \phi''_e &= \frac{1}{(x_E-x_P)}\left({
\frac{\phi_{EE}-\phi_P}{x_{EE}-x_P}-\frac{\phi_E-\phi_W}{x_E-x_W}}\right)\\
\phi'''_e &= \frac{1}{(x_E-x_P)}\left({
\frac{1}{(x_{ee}-x_e)}
\left({\frac{\phi_{EE}-\phi_E}{x_{EE}-x_E}-\frac{\phi_E-\phi_P}{x_E-x_P} }\right)
-\frac{1}{(x_e-x_w)}
\left({\frac{\phi_E-\phi_P}{x_E-x_P} - \frac{\phi_P-\phi_W}{x_P-x_W}  }\right)
}\right)
\end{align*}


Der Truncation Error des Diffusionsterms an der östlichen Grenze des Kontrollvolumens
lässt sich damit über folgende Gleichung beschreiben.

\begin{align*}
  {TE}_e &= \frac{1}{2 (x_E-x_P)}\left({
\frac{\phi_{EE}-\phi_P}{x_{EE}-x_P}-\frac{\phi_E-\phi_W}{x_E-x_W}}\right) \left({\frac{{(x_P-x_e)}^2-{(x_E-x_e)}^2}{x_E-x_P}}\right)\\
&+
\left({
\frac{1}{(x_{ee}-x_e)}
\left({\frac{\phi_{EE}-\phi_E}{x_{EE}-x_E}-\frac{\phi_E-\phi_P}{x_E-x_P} }\right)
-\frac{1}{(x_e-x_w)}
\left({\frac{\phi_E-\phi_P}{x_E-x_P} - \frac{\phi_P-\phi_W}{x_P-x_W}  }\right)
}\right)\\
&\frac{1}{6(x_E-x_P)}\left({\frac{{(x_P-x_e)}^3-{(x_E-x_e)}^3}{(x_E-x_P)}}\right)
+HOT
\end{align*}

Der Truncation Error im Westen ergibt sich äquivalent zu:

\begin{align*}
  \phi_P &= \phi_w + \phi'_w(x_P-x_w)+\frac{1}{2}\phi''_w(x_P-x_w)^2
  +\frac{1}{6}\phi'''_w(x_P-x_w)^3+HOT\\
  \phi_W &= \phi_w + \phi'_w(x_W-x_w)+\frac{1}{2}\phi''_w(x_W-x_w)^2
  +\frac{1}{6}\phi'''_w(x_W-x_w)^3+HOT
\end{align*}

\begin{equation}
  \phi'_w = \frac{\phi_P-\phi_W}{x_P-x_W}+\frac{1}{2}\phi''_w
\left({\frac{{(x_W-x_w)}^2-{(x_P-x_w)}^2}{x_P-x_W}}\right)+
\frac{1}{6} \phi'''_w \left({\frac{{(x_W-x_w)}^3-{(x_P-x_w)}^3}{(x_P-x_W)}}\right)+HOT
\end{equation}

\begin{align*}
  \phi''_w &= \frac{1}{(x_P-x_W)}\left({
\frac{\phi_{E}-\phi_W}{x_{E}-x_W}-\frac{\phi_P-\phi_{WW}}{x_P-x_{WW}}}\right)\\
 \phi'''_w &= \frac{1}{(x_P-x_W)}\left({
\frac{1}{(x_e-x_w)}
\left({\frac{\phi_E-\phi_P}{x_E-x_P}-\frac{\phi_P-\phi_W}{x_P-x_W} }\right)
-\frac{1}{(x_w-x_{ww})}
\left({\frac{\phi_P-\phi_W}{x_P-x_W} - \frac{\phi_W-\phi_{WW}}{x_W-x_{WW}}  }\right)
}\right)
\end{align*}

\begin{align*}
  TE_w &= \frac{1}{2 (x_P-x_W)} \left({
\frac{\phi_{E}-\phi_W}{x_{E}-x_W}-\frac{\phi_P-\phi_{WW}}{x_P-x_{WW}}}\right)
  \left({\frac{{(x_W-x_w)}^2-{(x_P-x_w)}^2}{x_P-x_W}}\right)\\
&+
\left({
\frac{1}{(x_e-x_w)}
\left({\frac{\phi_E-\phi_P}{x_E-x_P}-\frac{\phi_P-\phi_W}{x_P-x_W} }\right)
-\frac{1}{(x_w-x_{ww})}
\left({\frac{\phi_P-\phi_W}{x_P-x_W} - \frac{\phi_W-\phi_{WW}}{x_W-x_{WW}}  }\right)
}\right)\\
&\frac{1}{6(x_P-x_W)}\left({\frac{{(x_W-x_w)}^3-{(x_P-x_w)}^3}{(x_P-x_W)}}\right)
  +HOT
\end{align*}

\subsection{Äquidistante Gitter}

Für äquidistante Gitter vereinfache sich die Terme des Truncation Error. So löschen
sich beispielsweise die quadratischen Terme gegenseitig aus. Es ergeben sich die
folgenden Fehler:

\begin{align*}
  {TE}_e &= \left({
\frac{1}{6\Delta x^2}
\left({\frac{\phi_{EE}-\phi_E}{\Delta x}-\frac{\phi_E-\phi_P}{\Delta x} }\right)
-\frac{1}{6\Delta x^2}
\left({\frac{\phi_E-\phi_P}{\Delta x} - \frac{\phi_P-\phi_W}{\Delta x}  }\right)
}\right)\left({-\frac{\Delta x^2}{4} }\right)+HOT\\
&= -\frac{1}{24\Delta x}\left({
\phi_{EE}-3\phi_E+3\phi_P-\phi_W}\right)+HOT
\end{align*}

\begin{align*}
  TE_w &=\left({
\frac{1}{6 \Delta x^2}
\left({\frac{\phi_E-\phi_P}{\Delta x}-\frac{\phi_P-\phi_W}{\Delta x} }\right)
-\frac{1}{6\Delta x^2}
\left({\frac{\phi_P-\phi_W}{\Delta x} - \frac{\phi_W-\phi_{WW}}{\Delta x}  }\right)
}\right)
\left({-\frac{\Delta x^2}{4} }\right)+HOT\\
&= -\frac{1}{24 \Delta x}\left({
\phi_E-3\phi_P+3\phi_W-\phi_{WW}}\right)+HOT
\end{align*}


\subsection{Kontrollvolumen am Rand}


Wie man sieht, werden für die Differenzenquotienten die Funktionswerte benachbarter
Kontrollvolumen benötigt. Deshalb müssen die Approximationen der Ableitung für
Kontrollvolumen am Rand angepasst werden.

\paragraph{Westlicher Rand}

Es ergeben sich für gegebenes $\phi_w$:


\begin{equation*}
  \phi''_{w, W-Rand} = \frac{1}{(x_P-x_w)}\left({
\frac{\phi_{e}-\phi_w}{x_{e}-x_w}-\frac{\phi_P-\phi_w}{x_P-x_w}}\right)
\end{equation*}

Da bei der dritten Ableitung zwei linke Kontrollvolumen genutzt werden, müssen hier
vom Kontrollvolumen am Rand sowohl der westliche also auch der östliche Rand
betrachtet werden.

\begin{align*}
  \phi'''_{w, W-Rand} &= \frac{1}{(x_p-x_w)} \left({
  \frac{1}{(x_e-x_w)} \left({
    \frac{\phi_E-\phi_P}{x_E-x_P} - \frac{\phi_P-\phi_w}{x_P-x_w}
    }\right) -
  \frac{1}{x_P-x_w} \left({
    \frac{\phi_e-\phi_w}{x_e-x_w} - \frac{\phi_P-\phi_w}{x_P-x_w}
    }\right)
  }\right)
  \\
  \phi'''_{e, W-Rand} &= \frac{1}{(x_E-x_P)} \left({
  \frac{1}{(x_{ee}-x_e)} \left({
      \frac{\phi_{EE}-\phi_E}{x_{EE}-x_E} - \frac{\phi_E-\phi_P}{x_E-x_P}
    }\right) -
  \frac{1}{x_e-x_w} \left({
    \frac{\phi_E-\phi_P}{x_E-x_P} - \frac{\phi_P-\phi_w}{x_P-x_w}
    }\right)
  }\right)
\end{align*}

Hier wird noch der Wert $\phi_e$ benutzt, der aber nicht bekannt ist. Er wird deshalb
durch eine lineare Interploation von $\phi_E$ und $\phi_P$ bestimmt.

\begin{equation}
  \phi_e = \phi_P \frac{x_E-x_e}{x_E-x_P} + \phi_E \frac{x_e-x_P}{x_E-x_P}
\end{equation}

Damit ergibt sich für $\phi''_{w,Rand}$und $\phi'''_{w, Rand}$:

\begin{align}
  \phi''_{w, W-Rand} &= \frac{1}{(x_P-x_w)}\left({
\frac{
  \left({\phi_P \frac{x_E-x_e}{x_E-x_P} + \phi_E \frac{x_e-x_P}{x_E-x_P}
}\right)
-\phi_w}{x_{e}-x_w}-\frac{\phi_P-\phi_w}{x_P-x_w}}\right)\\
  \phi'''_{w, W-Rand} &= \frac{1}{(x_p-x_w)} \left({
  \frac{1}{(x_e-x_w)} \left({
    \frac{\phi_E-\phi_P}{x_E-x_P} - \frac{\phi_P-\phi_w}{x_P-x_w}
    }\right) -
  \frac{1}{x_P-x_w} \left({
    \frac{ \phi_P \frac{x_E-x_e}{x_E-x_P} + \phi_E \frac{x_e-x_P}{x_E-x_P}
-\phi_w}{x_e-x_w} - \frac{\phi_P-\phi_w}{x_P-x_w}
    }\right)
  }\right)
\end{align}


\paragraph{Östlicher Rand}
Äquivalent ergibt sich $\phi''_{e, E-Rand}$ bei gegebenem $\phi_e$:

\begin{align*}
  \phi''_{e, E-Rand} &= \frac{1}{(x_e-x_P)}\left({
\frac{\phi_{e}-\phi_P}{x_{e}-x_P}-\frac{\phi_e-\phi_w}{x_e-x_w}}\right)
\\
  \phi'''_{e, E-Rand} &= \frac{1}{(x_e-x_P)} \left({
  \frac{1}{(x_e-x_P)} \left({
    \frac{\phi_e-\phi_P}{x_e-x_P} - \frac{\phi_e-\phi_w}{x_e-x_w}
    }\right) -
  \frac{1}{x_e-x_w} \left({
    \frac{\phi_e-\phi_P}{x_e-x_P} - \frac{\phi_P-\phi_W}{x_P-x_W}
    }\right)
  }\right)
  \\
  \phi'''_{w, E-Rand} &= \frac{1}{(x_P-x_W)} \left({
  \frac{1}{(x_e-x_w)} \left({
      \frac{\phi_e-\phi_P}{x_e-x_P} - \frac{\phi_P-\phi_W}{x_P-x_W}
    }\right) -
    \frac{1}{x_w-x_{ww}} \left({
        \frac{\phi_P-\phi_W}{x_P-x_W} - \frac{\phi_W-\phi_{WW}}{x_W-x_{WW}}
    }\right)
  }\right)
\end{align*}

$\phi_w$ wird wie folgt linear interpoliert:

\begin{equation*}
  \phi_w = \phi_W \frac{x_P-x_w}{x_P-x_W} + \phi_P \frac{x_w-x_W}{x_P-x_W}
\end{equation*}

Damit ergibt sich für $\phi''_{e,Rand}$:

\begin{align}
  \phi''_{e,Rand} &= \frac{1}{(x_e-x_P)}\left({
\frac{\phi_{e}-\phi_P}{x_{e}-x_P}-\frac{\phi_e-
  \left({
   \phi_W \frac{x_P-x_w}{x_P-x_W} + \phi_P \frac{x_w-x_W}{x_P-x_W}
  }\right)
}{x_e-x_w}}\right)
\\
  \phi'''_{e, E-Rand} &= \frac{1}{(x_e-x_P)} \left({
  \frac{1}{(x_e-x_P)} \left({
    \frac{\phi_e-\phi_P}{x_e-x_P} - \frac{\phi_e- \phi_W \frac{x_P-x_w}{x_P-x_W} + \phi_P \frac{x_w-x_W}{x_P-x_W}
}{x_e-x_w}
    }\right) -
  \frac{1}{x_e-x_w} \left({
    \frac{\phi_e-\phi_P}{x_e-x_P} - \frac{\phi_P-\phi_W}{x_P-x_W}
    }\right)
  }\right)
\end{align}


\subsubsection{Äquidistante Gitter}

\paragraph{Westlicher Rand}

\begin{align}
  \phi''_{w, W-Rand} &= \frac{1}{\Delta x^2} (\phi_E-3\phi_P+2\phi_w)\\
  \phi'''_{w, W-Rand} &= 0\\
  \phi'''_{e, W-Rand} &= \frac{1}{\Delta x^3} (\phi_{EE} -3\phi_E + 4\phi_P -2\phi_w)
\end{align}

\paragraph{Östlicher Rand}

\begin{align}
  \phi''_{e, E-Rand} &= \frac{1}{\Delta x^2} (\phi_W-3\phi_P+2\phi_e)\\
  \phi'''_{e, E-Rand} &= 0\\
  \phi'''_{w, E-Rand} &= \frac{1}{\Delta x^3} (-\phi_{WW} +3\phi_W - 4\phi_P +2\phi_e)
\end{align}

\section{Diffusion 2D}

Äquivalent zu den Herleitungen des Truncation Error im Osten und Westen ergeben sich
sich für die Ableitungen $\phi'_n$ und $\phi'_s$ die folgenden Terme:

\begin{equation}
  \phi'_n = \frac{\phi_N-\phi_P}{y_N-y_P}+\frac{1}{2}\phi''_n
\left({\frac{{(y_P-y_n)}^2-{(y_N-y_n)}^2}{y_N-y_P}}\right)+
\frac{1}{6} \phi'''_n \left({\frac{{(y_P-y_n)}^3-{(y_N-y_n)}^3}{y_N-y_P}}\right)+HOT
\end{equation}


\begin{equation}
  \phi'_s = \frac{\phi_P-\phi_S}{y_P-y_S}+\frac{1}{2}\phi''_s
\left({\frac{{(y_S-y_s)}^2-{(y_P-y_s)}^2}{y_P-y_S}}\right)+
\frac{1}{6} \phi'''_s \left({\frac{{(y_S-y_s)}^3-{(y_P-y_s)}^3}{y_P-y_S}}\right)+HOT
\end{equation}

Nach Diskretisierung der auftretenden, unbekannten Ableitungen ergeben sich die folgenden
Terme für den Truncation Error:

\begin{align*}
  {TE}_n &= \frac{1}{2 (y_N-y_P)}\left({
\frac{\phi_{NN}-\phi_P}{y_{NN}-y_P}-\frac{\phi_N-\phi_S}{y_N-y_S}}\right) \left({\frac{{(y_P-y_n)}^2-{(y_N-y_n)}^2}{y_N-y_P}}\right)\\
&+
\left({
\frac{1}{(y_{nn}-y_n)}
\left({\frac{\phi_{NN}-\phi_N}{y_{NN}-y_N}-\frac{\phi_N-\phi_P}{y_N-y_P} }\right)
-\frac{1}{(y_n-y_s)}
\left({\frac{\phi_N-\phi_P}{y_N-y_P} - \frac{\phi_P-\phi_S}{y_P-y_S}  }\right)
}\right)\\
&\frac{1}{6(y_N-y_P)}\left({\frac{{(y_P-y_n)}^3-{(y_N-y_n)}^3}{y_N-y_P}}\right)
+HOT
\end{align*}


\begin{align*}
  TE_s &= \frac{1}{2 (y_P-y_S)} \left({
\frac{\phi_{N}-\phi_S}{y_{N}-y_S}-\frac{\phi_P-\phi_{SS}}{y_P-y_{SS}}}\right)
  \left({\frac{{(y_S-y_s)}^2-{(y_P-y_s)}^2}{y_P-y_S}}\right)\\
&+
\left({
\frac{1}{(y_n-y_s)}
\left({\frac{\phi_N-\phi_P}{y_N-y_P}-\frac{\phi_P-\phi_S}{y_P-y_S} }\right)
-\frac{1}{(y_s-y_{ss})}
\left({\frac{\phi_P-\phi_S}{y_P-y_S} - \frac{\phi_S-\phi_{SS}}{y_S-y_{SS}}  }\right)
}\right)\\
&\frac{1}{6(y_P-y_S)}\left({\frac{{(y_S-y_s)}^3-{(y_P-y_s)}^3}{y_P-y_S}}\right)
  +HOT
\end{align*}

\subsection{Äquidistante Gitter}

\begin{align*}
  {TE}_n &= \left({
\frac{1}{6\Delta y^2}
\left({\frac{\phi_{NN}-\phi_N}{\Delta y}-\frac{\phi_N-\phi_P}{\Delta y} }\right)
-\frac{1}{6\Delta y^2}
\left({\frac{\phi_N-\phi_P}{\Delta y} - \frac{\phi_P-\phi_S}{\Delta y}  }\right)
}\right)\left({-\frac{\Delta y^2}{4} }\right)+HOT\\
&= -\frac{1}{24\Delta y}\left({
\phi_{NN}-3\phi_N+3\phi_P-\phi_S}\right)+HOT
\end{align*}

\begin{align*}
  TE_s &=\left({
\frac{1}{6 \Delta y^2}
\left({\frac{\phi_N-\phi_P}{\Delta y}-\frac{\phi_P-\phi_S}{\Delta y} }\right)
-\frac{1}{6\Delta y^2}
\left({\frac{\phi_P-\phi_S}{\Delta y} - \frac{\phi_S-\phi_{SS}}{\Delta y}  }\right)
}\right)
\left({-\frac{\Delta y^2}{4} }\right)+HOT\\
&= -\frac{1}{24 \Delta y}\left({
\phi_N-3\phi_P+3\phi_S-\phi_{SS}}\right)+HOT
\end{align*}


\end{document}

  \section{Abbruchfehler von Konvektionstermen}

Wie in Abschnitt~\ref{sec:konv_fluss} beschrieben, existieren verschiedene Verfahren
um konvektive Flüsse abzuschätzen. Der Abbruchfehler ist vom genutzten Verfahren
abhängig und muss für jedes Verfahren hergeleitet werden.

Zu diesem Fehler kommt ab der zweiten Dimension wiederum der Abbruchfehler aus
der Integralapproximation, der bereits in Abschnitt~\ref{sec:source1d} hergeleitet wurde.

\subsection{CDS-Verfahren}
\label{sec:te_cds}

Die Herleitung des Abbruchfehlers für das Zentraldifferenzen-Verfahren sei hier am Beispiel des östlichen Randes dargestellt.
Man entwickelt zuerst die Taylorreihe von $\phi$ um den Entwicklungspunkt $x_P$ und wertet sie anschließend
an den Punkten $x_e$ und $x_E$ aus. Es ergeben sich die folgenden Reihendarstellungen:
\begin{align}
  \phi_e &= \phi_P + \phi'_P(x_e-x_P)+\frac{1}{2}\phi''_P(x_e-x_P)^2
  +\frac{1}{6}\phi'''_P(x_e-x_P)^3+HOT
  \label{eq:taylor_konv_eP}\\
  \phi_E &= \phi_P + \phi'_P(x_E-x_P)+\frac{1}{2}\phi''_P(x_E-x_P)^2
  +\frac{1}{6}\phi'''_P(x_E-x_P)^3+HOT
  \label{eq:taylor_konv_eE}
\end{align}
Werden die Gleichungen~\ref{eq:taylor_konv_eE} und \ref{eq:taylor_konv_eP} nun
voneinander subtrahiert und nach $\phi_e$ umgestellt, so ergibt sich mit dem
Interpolationsfaktor $\gamma_e$:

\begin{equation*}
  \begin{IEEEeqnarraybox}[][c]{rCl}
    \frac{\phi_e}{x_e-x_P} &=& \frac{\phi_E}{x_E-x_P} + \frac{\phi_P}{x_e-x_P} -
  \frac{\phi_P}{x_E-x_P} + \frac{1}{2} \phi''_P \left({(x_e-x_P)-(x_E-x_P)}\right)\\
  &&+ \frac{1}{6} \phi'''_P \left({(x_e-x_P)^2-(x_E-x_P)^2}\right)+HOT\\
  \phi_e &=& \phi_E \frac{x_e-x_P}{x_E-x_P} + \phi_P \left({1-\frac{x_e-x_P}{x_E-x_P} }\right)
  + \frac{1}{2} \phi''_P (x_e-x_E)(x_e-x_P)\\
  &&+ \frac{1}{6} \phi'''_P \left({(x_e-x_P)^2-(x_E-x_P)^2}\right)(x_e-x_P)+HOT\\
   &=& \phi_E \gamma_e + \phi_P (1-\gamma_e)+ \frac{1}{2} \phi''_P (x_e-x_E)(x_e-x_P)\\
         &&+ \frac{1}{6} \phi'''_P \left({(x_e-x_P)^2-(x_E-x_P)^2}\right)(x_e-x_P)+HOT
  \end{IEEEeqnarraybox}
\end{equation*}
Der Abbruchfehler lässt sich hier leicht ablesen. Er lautet:
\begin{equation*}
  TE_{e, CDS} =  \frac{1}{2} \phi''_P (x_e-x_E)(x_e-x_P)+ \frac{1}{6}
  \phi'''_P \left({(x_e-x_P)^2-(x_E-x_P)^2}\right)(x_e-x_P)+HOT
\end{equation*}
Die hier vorliegenden analytischen Ableitungen werden im nächsten Schritt durch Approximationen ersetzt.
\begin{align*}
  %\phi'_P &= \frac{\phi_E-\phi_W}{x_E-x_W}\\
  \phi''_P &= \frac{1}{(x_e-x_w)} \left({\frac{\phi_E-\phi_P}{x_E-x_P}
  - \frac{\phi_P-\phi_W}{x_P-x_W} }\right)\\
  %\label{eq:ddphip}\\
  \phi'''_P &= \frac{1}{(x_e-x_w)} \left({
  \frac{1}{(x_E-x_P)} \left({\frac{\phi_{EE}-\phi_P}{x_{EE}-x_P}- \frac{\phi_E-\phi_W}{x_E-x_W} }\right)-
  \frac{1}{(x_P-x_W)} \left({\frac{\phi_E-\phi_W}{x_E-x_W} - \frac{\phi_P-\phi_{WW}}{x_P-x_{WW}} }\right)
  }\right) \label{eq:dddphip}
\end{align*}
Der Abbruchfehler ergibt sich damit zu folgendem Term.
\begin{equation}
  \begin{IEEEeqnarraybox}{rCl}
    TE_{CDS,\,e} &=&  \frac{1}{2} \frac{1}{(x_e-x_w)} \left({\frac{\phi_E-\phi_P}{x_E-x_P}
  - \frac{\phi_P-\phi_W}{x_P-x_W} }\right) (x_e-x_E) (x_e-x_P) \nonumber \\
  &&+
 \frac{1}{(x_e-x_w)} \bigg(
  \frac{1}{(x_E-x_P)} \left({\frac{\phi_{EE}-\phi_P}{x_{EE}-x_P}- \frac{\phi_E-\phi_W}{x_E-x_W} }\right)
  \\&&-
  \frac{1}{(x_P-x_W)} \left({\frac{\phi_E-\phi_W}{x_E-x_W} - \frac{\phi_P-\phi_{WW}}{x_P-x_{WW}} }\right)
  \bigg) \nonumber \\
  &&\cdot \frac{1}{6} \left({(x_e-x_P)^2-(x_E-x_P)^2}\right)(x_e-x_P)+HOT
  \end{IEEEeqnarraybox}
\end{equation}
Im Westen wird die Ableitung äquivalent zum Osten durchgeführt. Für den Abbruchfehler ergibt sich letztendlich
folgender Term:
%\begin{align*}
  %\begin{IEEEeqnarraybox}{rCl}
  %\phi_w &= \left({1-\frac{x_w-x_P}{x_W-x_P}}\right)\phi_P + \left({\frac{x_w-x_P}{x_W-x_P} }\right) \phi_W
  %+ \frac{1}{2} \phi''_P \left({(x_w-x_P)-(x_W-x_P)}\right)(x_w-x_P)\\
  %&+ \frac{1}{6} \phi'''_P \left({(x_w-x_P)^2-(x_W-x_P)^2}\right)(x_w-x_P)\\
  %\phi_w &= \left({1-\frac{x_P-x_w}{x_P-x_W}}\right)\phi_P + \left({\frac{x_P-x_w}{x_P-x_W} }\right) \phi_W
  %+ \frac{1}{2} \phi''_P \left({x_w-x_W}\right)(x_w-x_P)\\
  %&+ \frac{1}{6} \phi'''_P \left({(x_w-x_P)^2-(x_W-x_P)^2}\right)(x_w-x_P)
  %\end{IEEEeqnarraybox}
%\end{align*}

%Der Abbruchfehler lässt sich nach der Diskretisierung der unbekannten Ableitungen nach Gleichung~\ref{eq:ddphip}
%und \ref{eq:dddphip} leicht ablesen und ergibt sich zu:

\begin{equation}
  \begin{IEEEeqnarraybox}[][b]{rCl}
    TE_{CDS,\,w} &=&  \frac{1}{2} \frac{1}{(x_e-x_w)} \left({\frac{\phi_E-\phi_P}{x_E-x_P}
  - \frac{\phi_P-\phi_W}{x_P-x_W} }\right) \left({x_w-x_W}\right)(x_w-x_P)  \nonumber\\
  &&+ \frac{1}{(x_e-x_w)} \bigg(
  \frac{1}{(x_E-x_P)} \left({\frac{\phi_{EE}-\phi_P}{x_{EE}-x_P}- \frac{\phi_E-\phi_W}{x_E-x_W} }\right)
  \\ && -
  \frac{1}{(x_P-x_W)} \left({\frac{\phi_E-\phi_W}{x_E-x_W} - \frac{\phi_P-\phi_{WW}}{x_P-x_{WW}} }\right)
  \bigg) \nonumber\\
  &&\frac{1}{6}  \left({(x_w-x_P)^2-(x_W-x_P)^2}\right)(x_w-x_P)
  \end{IEEEeqnarraybox}
\end{equation}

\subsubsection{Randwerte}
\label{sec:te_cds_rand}

Bei Kontrollvolumen direkt am Rand, wie zum Beispiel das in Abbildung~\ref{fig:kv1d_rand}
hervorgehobene Kontrollvolumen, müssen einige Besonderheiten beachtet werden.
So ist der konvektive Fluss, der direkt auf dem Rand verläuft, durch die Randbedingung ($\phi^w$) gegeben und
muss dementsprechend nicht angepasst werden. Am östlichen Rand muss hingegen die Approximation
der Ableitungen angepasst werden. Sie lauten dann:
\begin{align*}
  \phi''_P &= \frac{1}{(x_e-x_w)} \left({\frac{\phi_E-\phi_P}{x_E-x_P}
  - \frac{\phi_P-\phi^w}{x_P-x_w} }\right)\\
  \phi'''_P &= \frac{1}{(x_e-x_w)} \left({
  \frac{1}{(x_E-x_P)} \left({\frac{\phi_{EE}-\phi_P}{x_{EE}-x_P}- \frac{\phi_E-\phi^w}{x_E-x_w} }\right)-
  \frac{1}{(x_P-x_w)} \left({\frac{\phi_E-\phi^w}{x_E-x_w} - \frac{\phi_P-\phi^{w}}{x_P-x_{w}} }\right)
  }\right)
\end{align*}
Hat das betrachtete Kontrollvolumen  einen Abstand von eins zum Rand, so muss nur die Approximation der dritten
Ableitung angepasst werden.
\begin{equation*}
  \phi'''_P = \frac{1}{(x_e-x_w)} \left({
  \frac{1}{(x_E-x_P)} \left({\frac{\phi_{EE}-\phi_P}{x_{EE}-x_P}- \frac{\phi_E-\phi^w}{x_E-x_w} }\right)-
  \frac{1}{(x_P-x_W)} \left({\frac{\phi_E-\phi_W}{x_E-x_W} - \frac{\phi_P-\phi^{w}}{x_P-x_{w}} }\right)
  }\right)
\end{equation*}
Auf gleiche Art und Weise kann auch der östliche Rand behandelt werden.
%Hier muss abschließend $\phi_e$ aus $\phi_E$ und $\phi_P$ interpoliert werden. Letztendlich ergibt
%sich für äquidistante Gitter folgender Term:

%\begin{equation*}
  %TE_{e, CDS} = -\frac{1}{8} (\phi_E-2\phi_P+\phi_W) - \frac{1}{32}
  %(2\phi^{ee} - 3\phi_E + 2\phi_W - \phi_{WW})
%\end{equation*}

%Auf gleiche Art und Weise kann auch $TE_{w, CDS}$ berechnet werden.
\subsubsection{Äquidistante Gitter}

Für äquidistante Gitter ergeben sich die folgenden vereinfachten Formeln zur Berechnung der konvektiven Flüsse
mit dem CDS-Verfahren.

\begin{equation*}
  TE_{CDS,\,e} = -\frac{1}{8} (\phi_E-2\phi_P+\phi_W) - \frac{1}{32}
  (\phi_{EE} - 2\phi_E + 2\phi_W - \phi_{WW})
\end{equation*}
\begin{equation*}
  TE_{CDS,\,w} = -\frac{1}{8} (\phi_E-2\phi_P+\phi_W) + \frac{1}{32}
  (\phi_{EE} - 2\phi_E + 2\phi_W - \phi_{WW})
\end{equation*}








\subsection{UDS-Verfahren}

Um den Abbruchfehler des Upwind-Verfahrens herzuleiten, wertet man die
Taylorreihe von $\phi$ mit dem Entwicklungspunkt $x_P$ im Punkte $\phi_e$
aus.

\begin{equation*}
  \phi_e = \phi_P +(x_e-x_P) \phi'_P + \frac{1}{2} (x_e-x_P)^2 \phi''_P+HOT
\end{equation*}
Da im Falle einer positiven Geschwindigkeit das UDS-Verfahren $\phi_e$ mit $\phi_P$
gleichsetzt, lässt sich der Abbruchfehler hier direkt ablesen.
\begin{equation*}
  TE_{UDS,\,e} = (x_e-x_P) \phi'_P + \frac{1}{2} (x_e-x_P)^2 \phi''_P+HOT
\end{equation*}
Setzt man nun die diskretisierten Ableitungen aus Abschnitt~\ref{sec:te_cds} ein,
ergibt sich die Formel für den Abbruchfehler.
\begin{equation}
  TE_{UDS,\,e} = (x_e-x_P) \frac{\phi_E-\phi_W}{x_E-x_W}+
  \frac{1}{2} \frac{(x_e-x_P)^2}{(x_e-x_w)} \left({\frac{\phi_E-\phi_P}{x_E-x_P}
  - \frac{\phi_P-\phi_W}{x_P-x_W} }\right)+HOT
\end{equation}
Betrachtet man den westlichen Rand und geht von einer positiven Geschwindigkeit aus,
so ergibt sich der folgende Abbruchfehler:
\begin{equation*}
  TE_{UDS,\,w} = (x_w-x_W) \phi'_W + \frac{1}{2} (x_w-x_W)^2 \phi''_W + HOT
\end{equation*}
Hier werden nun wiederum die Ableitungen $\phi'_W$ und $\phi''_W$ diskretisiert. Der
Abbruchfehler lautet damit:
\begin{equation}
  TE_{UDS,\,w} = (x_w-x_W) \frac{\phi_P-\phi_{WW}}{x_P-x_{WW}}+
  \frac{1}{2} \frac{(x_w-x_W)^2}{(x_w-x_{ww})} \left({\frac{\phi_P-\phi_W}{x_P-x_W}
  - \frac{\phi_W-\phi_{WW}}{x_W-x_{WW}} }\right)+HOT
\end{equation}



\subsubsection{Randwerte}

Auch beim UDS-Verfahren müssen am Rand die Approximationen der verwendeten Ableitungen
so angepasst werden, dass nur vorhandene Werte benutzt werden.
Für den östlichen Rand des Kontrollvolumens können dabei die in Abschnitt~\ref{sec:te_cds_rand}
gezeigten Differenzenquotienten genutzt werden.

Für das Kontrollvolumen direkt am Rand ist der konvektive Fluss gegeben. Es gibt somit keinen
Abbruchfehler. Für das daneben liegende Kontrollvolumen müssen die Differenzenquotienten wie folgt angepasst werden:
\begin{align*}
  \phi'_W &= \frac{\phi_w-\phi^w}{x_w-x_{ww}}\\
  \phi''_W &= \frac{1}{x_W-x_{ww}} \left({\frac{\phi_w-\phi^w}{x_w-x_{ww}}
- \frac{\phi_W-\phi^w}{x_W-x_{ww}} }\right)+HOT
\end{align*}
$\phi_w$ wird hier beispielsweise aus der linearen Interpolation von $\phi_P$ und $\phi_W$ berechnet.
\begin{equation*}
  \phi_w = \phi_W \frac{x_P-x_w}{x_P-x_W} + \phi_P \frac{x_w-x_W}{x_P-x_W}
\end{equation*}
%etzt ergeben sich damit die folgenden Terme.
%\begin{align}
  %\phi'_W &= \frac{\phi_W \frac{x_P-x_w}{x_P-x_W} + \phi_P \frac{x_w-x_W}{x_P-x_W}
%-\phi^w}{x_w-x_{ww}}\\
  %\phi''_W &= \frac{1}{x_W-x_{ww}} \left({\frac{\phi_W \frac{x_P-x_w}{x_P-x_W} + \phi_P \frac{x_w-x_W}{x_P-x_W}
%-\phi^w}{x_w-x_{ww}}
%- \frac{\phi_W-\phi^w}{x_W-x_{ww}} }\right)+HOT
%\end{align}


\subsubsection{Äquidistante Gitter}

%\begin{align*}
  %\phi'_W  &= \frac{1}{2\Delta x} \phi_P + \frac{1}{2\Delta x} \phi_W - \phi^w\\
  %\phi''_W &= \frac{1}{\Delta x^2} \phi_P -\frac{1}{\Delta x^2} \phi_W
%\end{align*}
Für äquidistante Gitter ergeben sich folgende Gleichungen für den Abbruchfehler.

\begin{equation*}
  TE_{UDS,\,e} = \frac{3}{8} \phi_E-\frac{1}{4} \phi_P - \frac{1}{8} \phi_W
\end{equation*}
\begin{equation*}
  TE_{UDS,\,w} = \frac{3}{8} \phi_P-\frac{1}{4} \phi_W - \frac{1}{8} \phi_{WW}
\end{equation*}





\subsection{``Flux-Blending''-Verfahren}

Das Flux-Blending-Verfahren setzt sich aus UDS- und CDS-Verfahren zusammen. Beide
Verfahren werden dabei über den Faktor $\beta$ gewichtet. Dementsprechend müssen auch die
Abbruchfehler über $\beta$ gewichtet werden um den Abbruchfehler für das ``Flux-Blending''-Verfahren zu berechnen.
Mit Gleichung~\ref{eq:flux_blending} ergibt sich damit:

\begin{equation}
  TE_{e, Flux} = (1-\beta) TE_{e, UDS} + \beta TE_{e, CDS}
\end{equation}



%\section{Truncation Error eines Kontrollvolumens}
%\label{sec:Truncation Error eines Kontrollvolumens}

%Der Truncation Error für ein Kontrollvolumen setzt sich nun aus den Fehlern von Quell-
%und Diffusionstermen zusammen:

%\begin{equation*}
  %TE = \frac{TE_{source} - TE_e - TE_w}{\Delta x}
%\end{equation*}
%Für den äquidistanten Fall ergibt sich damit für zentrale Kontrollvolumen der folgende
%Truncation Error. Wichtig ist es, hier die durch den Gauß'schen Integralsatz 
%entstehenden Vorzeichen mit zu beachten. Weiterhin muss durch $\Delta x$ geteilt werden,
%da die gleiche Transformation beim Aufstellen des Gesamtgleichungsystem angewendet wird.

%\begin{align}
  %TE &= \frac{\frac{\Delta x}{24} \left({f_E-2f_P+f_W}\right)
   %+\frac{1}{24\Delta x}\left({
%\phi_{EE}-3\phi_E+3\phi_P-\phi_W}\right)
  %-\frac{1}{24 \Delta x}\left({
%\phi_E-3\phi_P+3\phi_W-\phi_{WW}}\right)}{\Delta x}
%\end{align}

\subsection{Zweidimensionale Probleme}
Die gezeigten Herleitungen lassen sich problemlos ins Zweidimensionale übertragen. Es muss
jedoch darauf geachtet werden, den Fehler aus der Integralapproximation nicht zu vergessen.
Dieser lautet für die Ostseite:
\begin{equation}
  TE_{C,\,int,\,e} = \frac{1}{6} \frac{\partial^2\phi}{\partial x^2}\bigg\vert_e
    \left[{{(y_n-y_P)}^3-{(y_s-y_P)}^3}\right] + HOT
\end{equation}
Er lässt sich leicht auf die anderen Seiten des Kontrollvolumens übertragen. Die Diskretisierung
der analytischen Ableitung wurde ebenfalls bereits gezeigt.
\clearpage


  \chapter{Auswertung des Abbruchfehlers}
  \section{Diffusionsprobleme}
  \section{Konvektionsprobleme}
  \subsection{UDS}
  \subsection{CDS}
  \subsection{Flux-Blending}

  \chapter{Fazit}


\appendix

\chapter{Quellcodes}

\begin{tikzpicture}[scale=1]
\coordinate[label=left:$A$] (A) at (0,0);
\coordinate[label=right:$B$] (B) at (4,1);
\draw (A) -- (B);
\fill (A) circle (2pt);
\fill (B) circle (2pt);
\end{tikzpicture}




%\lstinputlisting{create_plot_3d_data.m}
%\twocolumn
%\lstinputlisting{comb2d_cds_orth.m}

  \cleardoublepage

\bibliographystyle{gerplain}
\bibliography{bib}

\nocite{*}

\end{document}
