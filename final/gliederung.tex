\documentclass[bigchapter,twoside,report,11pt,type=bsc,colorback,accentcolor=tud2c]{tudthesis}
\usepackage{ngerman}
\usepackage[utf8]{inputenc}
\usepackage[ngerman]{babel}
\usepackage{amsmath}

\usepackage{microtype}
\usepackage[ngerman,pdfview=FitH,pdfstartview=FitV]{hyperref}

\usepackage{pgfplots}
\pgfplotsset{compat=newest}

%\KOMAoptions{cleardoublepage=realempty}

\usepackage{bibgerm}
\usepackage{listings}
\usepackage{inconsolata}
%\usepackage[T1]{fontenc}
\lstset{framextopmargin=50pt,frame=bottomline}
\lstset{language=Matlab,
   %keywords={break,case,catch,continue,else,elseif,end,for,function,
      %global,if,otherwise,persistent,return,switch,try,while},
   basicstyle=\ttfamily,
   breaklines=true,
   keywordstyle=\color{tud1c},
   commentstyle=\color{tud9c},
   stringstyle=\color{tud4c},
   numbers=left,
   numberstyle=\tiny\color{tud0c},
   stepnumber=1,
   numbersep=5pt,
   backgroundcolor=\color{white},
   tabsize=4,
   showspaces=false,
   showstringspaces=false}

\usepackage{tikz}


\newcommand{\getmydate}{%
  \ifcase\month%
    \or Januar\or Februar\or M\"arz%
    \or April\or Mai\or Juni\or Juli%
    \or August\or September\or Oktober%
    \or November\or Dezember%
  \fi\ \number\year%
}
\newcommand{\pder}[2][]{\frac{\partial#1}{\partial#2}}
\newcommand{\pderf}[1]{\frac{\partial f}{\partial#1}}
\newcommand{\pderfs}[1]{\frac{\partial^2 f}{\partial#1}}


\begin{document}
  %\thesistitle{Ableitung und Untersuchung des Abbruchfehlersch"atzers f"ur mittels
  %Finite Volumen diskretisierten Navier-Stokes Gleichungen\linebreak[1]}%
    %{Derivation and examination of the truncation error for finite volume discretised Navier-Stokes equations}
  %\author{Paul Lange}
  %\birthplace{Darmstadt}
  %\referee{Clemens v. Loewenich}{Ulrich Falk}
  %\department{Fachbereich Maschinenbau}
  %\group{Institut f"ur Numerische Berechnungsverfahren}
  %\dateofexam{\today}{\today}
  %\tuprints{12345}{1234}
  %\makethesistitle
  %\affidavit{P. Lange}

  %\newpage
  \tableofcontents
  %\cleardoublepage

  \chapter{Einleitung}
  \section{Motivation}
Durch steigende Computerleistung sowie Verbesserungen in den numerischen Methoden konnten
sich die simulationsbasierten Produktentwicklungsprozesse einen festen Platz neben
Experimenten erarbeiten. Neben Kostenvorteilen ist es vor allem die Möglichkeit
schnell Aussagen über Anpassungen und Modifikationen an Produkten treffen zu können, die
simulationsbasierten Methoden zum Wachstum verhelfen.

Für die ingenieurmäßige Berechnung von Strömungsproblemen hat sich die Methode der
finiten Volumen auf breitem Gebiet durchgesetzt. Ihr Fokus liegt hier auf der Simulation
der Navier-Stokes Gleichungen, oft auch unter Vernachlässigung kompressibler Strömungseffekte und
anderen zulässigen Vereinfachungen.

Um die Genauigkeit der Ergebnisse zu verbessern werden immer feinere Gebiete aufgelöst,
bei denen auch heutige Computersysteme an ihre Grenzen stoßen. Es ist deshalb notwendig
auf adaptive Methoden zurückzugreifen, die die Auflösung an interessanten Stellen wie
zum Beispiel Grenzschichten oder besonders interessanten Strömungsgebieten selektiv erhöhen können.
Weiterhin sollen Fehler, die durch die numerische Behandlung der kontinuierlichen Strömungsgebiete
entstehen, minimiert werden. Dafür ist es notwendig, Indikatoren für den Fehler zu definieren.

Heutige Methoden schätzen diesen Fehler ab, negieren aber seine Ursachen bei der Diskretisierung
und damit dem Übergang von der kontinuierlichen Differentialgleichung zur diskreten Gleichung.
Hier liegt der Ansatzpunkt der Fehlerbetrachtung über den Abbruchfehler. Dieser wird in der
vorliegenden Arbeit für Transportprobleme hergeleitet und seine Genauigkeit an Beispielen
verifiziert. Weiterhin wird er zur Gitteradaption eingesetzt.


\section{Stand der Forschung}

Der Abbruchfehler als Indikator für die Gitteradaption ist in der Literatur bereits bekannt.
Roy~\cite{roy2} stellt für die eindimensionale Burger-Gleichung mittels
Finiten-Differenzen her und vergleicht anschließend den abbruchfehlerbasierten Fehlerindikator
mit lösungsbasierten Indikatoren wie Gradient oder Krümmung. Weiterhin wird gezeigt,
dass für lineare Probleme der Abbruchfehler
als Quelle des Diskretisierungsfehlers interpretiert werden kann und
eine Gitterverfeinerung basierend auf dem Abbruchfehler
somit die Auflösung an der Quelle des Diskretisierungsfehler vergrößert.
Damit verkleinert sich dann der gesamte Diskretisierungsfehler.

Vor der Gitteradaption muss eine Lösung auf dem Problemgebiet berechnet werden. Diese
Lösung beziehungsweise das Lösungsprogramm muss verifiziert werden. Veluri~\cite{veluri} demonstriert
Methoden zur Verifikation von Programm über den Vergleich von formaler und beobachteter
Konvergenzordnung und stellt außerdem die Methode der konstruierten Lösung vor.
Auch Roache~\cite{roache} stellt diese vor und demonstriert ihre Nützlichkeit bei der Verifikation von
Lösungsprogrammen.

Der Einfluss nicht-orthogonaler Gitter auf die Lösungsgenauigkeit gehen Huang und Prosperetti~\cite{grid_ortho}
ein. Lee und Tsuei~\cite{lee} zeigen die Ableitung des Abbruchfehlers bei Finite-Differenzen Verfahren
auf nicht-orthogonalen Gittern. Die Transformation von Kontrollgebieten der Finite-Elemente Methode
wird von Fath~\cite{fath} gezeigt.

\section{Zielsetzung}

Im ersten Teil der Arbeit werden grundlegende Konzepte und Begriffe der Arbeit erläutert.
Anschließend wird der Abbruchfehler für alle Terme von skalaren Transportgleichungen
abgeleitet. Dabei werden ein- und zweidimensionale Probleme sowie kartesische, orthogonale
und nicht-orthogonale Gitter betrachtet.

Anschließend wird der abgeleitete Abbruchfehlerindikator an konkreten Testfällen verifiziert.
Die Funktionsweise und Genauigkeit der dazu nötigen Lösungsprogramme wird ebenfalls
untersucht. Abschließend wird dann der Abbruchfehler zur lokalen Gitteradaption genutzt
und hierbei mit anderen bekannten Verfahren verglichen.



\cleardoublepage


\chapter{Grundlagen}
\section{Skalare Probleme}

Eine Vielzahl praktisch relevanter Probleme lässt sich durch skalare (partielle)
Differential\-glei\-chungen beschreiben. Anwendungsgebiete reichen hier von Wärmetransportproblemen
über strukturmechanische Probleme (zum BEispiel die Auslenkung von Stäben und Membranen bei elastischem
Werkstoffverhalten) bis hin zu strömungsmechanischen Problemen (zum Beispiel das Geschwindigkeitspotential
wirbelfreier Strömungsfelder).

\subsection{Einfache Feldprobleme}

Einige kontinuumsmechanische Problemstellungen können durch eine Differentialgleichung
der Form

\begin{equation}
  -\frac{\partial}{\partial x_i}\left({\alpha \frac{\partial \phi}{\partial x_i} }\right)=g
  \label{eq:feldproblem_stat}
\end{equation}
 beschrieben werden. Diese muss auf dem gesamten betrachteten Gebiet $V$ gelten, das
vom Rand $S$ begrenzt wird. Bei $\phi=\phi(\mathbf{x})$ handelt es sich um die gesuchte,
skalare Feldgröße. $g=g(\mathbf{x})$ und $\alpha=\alpha(\mathbf{x})$ sind vorgegeben.


Weiterhin müssen auf dem gesamten Rand $S$ Randbedingungen
gegeben sein. Üblich sind hierbei folgende Typen von Randbedingungen:

\begin{itemize}
  \item Dirichletsche Randbedingung: $\quad \phi=\phi_S$
  \item Neumannsche Randbedingung: $\quad \alpha \frac{\partial \phi}{\partial x_i} n_i = b_s $
  \item Cauchysche Randbedingung: $\quad c_S \phi + \alpha \frac{\partial \phi}{\partial x_i} n_i = b_s$
\end{itemize}

Hierbei sind $\phi_S$, $b_S$ und $c_S$ vorgegebene Funktionen auf dem Rand $S$. Die
Komponenten des nach außen gerichteten Normaleneinheitsvektors an $S$ werden mit $n_i$ bezeichnet.

Treten mehrere Randbedingungstypen in einem Problem auf spricht man von gemischten Rand\-wert\-problemen.

Die mit Gleichung~\ref{eq:feldproblem_stat} beschriebenen Probleme beinhalten keine Zeitabhängigkeit
und werden deshalb als stationär bezeichnet.
Im instationären, also zeitabhängigem Fall, erhalten alle Größen neben der Ortsabhängigkeit
eine Zeitabhängigkeit. Die entsprechende Differentialgleichung lautet damit
\begin{equation}
  \frac{\partial \phi}{\partial t}
  -\frac{\partial}{\partial x_i}\left({\alpha \frac{\partial \phi}{\partial x_i} }\right)=g
  \label{eq:feldproblem_instat}
\end{equation}
für die unbekannte Größe $\phi=\phi(\mathbf{x}, t)$.
Für instationäre Probleme muss neben den Rand\-beding\-ungen auch eine Anfangsbedingung
$\phi(\mathbf{x}, t_0) = \phi_0(\mathbf{x})$ gegeben werden.

Im weiteren Verlauf der Arbeit werden nur stationäre Probleme betrachtet. Alle Ergebnisse
lassen sich aber auf instationäre Probleme übertragen.

\subsection{Allgemeine Transportgleichung}
\label{sec:transportgl}

Eine wichtige Problemklasse innerhalb des Maschinenbaus stellen Transportprobleme in
Festkörpern oder Fluiden dar. Bei der transportierten Größe kann es sich dabei beispielsweise
um joulsche Wärme (Wärmetransportprobleme) oder Stoffmengen (Stofftransportprobleme) handeln.

Stationäre Transportprobleme können in Differentialform durch die Gleichung
\begin{equation}
  \frac{\partial}{\partial x_i} \left({\rho v_i \phi
- \alpha \frac{\partial \phi}{\partial x_i} }\right) = f
\label{eq:transportgl}
\end{equation}
beschrieben werden. Die auftretenden Terme werden wie folgt benannt:
\begin{itemize}
  \item Quellterm $\quad f$
  \item Konvektionsterm $\quad\frac{\partial}{\partial x_i} \rho v_i \phi$
  \item Diffusionsterm $\quad\alpha \frac{\partial^2 \phi}{\partial x_i \partial x_i}$
\end{itemize}
Für spezielle Probleme müssen die entsprechenden Parameter angepasst werden
(für nähere Information siehe \cite{num_maschbau}).

Alle Betrachtungen in der vorliegenden Arbeit bauen auf Problemen auf, die durch
Transport\-gleich\-ungen beschrieben werden können.

\section{Navier-Stokes Gleichungen}

Für viele reale Probleme ist es nötig das Verhalten von Fluiden vorherzusagen.
Teilweise sind auch gleichzeitige Stoff- und Wärmetransportvorgänge interessant.
Die in praktisch relevanten Problemen am häufigsten auftretenden Fluide werden Newtonsche Fluide genannt.
Das bedeutet, dass sie linear-viskos und isotrop sind und damit durch den Cauchyschen
Spannungstensor $T$ beschrieben werden können, wobei $p$ für den Druck und $\mu$ für die dynamische Viskosität stehen.
\begin{equation}
  T_{ij} = \mu\left({\frac{\partial v_i}{\partial x_j}
  + \frac{\partial v_j}{\partial x_i}
-\frac{2}{3} \frac{\partial v_k}{\partial x_k} \delta_{ij}}\right)
-p\delta_{ij}
\end{equation}
Die
Erhaltungssätze für Masse, Impuls und innere Energie lauten mit diesem Materialgesetz dann:
\begin{align}
  \frac{\partial p}{\partial t} + \frac{\partial (\rho v_i)}{\partial x_i} &= 0\\
  \frac{\partial (\rho v_i)}{\partial t} + \frac{\partial (\rho v_i v_j)}{\partial x_j} &=
  \frac{\partial}{\partial x_j} \left[{\mu
  \left({\frac{\partial v_i}{\partial x_j}
  +\frac{\partial v_j}{\partial x_i}
  - \frac{2}{3} \frac{\partial v_k}{\partial x_k}\delta_{ij}}\right)}\right]
  -\frac{\partial p}{\partial x_i} + \rho f_i\\
  \frac{\partial (\rho e)}{\partial t} + \frac{\partial (\rho v_i e)}{\partial x_i} &=
  \mu \left[{\frac{\partial v_i}{\partial x_j}
  \left({\frac{\partial v_i}{\partial x_j}
  +\frac{\partial v_j}{\partial x_i}}\right)
  - \frac{2}{3} \left({\frac{\partial v_i}{\partial x_i}}\right)^2}\right]
  -p\frac{\partial v_i}{\partial x_i} +\frac{\partial}{\partial x_i}
  \left({\kappa \frac{\partial T}{\partial x_i}}\right)
  + \rho q
\end{align}
Neben diesen Erhaltungsgleichungen benötigt man weiterhin thermische und kalorische Zustands\-gleich\-ungen.
Diese definieren die thermodynamischen Eigenschaften des Fluids.

Bei
vielen realen Problemen müssen die obigen Gleichungen nicht in der allgemeinen Form gelöst werden, sondern
können durch zusätzliche Annahmen vereinfacht werden.
So wird beispielsweise bei vielen praktischen Anwendungen das Fluid nicht entscheidend komprimiert
und kann deshalb als inkompressibel angesehen werden. Als Kriterium gilt weithin,
das die Machzahl ($Ma = v/a$) kleiner~als~0,3 ist. Beispiele für inkompressible Probleme
sind Strömungen von Flüssigkeiten wie Rohrströmungen oder die Umströmung von Bootsrümpfen.
Auch die Strömung an langsam fliegenden Flugzeugen, zum Beispiel Segelflugzeugen, kann als
inkompressibel modelliert werden.

Ist die Bedingung für Inkompressibilität erfüllt,
vereinfachen sich die obigen Gleichungen für kompressible Fluide und es ergeben sich folgende Gleichungen:
\begin{align}
  \frac{\partial v_i}{\partial x_i} &= 0\\
  \frac{\partial (\rho v_i)}{\partial t} + \frac{\partial (\rho v_i v_j)}{\partial x_j} &=
  \frac{\partial}{\partial x_j} \left[{\mu
  \left({\frac{\partial v_i}{\partial x_j}
  +\frac{\partial v_j}{\partial x_i}}\right)}\right]
  -\frac{\partial p}{\partial x_i} + \rho f_i\\
  \frac{\partial (\rho e)}{\partial t} + \frac{\partial (\rho v_i e)}{\partial x_i} &=
  \mu \frac{\partial v_i}{\partial x_j}
  \left({\frac{\partial v_i}{\partial x_j}
  +\frac{\partial v_j}{\partial x_i}}\right)
  +\frac{\partial}{\partial x_i}
  \left({\kappa \frac{\partial T}{\partial x_i}}\right)
  + \rho q
\end{align}
\clearpage

  \section{Numerische Methoden}

Zum Erkenntnisgewinn in den Ingenieurwissenschaften stehen drei grundsätzlichen
Wege zur Verfügung \cite{num_maschbau}. Diese lauten:
\begin{itemize}
  \item Theoretische Methoden
  \item Experimentelle Versuche
  \item Numerische Simulationen
\end{itemize}

Theoretische Methoden, also insbesondere analytische Betrachtungen der das Problem
beschreibenden Gleichungen sind im allgemeinen nur sehr beschränkt möglich. Das heißt das
die Gleichungen, die zur Beschreibung realer Prozesse genutzt werden meist nur für
bestimmte Randbedingungen und unter bestimmten Vereinfachungen überhaupt analytisch lösbar
sind. Da jedoch diese Vereinfachungen in realen Prozessen nicht vernachlässigbar klein sind
sowie die geforderten Randbedingungen nicht erfüllbar oder präzise einhaltbar sind, ist es
für komplexere Problemstellungen unmöglich analytische Lösungen zu finden.

Die Intention bei experimentellen Untersuchungen ist über Versuche an Modellen oder
realen Bauteilen an die benötigten Systemgrößen heranzukommen. Diese Vorgehensweise
bereitet jedoch in vielen Fällen Probleme:
\begin{itemize}
  \item Messungen bestimmter Systemgrößen sind oft schwierig bis unmöglich. Gründe
    dafür können die Dimensionen der Objekte (z.B. Molekulare Prozesse, Weltmeere),
    die Geschwindigkeit der Zustandsänderung (z.B. Explosionen) oder moralische Gründe
    (z.B. Versuche an Mensch und Tieren, Versuche mit Gefahrenstoffen) sein.
  \item Experimente an Modellen lassen nur begrenzte Rückschlüsse auf das reale Objekt
    zu. So lassen sich beispielsweise Erkenntnisse aus dem Windkanal nur teilweise auf
    das reale Automobil oder Flugzeug übernehmen.
  \item Experimente sind oft teuer und zeitraubend. So muss beispielsweise um die Auswirkungen
    einer Änderung zu testen ein neues Modell bzw. Objekt hergestellt werden. In anderen Fällen
    ist das Modell nach dem Versuch zerstört (z.B. Crashtest). Generell ist auch der Betrieb
    von Messeinrichtungen teuer.
\end{itemize}

Aufgrund der Nachteile der oben genannten Verfahren setzt sich im Maschinenbau und der
Naturwissenschaft im allgemeinen die numerische Simulation immer weiter durch. Vorteile
gegenüber den anderen Verfahren liegen auf der Hand:

\begin{itemize}
  \item Simulationen sind meist schneller und kosteneffizienter zu erstellen als experimentelle
    Versuche.
  \item Änderungen am Objekt sowie Parameterstudien zur Optimierung lassen sich leicht
    erstellen.
  \item Die Ergebnisse der Simulation enthalten meist alle problemrelevanten Größen,
    deren Messung im Versuch mit viel Aufwand verbunden wäre.
\end{itemize}

Grundlegend für die Nutzung dieser Vorteile ist jedoch die Zuverlässigkeit der
berechneten Ergebnisse. Deshalb gehen mit der Verbesserung der numerischen Simulationsmethoden
immer auch experimentelle Validierung von Ergebnissen und verwendeten Modellen einher.

  \subsection{Numerische Gitter}

Sind das mathematische Modell mit seinen Differentialgleichungen sowie den nötigen
Rand- sowie gegebenenfalls Anfangsbedingungen festgelegt, besteht der nächste Schritt
bei der Anwendung eines numerischen Berechnungsverfahren darin, das kontinuierliche
Gebiet von Raum und ggf. Zeit durch eine endliche Menge von Teilgebieten zu approximieren,
in denen dann der Wert des gesuchten Größe berechnet wird.

Diese Teilgebiete werden im Allgemeinen in Form eines Gitters über den zu untersuchenden
Bereich $V$ gelegt, weshalb dieser Arbeitsschritt auch oft Gittergenerierung genannt wird.

Ein wichtiges Unterscheidungskriterium ist die Anordnung der Gitterpunkte.
Man unterscheidet grundsätzlich zwischen strukturierten und unstrukturierten Gittern.

Bei strukturierten Gittern ist die Lage der Gitterpunkte zueinander festgelegt. Diese
Lagebeziehungen müssen deshalb nicht aufwendig gespeichert werden und die Gittergenerierung
ist schnell und ohne großen Rechenaufwand nötig. Zwar ist es möglich bestimmte Gebiete
aus der Gittergenerierung auszuschließen, eine Anpassung an die Problemgeometrie ist
jedoch nur innerhalb der vorgegebenen Lagebeziehungen möglich.

Bei unstrukturierten Gittern hingegen gibt es keine Vorschriften zu den Lagebeziehungen
der einzelnen Gitterpunkte. Diese müssen deshalb aufwendig gespeichert werden, was
Gittergeneration sowie Lösung komplizierter macht. Andererseits ist es Problemlos möglich
das Gitter an der Problemgeometrie auszurichten oder in bestimmten Gebieten selektiv zu verfeinern.

\subsubsection{Physikalisches und logisches Gebiet}

Oftmals lässt die Geometrie des zu untersuchenden Problems keine Anwendung eines
strukturierten Gitters zu. Nicht in jeden Fall ist es jedoch nötig daraufhin unstrukturierte
Gitter zu verwenden. Vielmehr ist es möglich das physikalische Gebiet des Problems
mittels einer geeigneten mathematischen Transformation auf ein sogenanntes
logisches Gebiet abzubilden, auf welchem dann die Anwendung eines strukturierten Gitters
möglich ist.

Die Abbildung von physikalischem auf logisches Gebiet muss dabei umkehrbar sein, das heißt
die Abbildungsfunktion muss EINEINDEUTIG??? sein. Die Achsen des physikalischen Gebiets
werden im Allgemeinen mit lateinischen Buchstaben bezeichnet ($x$, $y$, $z$), während
für das logische Gebiet griechische Buchstaben verwendet werden ($\xi$, $\eta$, $\theta$).
Beispielhaft soll hier die formale, mathematische Beschreibung anhand eines ebenen
Beispiels gezeigt werden:
\begin{equation}
  f=f(x, y)\qquad \text{mit} \qquad x=x(\xi, \eta),\quad y=y(\xi, \eta)
\end{equation}



  \subsection{Eigenschaften}
  \subsection{Diskretisierungsmethoden}
  \section{Abbruchfehler}

Der Abbruchfehler ist der Fehler der beim Abschneiden einer unendlichen Summe
und deren Approximation durch eine endliche Summe entsteht.

In der Praxis werden Approximationen, wie die in Abschnitt~\ref{sec:konv_fluss} oder
\ref{sec:dif_fluss} beschriebenen, aus Taylorreihenentwicklungen um die betrachteten
Punkte gewonnen\footnote{Die Herleitungen sind in Kapitel~\ref{chap:herleitung} ausführlich ausgeführt.}. 
Da die Approximation jedoch zwangsläufig endlich sein muss, werden
dabei Glieder der unendlichen Taylorreihenentwicklung abgeschnitten. Der dadruch entstehende
Fehler wird Abbruchfehler (engl. Truncation Error) genannt.
Die restlichen, für die Approximation nicht genutzten Terme können zur Fehlerabschätzung
bei numerischen Berechnungen genutzt werden. Die grundlegenden Zusammenhänge sollen hier an einem einfachen Beispiel gezeigt werden.

Wir nehmen an, die Funktion $f=\sin(x)$ soll zur einfacheren Berechenbarkeit durch
ein Polynom approximiert werden. Anschließend soll ein Indikator für den dabei
entstehenden Fehler gefunden werden ohne dabei auf $f$
selbst zurückzugreifen.
Zuerst wird die Taylorreihe von $\sin(x)$ um den Entwicklungspunkt $x_0 =0$ berechnet:

\begin{equation}
  \sin(x) = \sum_{n=0}^{\infty}(-1)^n \frac{x^{2n+1}}{(2n+1)!} = 
  x-\frac{x^3}{3!} +\frac{x^5}{5!} -\frac{x^7}{7!} +\frac{x^9}{9!}\cdots
 \label{eq:taylor_example}
\end{equation}

Nun wird die Approximation für $\sin(x)$ festgelegt. Diese soll in diesem Beispiel
$x-\frac{x^3}{6} $ sein. In Abbildung~\ref{fig:taylor_example} sieht man, wie sich die Güte der
Approximation mit zunehmenden Abstand von $0$ verschlechtert.

\begin{figure}[h]
\begin{tikzpicture}
\begin{axis}[width=0.8*\textwidth, height=150pt, grid=major]
  \addplot[domain=-2*pi:2*pi, samples=100, color=tud2b, very thick]{sin(deg(x))};
  \addplot[domain=-3:3, samples=100, color=tud9b, very thick]{x - x*x*x/6};
  \legend{$\sin(x)$,$x-\frac{x^3}{6} $}
\end{axis}
\end{tikzpicture}
\centering
\caption{Vergleich von Funktion und Approximation}
 \label{fig:taylor_example}
\end{figure}

Der absolute Fehler kann nun einfach über die Formel $\sin(x) - x + \frac{x^3}{6}$
berechnet werden und entspricht den übrigen Gliedern der Taylorreihe~\eqref{eq:taylor_example}.
Der absolute Fehler entspricht also in diesem einfachen Beispiel dem Abbruchfehler.
Dieser wird nach seinem englischen Namen \textit{Truncation Error} mit $TE$ bezeichnet.
\begin{equation}
  TE = \frac{x^5}{5!} -\frac{x^7}{7!} +\frac{x^9}{9!}\cdots
 \label{eq:taylor_example_te}
\end{equation}
Da der Abbruchfehler selbst wieder eine unendliche Summe darstellt muss zur Auswertung
wieder eine endliche Anzahl von Termen abgetrennt werden. Dies stellt jedoch im allgemeinen
kein Problem dar, da der Einfluss der Terme mit steigendem Exponenten stark abnimmt.

Zur Abschätzund des Fehlers sollen im Beispiel zwei Glieder genutzt werden. Aus
Gleichung~\eqref{eq:taylor_example_te} ergibt sich damit der Fehlerindikator zu
$\frac{x^5}{5!} -\frac{x^7}{7!} $.
\begin{figure}[h]
\begin{tikzpicture}
\begin{axis}[width=0.8*\textwidth, height=150pt, grid=major, legend pos=north west]
  \addplot[domain=-6:6, samples=100, color=tud2b, very thick]{sin(deg(x))-x+x*x*x/6};
  \addplot[domain=-6:6, samples=100, color=tud9b, very thick]{x^5/120-x^7/(120*6*7)};
  \legend{Absoluter Fehler, Fehlerapproximation}
\end{axis}
\end{tikzpicture}
\centering
\caption{Vergleich des absoluten Fehlers mit der Fehlerschätzung}
 \label{fig:taylor_example_te}
\end{figure}
In Abbildung~\ref{fig:taylor_example_te} sieht man die gute Übereinstimmung von absolutem Fehler und Fehlerapproximation um den Entwicklungspunkt. Die Genauigkeit kann durch Hinzunahme weiterer Terme aus der Reihenentwicklung des Abbruchfehlers beliebig verbessert werden.

  \section{Manufactured Solution}
  %\cleardoublepage

  \chapter{Finite Volumen Methode}
  \section{Approximation von Volumenintegralen}
  \section{Approximation von Flächenintegralen}
  \section{Herleitung Diffusion}
  \section{Herleitung Konvektion}
  %\cleardoublepage


  \chapter{Ableitung des Abbruchfehlers}
  Um den Abbruchfehler eines Kontrollvolumens zu bestimmen, werden die
Abbruchfehler aller zu betrachtenden Terme addiert. Nach Gleichung~\eqref{eq:fvm3}
müssen demnach die Abbruchfehler für diffusive und konvektive Flüsse sowie
für Quellterme bestimmt werden.


\section{Abbruchfehler für Quellterme}
\label{sec:Quellterm}

\subsection{Eindimensionale Probleme}
\label{sec:source1d}
Die Taylorreihenentwicklung einer Funktion $f(x)$ um den Punkt $x_0$
liefert:
\begin{equation*}
  f(x) = f(x_0) + f'(x_0)(x-x_0) + \frac{1}{2} f''(x_0)(x-x_0)^2 + HOT
\end{equation*}
Nach Integration im Intervall $[a, b]$ ergibt sich:
\begin{align*}
  \int_a^b f(x) dx &= \int_a^b f(x_0) dx + \int_a^b f'(x_0)(x-x_0) dx
+ \int_a^b \frac{1}{2} f''(x_0)(x-x_0)^2 dx + HOT\\
&= f(x_0) (b-a) + f'(x_0) \frac{(b-x_0)^2-(a-x_0)^2}{2}
+ \frac{1}{2} f''(x_0) \frac{(b-x_0)^3-(a-x_0)^3}{3} +HOT
\end{align*}
Die Anwendung auf ein eindimensionales Kontrollvolumen (siehe Abbildung~\ref{fig:kv1d}) mit den Randpunkten $x_e$ und
$x_w$ sowie dem Mittelpunkt $x_P$ ergibt sich damit die folgende Formel:
\begin{equation*}
  \int_{x_w}^{x_e} f(x)dx = f_P(x_e-x_w)
  + \frac{1}{2} f'_P \underbrace{\left[{(x_e-x_P)^2-(x_w-x_P)^2}\right]}_{=0}
+ \frac{1}{6} f''_P \left[{{(x_e-x_P)}^3-{(x_w-x_P)}^3}\right] + HOT
\end{equation*}
Der sich der ersten Ableitung von $f$ anschließende Term wird immer Null, da nach
Definition der Finite-Volumen Methode der Mittelpunkt eines Kontrollvolumens immer
zwischen seinen Grenzen liegt.
Die verbleibende zweite Ableitung wird nun durch einen Differenzenquotienten
beschrieben.
\begin{equation}
  \label{eq:diskretisierung_f''P}
  f''_P = \frac{1}{x_e-x_w}\left(\frac{f_E-f_P}{x_E-x_P}-\frac{f_P-f_W}{x_P-x_W}\right)
\end{equation}
Der Truncation Error der Diskretisierung des Quelltermes lässt sich damit über die
folgende Gleichung beschreiben.
\begin{equation}
  TE_{source,1D} =
\frac{1}{6(x_e-x_w)}\left(\frac{f_E-f_P}{x_E-x_P}-\frac{f_P-f_W}{x_P-x_W}\right)
\left[{{(x_e-x_P)}^3-{(x_w-x_P)}^3}\right] + HOT\label{eq:te_source1}
\end{equation}

\begin{figure}[bt]

\begin{tikzpicture}[scale=0.8]
  %\draw[->] (6,0) -- (6.5,0) node[right] {$x$} coordinate(x axis);
  %\draw[->] (0,6) -- (0,6.5) node[above] {$y$} coordinate(y axis);
  \fill[tud0a] (2,2) rectangle +(2,2);
  \draw[step=2cm, thick] (0, 2) grid (6, 4);

  \fill (3, 3)  circle[radius=1.5pt];
  \node (N) at (3,3) [label=above:$P$]{};

  \fill (5, 3)  circle[radius=1.5pt];
  \node (N) at (5,3) [label=above:$E$]{};

  \fill (1, 3)  circle[radius=1.5pt];
  \node (N) at (1,3) [label=above:$W$]{};

  %\fill (7, 3)  circle[radius=1.5pt];
  %\node (N) at (7,3) [label=right:$EE$]{};

  %\fill (-1, 3)  circle[radius=1.5pt];
  %\node (N) at (-1,3) [label=left:$WW$]{};

  \fill (4, 3)  circle[radius=1.5pt];
  \node (N) at (3.84,3) [label={right=0pt}:$e$]{};

  \fill (2, 3)  circle[radius=1.5pt];
  \node (N) at (2.16,3) [label={left=0pt}:$w$]{};

  %\fill (6, 3)  circle[radius=1.5pt];
  %\node (N) at (6,3) [label=right:$ee$]{};

  %\fill (0, 3)  circle[radius=1.5pt];
  %\node (N) at (0,3) [label=left:$ww$]{};
\end{tikzpicture}

\centering
\caption{Allgemeines Kontrollvolumen im eindimensionalen Fall}
\label{fig:kv1d}
\end{figure}

\subsubsection{Kontrollvolumen am Rand}

Bei Kontrollvolumen am Rand sind nicht mehr alle in Gleichung~\eqref{eq:te_source1}
genutzten Werte gegeben. Der genutzte Differenzenquotient muss demnach angepasst werden.
Da man sich dabei auch von der Form des Zentraldifferenzenquotienten entfernt,
sinkt gleichzeitig die Genauigkeit in den Randgebieten.

Die Anpassung soll beispielhaft am westlichen Rand gezeigt werden. Der Differenzenquotient
ergibt sich bei Benutzung des mit $f^w$ bezeichneten Randwerts, der beispielsweise als
dirichletsche Randbedingung gegeben sein kann, zu:
\begin{equation}
  f''_P = \frac{1}{x_e-x_w}\left(\frac{f_E-f_P}{x_E-x_P}-\frac{f_P-f^w}{x_P-x_w}\right)
\end{equation}
Eingesetzt ergibt sich damit folgende Formel für den Abbruchfehler des Quellterms
für Kontrollvolumen am Rand.
\begin{equation}
  TE_{source, W-Rand} =
\frac{1}{6\,(x_e-x_w)}\left(\frac{f_E-f_P}{x_E-x_P}-\frac{f_P-f^w}{x_P-x_w}\right)
  \left[{{(x_e-x_P)}^3-{(x_w-x_P)}^3}\right] + HOT
\end{equation}
Mit der gleichen Vorgehensweise kann auch der der östliche Rand behandelt werden. Darauf
sei jedoch hier nicht weiter eingegangen.



\subsection{Zweidimensionale Probleme}


Bei gleichem Vorgehen wie in Abschnitt~\ref{sec:source1d} beschrieben, ergibt sich für
eine zweidimensionale Funktion $f(x, y)$ in einem in Abbildung~\ref{fig:kv2d} 
dargestelltem Kontrollvolumen der folgende Ausdruck:
\begin{equation}
  \begin{IEEEeqnarraybox}[][c]{rCl}
  \int_{y_s}^{y_n}\int_{x_w}^{x_e} f dx dy &=& f_P (x_e-x_w)(y_n-y_s)\\
  & &+ \pderf{x} \underbrace{\frac{(x_e-x_P)^2 - (x_w-x_P)^2}{2}}_{=0} (y_n-y_s)\\
  && + \pderf{y} \underbrace{\frac{(y_n-y_P)^2-(y_s-y_P)^2}{2}}_{=0} (x_e-x_w) \\
  & &+ \frac{1}{2} \pderfs{x^2}\frac{(x_e-x_P)^3 - (x_w-x_P)^3}{3} (y_n-y_s)\\
  &&+ \frac{1}{2} \pderfs{y^2} \frac{(y_n-y_P)^3-(y_s-y_P)^3}{3} (x_e-x_w) \\
  & &+ \pderfs{x\partial y} \underbrace{\frac{(x_e-x_P)^2 - (x_w-x_P)^2}{2}}_{=0} \cdot
  \underbrace{\frac{(y_n-y_P)^2-(y_s-y_P)^2}{2}}_{=0} + HOT
\end{IEEEeqnarraybox}
\end{equation}
Die sich den ersten Ableitungen anschließenden Terme ergeben bei Auswertung wiederum Null
und müssen deshalb nicht weiter beachtet werden.
Der Wert von $f$ ist im Mittelpunkt des Kontrollvolumens bekannt, nicht aber die auftretenden
Ableitungen von $f$. Diese müssen deshalb diskretisiert werden. Man kann hierbei auf
Gleichung~\eqref{eq:diskretisierung_f''P} zurückgreifen und ebenfalls für die $y$-Richtung anpassen.
Der Abbruchfehler ergibt sich damit zu:
\begin{align}
  \begin{split}
  TE_{source, 2D} &=                \frac{1}{2}
  \left[{\frac{f_E-f_P}{(x_E-x_P)(x_e-x_w)}-\frac{f_P-f_W}{(x_P-x_W)(x_e-x_w)}  }\right]
  \frac{(x_e-x_P)^3 - (x_w-x_P)^3}{3} (y_n-y_s)\\
  &+ \frac{1}{2}
  \left[{\frac{f_N-f_P}{(y_N-y_P)(y_n-y_s)}-\frac{f_P-f_S}{(y_P-y_S)(y_n-y_s)}  }\right]
  \frac{(y_n-y_P)^3-(y_s-y_P)^3}{3} (x_e-x_w)+HOT \\
  \end{split}
\end{align}
Die Betrachtung von Randkontrollvolumen erfolgt äquivalent zum Eindimensionalen. Es müssen
jedoch auch die Fälle für nördliche und südliche Ränder, als auch die Ecken betrachtet werden.
\begin{figure}[hbt]
\begin{tikzpicture}[scale=1.1]
  %\draw[->] (6,0) -- (6.5,0) node[right] {$x$} coordinate(x axis);
  %\draw[->] (0,6) -- (0,6.5) node[above] {$y$} coordinate(y axis);
  \fill[tud0a] (2,2) rectangle +(2,2);
  \draw[step=2cm, thick] (-2.00001, 0) grid (8, 6);

  \fill (3, 3)  circle[radius=1.5pt];
  \node (N) at (3,3) [label=above:$P$]{};

  \fill (3, 5)  circle[radius=1.5pt];
  \node (N) at (3,5) [label=above:$N$]{};

  \fill (3, 1)  circle[radius=1.5pt];
  \node (N) at (3,1) [label=below:$S$]{};

  \fill (5, 3)  circle[radius=1.5pt];
  \node (N) at (5,3) [label=right:$E$]{};

  \fill (1, 3)  circle[radius=1.5pt];
  \node (N) at (1,3) [label=left:$W$]{};

  \fill (7, 3)  circle[radius=1.5pt];
  \node (N) at (7,3) [label=right:$EE$]{};

  \fill (-1, 3)  circle[radius=1.5pt];
  \node (N) at (-1,3) [label=left:$WW$]{};

  \fill (5, 5)  circle[radius=1.5pt];
  \node (N) at (5,5) [label={above right}:$NE$]{};
  \fill (5, 1)  circle[radius=1.5pt];
  \node (N) at (5,1) [label={below right}:$SE$]{};
  \fill (1, 1)  circle[radius=1.5pt];
  \node (N) at (1,1) [label={below left}:$SW$]{};
  \fill (1, 5)  circle[radius=1.5pt];
  \node (N) at (1,5) [label={above left}:$NW$]{};

  \fill (4, 3)  circle[radius=1.5pt];
  \node (N) at (4,3) [label={right=0pt}:$e$]{};

  \fill (3, 2)  circle[radius=1.5pt];
  \node (N) at (3,2) [label={below=0pt}:$s$]{};

  \fill (2, 3)  circle[radius=1.5pt];
  \node (N) at (2,3) [label={left=0pt}:$w$]{};

  \fill (3, 4)  circle[radius=1.5pt];
  \node (N) at (3,4) [label={above=0pt}:$n$]{};

  \fill (6, 3)  circle[radius=1.5pt];
  \node (N) at (6,3) [label=right:$ee$]{};

  \fill (0, 3)  circle[radius=1.5pt];
  \node (N) at (0,3) [label=left:$ww$]{};


\end{tikzpicture}
\centering
\caption{Allgemeines Kontrollvolumen im zweidimensionalen Fall}
\label{fig:kv2d}
\end{figure}


\subsection{Äquidistante Gitter}

Für äquidistante Gitter vereinfacht sich der Terme des Abbruchfehlers und es ergeben sich
für ein- und zweidimensionale Fälle die folgenden Gleichungen.

\begin{align*}
  TE_{source,\ 1D} &= \frac{\Delta x}{24} \left({f_E-2f_P+f_W}\right) + HOT\\
  TE_{source,\ 2D} &= \frac{1}{24} \Delta x \Delta y \left[{f_E+f_W+f_N+f_S - 4f_P} \right] + HOT
\end{align*}

\newpage

  \section{Abbruchfehler von Diffusionstermen}
\label{sec:Diffusionsterme}

\subsection{Eindimensionale Probleme}

Um den Abbruchfehler von Diffusionstermen zu bestimmen werden die Differenzenquotienten
für die Ableitungen erster Ordnung hergeleitet. Dazu werden die Taylorreihendarstellungen der bekannten
Werte in der unmittelbaren Umgebung genutzt. Beispielhaft soll das anhand des östlichen
Randes des Kontrollvolumens geschehen.

Zuerst entwickeln wir die Taylordarstellungen vom Punkt $x_e$ aus in Richtung der anliegenden
Kontrollvolumenmittelpunkte $\phi_E$ und $\phi_P$.
\begin{align}
  \phi_E &= \phi_e + \phi'_e(x_E-x_e)+\frac{1}{2}\phi''_e(x_E-x_e)^2
  +\frac{1}{6}\phi'''_e(x_E-x_e)^3+HOT
  \label{eq:taylor_eE}\\
  \phi_P &= \phi_e + \phi'_e(x_P-x_e)+\frac{1}{2}\phi''_e(x_P-x_e)^2
  +\frac{1}{6}\phi'''_e(x_P-x_e)^3+HOT
  \label{eq:taylor_eP}
\end{align}
Wird nun Gleichung~\eqref{eq:taylor_eP} von Gleichung~\eqref{eq:taylor_eE} subtrahiert. Es ergibt sich:
\begin{equation*}
  \phi_E-\phi_P=\phi'_e(x_E-x_P)+
  \frac{1}{2}\phi''_e\left[{{(x_E-x_e)}^2-{(x_P-x_e)}^2}\right]+
  \frac{1}{6}\phi'''_e\left[{{(x_E-x_e)}^3-{(x_P-x_e)}^3}\right]+HOT
\end{equation*}
Nach Umstellen ergibt sich daraus für die Ableitung $\phi'_e$ der folgende Term, aus dem bereits
der Abbruchfehler abgelesen werden kann.
\begin{equation}
  \phi'_e = \frac{\phi_E-\phi_P}{x_E-x_P}+\frac{1}{2}\phi''_e
\left({\frac{{(x_P-x_e)}^2-{(x_E-x_e)}^2}{x_E-x_P}}\right)+
\frac{1}{6} \phi'''_e \left({\frac{{(x_P-x_e)}^3-{(x_E-x_e)}^3}{(x_E-x_P)}}\right)+HOT \label{eq:te_dif_e}
\end{equation}
Die hier auftretenden Ableitungen $\phi''_e$ und $\phi'''_e$ sind nicht bekannt und
müssen diskretisiert werden. Bei der Nutzung von
möglichst lokalen Differenzenquotienten ergeben sich die folgenden Approximationen der Ableitungen.
Die genutzten Punkte sind in Abbildung~\ref{fig:kv1d} ersichtlich.
\begin{align*}
  \phi''_e &= \frac{1}{(x_E-x_P)}\left({
\frac{\phi_{EE}-\phi_P}{x_{EE}-x_P}-\frac{\phi_E-\phi_W}{x_E-x_W}}\right)\\
\phi'''_e &= \frac{1}{(x_E-x_P)}\left({
\frac{1}{(x_{ee}-x_e)}
\left({\frac{\phi_{EE}-\phi_E}{x_{EE}-x_E}-\frac{\phi_E-\phi_P}{x_E-x_P} }\right)
-\frac{1}{(x_e-x_w)}
\left({\frac{\phi_E-\phi_P}{x_E-x_P} - \frac{\phi_P-\phi_W}{x_P-x_W}  }\right)
}\right)
\end{align*}
Der Abbruchfehler des Diffusionsterms an der östlichen Grenze eines Kontrollvolumens
lässt sich damit über folgende Gleichung beschreiben.
\begin{align}
\begin{split}
    \begin{IEEEeqnarraybox}[][c]{rCl}
      {TE}_e &=& \frac{1}{2 (x_E-x_P)}\left({
\frac{\phi_{EE}-\phi_P}{x_{EE}-x_P}-\frac{\phi_E-\phi_W}{x_E-x_W}}\right) \left({\frac{{(x_P-x_e)}^2-{(x_E-x_e)}^2}{x_E-x_P}}\right)\\
&&+
\left({
\frac{1}{(x_{ee}-x_e)}
\left({\frac{\phi_{EE}-\phi_E}{x_{EE}-x_E}-\frac{\phi_E-\phi_P}{x_E-x_P} }\right)
-\frac{1}{(x_e-x_w)}
\left({\frac{\phi_E-\phi_P}{x_E-x_P} - \frac{\phi_P-\phi_W}{x_P-x_W}  }\right)
}\right)\\
&&\frac{1}{6(x_E-x_P)}\left({\frac{{(x_P-x_e)}^3-{(x_E-x_e)}^3}{(x_E-x_P)}}\right)
+HOT
    \end{IEEEeqnarraybox}
\end{split}
\end{align}
Der Abbruchfehler für Diffusionsterme im Westen erfolgt äquivalent zur oben
gezeigten Herleitung im Osten. Es ergibt sich:
\begin{align}
  \begin{split}
    \begin{IEEEeqnarraybox}[][c]{rCl}
      TE_w &=& \frac{1}{2 (x_P-x_W)} \left({
\frac{\phi_{E}-\phi_W}{x_{E}-x_W}-\frac{\phi_P-\phi_{WW}}{x_P-x_{WW}}}\right)
  \left({\frac{{(x_W-x_w)}^2-{(x_P-x_w)}^2}{x_P-x_W}}\right)\\
  &&+
\left({
\frac{1}{(x_e-x_w)}
\left({\frac{\phi_E-\phi_P}{x_E-x_P}-\frac{\phi_P-\phi_W}{x_P-x_W} }\right)
-\frac{1}{(x_w-x_{ww})}
\left({\frac{\phi_P-\phi_W}{x_P-x_W} - \frac{\phi_W-\phi_{WW}}{x_W-x_{WW}}  }\right)
}\right)\\
&&\frac{1}{6(x_P-x_W)}\left({\frac{{(x_W-x_w)}^3-{(x_P-x_w)}^3}{(x_P-x_W)}}\right)
  +HOT
    \end{IEEEeqnarraybox}
\end{split}
\end{align}

%\subsection{Äquidistante Gitter}

%Für äquidistante Gitter vereinfache sich die Terme des Truncation Error. So löschen
%sich beispielsweise die quadratischen Terme gegenseitig aus. Es ergeben sich die
%folgenden Fehler:

%\begin{align*}
  %{TE}_e &= \left({
%\frac{1}{6\Delta x^2}
%\left({\frac{\phi_{EE}-\phi_E}{\Delta x}-\frac{\phi_E-\phi_P}{\Delta x} }\right)
%-\frac{1}{6\Delta x^2}
%\left({\frac{\phi_E-\phi_P}{\Delta x} - \frac{\phi_P-\phi_W}{\Delta x}  }\right)
%}\right)\left({-\frac{\Delta x^2}{4} }\right)+HOT\\
%&= -\frac{1}{24\Delta x}\left({
%\phi_{EE}-3\phi_E+3\phi_P-\phi_W}\right)+HOT
%\end{align*}

%\begin{align*}
  %TE_w &=\left({
%\frac{1}{6 \Delta x^2}
%\left({\frac{\phi_E-\phi_P}{\Delta x}-\frac{\phi_P-\phi_W}{\Delta x} }\right)
%-\frac{1}{6\Delta x^2}
%\left({\frac{\phi_P-\phi_W}{\Delta x} - \frac{\phi_W-\phi_{WW}}{\Delta x}  }\right)
%}\right)
%\left({-\frac{\Delta x^2}{4} }\right)+HOT\\
%&= -\frac{1}{24 \Delta x}\left({
%\phi_E-3\phi_P+3\phi_W-\phi_{WW}}\right)+HOT
%\end{align*}


\subsubsection{Kontrollvolumen am Rand}
\label{sec:te_dif_rand}

Kontrollvolumen, die in der Nähe des Randes des Problemgebietes liegen müssen wie schon beim
Abbruchfehler von Quelltermen besonders behandelt werden. Hierbei muss unterschieden werden ob es sich
bei der zu approximierenden Ableitung um die Ableitung genau am Rand oder weiter im Inneren des
Problemgebiets handelt.

Muss die Ableitung genau am Rand approximiert werden, so muss der in Gleichung~\ref{eq:te_dif_e} hergeleitete
Zentraldifferenzenquotient durch einen einseitigen Differenzenquotient ersetzt werden. Je nachdem ob das
betrachtete Volumen am Ost- oder Westrand liegt, müssen Vorwärts- oder Rückwärtsdifferenzenquotienten genutzt werden.
\begin{figure}[hb]
\begin{tikzpicture}[scale=1.1]
  %\draw[->] (6,0) -- (6.5,0) node[right] {$x$} coordinate(x axis);
  %\draw[->] (0,6) -- (0,6.5) node[above] {$y$} coordinate(y axis);
  \fill[tud0a] (2,2) rectangle +(2,2);
  \draw[step=2cm, thick] (2, 2) grid (8, 4);

  \filldraw[pattern=north east lines, thick] (1,2) rectangle +(1,2);

  \fill (3, 3)  circle[radius=1.5pt];
  \node (N) at (3,3) [label=above:$P$]{};

  \fill (5, 3)  circle[radius=1.5pt];
  \node (N) at (5,3) [label=right:$E$]{};


  \fill (7, 3)  circle[radius=1.5pt];
  \node (N) at (7,3) [label=right:$EE$]{};


  \fill (4, 3)  circle[radius=1.5pt];
  \node (N) at (4,3) [label={right=0pt}:$e$]{};

  \fill (2, 3)  circle[radius=1.5pt];
  \node (N) at (2,3) [label={right=0pt}:$w$]{};

  \fill (6, 3)  circle[radius=1.5pt];
  \node (N) at (6,3) [label=right:$ee$]{};

\end{tikzpicture}

\centering
\caption{Randkontrollvolumen im eindimensionalen Fall}
\label{fig:kv1d_rand}
\end{figure}
Das Vorgehen soll beispielhaft für die Ableitung $\phi'_w$ für das in Abbildung~\ref{fig:kv1d_rand} markierte Kontrollvolumen
vorgestellt werden.
Über die Randbedingung ist der Wert $\phi^w$ bekannt. Nun wird $\phi_P$ als Taylorreihe mit Entwicklungspunkt
$x_w$ dargestellt.
\begin{equation*}
  \phi_P = \phi^w + \phi'_w (x_P-x_w) + \frac{1}{2} \phi''_w (x_P-x_w)^2 + HOT
\end{equation*}
Nach Umstellen ergibt sich für $\phi'_w$ der folgende Wert:
\begin{equation}
  \phi'_w = \frac{\phi_P-\phi^w}{x_p-x_w} -\frac{1}{2} \phi''_w(x_p-x_w) + HOT
\end{equation}
Hier lässt sich ebenfalls der Abbruchfehler ablesen. Mit der Approximation der Ableitung
ergibt sich:
\begin{equation}
  TE_{diff, konv, Rand} = -\frac{1}{2} \left({\frac{\phi_e-\phi^w}{x_e-x_w}-
  \frac{\phi_P-\phi^w}{x_p-x_w} }\right)
\end{equation}
Wird die Ableitung nicht direkt auf dem Rand betrachtet, kann der Zentraldifferenzenquotient
aus Gleichung~\ref{eq:te_dif_e} genutzt werden. Hier müssen jedoch ebenfalls die verwendeten
Approximation der höheren Ableitungen angepasst werden, um nicht vorhandene Werte nicht zu verwenden.
Soll beispielsweise die Ableitung $\phi'_e$ für das in Abbildung~\ref{fig:kv1d_rand} markierte Kontrollvolumen
berechnet werden, so müssen die Ableitungen $\phi''_e$ und $\phi'''_e$ diskretisiert werden ohne den Wert
$\phi_W$ zu verwenden. Die sich ergebenden Approximationen lauten damit:
\begin{align*}
  \phi''_e &= \frac{1}{(x_E-x_P)}\left({
\frac{\phi_{EE}-\phi_P}{x_{EE}-x_P}-\frac{\phi_E-\phi^w}{x_E-x_w}}\right)\\
\phi'''_e &= \frac{1}{(x_E-x_P)}\left({
\frac{1}{(x_{ee}-x_e)}
\left({\frac{\phi_{EE}-\phi_E}{x_{EE}-x_E}-\frac{\phi_E-\phi_P}{x_E-x_P} }\right)
-\frac{1}{(x_e-x_w)}
\left({\frac{\phi_E-\phi_P}{x_E-x_P} - \frac{\phi_P-\phi^w}{x_P-x_w}  }\right)
}\right)
\end{align*}
Die oben vorgestellte Behandlung des westlichen Randes kann äquivalent auf die anderen Ränder übertragen werden.
Beim zweidimensionalen Fall ist zudem besonders auf die Ecken zu achten.


%\paragraph{Westlicher Rand}

%Es ergeben sich für gegebenes $\phi_w$:


%\begin{equation*}
  %\phi''_{w, W-Rand} = \frac{1}{(x_P-x_w)}\left({
%\frac{\phi_{e}-\phi_w}{x_{e}-x_w}-\frac{\phi_P-\phi_w}{x_P-x_w}}\right)
%\end{equation*}

%Da bei der dritten Ableitung zwei linke Kontrollvolumen genutzt werden, müssen hier
%vom Kontrollvolumen am Rand sowohl der westliche also auch der östliche Rand
%betrachtet werden.

%\begin{align*}
  %\phi'''_{w, W-Rand} &= \frac{1}{(x_p-x_w)} \left({
  %\frac{1}{(x_e-x_w)} \left({
    %\frac{\phi_E-\phi_P}{x_E-x_P} - \frac{\phi_P-\phi_w}{x_P-x_w}
    %}\right) -
  %\frac{1}{x_P-x_w} \left({
    %\frac{\phi_e-\phi_w}{x_e-x_w} - \frac{\phi_P-\phi_w}{x_P-x_w}
    %}\right)
  %}\right)
  %\\
  %\phi'''_{e, W-Rand} &= \frac{1}{(x_E-x_P)} \left({
  %\frac{1}{(x_{ee}-x_e)} \left({
      %\frac{\phi_{EE}-\phi_E}{x_{EE}-x_E} - \frac{\phi_E-\phi_P}{x_E-x_P}
    %}\right) -
  %\frac{1}{x_e-x_w} \left({
    %\frac{\phi_E-\phi_P}{x_E-x_P} - \frac{\phi_P-\phi_w}{x_P-x_w}
    %}\right)
  %}\right)
%\end{align*}

%Hier wird noch der Wert $\phi_e$ benutzt, der aber nicht bekannt ist. Er wird deshalb
%durch eine lineare Interpolation von $\phi_E$ und $\phi_P$ bestimmt.

%\begin{equation}
  %\phi_e = \phi_P \frac{x_E-x_e}{x_E-x_P} + \phi_E \frac{x_e-x_P}{x_E-x_P}
%\end{equation}

%Damit ergibt sich für $\phi''_{w,Rand}$und $\phi'''_{w, Rand}$:

%\begin{align}
  %\phi''_{w, W-Rand} &= \frac{1}{(x_P-x_w)}\left({
%\frac{
  %\left({\phi_P \frac{x_E-x_e}{x_E-x_P} + \phi_E \frac{x_e-x_P}{x_E-x_P}
%}\right)
%-\phi_w}{x_{e}-x_w}-\frac{\phi_P-\phi_w}{x_P-x_w}}\right)\\
  %\phi'''_{w, W-Rand} &= \frac{1}{(x_p-x_w)} \left({
  %\frac{1}{(x_e-x_w)} \left({
    %\frac{\phi_E-\phi_P}{x_E-x_P} - \frac{\phi_P-\phi_w}{x_P-x_w}
    %}\right) -
  %\frac{1}{x_P-x_w} \left({
    %\frac{ \phi_P \frac{x_E-x_e}{x_E-x_P} + \phi_E \frac{x_e-x_P}{x_E-x_P}
%-\phi_w}{x_e-x_w} - \frac{\phi_P-\phi_w}{x_P-x_w}
    %}\right)
  %}\right)
%\end{align}


%\paragraph{Östlicher Rand}
%Äquivalent ergibt sich $\phi''_{e, E-Rand}$ bei gegebenem $\phi_e$:

%\begin{align*}
  %\phi''_{e, E-Rand} &= \frac{1}{(x_e-x_P)}\left({
%\frac{\phi_{e}-\phi_P}{x_{e}-x_P}-\frac{\phi_e-\phi_w}{x_e-x_w}}\right)
%\\
  %\phi'''_{e, E-Rand} &= \frac{1}{(x_e-x_P)} \left({
  %\frac{1}{(x_e-x_P)} \left({
    %\frac{\phi_e-\phi_P}{x_e-x_P} - \frac{\phi_e-\phi_w}{x_e-x_w}
    %}\right) -
  %\frac{1}{x_e-x_w} \left({
    %\frac{\phi_e-\phi_P}{x_e-x_P} - \frac{\phi_P-\phi_W}{x_P-x_W}
    %}\right)
  %}\right)
  %\\
  %\phi'''_{w, E-Rand} &= \frac{1}{(x_P-x_W)} \left({
  %\frac{1}{(x_e-x_w)} \left({
      %\frac{\phi_e-\phi_P}{x_e-x_P} - \frac{\phi_P-\phi_W}{x_P-x_W}
    %}\right) -
    %\frac{1}{x_w-x_{ww}} \left({
        %\frac{\phi_P-\phi_W}{x_P-x_W} - \frac{\phi_W-\phi_{WW}}{x_W-x_{WW}}
    %}\right)
  %}\right)
%\end{align*}

%$\phi_w$ wird wie folgt linear interpoliert:

%\begin{equation*}
  %\phi_w = \phi_W \frac{x_P-x_w}{x_P-x_W} + \phi_P \frac{x_w-x_W}{x_P-x_W}
%\end{equation*}

%Damit ergibt sich für $\phi''_{e,Rand}$:

%\begin{align}
  %\phi''_{e,Rand} &= \frac{1}{(x_e-x_P)}\left({
%\frac{\phi_{e}-\phi_P}{x_{e}-x_P}-\frac{\phi_e-
  %\left({
   %\phi_W \frac{x_P-x_w}{x_P-x_W} + \phi_P \frac{x_w-x_W}{x_P-x_W}
  %}\right)
%}{x_e-x_w}}\right)
%\\
  %\phi'''_{e, E-Rand} &= \frac{1}{(x_e-x_P)} \left({
  %\frac{1}{(x_e-x_P)} \left({
    %\frac{\phi_e-\phi_P}{x_e-x_P} - \frac{\phi_e- \phi_W \frac{x_P-x_w}{x_P-x_W} + \phi_P \frac{x_w-x_W}{x_P-x_W}
%}{x_e-x_w}
    %}\right) -
  %\frac{1}{x_e-x_w} \left({
    %\frac{\phi_e-\phi_P}{x_e-x_P} - \frac{\phi_P-\phi_W}{x_P-x_W}
    %}\right)
  %}\right)
%\end{align}


%\subsubsection{Äquidistante Gitter}

%\paragraph{Westlicher Rand}

%\begin{align}
  %\phi''_{w, W-Rand} &= \frac{1}{\Delta x^2} (\phi_E-3\phi_P+2\phi_w)\\
  %\phi'''_{w, W-Rand} &= 0\\
  %\phi'''_{e, W-Rand} &= \frac{1}{\Delta x^3} (\phi_{EE} -3\phi_E + 4\phi_P -2\phi_w)
%\end{align}

%\paragraph{Östlicher Rand}

%\begin{align}
  %\phi''_{e, E-Rand} &= \frac{1}{\Delta x^2} (\phi_W-3\phi_P+2\phi_e)\\
  %\phi'''_{e, E-Rand} &= 0\\
  %\phi'''_{w, E-Rand} &= \frac{1}{\Delta x^3} (-\phi_{WW} +3\phi_W - 4\phi_P +2\phi_e)
%\end{align}

\subsection{Zweidimensionale Probleme}

Äquivalent zu den Herleitungen des Abbruchfehlersim Osten und Westen ergeben sich
sich für die Ableitungen $\phi'_n$ und $\phi'_s$ die folgenden Reihenentwicklungen:

\begin{equation}
  \phi'_n = \frac{\phi_N-\phi_P}{y_N-y_P}+\frac{1}{2}\phi''_n
\left({\frac{{(y_P-y_n)}^2-{(y_N-y_n)}^2}{y_N-y_P}}\right)+
\frac{1}{6} \phi'''_n \left({\frac{{(y_P-y_n)}^3-{(y_N-y_n)}^3}{y_N-y_P}}\right)+HOT
\end{equation}


\begin{equation}
  \phi'_s = \frac{\phi_P-\phi_S}{y_P-y_S}+\frac{1}{2}\phi''_s
\left({\frac{{(y_S-y_s)}^2-{(y_P-y_s)}^2}{y_P-y_S}}\right)+
\frac{1}{6} \phi'''_s \left({\frac{{(y_S-y_s)}^3-{(y_P-y_s)}^3}{y_P-y_S}}\right)+HOT
\end{equation}
Nach der Diskretisierung der auftretenden, unbekannten Ableitungen ergeben sich die folgenden
Terme für den Abbruchfehler:

\begin{equation}
  \begin{IEEEeqnarraybox}[][c]{rCl}
    {TE}_n &=& \frac{1}{2\,(y_N-y_P)}\left({
\frac{\phi_{NN}-\phi_P}{y_{NN}-y_P}-\frac{\phi_N-\phi_S}{y_N-y_S}}\right) \left({\frac{{(y_P-y_n)}^2-{(y_N-y_n)}^2}{y_N-y_P}}\right)\\
&&+
\left({
\frac{1}{(y_{nn}-y_n)}
\left({\frac{\phi_{NN}-\phi_N}{y_{NN}-y_N}-\frac{\phi_N-\phi_P}{y_N-y_P} }\right)
-\frac{1}{(y_n-y_s)}
\left({\frac{\phi_N-\phi_P}{y_N-y_P} - \frac{\phi_P-\phi_S}{y_P-y_S}  }\right)
}\right)\\
&&\frac{1}{6\,(y_N-y_P)}\left({\frac{{(y_P-y_n)}^3-{(y_N-y_n)}^3}{y_N-y_P}}\right)
+HOT
\end{IEEEeqnarraybox}
\end{equation}


\begin{equation}
\begin{IEEEeqnarraybox}[][c]{rCl}
  TE_s &=& \frac{1}{2\,(y_P-y_S)} \left({
\frac{\phi_{N}-\phi_S}{y_{N}-y_S}-\frac{\phi_P-\phi_{SS}}{y_P-y_{SS}}}\right)
  \left({\frac{{(y_S-y_s)}^2-{(y_P-y_s)}^2}{y_P-y_S}}\right)\\
  &&+
\left({
\frac{1}{(y_n-y_s)}
\left({\frac{\phi_N-\phi_P}{y_N-y_P}-\frac{\phi_P-\phi_S}{y_P-y_S} }\right)
-\frac{1}{(y_s-y_{ss})}
\left({\frac{\phi_P-\phi_S}{y_P-y_S} - \frac{\phi_S-\phi_{SS}}{y_S-y_{SS}}  }\right)
}\right)\\
&&\frac{1}{6\,(y_P-y_S)}\left({\frac{{(y_S-y_s)}^3-{(y_P-y_s)}^3}{y_P-y_S}}\right)
  +HOT
\end{IEEEeqnarraybox}
\end{equation}
Die Behandlung der Randwerte erfolgt ebenfalls wir in Abschnitt~\ref{sec:te_dif_rand} erläutert und wird hier
nicht wiederholt.

\subsection{Äquidistante Gitter}

\begin{align*}
  {TE}_n &= \left({
\frac{1}{6\Delta y^2}
\left({\frac{\phi_{NN}-\phi_N}{\Delta y}-\frac{\phi_N-\phi_P}{\Delta y} }\right)
-\frac{1}{6\Delta y^2}
\left({\frac{\phi_N-\phi_P}{\Delta y} - \frac{\phi_P-\phi_S}{\Delta y}  }\right)
}\right)\left({-\frac{\Delta y^2}{4} }\right)+HOT\\
&= -\frac{1}{24\Delta y}\left({
\phi_{NN}-3\phi_N+3\phi_P-\phi_S}\right)+HOT
\end{align*}

\begin{align*}
  TE_s &=\left({
\frac{1}{6 \Delta y^2}
\left({\frac{\phi_N-\phi_P}{\Delta y}-\frac{\phi_P-\phi_S}{\Delta y} }\right)
-\frac{1}{6\Delta y^2}
\left({\frac{\phi_P-\phi_S}{\Delta y} - \frac{\phi_S-\phi_{SS}}{\Delta y}  }\right)
}\right)
\left({-\frac{\Delta y^2}{4} }\right)+HOT\\
&= -\frac{1}{24 \Delta y}\left({
\phi_N-3\phi_P+3\phi_S-\phi_{SS}}\right)+HOT
\end{align*}

  \section{Abbruchfehler von Konvektionstermen}

Wie in Abschnitt~\ref{sec:konv_fluss} beschrieben, existieren verschiedene Verfahren
um konvektive Flüsse abzuschätzen. Der Abbruchfehler ist vom genutzten Verfahren
abhängig und müss für jedes Verfahren hergeleitet werden.

Zu diesem Fehler kommt ab der zweiten Dimension wiederum der Abbruchfehler aus
der Integralapproximation, der bereits in Abschnitt~\ref{sec:source1d} hergeleitet wurde.

\subsection{CDS-Verfahren}
\label{sec:te_cds}

Für die Herleitung des Abbruchfehlers für das Zentraldifferenzen-Verfahren sei hier am Beispiel des östlichen Randes dargestellt.
Man entwickelt zuerst die Taylorreihe von $\phi$ um den Entwicklunspunktunkt $x_P$ und wertet sie anschließend
an den Punkten $x_e$ und $x_E$ aus. Es ergeben sich die folgenden Reihendarstellungen.
\begin{align}
  \phi_e &= \phi_P + \phi'_P(x_e-x_P)+\frac{1}{2}\phi''_P(x_e-x_P)^2
  +\frac{1}{6}\phi'''_P(x_e-x_P)^3+HOT
  \label{eq:taylor_konv_eP}\\
  \phi_E &= \phi_P + \phi'_P(x_E-x_P)+\frac{1}{2}\phi''_P(x_E-x_P)^2
  +\frac{1}{6}\phi'''_P(x_E-x_P)^3+HOT
  \label{eq:taylor_konv_eE}
\end{align}
Werden die Gleichungen~\eqref{eq:taylor_konv_eE} und \eqref{eq:taylor_konv_eP} nun
voneinander subtrahiert und nach $\phi_e$ umgestellt, so ergibt sich:

\begin{equation*}
  \begin{IEEEeqnarraybox}[][c]{rCl}
    \frac{\phi_e}{x_e-x_P} &=& \frac{\phi_E}{x_E-x_P} + \frac{\phi_P}{x_e-x_P} -
  \frac{\phi_P}{x_E-x_P} + \frac{1}{2} \phi''_P \left({(x_e-x_P)-(x_E-x_P)}\right)\\
  &&+ \frac{1}{6} \phi'''_P \left({(x_e-x_P)^2-(x_E-x_P)^2}\right)\\
  \phi_e &=& \phi_E \frac{x_e-x_P}{x_E-x_P} + \phi_P \left({1-\frac{x_e-x_P}{x_E-x_P} }\right)+HOT
  + \frac{1}{2} \phi''_P (x_e-x_E)(x_e-x_P)+HOT\\
  &&+ \frac{1}{6} \phi'''_P \left({(x_e-x_P)^2-(x_E-x_P)^2}\right)(x_e-x_P)\\
   &=& \phi_E \gamma_e + \phi_P (1-\gamma_e)+ \frac{1}{2} \phi''_P (x_e-x_E)(x_e-x_P)\\
         &&+ \frac{1}{6} \phi'''_P \left({(x_e-x_P)^2-(x_E-x_P)^2}\right)(x_e-x_P)+HOT
  \end{IEEEeqnarraybox}
\end{equation*}
Der Abbruchfehler lässt sich hier leicht ablesen. Er lautet:
\begin{equation*}
  TE_{e, CDS} =  \frac{1}{2} \phi''_P (x_e-x_E)(x_e-x_P)+ \frac{1}{6}
  \phi'''_P \left({(x_e-x_P)^2-(x_E-x_P)^2}\right)(x_e-x_P)+HOT
\end{equation*}
Die hier genutzten analytischen Ableitungen werden im nächsten Schritt durch Approximationen ersetzt.
\begin{align*}
  %\phi'_P &= \frac{\phi_E-\phi_W}{x_E-x_W}\\
  \phi''_P &= \frac{1}{(x_e-x_w)} \left({\frac{\phi_E-\phi_P}{x_E-x_P}
  - \frac{\phi_P-\phi_W}{x_P-x_W} }\right)\\
  %\label{eq:ddphip}\\
  \phi'''_P &= \frac{1}{(x_e-x_w)} \left({
  \frac{1}{(x_E-x_P)} \left({\frac{\phi_{EE}-\phi_P}{x_{EE}-x_P}- \frac{\phi_E-\phi_W}{x_E-x_W} }\right)-
  \frac{1}{(x_P-x_W)} \left({\frac{\phi_E-\phi_W}{x_E-x_W} - \frac{\phi_P-\phi_{WW}}{x_P-x_{WW}} }\right)
  }\right) \label{eq:dddphip}
\end{align*}
Der Abbruchfehler ergibt sich damit zu folgendem Term.
\begin{equation}
  \begin{IEEEeqnarraybox}{rCl}
    TE_{e, CDS} &=&  \frac{1}{2} \frac{1}{(x_e-x_w)} \left({\frac{\phi_E-\phi_P}{x_E-x_P}
  - \frac{\phi_P-\phi_W}{x_P-x_W} }\right) (x_e-x_E) (x_e-x_P) \nonumber \\
  &&+
 \frac{1}{(x_e-x_w)} \left({
  \frac{1}{(x_E-x_P)} \left({\frac{\phi_{EE}-\phi_P}{x_{EE}-x_P}- \frac{\phi_E-\phi_W}{x_E-x_W} }\right)-
  \frac{1}{(x_P-x_W)} \left({\frac{\phi_E-\phi_W}{x_E-x_W} - \frac{\phi_P-\phi_{WW}}{x_P-x_{WW}} }\right)
  }\right) \nonumber \\
  &&\frac{1}{6} \left({(x_e-x_P)^2-(x_E-x_P)^2}\right)(x_e-x_P)+HOT
  \end{IEEEeqnarraybox}
\end{equation}
Im Westen wird die Ableitung äquivalent zum Osten durchgeführt. Für den Abbruchfehler ergibt sich letztendlich
folgender Term:
%\begin{align*}
  %\begin{IEEEeqnarraybox}{rCl}
  %\phi_w &= \left({1-\frac{x_w-x_P}{x_W-x_P}}\right)\phi_P + \left({\frac{x_w-x_P}{x_W-x_P} }\right) \phi_W
  %+ \frac{1}{2} \phi''_P \left({(x_w-x_P)-(x_W-x_P)}\right)(x_w-x_P)\\
  %&+ \frac{1}{6} \phi'''_P \left({(x_w-x_P)^2-(x_W-x_P)^2}\right)(x_w-x_P)\\
  %\phi_w &= \left({1-\frac{x_P-x_w}{x_P-x_W}}\right)\phi_P + \left({\frac{x_P-x_w}{x_P-x_W} }\right) \phi_W
  %+ \frac{1}{2} \phi''_P \left({x_w-x_W}\right)(x_w-x_P)\\
  %&+ \frac{1}{6} \phi'''_P \left({(x_w-x_P)^2-(x_W-x_P)^2}\right)(x_w-x_P)
  %\end{IEEEeqnarraybox}
%\end{align*}

%Der Abbruchfehler lässt sich nach der Diskretisierung der unbekannten Ableitungen nach Gleichung~\eqref{eq:ddphip}
%und \eqref{eq:dddphip} leicht ablesen und ergibt sich zu:

\begin{equation}
  \begin{IEEEeqnarraybox}[][b]{rCl}
    TE_{w, CDS} &=&  \frac{1}{2} \frac{1}{(x_e-x_w)} \left({\frac{\phi_E-\phi_P}{x_E-x_P}
  - \frac{\phi_P-\phi_W}{x_P-x_W} }\right) \left({x_w-x_W}\right)(x_w-x_P)  \nonumber\\
  &&+ \frac{1}{(x_e-x_w)} \left({
  \frac{1}{(x_E-x_P)} \left({\frac{\phi_{EE}-\phi_P}{x_{EE}-x_P}- \frac{\phi_E-\phi_W}{x_E-x_W} }\right)-
  \frac{1}{(x_P-x_W)} \left({\frac{\phi_E-\phi_W}{x_E-x_W} - \frac{\phi_P-\phi_{WW}}{x_P-x_{WW}} }\right)
  }\right) \nonumber\\
  &&\frac{1}{6}  \left({(x_w-x_P)^2-(x_W-x_P)^2}\right)(x_w-x_P)
  \end{IEEEeqnarraybox}
\end{equation}

\subsubsection{Randwerte}
\label{sec:te_cds_rand}

Bei Kontrollvolumen direkt am Rand, wie zum Beispiel das in Abbildung~\ref{fig:kv1d_rand}
hervorgehobene Kontrollvolumen, müssen einige Besonderheiten beachtet werden.
So ist der konvektive Fluss, der direkt auf dem Rand verläuft, durch die Randbedingung ($\phi^w$) gegeben und
muss dementsprechend nicht angepasst werden. Am östlichen Rand muss hingegen die Approximation
der Ableitungen angepasst werden. Sie lauten dann:
\begin{align*}
  \phi''_P &= \frac{1}{(x_e-x_w)} \left({\frac{\phi_E-\phi_P}{x_E-x_P}
  - \frac{\phi_P-\phi^w}{x_P-x_w} }\right)\\
  \phi'''_P &= \frac{1}{(x_e-x_w)} \left({
  \frac{1}{(x_E-x_P)} \left({\frac{\phi_{EE}-\phi_P}{x_{EE}-x_P}- \frac{\phi_E-\phi^w}{x_E-x_w} }\right)-
  \frac{1}{(x_P-x_w)} \left({\frac{\phi_E-\phi^w}{x_E-x_w} - \frac{\phi_P-\phi^{w}}{x_P-x_{w}} }\right)
  }\right)
\end{align*}
Hat das betrachtete Kontrollvolumen  einen Abstand von eins zum Rand, so muss nur die Approximation der dritten
Ableitung angepasst werden.
\begin{equation*}
  \phi'''_P = \frac{1}{(x_e-x_w)} \left({
  \frac{1}{(x_E-x_P)} \left({\frac{\phi_{EE}-\phi_P}{x_{EE}-x_P}- \frac{\phi_E-\phi^w}{x_E-x_w} }\right)-
  \frac{1}{(x_P-x_W)} \left({\frac{\phi_E-\phi_W}{x_E-x_W} - \frac{\phi_P-\phi^{w}}{x_P-x_{w}} }\right)
  }\right)
\end{equation*}
Auf gleiche Art und Weise kann auch der östliche Rand behandelt werden.
%Hier muss abschließend $\phi_e$ aus $\phi_E$ und $\phi_P$ interpoliert werden. Letztendlich ergibt
%sich für äquidistante Gitter folgender Term:

%\begin{equation*}
  %TE_{e, CDS} = -\frac{1}{8} (\phi_E-2\phi_P+\phi_W) - \frac{1}{32}
  %(2\phi^{ee} - 3\phi_E + 2\phi_W - \phi_{WW})
%\end{equation*}

%Auf gleiche Art und Weise kann auch $TE_{w, CDS}$ berechnet werden.
\subsubsection{Äquidistante Gitter}

Für äquidistante Gitter ergeben sich die folgenden vereinfachten Formeln zur Berechnung der konvektiven Flüsse
mit dem CDS-Verfahren.

\begin{equation*}
  TE_{e, CDS} = -\frac{1}{8} (\phi_E-2\phi_P+\phi_W) - \frac{1}{32}
  (\phi_{EE} - 2\phi_E + 2\phi_W - \phi_{WW})
\end{equation*}
\begin{equation*}
  TE_{w, CDS} = -\frac{1}{8} (\phi_E-2\phi_P+\phi_W) + \frac{1}{32}
  (\phi_{EE} - 2\phi_E + 2\phi_W - \phi_{WW})
\end{equation*}








\subsection{UDS-Verfahren}

Um der Abbruchfehler des Upwind-Verfahrens herzuleiten, wertet man die
Taylorreihe von $\phi$ mit dem Entwicklungspunkt $x_P$ im Punkte $\phi_e$
aus.

\begin{equation*}
  \phi_e = \phi_P +(x_e-x_P) \phi'_P + \frac{1}{2} (x_e-x_P)^2 \phi''_P+HOT
\end{equation*}
Da im Falle einer positiven Geschwindigkeit das UDS-Verfahren $\phi_e$ mit $\phi_P$
gleichsetzt, lässt sich der Abbruchfehler hier direkt ablesen.
\begin{equation*}
  TE_{e, UDS} = (x_e-x_P) \phi'_P + \frac{1}{2} (x_e-x_P)^2 \phi''_P+HOT
\end{equation*}
Setzt man nun die diskretisierten Ableitungen aus Abschnitt~\ref{sec:te_cds} ein,
ergibt sich die Formel für den Abbruchfehler.
\begin{equation}
  TE_{e, UDS} = (x_e-x_P) \frac{\phi_E-\phi_W}{x_E-x_W}+
  \frac{1}{2} \frac{(x_e-x_P)^2}{(x_e-x_w)} \left({\frac{\phi_E-\phi_P}{x_E-x_P}
  - \frac{\phi_P-\phi_W}{x_P-x_W} }\right)+HOT
\end{equation}
Betrachtet man den westlichen Rand und geht von einer positiven Geschwindigkeit aus,
so ergibt sich der folgende Abbruchfehler:
\begin{equation*}
  TE_{w, UDS} = (x_w-x_W) \phi'_W + \frac{1}{2} (x_w-x_W)^2 \phi''_W + HOT
\end{equation*}
Hier werden nun wiederum die Ableitungen $\phi'_W$ und $\phi''_W$ diskretisiert. Der
Abbruchfehler lautet damit:
\begin{equation}
  TE_{w, UDS} = (x_w-x_W) \frac{\phi_P-\phi_{WW}}{x_P-x_{WW}}+
  \frac{1}{2} \frac{(x_w-x_W)^2}{(x_w-x_{ww})} \left({\frac{\phi_P-\phi_W}{x_P-x_W}
  - \frac{\phi_W-\phi_{WW}}{x_W-x_{WW}} }\right)+HOT
\end{equation}



\subsubsection{Randwerte}

Auch beim UDS-Verfahren müssen am Rand die Approximationen der verwendeten Ableitungen
so angepasst werden, das nur vorhandene Werte benutzt werden.
Für den östlichen Rand der Kontrollvolumens können dabei die in Abschnitt~\ref{sec:te_cds_rand}
gezeigten Differenzenquotienten genutzt werden.

Für das Kontrollvolumen direkt am Rand ist der konvektive Fluss gegeben. Es gibt somit keinen
Abbruchfehler. Für das daneben liegende Kontrollvolumen müssen die Differenzenquotienten wie folgt angepasst werden:
\begin{align*}
  \phi'_W &= \frac{\phi_w-\phi^w}{x_w-x_{ww}}\\
  \phi''_W &= \frac{1}{x_W-x_{ww}} \left({\frac{\phi_w-\phi^w}{x_w-x_{ww}}
- \frac{\phi_W-\phi^w}{x_W-x_{ww}} }\right)+HOT
\end{align*}
$\phi_w$ wird hier beispielsweise aus der linearen Interpolation von $\phi_P$ und $\phi_W$ berechnet.
\begin{equation*}
  \phi_w = \phi_W \frac{x_P-x_w}{x_P-x_W} + \phi_P \frac{x_w-x_W}{x_P-x_W}
\end{equation*}
%etzt ergeben sich damit die folgenden Terme.
%\begin{align}
  %\phi'_W &= \frac{\phi_W \frac{x_P-x_w}{x_P-x_W} + \phi_P \frac{x_w-x_W}{x_P-x_W}
%-\phi^w}{x_w-x_{ww}}\\
  %\phi''_W &= \frac{1}{x_W-x_{ww}} \left({\frac{\phi_W \frac{x_P-x_w}{x_P-x_W} + \phi_P \frac{x_w-x_W}{x_P-x_W}
%-\phi^w}{x_w-x_{ww}}
%- \frac{\phi_W-\phi^w}{x_W-x_{ww}} }\right)+HOT
%\end{align}


\subsubsection{Äquidistante Gitter}

%\begin{align*}
  %\phi'_W  &= \frac{1}{2\Delta x} \phi_P + \frac{1}{2\Delta x} \phi_W - \phi^w\\
  %\phi''_W &= \frac{1}{\Delta x^2} \phi_P -\frac{1}{\Delta x^2} \phi_W
%\end{align*}
Für äquidistante Gitter ereben sich folgende Gleichungen für den Abbruchfehler.

\begin{equation*}
  TE_{e, UDS} = \frac{3}{8} \phi_E-\frac{1}{4} \phi_P - \frac{1}{8} \phi_W
\end{equation*}
\begin{equation*}
  TE_{w, UDS} = \frac{3}{8} \phi_P-\frac{1}{4} \phi_W - \frac{1}{8} \phi_{WW}
\end{equation*}





\subsection{``Flux-Blending''-Verfahren}

Das Flux-Blending-Verfahren setzt sich aus UDS- und CDS-Verfahren zusammen. Beide
Verfahren werden dabei über den Faktor $\beta$ gewichtet. Dementsprechend müssen auch die
Abbruchfehler über $\beta$ gewichtet werden um den Abbruchfehler für das ``Flux-Blending''-Verfahren zu berechnen.
Mit Gleichung~\ref{eq:flux_blending} ergibt sich damit:

\begin{equation}
  TE_{e, Flux} = (1-\beta) TE_{e, UDS} + \beta TE_{e, CDS}
\end{equation}



%\section{Truncation Error eines Kontrollvolumens}
%\label{sec:Truncation Error eines Kontrollvolumens}

%Der Truncation Error für ein Kontrollvolumen setzt sich nun aus den Fehlern von Quell-
%und Diffusionstermen zusammen:

%\begin{equation*}
  %TE = \frac{TE_{source} - TE_e - TE_w}{\Delta x}
%\end{equation*}
%Für den äquidistanten Fall ergibt sich damit für zentrale Kontrollvolumen der folgende
%Truncation Error. Wichtig ist es, hier die durch den Gauß'schen Integralsatz 
%entstehenden Vorzeichen mit zu beachten. Weiterhin muss durch $\Delta x$ geteilt werden,
%da die gleiche Transformation beim Aufstellen des Gesamtgleichungsystem angewendet wird.

%\begin{align}
  %TE &= \frac{\frac{\Delta x}{24} \left({f_E-2f_P+f_W}\right)
   %+\frac{1}{24\Delta x}\left({
%\phi_{EE}-3\phi_E+3\phi_P-\phi_W}\right)
  %-\frac{1}{24 \Delta x}\left({
%\phi_E-3\phi_P+3\phi_W-\phi_{WW}}\right)}{\Delta x}
%\end{align}

\subsection{Zweidimensionale Probleme}
Die gezeigten Herleitungen lassen sich problemlos ins Zweidimensionale übertragen. Es muss
jedoch darauf geachtet werdenn den Fehler aus der Integralapproximation nciht zu vergessen.
Dieser lautet für die Ostseite:
\begin{equation}
  TE_{e,int} = \frac{1}{6} \frac{\partial^2\phi}{\partial x^2}\bigg\vert_e
    \left[{{(y_n-y_P)}^3-{(y_s-y_P)}^3}\right] + HOT
\end{equation}
Er lässt sich leicht auf die anderen Seiten des Kontrollvolumens übertragen.
\clearpage


  \chapter{Auswertung des Abbruchfehlers}
  \section{Diffusionsprobleme}
  \section{Konvektionsprobleme}
  \subsection{UDS}
  \subsection{CDS}
  \subsection{Flux-Blending}

  \chapter{Fazit}


\appendix

\chapter{Quellcodes}

\begin{tikzpicture}[scale=1]
\coordinate[label=left:$A$] (A) at (0,0);
\coordinate[label=right:$B$] (B) at (4,1);
\draw (A) -- (B);
\fill (A) circle (2pt);
\fill (B) circle (2pt);
\end{tikzpicture}




%\lstinputlisting{create_plot_3d_data.m}
%\lstinputlisting{dif2d.m}

  \cleardoublepage

\bibliographystyle{gerplain}
\bibliography{bib}

\nocite{*}

\end{document}
