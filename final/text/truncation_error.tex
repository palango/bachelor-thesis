\section{Abbruchfehler}

Der Abbruchfehler ist der Fehler der beim Abschneiden einer unendlichen Summe
und deren Approximation durch eine endliche Summe entsteht.

In der Praxis werden Approximationen, wie die in Abschnitt~\ref{sec:konv_fluss} oder
\ref{sec:dif_fluss} beschriebenen, aus Taylorreihenentwicklungen um die betrachteten
Punkte gewonnen\footnote{Die Herleitungen sind in Kapitel~\ref{chap:herleitung} ausführlich ausgeführt.}. 
Da die Approximation jedoch zwangsläufig endlich sein muss, werden
dabei Glieder der unendlichen Taylorreihenentwicklung abgeschnitten. Der dadurch entstehende
Fehler wird Abbruchfehler (engl. Truncation Error) genannt.
Die restlichen, für die Approximation nicht genutzten Terme können zur Fehlerabschätzung
bei numerischen Berechnungen genutzt werden. Die grundlegenden Zusammenhänge sollen hier an einem einfachen Beispiel gezeigt werden.

Wir nehmen an, die Funktion $f=\sin(x)$ soll zur einfacheren Berechenbarkeit durch
ein Polynom approximiert werden. Anschließend soll ein Indikator für den dabei
entstehenden Fehler gefunden werden, ohne dabei auf $f$
selbst zurückzugreifen.
Zuerst wird die Taylorreihe von $\sin(x)$ um den Entwicklungspunkt $x_0 =0$ berechnet:

\begin{equation}
  \sin(x) = \sum_{n=0}^{\infty}(-1)^n \frac{x^{2n+1}}{(2n+1)!} = 
  x-\frac{x^3}{3!} +\frac{x^5}{5!} -\frac{x^7}{7!} +\frac{x^9}{9!}\cdots
 \label{eq:taylor_example}
\end{equation}

Nun wird die Approximation für $\sin(x)$ festgelegt. Diese soll in diesem Beispiel
$x-\frac{x^3}{6} $ sein. In Abbildung~\ref{fig:taylor_example} sieht man, wie sich die Güte der
Approximation mit zunehmendem Abstand von $0$ verschlechtert.

\begin{figure}[h]
\begin{tikzpicture}
\begin{axis}[width=0.8*\textwidth, height=150pt, grid=major]
  \addplot[domain=-2*pi:2*pi, samples=100, color=tud2b, very thick]{sin(deg(x))};
  \addplot[domain=-3:3, samples=100, color=tud9b, very thick]{x - x*x*x/6};
  \legend{$\sin(x)$,$x-\frac{x^3}{6} $}
\end{axis}
\end{tikzpicture}
\centering
\caption{Vergleich von Funktion und Approximation}
 \label{fig:taylor_example}
\end{figure}

Der absolute Fehler kann nun einfach über die Formel $\sin(x) - x + \frac{x^3}{6}$
berechnet werden und entspricht den übrigen Gliedern der Taylorreihe~\eqref{eq:taylor_example}.
Der absolute Fehler entspricht also in diesem einfachen Beispiel dem Abbruchfehler.
Dieser wird nach seinem englischen Namen \textit{Truncation Error} mit $TE$ bezeichnet.
\begin{equation}
  TE = \frac{x^5}{5!} -\frac{x^7}{7!} +\frac{x^9}{9!}\cdots
 \label{eq:taylor_example_te}
\end{equation}
Da der Abbruchfehler selbst wieder eine unendliche Summe darstellt muss zur Auswertung
wieder eine endliche Anzahl von Termen abgetrennt werden. Dies stellt jedoch im allgemeinen
kein Problem dar, da der Einfluss der Terme mit steigendem Exponenten stark abnimmt.

Zur Abschätzung des Fehlers sollen im Beispiel zwei Glieder genutzt werden. Aus
Gleichung~\eqref{eq:taylor_example_te} ergibt sich damit der Fehlerindikator zu
$\frac{x^5}{5!} -\frac{x^7}{7!} $.
\begin{figure}[h]
\begin{tikzpicture}
\begin{axis}[width=0.8*\textwidth, height=150pt, grid=major, legend pos=north west]
  \addplot[domain=-6:6, samples=100, color=tud2b, very thick]{sin(deg(x))-x+x*x*x/6};
  \addplot[domain=-6:6, samples=100, color=tud9b, very thick]{x^5/120-x^7/(120*6*7)};
  \legend{Absoluter Fehler, Fehlerapproximation}
\end{axis}
\end{tikzpicture}
\centering
\caption{Vergleich des absoluten Fehlers mit der Fehlerschätzung}
 \label{fig:taylor_example_te}
\end{figure}
In Abbildung~\ref{fig:taylor_example_te} sieht man die gute Übereinstimmung von absolutem Fehler und Fehlerapproximation um den Entwicklungspunkt. Die Genauigkeit kann durch Hinzunahme weiterer Terme aus der Reihenentwicklung des Abbruchfehlers beliebig verbessert werden.
