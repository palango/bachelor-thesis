  \section{Motivation}
Für die ingenieurmäßige Berechnung von Strömungsproblemen hat sich die Methode der
finiten Volumen auf breitem Gebiet durchgesetzt. Der Fokus liegt hier auf der Simulation
der Navier-Stokes Gleichungen, oft auch unter Vernachlässigung kompressibler Strömungseffekte.
Um die Genauigkeit der Ergebnisse zu verbessern werden immer feinere Gebiete aufgelöst,
bei denen auch heutige Computersystem an ihre Grenzen stoßen. Es ist deshlab notwendig
auf adaptive Methoden zurückzugreifen, die die Auflösung an interessanten Stellen wie
zum Beispiel Grenzschichten, besonders belasteten Bauteilen oder Stellen mit hohem Fehler
erhöhen können. Dazu ist es aber nötig diesen Fehler zu kennen.
Heutige Methoden schätzen diesen Fehler ab, negieren aber seine Ursachen bei der Diskretisierung
und damit dem Übergang vonder kontinuierlichen Differentialgleichung zur diskreten Gleichung.
Hier liegt der Ansatzpunkt der Feherbetrachtung über den Abbruchfehler. Dieser wird in der
vorliegenden Arbeit für Feldprobleme hergeleitet und seine Tauglichkeit anhand
von Beispielen nachgewiesen.

  \section{Stand der Forschung}
  \section{Zielsetzung}
  \cleardoublepage
