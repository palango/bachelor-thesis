\section{Motivation}
Durch steigende Computerleistung sowie Verbesserungen in den numerischen Methoden konnten
sich die simulationsbasierten Produktentwicklungsprozesse einen festen Platz neben
Experimenten erarbeiten. Neben Kostenvorteilen ist es vor allem die Möglichkeit,
schnell Aussagen über Anpassungen und Modifikationen an Produkten treffen zu können, die
simulationsbasierten Methoden zu einem stetigen Wachstum verhelfen.

Für die ingenieurmäßige Berechnung von Strömungsproblemen hat sich die Methode der
Finiten Volumen auf breitem Gebiet durchgesetzt. Der Fokus liegt hierbei auf der Simulation
der Navier-Stokes Gleichungen, oft auch bei der Ausnutzung von zulässigen Vereinfachungen
wie der Vernachlässigung kompressibler Strömungseffekte.

Um die Genauigkeit der Ergebnisse zu verbessern werden immer feinere Gebiete aufgelöst,
bei denen auch heutige Computersysteme an ihre Grenzen ihrer Leistungsfähigkeit stoßen. Es ist deshalb notwendig
auf adaptive Methoden zurückzugreifen, die die Auflösung an interessanten Stellen wie
zum Beispiel Grenzschichten oder besonders interessanten Strömungsgebieten selektiv erhöhen können.
Weiterhin sollen Fehler, die durch die numerische Behandlung der kontinuierlichen Strömungsgebiete
entstehen, minimiert werden. Dafür ist es notwendig, Indikatoren für den lokalen Fehler
der berechneten Lösung zu definieren.

Heutige Methoden schätzen diesen Fehler ab, negieren aber seine Ursachen bei der Diskretisierung
und damit dem Übergang von der kontinuierlichen Differentialgleichung zur diskreten Gleichung.
Hier liegt der Ansatzpunkt der Fehlerbetrachtung über den Abbruchfehler. Er schätzt
den durch Diskretisierungen entstandenen Fehler auf Basis von Taylorreihenentwicklungen ab.
Der Abbruchfehler wird in der
vorliegenden Arbeit für skalare Transportprobleme hergeleitet und seine Genauigkeit an Beispielen
verifiziert. Weiterhin wird er zur Gitteradaption eingesetzt.


\section{Stand der Forschung}

Der Abbruchfehler als Indikator für die Gitteradaption ist in der Literatur bereits bekannt.
Roy~\cite{roy2} leitet ihn für die eindimensionale Burger-Gleichung mittels
Finiten-Differenzen her und vergleicht anschließend den abbruchfehlerbasierten Fehlerindikator
mit lösungsbasierten Indikatoren wie dem Lösungsgradient oder der Lösungskrümmung. Weiterhin wird gezeigt,
dass für lineare Probleme der Abbruchfehler
als Quelle des Diskretisierungsfehlers interpretiert werden kann.
Dadurch bewirkt eine Gitterverfeinerung basierend auf dem Abbruchfehler
eine Erhöhung der Auflösung an der Quelle des Diskretisierungsfehlers und
verringert damit den gesamten Diskretisierungsfehler.

Vor der Gitteradaption muss eine Lösung auf dem Problemgebiet berechnet werden. Diese
Lösung und das zu ihrer Berechnung verwendete Lösungsprogramm müssen
auf ihre Korrektheit überprüft und verifiziert werden. Veluri~\cite{veluri} demonstriert
Methoden zur Verifikation von Lösungsprogrammen. Diese basieren auf dem Vergleich von formaler und beobachteter
Konvergenzordnung und stellt außerdem die Methode der konstruierten Lösung vor, die genutzt werden kann
um analytische Lösungen zum Zweck der Verifikation von Computerprogrammen zu erzeugen.
Auch Roache~\cite{roache} stellt diese vor und demonstriert ihre Nützlichkeit bei der Verifikation von
Lösungsprogrammen.

In praktisch relevanten Problemfällen kann nur in den wenigsten Fällen mit
kartesischen oder orthogonalen Gittern gerechnet werden. Die Nichtorthogonalität
von Gittern bewirkt neue Fehlerterme, die bereits untersucht wurden. So gehen
Huang und Prosperetti~\cite{grid_ortho} auf den Einfluss nicht-orthogonaler Gitter
auf die Lösungsgenauigkeit gehen 
ein. Lee und Tsuei~\cite{lee} demonstrieren die Ableitung des Abbruchfehlers für Finite-Differenzen Verfahren
auf nicht-orthogonalen Gittern.

Einen Überblick über Methoden zur Gitteradaption gibt Fath~\cite{fath}.
Neben den Grundlagen geht sie auch auf die Ableitung des Abbruchfehlers für
orthogonale Gitter sowie die Transformation von physikalischen zu logischen Gebiet ein.



\section{Zielsetzung}

Im ersten Teil der vorliegenden Arbeit werden grundlegende Konzepte und Begriffe vorgestellt,
die für das Verständnis der Thesis notwendig sind. Grundlegende Formeln wie
die skalare Transportgleichung oder die Navier-Stokes Gleichungen werden vorgestellt
und ein Überblick über die verwendeten numerischen Methoden gegeben. Außerdem
wird der Abbruchfehler eingeführt und die Grundlagen der Verifikation von
Lösungsprogrammen erläutert.

Anschließend wird der Abbruchfehler formal
abgeleitet. Dabei werden ein- und zweidimensionale Probleme betrachtet.
 Die geschieht auf kartesischen, orthogonalen
und nicht-orthogonalen Gittern. Der Abbruchfehler wird für Konvektions-
und Diffusionsterme sowie für Quellterme hergeleitet. Die Herleitung beschränkt sich damit auf
skalare Transportgleichungen. Diese decken jedoch bereits die wesentlichen Terme der Navier-Stokes Gleichungen
ab. Zudem lassen sich die erzielten Ergebnisse leichter untersuchen und verifizieren.

Anschließend wird der abgeleitete Abbruchfehlerindikator an konkreten Testfällen untersucht.
Dazu werden Lösungsprogramme für ein- und zweidimensionale Problemgebiete implementiert
und ihre Korrektheit nachgewiesen. Darauf folgt die Implementation
der Abbruchfehlers sowie die Untersuchung auf den verschiedenen Gittertypen.

Abschließend wird dann der Abbruchfehlerindikator zur lokalen Gitteradaption genutzt.
Zum Vergleich werden weitere gängige Indikatoren wie der Lösungsgradient oder
die Fehlerabschätzung mit der Richardson-Extrapolation implementiert und mit dem Abbruchfehler verglichen.

\cleardoublepage
