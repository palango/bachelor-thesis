\section{Motivation}
Für die ingenieurmäßige Berechnung von Strömungsproblemen hat sich die Methode der
finiten Volumen auf breitem Gebiet durchgesetzt. Ihr Fokus liegt hier auf der Simulation
der Navier-Stokes Gleichungen, oft auch unter Vernachlässigung kompressibler Strömungseffekte und
anderen zulässigen Vereinfachungen.

Durch steigende Computerleistung sowie Verbesserungen in den numerischen Methoden konnten
sich Simulation

Um die Genauigkeit der Ergebnisse zu verbessern werden immer feinere Gebiete aufgelöst,
bei denen auch heutige Computersystem an ihre Grenzen stoßen. Es ist deshalb notwendig
auf adaptive Methoden zurückzugreifen, die die Auflösung an interessanten Stellen wie
zum Beispiel Grenzschichten, besonders belasteten Bauteilen oder Stellen mit hohem Fehler
erhöhen können. Dazu ist es aber nötig diesen Fehler zu kennen.

Heutige Methoden schätzen diesen Fehler ab, negieren aber seine Ursachen bei der Diskretisierung
und damit dem Übergang von der kontinuierlichen Differentialgleichung zur diskreten Gleichung.
Hier liegt der Ansatzpunkt der Fehlerbetrachtung über den Abbruchfehler. Dieser wird in der
vorliegenden Arbeit für Feldprobleme hergeleitet und seine Tauglichkeit anhand
von Beispielen nachgewiesen.

\section{Stand der Forschung}

ZUr Adaptiven Gitterverfeinerung sind das h-Verfahren, das r-Verfahrung sowie das p-Verfahren bekannt.
Während das p-Verfahren keine größere Anwendung in der STrömungssimulatio ngefunden hat


IN ROY wird gezeiugt das für lineare Probleme der Abbruchfehler
als Quelle des Diskretisierungsfehlers interpretiert werden kann
Eine Giztterverfeinerund basierend auf dem Abbruchfehler
vergrößert somit die Auflösung an der Quelle des Diskretisierungsfehler
und verkleinert somit den gesamten Diskretisierungsfehler.
Der Ansatz von Roy wird auf skalare Transportgleichen übertragen..

\section{Zielsetzung}
\cleardoublepage
