\section{Motivation}
Durch steigende Computerleistung sowie Verbesserungen in den numerischen Methoden konnten
sich die simulationsbasierten Produktentwicklungsprozesse einen festen Platz neben
Experimenten erarbeiten. Neben Kostenvorteilen ist es vor allem die Möglichkeit
schnell Aussagen über Anpassungen und Modifikationen an Produkten treffen zu können, die
simulationsbasierten Methoden zum Wachstum verhelfen.

Für die ingenieurmäßige Berechnung von Strömungsproblemen hat sich die Methode der
finiten Volumen auf breitem Gebiet durchgesetzt. Ihr Fokus liegt hier auf der Simulation
der Navier-Stokes Gleichungen, oft auch unter Vernachlässigung kompressibler Strömungseffekte und
anderen zulässigen Vereinfachungen.

Um die Genauigkeit der Ergebnisse zu verbessern werden immer feinere Gebiete aufgelöst,
bei denen auch heutige Computersysteme an ihre Grenzen stoßen. Es ist deshalb notwendig
auf adaptive Methoden zurückzugreifen, die die Auflösung an interessanten Stellen wie
zum Beispiel Grenzschichten oder besonders interessanten Strömungsgebieten selektiv erhöhen können.
Weiterhin sollen Fehler, die durch die numerische Behandlung der kontinuierlichen Strömungsgebiete
entstehen, minimiert werden. Dafür ist es notwendig, Indikatoren für den Fehler zu definieren.

Heutige Methoden schätzen diesen Fehler ab, negieren aber seine Ursachen bei der Diskretisierung
und damit dem Übergang von der kontinuierlichen Differentialgleichung zur diskreten Gleichung.
Hier liegt der Ansatzpunkt der Fehlerbetrachtung über den Abbruchfehler. Dieser wird in der
vorliegenden Arbeit für Transportprobleme hergeleitet und seine Genauigkeit an Beispielen
verifiziert. Weiterhin wird er zur Gitteradaption eingesetzt.


\section{Stand der Forschung}

Der Abbruchfehler als Indikator für die Gitteradaption ist in der Literatur bereits bekannt.
Roy~\cite{roy2} stellt für die eindimensionale Burger-Gleichung mittels
Finiten-Differenzen her und vergleicht anschließend den abbruchfehlerbasierten Fehlerindikator
mit lösungsbasierten Indikatoren wie Gradient oder Krümmung. Weiterhin wird gezeigt,
dass für lineare Probleme der Abbruchfehler
als Quelle des Diskretisierungsfehlers interpretiert werden kann und
eine Gitterverfeinerung basierend auf dem Abbruchfehler
somit die Auflösung an der Quelle des Diskretisierungsfehler vergrößert.
Damit verkleinert sich dann der gesamte Diskretisierungsfehler.

Vor der Gitteradaption muss eine Lösung auf dem Problemgebiet berechnet werden. Diese
Lösung beziehungsweise das Lösungsprogramm muss verifiziert werden. Veluri~\cite{veluri} demonstriert
Methoden zur Verifikation von Programm über den Vergleich von formaler und beobachteter
Konvergenzordnung und stellt außerdem die Methode der konstruierten Lösung vor.
Auch Roache~\cite{roache} stellt diese vor und demonstriert ihre Nützlichkeit bei der Verifikation von
Lösungsprogrammen.

Der Einfluss nicht-orthogonaler Gitter auf die Lösungsgenauigkeit gehen Huang und Prosperetti~\cite{grid_ortho}
ein. Lee und Tsuei~\cite{lee} zeigen die Ableitung des Abbruchfehlers bei Finite-Differenzen Verfahren
auf nicht-orthogonalen Gittern. Die Transformation von Kontrollgebieten der Finite-Elemente Methode
wird von Fath~\cite{fath} gezeigt.

\section{Zielsetzung}

Im ersten Teil der Arbeit werden grundlegende Konzepte und Begriffe der Arbeit erläutert.
Anschließend wird der Abbruchfehler für alle Terme von skalaren Transportgleichungen
abgeleitet. Dabei werden ein- und zweidimensionale Probleme sowie kartesische, orthogonale
und nicht-orthogonale Gitter betrachtet.

Anschließend wird der abgeleitete Abbruchfehlerindikator an konkreten Testfällen verifiziert.
Die Funktionsweise und Genauigkeit der dazu nötigen Lösungsprogramme wird ebenfalls
untersucht. Abschließend wird dann der Abbruchfehler zur lokalen Gitteradaption genutzt
und hierbei mit anderen bekannten Verfahren verglichen.



\cleardoublepage
