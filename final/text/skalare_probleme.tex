\section{Skalare Probleme}

Eine Vielzahl praktisch relevanter Probleme lässt sich durch skalare (partielle)
Differentialgleichungen beschreiben. Anwendungsgebiete reichen hier von Wärmetransportproblemen
über strukturmechanische Probleme (z.B. Auslenkung von Stäben und Membranen bei elastischem
Werkstoffverhalten) bis hin zu strömungsmechanischen Problemen (z.B. Geschwindigkeitspotential
wirbelfreier Strömungsfelder).

\subsection{Einfache Feldprobleme}

Einige kontinuumsmechanische Problemstellungen können durch eine Differentialgleichung
der Form

\begin{equation}
  -\frac{\partial}{\partial x_i}\left({\alpha \frac{\partial \phi}{\partial x_i} }\right)=g
  \label{eq:feldproblem_stat}
\end{equation}
 beschrieben werden. Diese muss auf dem gesamten betrachteten Gebiet $V$ gelten, das
vom Rand $S$ begrenzt wird. Bei $\phi=\phi(\mathbf{x})$ handelt es sich um die gesuchte,
skalare Feldgröße. $g=g(\mathbf{x})$ und $\alpha=\alpha(\mathbf{x})$ sind vorgegeben.


Weiterhin müssen auf dem gesamten Rand $S$ Randbedingungen
gegeben sein. Üblich sind hierbei folgende Typen von Randbedingungen:

\begin{itemize}
  \item Dirichletsche Randbedingung: $\quad \phi=\phi_S$
  \item Neumannsche Randbedingung: $\quad \alpha \frac{\partial \phi}{\partial x_i} n_i = b_s $
  \item Cauchysche Randbedingung: $\quad c_S \phi + \alpha \frac{\partial \phi}{\partial x_i} n_i = b_s$
\end{itemize}

Hierbei sind $\phi_S$, $b_S$ und $c_S$ vorgegebene Funktionen auf dem Rand $S$. Die
Komponenten des nach außen gerichteten Normaleneinheitsvektors an $S$ werden mit $n_i$ bezeichnet.

Treten mehrere Randbedingungstypen in einem Problem auf spricht man von gemischten Randwertproblemen.

Die mit Gleichung~\eqref{eq:feldproblem_stat} beschriebenen Probleme beinhalten keine Zeitabhängigkeit
und werden deshalb als stationär bezeichnet.
Im instationären, also zeitabhängigem Fall, erhalten alle Größen neben der Ortsabhängigkeit
eine Zeitabhängigkeit. Die entsprechende Differentialgleichung lautet damit
\begin{equation}
  \frac{\partial \phi}{\partial t}
  -\frac{\partial}{\partial x_i}\left({\alpha \frac{\partial \phi}{\partial x_i} }\right)=g
  \label{eq:feldproblem_instat}
\end{equation}
für die unbekannte Größe $\phi=\phi(\mathbf{x}, t)$.
Für instationäre Probleme muss neben den Randbedingungen auch eine Anfangsbedingung
$\phi(\mathbf{x}, t_0) = \phi_0(\mathbf{x})$ gegeben werden.

Im weiteren Verlauf der Arbeit werden nur stationäre Probleme betrachtet. Alle Ergebnisse
lassen sich aber auf instationäre Probleme übertragen.

\subsection{Allgemeine Transportgleichung}

Eine wichtige Problemklasse innerhalb des Maschinenbaus stellen Transportprobleme in
Festkörpern oder Fluiden dar. Bei der transportierten Größe kann es sich dabei beispielsweise
um joulsche Wärme (Wärmetransportprobleme) oder Stoffmengen (Stofftransportprobleme) handeln.

Stationäre Transportprobleme können in Differentialform durch die Gleichung
\begin{equation}
  \frac{\partial}{\partial x_i} \left({\rho v_i \phi
- \alpha \frac{\partial \phi}{\partial x_i} }\right) = f
\end{equation}
beschrieben werden. Hierbei stellt $f$ einen allgemeinen Quellterm,
$\frac{\partial}{\partial x_i} \rho v_i \phi$ den Konvektionsterm sowie
$\alpha \frac{\partial^2 \phi}{\partial x_i \partial x_i}$ den Diffusionsterm dar.
Für spezielle Probleme müssen die entsprechenden Parameter angepasst werden (s. z.B. \cite{num_maschbau}).

Alle Betrachtungen in der vorliegenden Arbeit bauen auf Problemen auf, die durch die
Transportgleichung beschrieben werden können.
