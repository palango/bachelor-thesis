%\newcommand{\pder}[2][]{\frac{\partial#1}{\partial#2}}
%\newcommand{\pderf}[1]{\frac{\partial f}{\partial#1}}
%\newcommand{\pderfs}[1]{\frac{\partial^2 f}{\partial#1}}

\newcommand{\fxi}{f_{\xi}}
\newcommand{\fxxi}{f_{\xi\xi}}
\newcommand{\fxxxi}{f_{\xi\xi\xi}}
\newcommand{\fxxxxi}{f_{\xi\xi\xi\xi}}
\newcommand{\xxi}{x_{\xi}}
\newcommand{\xxxi}{x_{\xi\xi}}
\newcommand{\xxxxi}{x_{\xi\xi\xi}}
\newcommand{\xxxxxi}{x_{\xi\xi\xi\xi}}
\newcommand{\yxi}{y_{\xi}}
\newcommand{\yxxi}{y_{\xi\xi}}
\newcommand{\yxxxi}{y_{\xi\xi\xi}}
\newcommand{\yxxxxi}{y_{\xi\xi\xi\xi}}

\newcommand{\feta}{f_{\eta}}
\newcommand{\feeta}{f_{\eta\eta}}
\newcommand{\feeeta}{f_{\eta\eta\eta}}
\newcommand{\feeeeta}{f_{\eta\eta\eta\eta}}
\newcommand{\xeta}{x_{\eta}}
\newcommand{\xeeta}{x_{\eta\eta}}
\newcommand{\xeeeta}{x_{\eta\eta\eta}}
\newcommand{\xeeeeta}{x_{\eta\eta\eta\eta}}
\newcommand{\yeta}{y_{\eta}}
\newcommand{\yeeta}{y_{\eta\eta}}
\newcommand{\yeeeta}{y_{\eta\eta\eta}}
\newcommand{\yeeeeta}{y_{\eta\eta\eta\eta}}


\newcommand{\etai}{\eta_{e1}}
\newcommand{\etaii}{\eta_{e2}}
\newcommand{\etilde}{\tilde{e}}


%\chapter{Transformation}

%Die Transformation vom physikalischen auf den logischen Bereich mit
%den physikalischen Koordinaten $(x, y)$ sowie den logischen
%Koordinaten $(\xi, \eta)$ kann wie folgt definiert werden:
%\begin{equation}
  %x=x(\xi,\eta),\quad y=y(\xi, \eta)
%\end{equation}
%\begin{equation}
  %\pder[x]{\xi}=\xi_x=\frac{y_{\eta}}{J}
%\end{equation}
%Damit folgt für für die bei Konvektion und Diffusion auftretende
%erste und zweite Ableitung:

%\begin{align*}
  %\pderf{x}&=\pder[\xi]{x}\pderf{\xi}+\pder[\eta]{x}\pderf{\eta}
  %=\frac{y_{\eta}}{J} f_{\xi} - \frac{y_{\xi}}{J}f_{\eta}\\
  %\pderf{y}&=\pder[\xi]{y}\pderf{\xi}+\pder[\eta]{y}\pderf{\eta}
  %=\frac{x_{\xi}}{J}f_{\eta}-\frac{x_{\eta}}{J} f_{\xi} \\
  %\pderfs{x^2} &= \pder{x}\left({\frac{y_{\eta}}{J} f_{\xi} - \frac{y_{\xi}}{J}f_{\eta}}\right)\\
               %&= \frac{\yeta^2}{J^2} \fxxi + \frac{\yxi^2}{J^2} \feeta-2 \frac{\yxi \yeta}{J^2} f_{\xi\eta}\\
               %&+ \frac{1}{J^3} \Big[\fxi
  %\Big(-2\xxi\yxi\yeta\yeeta + 2\xxi\yeta^2y_{\xi\eta} + \xxxi \yeta^3 - 2x_{\xi \eta}\yxi\yeta^2
  %-\xeta\yeta^2\yxxi + \xeta\yxi^2\yeeta + \yeeta\yxi^2\yeta\Big)\\
  %&+ \feta\Big(
  %2\xeta\yxi\yeta\yxxi - 2\xeta\yxi^2y_{\xi\eta} -\xeeta\yxi^3+2x_{\xi\eta}\yeta\yxi^2
  %+\xxi\yxi^2\yeeta-\xxi\yeta^2\yxxi-\xxxi\yeta^2\yxi
%\Big)\Big]
%\end{align*}


%Im logischen Gebiet können nun die vorhandenen Ableitungen von $f$
%($f_{\xi}$, $f_{\eta}$, $\dots$) diskretisiert werden.
%Es werden hierbei Zentraldifferenzen zweiter Ordnung gewählt.

%\begin{equation}
  %f_{\xi}=\frac{f_{i+1,j}-f_{i-1,j}}{2\Delta \xi} - \frac{f_{\xi\xi\xi}}{6}\Delta \xi^2 + HOT
%\end{equation}
%\begin{equation}
  %f_{\eta}=\frac{f_{i,j+1}-f_{i,j-1}}{2 \Delta \eta} - \frac{f_{\eta\eta\eta}}{6}\Delta \eta^2 + HOT
%\end{equation}
%\begin{equation}
  %\fxxi=\frac{f_{i+1,j}-2f_{i,j}+f_{i-1,j}}{\Delta \xi^2} -\frac{\fxxxxi}{12} \Delta \xi^2 + HOT
%\end{equation}
%\begin{equation}
  %\feeta=\frac{f_{i,j+1}-2f_{i,j}+f_{i,j-1}}{\Delta \eta^2} -\frac{\feeeeta}{12} \Delta \eta^2 + HOT
%\end{equation}
%\begin{equation}
  %f_{\xi\eta}=\frac{f_{i+1,j+1}-2f_{i,j}+f_{i-1,j-1}}{\Delta \xi\Delta \eta} -\frac{\fxxi+\feeta}{4} \Delta \eta \Delta \xi + HOT
%\end{equation}

%Werden diese nun eingesetzt ergibt sich für $\Delta \xi = 1$ folgende Form:

%\begin{equation}
  %\pderf{x}=\frac{y_{\eta}}{J}\frac{f_{i+1,j}-f_{i-1,j}}{2}
  %- \frac{y_{\xi}}{J}\frac{f_{i,j+1}-f_{i,j-1}}{2}
%\end{equation}

%sowie der entstehende Abbruchfehler:

%\begin{equation}
  %TE_x = -\frac{y_{\eta}}{J}\frac{f_{\xi\xi\xi}}{6}
  %+ \frac{y_{\xi}}{J}\frac{f_{\eta\eta\eta}}{6}+HOT
%\end{equation}

%Anschließend können die berechneten Werte für $f_{\xi},\dots$
%eingesetzt werden.

%\begin{align*}
  %\fxi &= \xxi f_x + \yxi f_y\\
  %\fxxi &= \xxxi f_x + \xxi^2f_{xx}+2\xxi \yxi f_{xy}
  %+\yxi^2 f_{yy} + \yxxi f_y\\
  %\fxxxi &= \xxxxi f_x + \yxxxi f_y + 3 \xxxi\xxi f_{xx}+
  %3 \yxxi\yxi f_{yy} + \left({3\xxxi\yxi+3\yxxi\xxi}\right)f_{xy}\\
  %&+ \xxi^3f_{xxx}+\yxi^3f_{yyy}+3\xxi^2\yxi f_{xxy}+3\xxi\yxi^2f_{xyy}\\
 %\feeeta &= \xeeeta f_x + \yeeeta f_y + 3 \xeeta\xeta f_{xx}+
  %3 \yeeta\yeta f_{yy} + \left({3\xeeta\yeta+3\yeeta\xeta}\right)f_{xy}\\
  %&+ \xeta^3f_{xxx}+\yeta^3f_{yyy}+3\xeta^2\yeta f_{xxy}+3\xeta\yeta^2f_{xyy}
%\end{align*}

%Es ergibt sich:

%\begin{align*}
  %TE_x&= -\frac{y_{\eta}}{J} \frac{1}{6}\Big[\xxxxi f_x + \yxxxi f_y + 3 \xxxi\xxi f_{xx}+
  %3 \yxxi\yxi f_{yy} + \left({3\xxxi\yxi+3\yxxi\xxi}\right)f_{xy}\\
%&+ \xxi^3f_{xxx}+\yxi^3f_{yyy}+3\xxi^2\yxi f_{xxy}+3\xxi\yxi^2f_{xyy}\Big]\\
%&+  \frac{y_{\xi}}{J}\frac{1}{6} \Big[\xeeeta f_x + \yeeeta f_y + 3 \xeeta\xeta f_{xx}+
  %3 \yeeta\yeta f_{yy} + \left({3\xeeta\yeta+3\yeeta\xeta}\right)f_{xy}\\
  %&+ \xeta^3f_{xxx}+\yeta^3f_{yyy}+3\xeta^2\yeta f_{xxy}+3\xeta\yeta^2f_{xyy}
%\Big]\\
  %&= TE_{x1} + TE_{x2} + TE_{x3}
%\end{align*}

%wobei $TE_{x1}$die ersten Ableitungen von $f$ enthält, usw.:

%\begin{align*}
  %TE_{x1} &= \frac{1}{6\ J}\left[{
  %\left({-\yeta\xxxxi + \yxi\xeeeta}\right) f_x +
  %\left({-\yeta\yxxxi + \yxi\yeeeta}\right) f_y
  %}\right]\\
  %TE_{x2} &= \frac{1}{2\ J} \Big[
  %\left({-\yeta\xxxi\xxi + \yxi\xeeta\xeta}\right) f_{xx}+
  %\left({-\yeta\yxxi\yxi + \yxi\yeeta\yeta}\right) f_{yy}\\&+
  %\left({-\yeta \left({\xxxi\yxi+\yxxi\xxi}\right) +
  %\yxi \left({\xeeta\yeta+\yeeta\xeta}\right)}\right) f_{xy}
  %\Big]\\
  %TE_{x3}&=\frac{1}{6\ J} \Big[
  %\left({-\yeta\xxi^3+\yxi\xeta^3}\right) f_{xxx}+
  %\yxi \yeta\left({-\yxi^2+\yeta^2}\right) f_{yyy}\\&+
  %3 \yxi \yeta \left({-\xxi^2+\xeta^2}\right) f_{xxy}+
  %3 \yxi \yeta \left({-\xxi\yxi+\xeta\yeta}\right) f_{xyy}
  %\Big]
%\end{align*}




\section{Abbruchfehler für nicht-orthogonale Gitter}

Mit der Anwendung des Gaußschen Integralsatzes müssen nun die Integrale über
die Seitenflächen bestimmt werden. Dies soll hier beispielhaft an der Ostseite geschehen.

\begin{equation}
  \int_{x_{e1}}^{x_{e2}}\left({\rho v_i\phi+\alpha \pder[\phi]{x_i}}\right)n_{ei}ds
  \label{eq:int_trans}
\end{equation}

Anstatt das Integral im physikalischen Bereich auszuwerten wird es zuerst in den
logischen Bereich transformiert. Auf dem dortigen kartesischen Gitter können dort einfacherer
Differenzenquotienten genutzt werden. Nachteil ist die höhere Komplexität durch
die Transformation.
\begin{equation}
  \int_{\gamma}f\ ds=\int_a^bf(\gamma (t)) \lVert \dot{\gamma}(t)\rVert_2 dt
\end{equation}

Damit folgt für das Integral aus Gleichung~\eqref{eq:int_trans}:

\begin{align*}
  \underbrace{
  \int_{\eta_{e1}}^{\eta_{e2}} (\rho v_1 \phi n_{e1})
  \left\lVert \frac{\partial x(\xi=e, \eta)}{\partial \eta} \right\rVert_2 d\eta
  }_I
  &+ \underbrace{
  \int_{\eta_{e1}}^{\eta_{e2}} (\alpha \pder[\phi]{x} n_{e1})
  \left\lVert \frac{\partial x(\xi=e, \eta)}{\partial \eta} \right\rVert_2 d\eta
}_{II}\\
  + \underbrace{
  \int_{\eta_{e1}}^{\eta_{e2}} (\rho v_2 \phi n_{e2})
  \left\lVert \frac{\partial x(\xi=e, \eta)}{\partial \eta} \right\rVert_2 d\eta
  }_{III}
  &+ \underbrace{
  \int_{\eta_{e1}}^{\eta_{e2}} (\alpha \pder[\phi]{x} n_{e2})
  \left\lVert \frac{\partial x(\xi=e, \eta)}{\partial \eta} \right\rVert_2 d\eta
  }_{IV}
\end{align*}

Bei Integration mit der Mittelpunktsregel folgt für Term $I$:
\begin{equation*}
  C=(\rho v_1 \phi n_{e1})
  \left\lVert \frac{\partial x(\xi=e, \eta)}{\partial \eta} \right\rVert_2
\end{equation*}

\begin{align*}
  I&= C\big\vert_{\tilde{e}} \underbrace{(\etaii-\etai)}_{=1} + \frac{1}{2} \pder[C]{\eta}\bigg\vert_{\tilde{e}}
  \underbrace{\left({(\etaii-\etilde)^2-(\etai-\etilde)^2}\right)}_{=0}
  +\frac{1}{6} \frac{\partial^2 C}{\partial \eta^2}\bigg\vert_{\tilde{e}}
  \underbrace{\left({(\etaii-\etilde)^3-(\etai-\etilde)^3}\right)}_{=\frac{1}{4}} + HOT\\
  &= C \big\vert_{\tilde{e}}+ \underbrace{\frac{1}{24}  \frac{\partial^2 C}{\partial \eta^2}\bigg\vert_{\tilde{e}}+HOT}_{TE_{int}}
\end{align*}

Im nächsten Schritt muss nun die $C$ durch die umgebenden Zellenmittelpunkte ausgedrückt werden.
\begin{align*}
   C\big\vert_{\tilde{e}}
   &= C\big\vert_{\tilde{E}}
   \left({\frac{\xi_{\tilde{e}}-\xi_{\tilde{P}}}{\xi_{\tilde{E}}-\xi_{\tilde{P}}}}\right)
   + C\big\vert_{\tilde{P}} \left({1-\frac{\xi_{\tilde{e}}-\xi_{\tilde{P}}}{\xi_{\tilde{E}}-\xi_{\tilde{P}}} }\right)
   + \frac{1}{2} \frac{\partial^2 C}{\partial \xi^2}\bigg\vert_{\tilde{P}}
   (\xi_{\tilde{e}}-\xi_{\tilde{E}})(\xi_{\tilde{e}}-\xi_{\tilde{P}})\\
  &+ \frac{1}{6}  \frac{\partial^3 C}{\partial \xi^3}\bigg\vert_{\tilde{P}}
   \left({(\xi_{\tilde{e}}-\xi_{\tilde{P}})^2-(\xi_{\tilde{E}}-\xi_{\tilde{P}})^2}\right)
   (\xi_{\tilde{e}}-\xi_{\tilde{P}}) + HOT\\
   &=\frac{1}{2} C \big\vert_{\tilde{E}} + \frac{1}{2} C \big\vert_{\tilde{P}}
   \underbrace{- \frac{1}{8} \frac{\partial^2 C}{\partial \xi^2}\bigg\vert_{\tilde{P}}
   - \frac{1}{16} \frac{\partial^3 C}{\partial \xi^3}\bigg\vert_{\tilde{P}} + HOT}_{TE_{konv}}
\end{align*}


Bei dem Diffusionsterm $II$ muss die Ableitung nach $x$ durch Ableitungen ins logische
Gebiet ersetzt werden:

\begin{equation}
  \pder[\phi]{x}=\pder[\xi]{x}\pder[\phi]{\xi}+\pder[\eta]{x}\pder[\phi]{\eta}
  = \frac{\yeta}{det\ J} \phi_{\xi} - \frac{\yxi}{det\ J} \phi_{\eta}
\end{equation}

\begin{equation}
  D =\left({\frac{\alpha\ n_{e1}}{det\ J}}\right)
  \left\lVert \frac{\partial x(\xi=e, \eta)}{\partial \eta} \right\rVert_2 
\left({\yeta(\phi_{\tilde{E}}-\phi_{\tilde{P}}) - \yxi(\phi_{\tilde{ne}}-\phi_{\tilde{se}})}\right)
\end{equation}

\begin{equation*}
  D\big\vert_{\etilde} =
  \left({\frac{\alpha\ n_{e1}}{det\ J}}\right)
  \left\lVert \frac{\partial x(\xi=e, \eta)}{\partial \eta} \right\rVert_2 
  \Bigg(\yeta(\phi_{\tilde{E}}-\phi_{\tilde{P}}) - \yxi(\phi_{\tilde{ne}}-\phi_{\tilde{se}})
      \underbrace{ -\frac{\yeta}{6} \frac{\partial^3 \phi}{\partial \xi^3} \bigg\vert_{\etilde}
      +\frac{\yeta}{6} \frac{\partial^3 \phi}{\partial \xi^3} \bigg\vert_{\etilde}
+ HOT}_{TE_{konv}}\Bigg)
\end{equation*}

Auch $\phi_{\tilde{ne}}$ und $\phi_{\tilde{se}}$ müssen durch die umliegenden Zellenwerte ausgedrückt werden.
Dies kann beispielsweise durch Interpolation erfolgen.

In der oben stehenden Gleichung treten auch $\yxi$ und $\yeta$ auf. Diese werden Metriken der Transformation genannt und
sind nur vom Gitter, nicht aber von der Lösungsfunktion abhängig.


Ab hier können wieder die bereits gezeigten Herleitungen genutzt werden.
\begin{equation}
  \begin{IEEEeqnarraybox}[][c]{rCl}
    \int\int_{P} \underbrace{\Pi_{\phi}(\xi, \eta) J }_{h} d\xi d\eta
    &=& h\big\vert_{\tilde{P}} (\xi_e-\xi_w)(\eta_n-\eta_s)\\
  & &+ h_{\xi}\big\vert_{\tilde{P}} \frac{(\xi_e-\xi_P)^2 - (\xi_w-\xi_P)^2}{2} (\eta_n-\eta_s)\\
  && + h_{\eta}\big\vert_{\tilde{P}} \frac{(\eta_n-\eta_P)^2-(\eta_s-\eta_P)^2}{2} (\xi_e-\xi_w) \\
  & &+ \frac{1}{2} h_{\xi\xi}\big\vert_{\tilde{P}}\frac{(\xi_e-\xi_P)^3 - (\xi_w-\xi_P)^3}{3} (\eta_n-\eta_s)\\
  &&+ \frac{1}{2} h_{\eta\eta}\big\vert_{\tilde{P}} \frac{(\eta_n-\eta_P)^3-(\eta_s-\eta_P)^3}{3} (\xi_e-\xi_w) \\
  & &+ h_{\xi\eta}\big\vert_{\tilde{P}} \frac{(\xi_e-\xi_P)^2 - (\xi_w-\xi_P)^2}{2} \cdot
  \frac{(\eta_n-\eta_P)^2-(\eta_s-\eta_P)^2}{2} + HOT\\
  &=&h\big\vert_{\tilde{P}} + \underbrace{\frac{1}{24} \left({h_{\xi\xi}\big\vert_{\tilde{P}}+h_{\eta\eta}\big\vert_{\tilde{P}}}\right)
+ HOT}_{TE_{source}}
\end{IEEEeqnarraybox}
\end{equation}






\subsection{Diskretisierung der Metriken}

Größen, die nur vom numerischen Gitter, nicht aber der Lösungsfunktion abhängen, werden Metriken genannt.
Meistens handelt es sich hierbei um physikalische Variablen, die nach logischen Koordinaten abgeleitet werden.
Beispiele sind $x_{\xi}$ oder $y_{\eta}$.

Die Metriken müssen für jeden Kontrollvolumen berechnet werden. Hierfür werden Approximation über
Differenzenquotienten genutzt. Beispielsweise ergibt sich auf einem zweidimensionalen nicht-orthogonalen
Gitter mit $\Delta\xi = \Delta \eta = 1$:
\begin{equation}
  x_{\xi} = \frac{x_{i+1, j} - x_{i-1,j}}{\Delta \xi} = x_{i+1, j} - x_{i-1,j} = a \cos \gamma
\end{equation}
Für die anderen Metriken erster Ordnung ergibt sich:
\begin{align*}
  x_{\eta} &= b\sin \beta\\
  y_{\xi} &= a \sin \gamma\\
  y_{\eta} &= b \cos \beta\\
  J &= x_{\xi}y_{\eta}-x_{\eta}y_{\xi} = a b \sin \theta
\end{align*}
