%\newcommand{\pder}[2][]{\frac{\partial#1}{\partial#2}}
%\newcommand{\pderf}[1]{\frac{\partial f}{\partial#1}}
%\newcommand{\pderfs}[1]{\frac{\partial^2 f}{\partial#1}}

\newcommand{\fxi}{f_{\xi}}
\newcommand{\fxxi}{f_{\xi\xi}}
\newcommand{\fxxxi}{f_{\xi\xi\xi}}
\newcommand{\fxxxxi}{f_{\xi\xi\xi\xi}}
\newcommand{\xxi}{x_{\xi}}
\newcommand{\xxxi}{x_{\xi\xi}}
\newcommand{\xxxxi}{x_{\xi\xi\xi}}
\newcommand{\xxxxxi}{x_{\xi\xi\xi\xi}}
\newcommand{\yxi}{y_{\xi}}
\newcommand{\yxxi}{y_{\xi\xi}}
\newcommand{\yxxxi}{y_{\xi\xi\xi}}
\newcommand{\yxxxxi}{y_{\xi\xi\xi\xi}}

\newcommand{\feta}{f_{\eta}}
\newcommand{\feeta}{f_{\eta\eta}}
\newcommand{\feeeta}{f_{\eta\eta\eta}}
\newcommand{\feeeeta}{f_{\eta\eta\eta\eta}}
\newcommand{\xeta}{x_{\eta}}
\newcommand{\xeeta}{x_{\eta\eta}}
\newcommand{\xeeeta}{x_{\eta\eta\eta}}
\newcommand{\xeeeeta}{x_{\eta\eta\eta\eta}}
\newcommand{\yeta}{y_{\eta}}
\newcommand{\yeeta}{y_{\eta\eta}}
\newcommand{\yeeeta}{y_{\eta\eta\eta}}
\newcommand{\yeeeeta}{y_{\eta\eta\eta\eta}}


\newcommand{\etai}{\eta_{e1}}
\newcommand{\etaii}{\eta_{e2}}
\newcommand{\etilde}{\tilde{e}}


%\chapter{Transformation}

%Die Transformation vom physikalischen auf den logischen Bereich mit
%den physikalischen Koordinaten $(x, y)$ sowie den logischen
%Koordinaten $(\xi, \eta)$ kann wie folgt definiert werden:
%\begin{equation}
  %x=x(\xi,\eta),\quad y=y(\xi, \eta)
%\end{equation}
%\begin{equation}
  %\pder[x]{\xi}=\xi_x=\frac{y_{\eta}}{J}
%\end{equation}
%Damit folgt für für die bei Konvektion und Diffusion auftretende
%erste und zweite Ableitung:

%\begin{align*}
  %\pderf{x}&=\pder[\xi]{x}\pderf{\xi}+\pder[\eta]{x}\pderf{\eta}
  %=\frac{y_{\eta}}{J} f_{\xi} - \frac{y_{\xi}}{J}f_{\eta}\\
  %\pderf{y}&=\pder[\xi]{y}\pderf{\xi}+\pder[\eta]{y}\pderf{\eta}
  %=\frac{x_{\xi}}{J}f_{\eta}-\frac{x_{\eta}}{J} f_{\xi} \\
  %\pderfs{x^2} &= \pder{x}\left({\frac{y_{\eta}}{J} f_{\xi} - \frac{y_{\xi}}{J}f_{\eta}}\right)\\
               %&= \frac{\yeta^2}{J^2} \fxxi + \frac{\yxi^2}{J^2} \feeta-2 \frac{\yxi \yeta}{J^2} f_{\xi\eta}\\
               %&+ \frac{1}{J^3} \Big[\fxi
  %\Big(-2\xxi\yxi\yeta\yeeta + 2\xxi\yeta^2y_{\xi\eta} + \xxxi \yeta^3 - 2x_{\xi \eta}\yxi\yeta^2
  %-\xeta\yeta^2\yxxi + \xeta\yxi^2\yeeta + \yeeta\yxi^2\yeta\Big)\\
  %&+ \feta\Big(
  %2\xeta\yxi\yeta\yxxi - 2\xeta\yxi^2y_{\xi\eta} -\xeeta\yxi^3+2x_{\xi\eta}\yeta\yxi^2
  %+\xxi\yxi^2\yeeta-\xxi\yeta^2\yxxi-\xxxi\yeta^2\yxi
%\Big)\Big]
%\end{align*}


%Im logischen Gebiet können nun die vorhandenen Ableitungen von $f$
%($f_{\xi}$, $f_{\eta}$, $\dots$) diskretisiert werden.
%Es werden hierbei Zentraldifferenzen zweiter Ordnung gewählt.

%\begin{equation}
  %f_{\xi}=\frac{f_{i+1,j}-f_{i-1,j}}{2\Delta \xi} - \frac{f_{\xi\xi\xi}}{6}\Delta \xi^2 + HOT
%\end{equation}
%\begin{equation}
  %f_{\eta}=\frac{f_{i,j+1}-f_{i,j-1}}{2 \Delta \eta} - \frac{f_{\eta\eta\eta}}{6}\Delta \eta^2 + HOT
%\end{equation}
%\begin{equation}
  %\fxxi=\frac{f_{i+1,j}-2f_{i,j}+f_{i-1,j}}{\Delta \xi^2} -\frac{\fxxxxi}{12} \Delta \xi^2 + HOT
%\end{equation}
%\begin{equation}
  %\feeta=\frac{f_{i,j+1}-2f_{i,j}+f_{i,j-1}}{\Delta \eta^2} -\frac{\feeeeta}{12} \Delta \eta^2 + HOT
%\end{equation}
%\begin{equation}
  %f_{\xi\eta}=\frac{f_{i+1,j+1}-2f_{i,j}+f_{i-1,j-1}}{\Delta \xi\Delta \eta} -\frac{\fxxi+\feeta}{4} \Delta \eta \Delta \xi + HOT
%\end{equation}

%Werden diese nun eingesetzt ergibt sich für $\Delta \xi = 1$ folgende Form:

%\begin{equation}
  %\pderf{x}=\frac{y_{\eta}}{J}\frac{f_{i+1,j}-f_{i-1,j}}{2}
  %- \frac{y_{\xi}}{J}\frac{f_{i,j+1}-f_{i,j-1}}{2}
%\end{equation}

%sowie der entstehende Abbruchfehler:

%\begin{equation}
  %TE_x = -\frac{y_{\eta}}{J}\frac{f_{\xi\xi\xi}}{6}
  %+ \frac{y_{\xi}}{J}\frac{f_{\eta\eta\eta}}{6}+HOT
%\end{equation}

%Anschließend können die berechneten Werte für $f_{\xi},\dots$
%eingesetzt werden.

%\begin{align*}
  %\fxi &= \xxi f_x + \yxi f_y\\
  %\fxxi &= \xxxi f_x + \xxi^2f_{xx}+2\xxi \yxi f_{xy}
  %+\yxi^2 f_{yy} + \yxxi f_y\\
  %\fxxxi &= \xxxxi f_x + \yxxxi f_y + 3 \xxxi\xxi f_{xx}+
  %3 \yxxi\yxi f_{yy} + \left({3\xxxi\yxi+3\yxxi\xxi}\right)f_{xy}\\
  %&+ \xxi^3f_{xxx}+\yxi^3f_{yyy}+3\xxi^2\yxi f_{xxy}+3\xxi\yxi^2f_{xyy}\\
 %\feeeta &= \xeeeta f_x + \yeeeta f_y + 3 \xeeta\xeta f_{xx}+
  %3 \yeeta\yeta f_{yy} + \left({3\xeeta\yeta+3\yeeta\xeta}\right)f_{xy}\\
  %&+ \xeta^3f_{xxx}+\yeta^3f_{yyy}+3\xeta^2\yeta f_{xxy}+3\xeta\yeta^2f_{xyy}
%\end{align*}

%Es ergibt sich:

%\begin{align*}
  %TE_x&= -\frac{y_{\eta}}{J} \frac{1}{6}\Big[\xxxxi f_x + \yxxxi f_y + 3 \xxxi\xxi f_{xx}+
  %3 \yxxi\yxi f_{yy} + \left({3\xxxi\yxi+3\yxxi\xxi}\right)f_{xy}\\
%&+ \xxi^3f_{xxx}+\yxi^3f_{yyy}+3\xxi^2\yxi f_{xxy}+3\xxi\yxi^2f_{xyy}\Big]\\
%&+  \frac{y_{\xi}}{J}\frac{1}{6} \Big[\xeeeta f_x + \yeeeta f_y + 3 \xeeta\xeta f_{xx}+
  %3 \yeeta\yeta f_{yy} + \left({3\xeeta\yeta+3\yeeta\xeta}\right)f_{xy}\\
  %&+ \xeta^3f_{xxx}+\yeta^3f_{yyy}+3\xeta^2\yeta f_{xxy}+3\xeta\yeta^2f_{xyy}
%\Big]\\
  %&= TE_{x1} + TE_{x2} + TE_{x3}
%\end{align*}

%wobei $TE_{x1}$die ersten Ableitungen von $f$ enthält, usw.:

%\begin{align*}
  %TE_{x1} &= \frac{1}{6\ J}\left[{
  %\left({-\yeta\xxxxi + \yxi\xeeeta}\right) f_x +
  %\left({-\yeta\yxxxi + \yxi\yeeeta}\right) f_y
  %}\right]\\
  %TE_{x2} &= \frac{1}{2\ J} \Big[
  %\left({-\yeta\xxxi\xxi + \yxi\xeeta\xeta}\right) f_{xx}+
  %\left({-\yeta\yxxi\yxi + \yxi\yeeta\yeta}\right) f_{yy}\\&+
  %\left({-\yeta \left({\xxxi\yxi+\yxxi\xxi}\right) +
  %\yxi \left({\xeeta\yeta+\yeeta\xeta}\right)}\right) f_{xy}
  %\Big]\\
  %TE_{x3}&=\frac{1}{6\ J} \Big[
  %\left({-\yeta\xxi^3+\yxi\xeta^3}\right) f_{xxx}+
  %\yxi \yeta\left({-\yxi^2+\yeta^2}\right) f_{yyy}\\&+
  %3 \yxi \yeta \left({-\xxi^2+\xeta^2}\right) f_{xxy}+
  %3 \yxi \yeta \left({-\xxi\yxi+\xeta\yeta}\right) f_{xyy}
  %\Big]
%\end{align*}




\section{Abbruchfehler für nicht-orthogonale Gitter}

Bei der Verwendung von orthogonalen Gittern vereinfachen sich viele Terme oder
heben sich mit der Gegenseite auf. Diese müssen im Falle eines nicht-orthogonalen
numerischen Gitters beachtet werden, soll die Berechnung realistische Werte liefern.

Da viele der zusätzlichen Terme auf der Verzerrtheit des Gitters beruhen und eine
Betrachtung im physikalischen Bereich verkomplizieren, wird eine Transformation auf
den logischen Bereich durchgeführt. Der Vorteil besteht darin, dass auf dem dortigen kartesischen
Gitter die Anwendung von Differenzenschemata einfach ist und man die zusätzlichen Fehler des
verzerrten Gitters vermeiden kann. Der Nachteil ist die zusätzliche Komplexität durch
die Transformation. Die Transformation eines Kontrollvolumens vom physikalischen in
den logischen Bereich ist in Abbildung~\ref{fig:no-trans} dargestellt.
\begin{figure}[ht]
  \begin{tikzpicture}[scale=1.3]
  \draw[->, thick] (-1,-0.5) -- (-0.5,-0.5) node[right] {$x$} coordinate(x axis);
  \draw[->, thick] (-1,-0.5) -- (-1,0) node[above] {$y$} coordinate(y axis);
  \fill[tud0a] (0,0) -- (2,0) -- (3,2) -- (-1, 3) --cycle;
  \draw[thick] (0,0) -- (2,0) -- (3,2) -- (-1, 3) --cycle;

  \fill (1,1.25) circle[radius=1.5pt];
  \node (x) at (1,1.25) [label=above:$P$] {};
  \fill (2,0) circle[radius=1.5pt];
  \node (x) at (2,0) [label=below right:$\vec x_{e1}$] {};
  \fill (3,2) circle[radius=1.5pt];
  \node (x) at (3,2) [label=right:$\vec x_{e2}$] {};

  \fill (-1,3) circle[radius=1.5pt];
  \fill (0,0) circle[radius=1.5pt];

  \fill (1,0) circle[radius=1pt];
  \node (x) at (1,0) [label=below:$s$] {};
  \fill (2.5,1) circle[radius=1.5pt];
  \node (x) at (2.5,1) [label=left:$e$] {};
  \fill (1,2.5) circle[radius=1pt];
  \node (x) at (1,2.5) [label=above:$n$] {};
  \fill (-0.5,1.5) circle[radius=1pt];
  \node (x) at (-0.5,1.5) [label=left:$w$] {};


  \draw[->, thick] (2.5,1) -> (3.5,0.5);
  \node (x) at (3.5,0.5) [label=above:$\vec n_e$] {};



  % xi-eta
  \draw[->, thick] (4.5,-0.5) -- (5,-0.5) node[right] {$\xi$} coordinate(x axis);
  \draw[->, thick] (4.5,-0.5) -- (4.5,0) node[above] {$\eta$} coordinate(y axis);
  \fill[tud0a] (5, 0) rectangle (7, 2);
  \draw[thick] (5, 0) rectangle (7, 2);
  \fill (7, 1)  circle[radius=1.5pt];
  \node (x) at (7,1) [label={right}:{$\tilde{e}$}] {};

  \fill (5, 0)  circle[radius=1.5pt];
  \fill (5, 2)  circle[radius=1.5pt];
  \fill (5, 1)  circle[radius=1pt];
  \fill (6, 2)  circle[radius=1pt];
  \fill (6, 0)  circle[radius=1pt];

  \fill (7, 0)  circle[radius=1.5pt];
  \node (x) at (7,0) [label=below:{$(\xi_{e1},\eta_{e1})$}] {};
  \fill (7, 2)  circle[radius=1.5pt];
  \node (x) at (7,2) [label=above:{$(\xi_{e2}, \eta_{e2})$}] {};

  \fill (6, 1)  circle[radius=1.5pt];
  \node (x) at (6,1) [label=above:{$\tilde{P}$}] {};
\end{tikzpicture}

\centering
\caption{Transformation eines Kontrollvolumens}
\label{fig:no-trans}
\end{figure}

\noindent Die Ableitung des Abbruchfehlers soll hier beispielhaft an der östlichen Seite des Kontrollvolumens
gezeigt werden. Bei einer Transportgleichung mit konvektivem und diffusivem Term entsteht damit folgendes
Integral:
\begin{equation}
  \int_{\mathbf{x}_{e1}}^{\mathbf{x}_{e2}}\left({\rho v_i\phi+\alpha \pder[\phi]{x_i}}\right)n_{ei}ds
  \label{eq:int_trans}
\end{equation}
Hierbei bezeichnen $\mathbf{x}_{e1}$ und $\mathbf{x}_{e2}$ Anfangs- und Endpunkt, sowie $\mathbf{n}_e$
den Einheitsnormalenvektor der Ostseite, wie in Abbildung~\ref{fig:no-trans} gezeigt.
Statt der direkten Auswertung erfolgt nun die Transformation ins logische Gebiet. Dazu wird die
Transformationsregel eines Wegintegrals genutzt.
\begin{equation}
  \int_{\gamma}f\ ds=\int_a^bf(\gamma (t)) \lVert \dot{\gamma}(t)\rVert_2 dt
\end{equation}
Bei der Koordinatentransformation ergibt sich, dass der Wert von $\xi$ auf der Ostseite konstant sein muss.
Die Transformation von Gleichung~\eqref{eq:int_trans} ergibt:
\begin{align*}
  \underbrace{
    \int_{\eta_{e1}}^{\eta_{e2}} \left({\rho v_1 \phi n_{e1}}\right)
  \left\lVert \frac{\partial x(\xi=e, \eta)}{\partial \eta} \right\rVert_2 d\eta
  }_I
  &+ \underbrace{
  \int_{\eta_{e1}}^{\eta_{e2}} \left({\alpha \pder[\phi]{x} n_{e1}}\right)
  \left\lVert \frac{\partial x(\xi=e, \eta)}{\partial \eta} \right\rVert_2 d\eta
}_{II}\\
  + \underbrace{
  \int_{\eta_{e1}}^{\eta_{e2}} (\rho v_2 \phi n_{e2})
  \left\lVert \frac{\partial x(\xi=e, \eta)}{\partial \eta} \right\rVert_2 d\eta
  }_{III}
  &+ \underbrace{
  \int_{\eta_{e1}}^{\eta_{e2}} \left({\alpha \pder[\phi]{x} n_{e2}}\right)
  \left\lVert \frac{\partial x(\xi=e, \eta)}{\partial \eta} \right\rVert_2 d\eta
  }_{IV}
\end{align*}
Hierbei beschreiben die Terme $I$ und $III$ die konvektiven Flüsse, die Terme $II$ und $IV$ sind
diffusive Flüsse. Bei $III$ und $IV$ handelt es sich um Terme, die auf orthogonalen Gittern wegfallen würden,
da hier die zweite Komponente des Einheitsnormalenvektors Null wäre.
Im Folgenden werden nun wieder Abbruchfehler der einzelnen Terme abgeleitet. Bei den Flüssen muss dabei
einerseits der Abbruchfehler der Integralapproximation über die Mittelpunktsregel ermittelt werden.
Andererseits muss der Abbruchfehler aus der Fluss- bzw. Ableitungsapproximation ermittelt werden.



\paragraph{Konvektive Terme}

\noindent
Bei Integration mit der Mittelpunktsregel folgt für Term $I$:
\begin{equation*}
  C=(\rho v_1 \phi n_{e1})
  \left\lVert \frac{\partial x(\xi=e, \eta)}{\partial \eta} \right\rVert_2
\end{equation*}

\begin{align*}
  I&= C\big\vert_{\tilde{e}} \underbrace{(\etaii-\etai)}_{=1} + \frac{1}{2} \pder[C]{\eta}\bigg\vert_{\tilde{e}}
  \underbrace{\left({(\etaii-\etilde)^2-(\etai-\etilde)^2}\right)}_{=0}
  +\frac{1}{6} \frac{\partial^2 C}{\partial \eta^2}\bigg\vert_{\tilde{e}}
  \underbrace{\left({(\etaii-\etilde)^3-(\etai-\etilde)^3}\right)}_{=\frac{1}{4}} + HOT\\
  &= C \big\vert_{\tilde{e}}+ \underbrace{\frac{1}{24}  \frac{\partial^2 C}{\partial \eta^2}
\bigg\vert_{\tilde{e}}+HOT}_{TE_{C,\,int,\,e}}
\end{align*}
Hierbei wird die Abkürzung $C$ zur Übersichtlichkeit eingeführt und $\tilde{e}$ bezeichnet den Mittelpunkt
der transformierten Ostseite (siehe Abbildung~\ref{fig:no-trans}).  Nach dem allgemeinen Vorgehen bei der
Methode der Finiten Volumen muss nun $C\vert_{\tilde{e}}$ durch die umgebenden Zellenmittelpunkte ausgedrückt werden.
Dies soll hier mit dem Zentraldifferenzenschema gezeigt werden.
\begin{align*}
   C\big\vert_{\tilde{e}}
   =& C\big\vert_{\tilde{E}}
   \left({\frac{\xi_{\tilde{e}}-\xi_{\tilde{P}}}{\xi_{\tilde{E}}-\xi_{\tilde{P}}}}\right)
   + C\big\vert_{\tilde{P}} \left({1-\frac{\xi_{\tilde{e}}-\xi_{\tilde{P}}}{\xi_{\tilde{E}}-\xi_{\tilde{P}}} }\right)
   + \frac{1}{2} \frac{\partial^2 C}{\partial \xi^2}\bigg\vert_{\tilde{P}}
   (\xi_{\tilde{e}}-\xi_{\tilde{E}})(\xi_{\tilde{e}}-\xi_{\tilde{P}})\\
   &+ \frac{1}{6}  \frac{\partial^3 C}{\partial \xi^3}\bigg\vert_{\tilde{P}}
   \left({(\xi_{\tilde{e}}-\xi_{\tilde{P}})^2-(\xi_{\tilde{E}}-\xi_{\tilde{P}})^2}\right)
   (\xi_{\tilde{e}}-\xi_{\tilde{P}}) + HOT\\
   =&\frac{1}{2} C \big\vert_{\tilde{E}} + \frac{1}{2} C \big\vert_{\tilde{P}}
   \underbrace{- \frac{1}{8} \frac{\partial^2 C}{\partial \xi^2}\bigg\vert_{\tilde{P}}
   - \frac{1}{16} \frac{\partial^3 C}{\partial \xi^3}\bigg\vert_{\tilde{P}} + HOT}_{TE_{CDS,\,e}}
\end{align*}



\paragraph{Diffusive Terme}
\noindent
Bei Term $II$ wird ebenfalls die Mittelpunktsregel angewandt. Mit der Abkürzung $D$
ergibt sich hier:
\begin{equation*}
  D=  \left({\alpha \pder[\phi]{x} n_{e1}}\right)
  \left\lVert \frac{\partial x(\xi=e, \eta)}{\partial \eta} \right\rVert_2 d\eta
\end{equation*}

\begin{align*}
  II&= D\big\vert_{\tilde{e}} \underbrace{(\etaii-\etai)}_{=1} + \frac{1}{2}
  \frac{\partial^2 D}{\partial \eta^2} \bigg\vert_{\tilde{e}}
  \underbrace{\left({(\etaii-\etilde)^2-(\etai-\etilde)^2}\right)}_{=0}
  +\frac{1}{6} \frac{\partial^3 D}{\partial \eta^3}\bigg\vert_{\tilde{e}}
  \underbrace{\left({(\etaii-\etilde)^3-(\etai-\etilde)^3}\right)}_{=\frac{1}{4}} + HOT\\
  &= D \big\vert_{\tilde{e}}+ \underbrace{\frac{1}{24}  \frac{\partial^3 D}{\partial \eta^3}
\bigg\vert_{\tilde{e}}+HOT}_{TE_{D,\,int,\,e}}
\end{align*}
Hier muss nun beachtet werden, dass $D$ selbst noch die Ableitung $\pder[\phi]{x}$ im physikalischen Gebiet enthält.
Diese muss wie das Kontrollvolumen ins logische Gebiet transformiert werden. Die Transformation ergibt
mit der Jakobimatrix aus Gleichung~\eqref{eq:detj} für
die Ableitung $\pder[\phi]{x}$:% und $\pder[\phi]{y}$:
\begin{align}
  \pder[\phi]{x}&=\pder[\xi]{x}\pder[\phi]{\xi}+\pder[\eta]{x}\pder[\phi]{\eta}
  = \frac{\yeta}{det(J)} \phi_{\xi} - \frac{\yxi}{det(J)} \phi_{\eta}
\end{align}
  %\pder[\phi]{y}&=\pder[\xi]{y}\pderf{\xi}+\pder[\eta]{y}\pderf{\eta}
  %=\frac{x_{\xi}}{det(J)}\phi_{\eta}-\frac{x_{\eta}}{det(J)} \phi_{\xi}
%\end{align}

Bei der anschließenden Approximation von $D_{\tilde{e}}$ durch die Mittelpunkte der Nachbarkontrollvolumen
muss ebenfalls die $\frac{\partial \phi}{\partial x}$ transformiert werden. Damit ergibt sich folgende Gleichung:



%\begin{equation}
  %D =\left({\frac{\alpha\ n_{e1}}{det\ J}}\right)
  %\left\lVert \frac{\partial x(\xi=e, \eta)}{\partial \eta} \right\rVert_2 
%\left({\yeta(\phi_{\tilde{E}}-\phi_{\tilde{P}}) - \yxi(\phi_{\tilde{ne}}-\phi_{\tilde{se}})}\right)
%\end{equation}

\begin{align*}
  D\big\vert_{\etilde} &=
  \left({\frac{\alpha\ n_{e1}}{det\ J}}\right)
  \left\lVert \frac{\partial x(\xi=e, \eta)}{\partial \eta} \right\rVert_2 
  \Bigg(\yeta(\phi_{\tilde{E}}-\phi_{\tilde{P}}) - \yxi(\phi_{\tilde{ne}}-\phi_{\tilde{se}})
      \underbrace{ -\frac{\yeta}{6} \frac{\partial^3 \phi}{\partial \xi^3} \bigg\vert_{\etilde}
      +\frac{\yeta}{6} \frac{\partial^3 \phi}{\partial \xi^3} \bigg\vert_{\etilde}
+ HOT}_{TE_{D,\,e}}\Bigg)
\end{align*}
Auch $\phi_{\tilde{ne}}$ und $\phi_{\tilde{se}}$ müssen durch die umliegenden Zellenwerte ausgedrückt werden.
Dies kann beispielsweise durch lineare Interpolation erfolgen.
\begin{equation}
\phi_{\tilde{ne}} = \frac{\phi_{\tilde{P}} +\phi_{\tilde{E}} + \phi_{\tilde{NE}}+ \phi_{\tilde{N}}}{4} 
\end{equation}

In der oben stehenden Gleichung treten auch $\yxi$ und $\yeta$ auf. Diese werden Metriken der Transformation genannt und
sind nur vom Gitter, nicht aber von der Lösungsfunktion abhängig. Ihre Approximation
wird in Abschnitt~\ref{sec:verz-metrik} gezeigt.





\paragraph{Quellterm}
\noindent
Zur Berechnung des Abbruchfehlers des Quellterms von nicht-orthogonalen Gittern wird der Transformationssatz benötigt.
Dieser lautet im zweidimensionalen Fall:
\begin{equation}
  \int \int_A f(x,y)\ dx\,dy = \int \int_P f\left({x(\xi,\,\eta),\ y(\xi,\,\eta)}\right)\ det(J)\ d\xi\,d\eta
\end{equation}
Angewandt auf den Quellterm und der Abkürzung $Q$ ergibt sich für den Quellterm:
\begin{equation}
  \begin{IEEEeqnarraybox}[][c]{rCl}
    \int\int_{P} \underbrace{\Pi_{\phi}(\xi, \eta) det(J) }_{Q} d\xi d\eta
    &=& Q\big\vert_{\tilde{P}} (\xi_e-\xi_w)(\eta_n-\eta_s)\\
  & &+ Q_{\xi}\big\vert_{\tilde{P}} \frac{(\xi_e-\xi_P)^2 - (\xi_w-\xi_P)^2}{2} (\eta_n-\eta_s)\\
  && + Q_{\eta}\big\vert_{\tilde{P}} \frac{(\eta_n-\eta_P)^2-(\eta_s-\eta_P)^2}{2} (\xi_e-\xi_w) \\
  & &+ \frac{1}{2} Q_{\xi\xi}\big\vert_{\tilde{P}}\frac{(\xi_e-\xi_P)^3 - (\xi_w-\xi_P)^3}{3} (\eta_n-\eta_s)\\
  &&+ \frac{1}{2} Q_{\eta\eta}\big\vert_{\tilde{P}} \frac{(\eta_n-\eta_P)^3-(\eta_s-\eta_P)^3}{3} (\xi_e-\xi_w) \\
  & &+ Q_{\xi\eta}\big\vert_{\tilde{P}} \frac{(\xi_e-\xi_P)^2 - (\xi_w-\xi_P)^2}{2} \cdot
  \frac{(\eta_n-\eta_P)^2-(\eta_s-\eta_P)^2}{2} + HOT\\
  &=&Q\big\vert_{\tilde{P}} + \underbrace{\frac{1}{24} \left({Q_{\xi\xi}\big\vert_{\tilde{P}}+Q_{\eta\eta}\big\vert_{\tilde{P}}}\right)
+ HOT}_{TE_{Q}}
\end{IEEEeqnarraybox}
\end{equation}
Die gezeigten Ableitungen müssen nun für die anderen Seiten des Kontrollvolumens wiederholt werden.
Anschließend werden sie summiert und ergeben dann den Abbruchfehler eines Kontrollvolumens auf
einem nicht-orthogonalem Gitter.





\subsection{Diskretisierung der Metriken}
\label{sec:verz-metrik}

In den oben gezeigten Ableitungen treten mehrfach Größen auf, die nur vom numerischen Gitter, nicht aber
von der Lösungsfunktion abhängen. Diese werden Metriken genannt.
Meistens handelt es sich hierbei um physikalische Variablen, die nach logischen Koordinaten abgeleitet werden.
Beispiele sind $x_{\xi}$ oder $y_{\eta\eta}$.

Metriken sind analytische Größen und müssen im Rahmen der Berechnung diskretisiert werden.
Treten die Metriken bei einer Approximation über die Grenzen des Kontrollvolumens auf,
ist die Diskretisierung verschieden zur Approximation innerhalb der Grenzen eines Kontrollvolumens.

\paragraph{Diskretisierung innerhalb eines Kontrollvolumens}
\begin{figure}[ht]
  \begin{tikzpicture}[scale=1.2]
  \draw[->, thick] (-2,0) -- (-1.5,0) node[right] {$x$} coordinate(x axis);
  \draw[->, thick] (-2,0) -- (-2,0.5) node[above] {$y$} coordinate(y axis);

  \fill[tud0a] (0,0) -- (3,0.5) -- (3.5,3.5) -- (0.5, 3) --cycle;

  \draw[thick] (-0.5,-0.08333) -- (4,0.66666);
  \draw[thick] (-0.08333,-0.5) -- (0.66666,4);
  \draw[thick] (-0.5,-0.16666+3) -- (4.5,3.66666);
  \draw[thick] (-.08333+3,0) -- (0.5833+3,4);

  \draw[thick,<->] (15mm,0mm) arc (0:9.5:15mm);
  \draw[thick,<->] (0mm,15mm) arc (90:80.5:15mm);
  \draw[thick,<->] (9mm,1.5mm) arc (9.5:81:9mm);

  \node (x) at (1.5,0.05) [label=below:$\gamma$]{};
  \node (x) at (0.05,1.5) [label=left:$\beta$]{};
  \node (x) at (0.5,0.5) [label=above right:$\theta$]{};

  \node (x) at (2.5,0.3) [label=above left:$a$]{};
  \node (x) at (0.3,2.5) [label=below right:$b$]{};

  %\node (x) at (4,0.66666) [label=right:{$\eta=j-\frac{1}{2}$}]{};
  %\node (x) at (4.5,3.66666) [label=right:{$\eta=j+\frac{1}{2}$}]{};
  %\node (x) at (0.66666,4) [label=left:{$\xi=i-\frac{1}{2}$}]{};
  %\node (x) at (0.66666+3,4) [label=left:{$\xi=i+\frac{1}{2}$}]{};

  \draw[thick] (0,0) -- (2,0);
  \draw[thick] (0,0) -- (0,2);
\end{tikzpicture}

\centering
\caption{Größen für die Berechnung der Metriken innerhalb des Kontrollvolumens}
\end{figure}

\noindent
Die Metriken innerhalb eines Kontrollvolumens sind beispielsweise für die Transformation
des Quellterms nötig. Zur Diskretisierung der Metriken werden
Differenzenquotienten als Approximationen genutzt \cite{lee}.
Beispielsweise ergibt sich auf einem zweidimensionalen nicht-orthogonalen
Gitter mit $\Delta\xi = \Delta \eta = 1$:
\begin{equation}
  x_{\xi} = \frac{x_{i+1, j} - x_{i-1,j}}{\Delta \xi} = x_{i+1, j} - x_{i-1,j} = a \cos \gamma
\end{equation}
Für die anderen Metriken erster Ordnung ergeben sich:
\begin{align*}
  x_{\eta} &= b\sin \beta\\
  y_{\xi} &= a \sin \gamma\\
  y_{\eta} &= b \cos \beta
\end{align*}
Eine weitere benötige Größe stellt die Determinante der Jakobimatrix
dar. Sie ist für jedes Kon\-troll\-volumen unterschiedlich. Werden
die diskretisierten Metriken eingesetzt ergibt sich folgende Gleichung:
\begin{align}
  det(J) &= x_{\xi}y_{\eta}-x_{\eta}y_{\xi}\nonumber\\
    &= a \cos \gamma \, b \cos \beta - 
       a \sin \beta \, b \sin \gamma\nonumber\\
       &= ab(\cos\gamma\cos\beta-\sin\beta\sin\gamma)
\end{align}
Hier kann nun das folgende Additionstheorem
angewendet werden.
\begin{equation*}
\cos(\beta+\gamma)=\cos\gamma\cos\beta-\sin\beta\sin\gamma
\end{equation*}
Da weiterhin bekannt ist, dass $\beta+\gamma
+\theta = 90^{\circ}$ ist, ergibt sich mit der Beziehung
$\sin(x)=\cos(\frac{\pi}{2} -x)$ letztendlich aus
Gleichung~\eqref{eq:detj} der folgende Term für $J$. Er stellt den Betrag des
Kreuzprodukts der beiden
Vektoren dar,
die das Kontrollvolumen aufspannen.
Es handelt sich somit um eine Approximation des Flächeninhalts
des Kontrollvolumens.
\begin{equation}
  det(J) = a b \sin \theta
\end{equation}

\paragraph{Diskretisierung über Kontrollvolumengrenzen}

\noindent
Bei der Approximation der diffusiven und konvektiven Flüsse werden meist
auch die Grenzen eines Kontrollvolumens überschritten. Damit kommt auch
die Geometrie des Nachbarkontrollvolumens zum Tragen und die Diskretisierung der
Metriken muss angepasst werden.
\begin{figure}[ht]
  \begin{tikzpicture}[scale=0.8]
  \draw[->, thick] (-1.5,0) -- (-1,0) node[right] {$x$} coordinate(x axis);
  \draw[->, thick] (-1.5,0) -- (-1.5,0.5) node[above] {$y$} coordinate(y axis);

  \fill[tud0a] (0,1) -- (2,0.5) -- (2.5,2) -- (0.5, 3) --cycle;
  \draw[thick] (0,1) -- (2,0.5) -- (2.5,2) -- (0.5, 3) --cycle;
  \draw[thick] (4,-0.5) -- (2,0.5) -- (2.5,2) -- (5, 2.5) --cycle;

  %\fill (2,0.5,0) circle[radius=1.5pt];
  %\node (x) at (2,0.5) [label=below left:$se$] {};
  %\fill (2.5,2) circle[radius=1.5pt];
  %\node (x) at (2.5,2) [label=below right:$ne$] {};
  \fill (1.25,1.625) circle[radius=1.5pt];
  \node (x) at (1.25,1.625) [label=above:$P$] {};
  %\node (x) at (1.25,1.625) [label=below:{$(\xi_{i,j},\ \eta_{i,j})$}] {};
  \fill (3.25,1.125) circle[radius=1.5pt];
  \node (x) at (3.25,1.125) [label=above:$E$] {};
  %\node (x) at (3.25,1.125) [label=below:{$(\xi_{i+1,j},\ \eta_{i+1,j})$}] {};
  \fill (2.25,1.25) circle[radius=1.5pt];
  \node (x) at (2.25,1.25) [label=above left:$e$] {};

  %\draw[thick,->] (-0.75,2.125) -- (5.25,0.625);
  %\node (x) at (5.25,0.625) [label=right:$\xi$] {};

  %\draw[thick,->] (1.75,-0.25) -- (2.75,2.75);
  %\node (x) at (2.75,2.75) [label=right:$\eta$] {};
\end{tikzpicture}

\centering
\caption{Berechnung der Metriken über Grenzen von Kontrollvolumen}
\end{figure}

Die Ableitungen nach $\xi$ werden nach folgendem Schema approximiert, wobei
$\vert \mathbf{x}_E -\mathbf{x}_P \vert$ den Abstand der Punkte $E$ und $P$ beschreibt.
\begin{equation}
  \pder[x]{\xi} \approx \frac{x_E-x_P}{\vert \mathbf{x}_E -\mathbf{x}_P \vert} \qquad
  \text{und} \qquad \pder[y]{\xi} \approx \frac{y_E-y_P}{\vert \mathbf{x}_E -\mathbf{x}_P \vert}
\end{equation}
Die Ableitungen nach $\eta$ ergeben sich mit der Seitenlänge $\delta S_e$ der Ostseite
zu:
\begin{equation}
  \pder[x]{\eta} \approx \frac{x_{ne}-x_{se}}{\delta S_e} \qquad
  \text{und} \qquad \pder[y]{\eta} \approx \frac{y_{ne}-y_{se}}{\delta S_e}
\end{equation}
