\section{Manufactured Solution}



Wir wählen die folgende Lösung für $\phi$ und berechnen die Ableitungen.

\begin{align*}
  \phi(x) &= A + \sin(Bx)\\
  \phi'(x) &= B \cos(Bx)\\
  \phi''(x) &= -B^2\sin(Bx)
\end{align*}

Nun führen wir den zusätzlichen Quellterm $f_{ad}$ ein und setzen $\phi(x)$ ein.

\begin{equation*}
  \alpha\frac{\partial^2 \phi}{\partial x^2}-f=f_{ad}
\end{equation*}

so erhält man:

\begin{equation*}
  f_{ad} = -\alpha B^2\sin(BX)-f
\end{equation*}

Damit ergibt sich als Manufactured Solution:
\begin{align}
  \alpha\frac{\partial^2 \phi}{\partial x^2} &= f -\alpha B^2\sin(Bx)-f\\
                                             &= -\alpha B^2\sin(Bx)
\end{align}

\subsection{Randbedingungen}
\label{sec:Randbedingungen}

Um die Konstanten $A$ und $B$ in der Manufactured Solution bestimmen zu können,
müssen konkrete Randbedingungen für das Problem festgelegt werden. Beispielhaft
soll das hier mit den Randwerten $\phi(0) = 1$ und $\phi(1) = 1$ geschehen.
Damit ergeben sich die Konstanten zu $A = 1$ sowie $B = \pi$. Die 
konstruierte Lösung $\phi(x)$ hat damit die folgende Form und der Quellterm $f$
ergibt sich zu:
\begin{equation*}
  \phi(x) = 1 + \sin(\pi x)
\end{equation*}
\begin{equation}
  f=-\pi^2 \sin(\pi x)
  \label{eq:quellterm}
\end{equation}

\subsection{Konvergenzuntersuchung}
\label{sec:Konvergenzuntersuchung}

Der Fehler $E$ wird definiert als Differenz zwischen diskreter Lösung $f(\Delta)$ sowie
der exakten Lösung $f_{exakt}$.
\begin{equation*}
  E=f(\Delta) - f_{exakt}
\end{equation*}

Für finite Methoden der Ordnung $p$ sollte sich der Fehler $DE$ proportional zu
$h^p$ verhalten. Daraus folgt mit dem Proportionalitätskoeffizienten $C$:

\begin{equation*}
  DE=f(\Delta) - f_{exakt}=C h^p + HOT
\end{equation*}

Wird nun die Gitterweite $h$ systematisch verkleinert, so lässst sich der beobachtete
Genauigkeitsgrad $p$ wie folgt berechnen.

\begin{equation}
  p=\frac{\ln \left(\frac{DE_2}{DE_1}\right)}{\ln \left(\frac{h_2}{h_1}\right)}
\end{equation}

