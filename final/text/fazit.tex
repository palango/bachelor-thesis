In der vorliegenden Arbeit wurde der Abbruchfehler zur Fehlerabschätzung
in Finite-Volumen Berechnungen und als Indikator zur
adaptiven Gitterverfeinerung untersucht.

Im ersten Teil der Arbeit wurden die theoretischen Grundlagen zusammengefasst
und für das Verständnis der Arbeit wichtige Begriffe eingeführt.
Anschließend wurde der Abbruchfehler für skalare Transportgleichungen abgeleitet.
Dabei musste er für Quellterme, Diffusionsterm und Konvektionsterme hergeleitet werden.
Die Herleitung umfasste dabei ein- und zweidimensionale Problemgebiete sowie kartesische und orthogonale
numerische Gitter. Für die Herleitung des Abbruchfehlers auf nicht-orthogonalen Gittern
wurde die Transformation von physikalischen ins logische Gebiet durchgeführt. Für die
dabei auftretenden Metriken wurden Diskretisierungen vorgeschlagen.


Zur Implementierung des Abbruchfehlerindikators ist die Lösung der Transportgleichung
auf dem Problemgebiet nötig. Hierfür wurden Lösungsprogramme für ein- und zweidimensionale
Problemgebiete in Matlab implementiert. Die Gitter können dabei kartesische und orthogonale
Form annehmen. Das Lösungsprogramm für nicht-orthogonale Gitter konnte aus Zeitgründen nicht vollendet werden.

Die Lösungsprogramme wurden anschließend nach den vorher gezeigten Methoden
verifiziert und ihre Korrektheit bewiesen.
Anschließend wurde der Abbruchfehlerindikator implementiert und mit anderen Fehlerindikatoren verglichen.
Hier stellte sich eine sehr gute Übereinstimmung heraus.

Schlussendlich
wurde der Abbruchfehler als Indikator zur adaptiven Gitterverfeinerung mit dem r-Verfahren genutzt.
Er wurde dabei mit anderen Indikatoren verglichen. Bei diesen handelte es sich um den
Lösungsgradienten, den Zweigitter-Fehlerschätzer sowie den Richardson-Extrapolations-Indikator.
Der Abbruchfehlerindikator konnte hier das beste Ergebnis liefern und so seine Eignung
als Indikator zur Gitteradaption zeigen.

Der in der vorliegenden Arbeit abgeleite Abbruchfehler verwendet zur Diskretisierung von Ableitungen
Finite Differenzen. Dies führt gerade am Rand des Problemgebietes zu ungenauen Vorhersagen, was
auch im Vergleich mit dem Residuum bemerkt wurde.
Alternativ kann hier die Verwendung von Interpolationspolynomen 
zur Ableitungsdiskretisierung geprüft werden, um die Ergebnisse am Rand zu verbessern.
Desweiteren steht eine Implementation des Abbruchfehlers für nicht-orthogonale Gitter
noch aus. Dies ist gerade im Hinblick auf die Nutzung gängiger Programmpakete zur Gitteradaption
sinnvoll.
