In der vorliegenden Arbeit wurde der Abbruchfehler als Indikator zur
Gitteradaption untersucht. Dazu wurde er für skalare Tranbsportgleichungen 
im Ein- und Zweidimensionalen hergeleitet. Dies geschah für kartesische,
orthogonale und nicht-orthogonale Gitter.
Anschließend wurden die Ergebnisse verifiziert. Die dafür nötigen Lösungsprogramme
wurden implementiert und nach gängigen Methoden auf ihre Korrektheit überprüft.
Anschließend wurde der Abbruchfehler berechnet und mit anderen fehlerindikatoren verglichen.
Hier stellte sich eine gute Übereinstimmung heraus. Schlussendlich
wurde der Abbruchfehler als indikator dur adaptiven Gitterverfeinerung mit dem r-Verfahren genutzt.
Auch hier konnten Vorteile gegenüber anderen gängigen Indikatoren nachgewiesen werden.

Der abgeleite Abbruchfehler verwendet zur Diskretisierung von Ableitungen
finite Differenzen. Dies führt gerade am Rand des problemgebietes zu ungenauen Vorhersagen.
Alternativ kann hier die Verwendung von Interpolationspolynomen 
zur Ableitungsdiskretisierung geprüft werden.
Desweiteren steht eine Implementation des Abbruchfehlers für nicht-orthogonale Gitter
noch aus. Dies ist gerade im Hinblick auf die Nutzung gängiger Programmpakete zur Gitteradaption
sinnvoll.
