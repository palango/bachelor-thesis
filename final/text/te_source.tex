\newcommand{\pder}[2][]{\frac{\partial#1}{\partial#2}}
\newcommand{\pderf}[1]{\frac{\partial f}{\partial#1}}
\newcommand{\pderfs}[1]{\frac{\partial^2 f}{\partial#1}}


\section{Abbruchfehler für Quellterme}
\label{sec:Quellterm}

\subsection{Eindimensionale Probleme}

Die Taylorreihenentwicklung einer Funktion $f(x)$ um den Punkt $x_0$
liefert:
\begin{equation*}
  f(x) = f(x_0) + f'(x_0)(x-x_0) + \frac{1}{2} f''(x_0)(x-x_0)^2 + HOT
\end{equation*}
Nach Integration im Intervall $[a, b]$ ergibt sich:
\begin{align*}
  \int_a^b f(x) dx &= \int_a^b f(x_0) dx + \int_a^b f'(x_0)(x-x_0) dx
+ \int_a^b \frac{1}{2} f''(x_0)(x-x_0)^2 dx + HOT\\
&= f(x_0) (b-a) + f'(x_0) \frac{(b-x_0)^2-(a-x_0)^2}{2}
+ \frac{1}{2} f''(x_0) \frac{(b-x_0)^3-(a-x_0)^3}{3} +HOT
\end{align*}
Die Anwendung auf ein Kontrollvolumen mit den Randpunkten $x_e$ und
$x_w$ sowie dem Mittelpunkt $x_P$ ergibt sich damit die folgende Formel:
\begin{equation*}
  \int_{x_w}^{x_e} f(x)dx = f_P(x_e-x_w)
  + \frac{1}{2} f'_P \underbrace{\left[{(x_e-x_P)^2-(x_w-x_P)^2}\right]}_{=0}
+ \frac{1}{6} f''_P \left[{{(x_e-x_P)}^3-{(x_w-x_P)}^3}\right] + HOT
\end{equation*}
Der sich der ersten Ableitung von $f$ anschließende Term wird immer Null, da nach
Definition der Finiten-Volumen Methode der Mittelpunkt eines Kontrollvolumens immer
zwischen seinen Grenzen liegt.
Die verbleibende, auftretenden Ableitung muss nun durch einen Differenzenquotienten
beschrieben werden.
\begin{equation}
  \label{eq:diskretisierung_f''P}
  f''_P = \frac{1}{x_e-x_w}\left(\frac{f_E-f_P}{x_E-x_P}-\frac{f_P-f_W}{x_P-x_W}\right)
\end{equation}
Der Truncation Error der Diskretisierung des Quelltermes lässt sich damit über die
folgende Gleichung beschreiben.
\begin{equation}
  TE_{source,1D} =
\frac{1}{6(x_e-x_w)}\left(\frac{f_E-f_P}{x_E-x_P}-\frac{f_P-f_W}{x_P-x_W}\right)
  \left[{{(x_e-x_P)}^3-{(x_w-x_P)}^3}\right]
\end{equation}

\subsubsection{Äquidistante Gitter}

Für äquidistante Gitter vereinfacht sich der Term und es ergibt sich:

\begin{equation*}
  TE_{source, 1D} = \frac{\Delta x}{24} \left({f_E-2f_P+f_W}\right)
\end{equation*}

\subsubsection{Randkontrollvolumen}

\paragraph{Westrand}

%Mit der Interpolation:

%\begin{equation}
  %\phi_e = \phi_P \frac{x_E-x_e}{x_E-x_P} + \phi_E \frac{x_e-x_P}{x_E-x_P}
%\end{equation}

\begin{align*}
  f'_P &=\frac{f_e-f_w}{x_e-x_w}\\
  f''_P &= \frac{1}{x_e-x_w}\left(\frac{f_E-f_P}{x_E-x_P}-\frac{f_P-f_w}{x_P-x_w}\right)
\end{align*}

\begin{align}
  TE_{source, W-Rand} &=
 \frac{f_e-f_w}{2(x_e-x_w)}  \left[{(x_e-x_P)^2-(x_w-x_P)^2}\right]\nonumber\\
&+\frac{1}{6(x_e-x_w)}\left(\frac{f_E-f_P}{x_E-x_P}-\frac{f_P-f_w}{x_P-x_w}\right)
  \left[{{(x_e-x_P)}^3-{(x_w-x_P)}^3}\right] + HOT
\end{align}

\begin{equation*}
  TE_{source, W-Rand, äquidistant} = \frac{\Delta x}{24} \left({f_E-3f_P+2f_w}\right)
\end{equation*}


\paragraph{Ostrand}

\begin{align*}
  f'_P &=\frac{f_e-f_w}{x_e-x_w}\\
  f''_P &= \frac{1}{x_e-x_w}\left(\frac{f_e-f_P}{x_e-x_P}-\frac{f_P-f_W}{x_P-x_W}\right)
\end{align*}

\begin{align}
  TE_{source, E-Rand} &=
 \frac{f_e-f_w}{2(x_e-x_w)}  \left[{(x_e-x_P)^2-(x_w-x_P)^2}\right]\nonumber\\
&+\frac{1}{6(x_e-x_w)}\left(\frac{f_e-f_P}{x_e-x_P}-\frac{f_P-f_W}{x_P-x_W}\right)
  \left[{{(x_e-x_P)}^3-{(x_w-x_P)}^3}\right] + HOT
\end{align}

\begin{equation*}
  TE_{source, E-Rand, äquidistant} = \frac{\Delta x}{24} \left({2f_e-3f_P+f_W}\right)
\end{equation*}








\subsection{Zweidimensionale Probleme}

Bei gleichem Vorgehen wie in Abschnitt~\ref{sec:Quellterm} beschrieben, ergibt sich für
eine zweidimensionale Funktion $f(x, y)$ der folgende Ausdruck:
\begin{align*}
  \int_{y_s}^{y_n}\int_{x_w}^{x_e} f dx dy &= f_P (x_e-x_w)(y_n-y_s)\\
                                           &+ \pderf{x} \underbrace{\frac{(x_e-x_P)^2 - (x_w-x_P)^2}{2}}_{=0} (y_n-y_s)
  + \pderf{y} \underbrace{\frac{(y_n-y_P)^2-(y_s-y_P)^2}{2}}_{=0} (x_e-x_w) \\
  &+ \frac{1}{2} \pderfs{x^2}\frac{(x_e-x_P)^3 - (x_w-x_P)^3}{3} (y_n-y_s)
  + \frac{1}{2} \pderfs{y^2} \frac{(y_n-y_P)^3-(y_s-y_P)^3}{3} (x_e-x_w) \\
  &+ \pderfs{x\partial y} \underbrace{\frac{(x_e-x_P)^2 - (x_w-x_P)^2}{2}}_{=0} \cdot
  \underbrace{\frac{(y_n-y_P)^2-(y_s-y_P)^2}{2}}_{=0} + HOT
\end{align*}
Die sich den ersten Ableitungen anschließenden Terme ergeben bei Auswertung wiederum Null
und müssen deshalb nicht weiter beachtet werden.
Der Wert von $f$ ist im Mittelpunkt des Kontrollvolumens bekannt, nicht aber die auftretenden
Ableitungen von $f$. Diese müssen deshalb diskretisiert werden. Man kann hierbei auf
Gleichung~\eqref{eq:diskretisierung_f''P} zurückgreifen und ebenfalls für die $y$-Richtung anpassen.
Der Fehler ergibt sich damit zu:

\begin{align}
  \begin{split}
  TE_{source, 2D} &=                \frac{1}{2}
  \left[{\frac{f_E-f_P}{(x_E-x_P)(x_e-x_w)}-\frac{f_P-f_W}{(x_P-x_W)(x_e-x_w)}  }\right]
  \frac{(x_e-x_P)^3 - (x_w-x_P)^3}{3} (y_n-y_s)\\
  &+ \frac{1}{2}
  \left[{\frac{f_N-f_P}{(y_N-y_P)(y_n-y_s)}-\frac{f_P-f_S}{(y_P-y_S)(y_n-y_s)}  }\right]
  \frac{(y_n-y_P)^3-(y_s-y_P)^3}{3} (x_e-x_w) \\
  \end{split}
\end{align}

\subsubsection{Äquidistante Gitter}

Für äquidistante Gitter vereinfacht sich der Term zu:
\begin{equation}
  TE_{source,2D} = \frac{1}{24} \Delta x \Delta y \left[{f_E+f_W+f_N+f_S - 4f_P}\right]
\end{equation}
