\section{Numerische Methoden}

Zum Erkenntnisgewinn in den Ingenieurwissenschaften stehen drei grundsätzlichen
Wege zur Verfügung \cite{num_maschbau}:
\begin{itemize}
  \item Theoretische Methoden
  \item Experimentelle Versuche
  \item Numerische Simulationen
\end{itemize}

Theoretische Methoden, also insbesondere analytische Betrachtungen der das Problem
beschreibenden Gleichungen sind im allgemeinen nur sehr beschränkt möglich. Das heißt das
die Gleichungen, die zur Beschreibung realer Prozesse genutzt werden meist nur für
bestimmte Randbedingungen sowie gezielten Vereinfachungen überhaupt analytisch lösbar
sind. Da jedoch diese Vereinfachungen in realen Prozessen nicht vernachlässigbar klein sind
sowie die geforderten Randbedingungen nicht erfüllbar oder präzise einhaltbar sind, ist es
für komplexere Problemstellungen unmöglich analytische Lösungen zu finden.

Die Intention bei experimentellen Untersuchungen ist über Versuche an Modellen oder
realen Bauteilen an die benötigten Systemgrößen heranzukommen. Diese Vorgehensweise
bereitet jedoch in vielen Fällen Probleme:
\begin{itemize}
  \item Messungen bestimmter Systemgrößen sind oft schwierig bis unmöglich. Gründe
    dafür können die Dimensionen der Objekte (z.B. Molekulare Prozesse, Weltmeere),
    die Geschwindigkeit der Zustandsänderung (z.B. Explosionen) oder moralischen Gründen
    (z.B. Versuche an Mensch und Tieren, Versuche mit Gefahrenstoffen).
  \item Experimente an Modellen lassen nur begrenzte Rückschlüsse auf das reale Objekt
    zu. So lassen sich beispielsweise Erkenntnisse aus dem Windkanal nur teilweise auf
    das reale Automobil oder Flugzeug übernehmen.
  \item Experimente sind oft teuer und zeitraubend. So muss beispielsweise um die Auswirkungen
    einer Änderung zu testen ein neues Modell bzw. Objekt hergestellt werden. In anderen Fällen
    ist das Modell nach dem Versuch zerstört (z.B. Crashtest). Generell ist auch der Betrieb
    von Messeinrichtungen teuer.
\end{itemize}

Aufgrund der Nachteile der oben genannten Verfahren setzt sich im Maschinenbau und der
Naturwissenschaft im allgemeinen die numerische Simulation immer weiter durch. Vorteile
gegenüber den anderen Verfahren liegen auf der Hand:

\begin{itemize}
  \item Simulationen sind meist schneller und kosteneffizienter zu erstellen als experimentelle
    Versuche.
  \item Änderungen am Objekt sowie Parameterstudien zur Optimierung lassen sich leicht
    erstellen.
  \item Die Ergebnisse der Simulation enthalten meist alle problemrelevanten Größen,
    deren Messung im Versuch mit viel Aufwand verbunden wäre.
\end{itemize}

Grundlegend für die Nutzung dieser Vorteile ist jedoch die Zuverlässigkeit der
berechneten Ergebnisse. Deshalb gehen mit der Verbesserung der numerischen Simulationsmethoden
immer auch experimentelle Validierung von Ergebnissen und verwendeten Modellen einher.
