\section{Abbruchfehler von Diffusionstermen}
\label{sec:Diffusionsterme}

\subsection{Eindimensionale Probleme}

Um den Abbruchfehler von Diffusionstermen zu bestimmen werden die Differenzenquotienten
für die Ableitungen erster Ordnung hergeleitet. Dazu werden die Taylorreihendarstellungen der bekannten
Werte in der unmittelbaren Umgebung genutzt. Beispielhaft soll das anhand des östlichen
Randes des Kontrollvolumens geschehen.

Zuerst entwickeln wir die Taylordarstellungen vom Punkt $x_e$ aus in Richtung der anliegenden
Kontrollvolumenmittelpunkte $\phi_E$ und $\phi_P$.
\begin{align}
  \phi_E &= \phi_e + \phi'_e(x_E-x_e)+\frac{1}{2}\phi''_e(x_E-x_e)^2
  +\frac{1}{6}\phi'''_e(x_E-x_e)^3+HOT
  \label{eq:taylor_eE}\\
  \phi_P &= \phi_e + \phi'_e(x_P-x_e)+\frac{1}{2}\phi''_e(x_P-x_e)^2
  +\frac{1}{6}\phi'''_e(x_P-x_e)^3+HOT
  \label{eq:taylor_eP}
\end{align}
Wird nun Gleichung~\eqref{eq:taylor_eP} von Gleichung~\eqref{eq:taylor_eE} subtrahiert. Es ergibt sich:
\begin{equation*}
  \phi_E-\phi_P=\phi'_e(x_E-x_P)+
  \frac{1}{2}\phi''_e\left[{{(x_E-x_e)}^2-{(x_P-x_e)}^2}\right]+
  \frac{1}{6}\phi'''_e\left[{{(x_E-x_e)}^3-{(x_P-x_e)}^3}\right]+HOT
\end{equation*}
Nach Umstellen ergibt sich daraus für die Ableitung $\phi'_e$ der folgende Term, aus dem bereits
der Abbruchfehler abgelesen werden kann.
\begin{equation}
  \phi'_e = \frac{\phi_E-\phi_P}{x_E-x_P}+\frac{1}{2}\phi''_e
\left({\frac{{(x_P-x_e)}^2-{(x_E-x_e)}^2}{x_E-x_P}}\right)+
\frac{1}{6} \phi'''_e \left({\frac{{(x_P-x_e)}^3-{(x_E-x_e)}^3}{(x_E-x_P)}}\right)+HOT \label{eq:te_dif_e}
\end{equation}
Die hier auftretenden Ableitungen $\phi''_e$ und $\phi'''_e$ sind nicht bekannt und
müssen diskretisiert werden. Bei der Nutzung von
möglichst lokalen Differenzenquotienten ergeben sich die folgenden Approximationen der Ableitungen.
Die genutzten Punkte sind in Abbildung~\ref{fig:kv1d} ersichtlich.
\begin{align*}
  \phi''_e &= \frac{1}{(x_E-x_P)}\left({
\frac{\phi_{EE}-\phi_P}{x_{EE}-x_P}-\frac{\phi_E-\phi_W}{x_E-x_W}}\right)\\
\phi'''_e &= \frac{1}{(x_E-x_P)}\left({
\frac{1}{(x_{ee}-x_e)}
\left({\frac{\phi_{EE}-\phi_E}{x_{EE}-x_E}-\frac{\phi_E-\phi_P}{x_E-x_P} }\right)
-\frac{1}{(x_e-x_w)}
\left({\frac{\phi_E-\phi_P}{x_E-x_P} - \frac{\phi_P-\phi_W}{x_P-x_W}  }\right)
}\right)
\end{align*}
Der Abbruchfehler des Diffusionsterms an der östlichen Grenze eines Kontrollvolumens
lässt sich damit über folgende Gleichung beschreiben.
\begin{align}
\begin{split}
    \begin{IEEEeqnarraybox}[][c]{rCl}
      {TE}_e &=& \frac{1}{2 (x_E-x_P)}\left({
\frac{\phi_{EE}-\phi_P}{x_{EE}-x_P}-\frac{\phi_E-\phi_W}{x_E-x_W}}\right) \left({\frac{{(x_P-x_e)}^2-{(x_E-x_e)}^2}{x_E-x_P}}\right)\\
&&+
\left({
\frac{1}{(x_{ee}-x_e)}
\left({\frac{\phi_{EE}-\phi_E}{x_{EE}-x_E}-\frac{\phi_E-\phi_P}{x_E-x_P} }\right)
-\frac{1}{(x_e-x_w)}
\left({\frac{\phi_E-\phi_P}{x_E-x_P} - \frac{\phi_P-\phi_W}{x_P-x_W}  }\right)
}\right)\\
&&\frac{1}{6(x_E-x_P)}\left({\frac{{(x_P-x_e)}^3-{(x_E-x_e)}^3}{(x_E-x_P)}}\right)
+HOT
    \end{IEEEeqnarraybox}
\end{split}
\end{align}
Der Abbruchfehler für Diffusionsterme im Westen erfolgt äquivalent zur oben
gezeigten Herleitung im Osten. Es ergibt sich:
\begin{align}
  \begin{split}
    \begin{IEEEeqnarraybox}[][c]{rCl}
      TE_w &=& \frac{1}{2 (x_P-x_W)} \left({
\frac{\phi_{E}-\phi_W}{x_{E}-x_W}-\frac{\phi_P-\phi_{WW}}{x_P-x_{WW}}}\right)
  \left({\frac{{(x_W-x_w)}^2-{(x_P-x_w)}^2}{x_P-x_W}}\right)\\
  &&+
\left({
\frac{1}{(x_e-x_w)}
\left({\frac{\phi_E-\phi_P}{x_E-x_P}-\frac{\phi_P-\phi_W}{x_P-x_W} }\right)
-\frac{1}{(x_w-x_{ww})}
\left({\frac{\phi_P-\phi_W}{x_P-x_W} - \frac{\phi_W-\phi_{WW}}{x_W-x_{WW}}  }\right)
}\right)\\
&&\frac{1}{6(x_P-x_W)}\left({\frac{{(x_W-x_w)}^3-{(x_P-x_w)}^3}{(x_P-x_W)}}\right)
  +HOT
    \end{IEEEeqnarraybox}
\end{split}
\end{align}

%\subsection{Äquidistante Gitter}

%Für äquidistante Gitter vereinfache sich die Terme des Truncation Error. So löschen
%sich beispielsweise die quadratischen Terme gegenseitig aus. Es ergeben sich die
%folgenden Fehler:

%\begin{align*}
  %{TE}_e &= \left({
%\frac{1}{6\Delta x^2}
%\left({\frac{\phi_{EE}-\phi_E}{\Delta x}-\frac{\phi_E-\phi_P}{\Delta x} }\right)
%-\frac{1}{6\Delta x^2}
%\left({\frac{\phi_E-\phi_P}{\Delta x} - \frac{\phi_P-\phi_W}{\Delta x}  }\right)
%}\right)\left({-\frac{\Delta x^2}{4} }\right)+HOT\\
%&= -\frac{1}{24\Delta x}\left({
%\phi_{EE}-3\phi_E+3\phi_P-\phi_W}\right)+HOT
%\end{align*}

%\begin{align*}
  %TE_w &=\left({
%\frac{1}{6 \Delta x^2}
%\left({\frac{\phi_E-\phi_P}{\Delta x}-\frac{\phi_P-\phi_W}{\Delta x} }\right)
%-\frac{1}{6\Delta x^2}
%\left({\frac{\phi_P-\phi_W}{\Delta x} - \frac{\phi_W-\phi_{WW}}{\Delta x}  }\right)
%}\right)
%\left({-\frac{\Delta x^2}{4} }\right)+HOT\\
%&= -\frac{1}{24 \Delta x}\left({
%\phi_E-3\phi_P+3\phi_W-\phi_{WW}}\right)+HOT
%\end{align*}


\subsubsection{Kontrollvolumen am Rand}
\label{sec:te_dif_rand}

Kontrollvolumen, die in der Nähe des Randes des Problemgebietes liegen müssen wie schon beim
Abbruchfehler von Quelltermen besonders behandelt werden. Hierbei muss unterschieden werden ob es sich
bei der zu approximierenden Ableitung um die Ableitung genau am Rand oder weiter im Inneren des
Problemgebiets handelt.

Muss die Ableitung genau am Rand approximiert werden, so muss der in Gleichung~\ref{eq:te_dif_e} hergeleitete
Zentraldifferenzenquotient durch einen einseitigen Differenzenquotient ersetzt werden. Je nachdem ob das
betrachtete Volumen am Ost- oder Westrand liegt, müssen Vorwärts- oder Rückwärtsdifferenzenquotienten genutzt werden.
\begin{figure}[hb]
\begin{tikzpicture}[scale=1.1]
  %\draw[->] (6,0) -- (6.5,0) node[right] {$x$} coordinate(x axis);
  %\draw[->] (0,6) -- (0,6.5) node[above] {$y$} coordinate(y axis);
  \fill[tud0a] (2,2) rectangle +(2,2);
  \draw[step=2cm, thick] (2, 2) grid (8, 4);

  \filldraw[pattern=north east lines, thick] (1,2) rectangle +(1,2);

  \fill (3, 3)  circle[radius=1.5pt];
  \node (N) at (3,3) [label=above:$P$]{};

  \fill (5, 3)  circle[radius=1.5pt];
  \node (N) at (5,3) [label=right:$E$]{};


  \fill (7, 3)  circle[radius=1.5pt];
  \node (N) at (7,3) [label=right:$EE$]{};


  \fill (4, 3)  circle[radius=1.5pt];
  \node (N) at (4,3) [label={right=0pt}:$e$]{};

  \fill (2, 3)  circle[radius=1.5pt];
  \node (N) at (2,3) [label={right=0pt}:$w$]{};

  \fill (6, 3)  circle[radius=1.5pt];
  \node (N) at (6,3) [label=right:$ee$]{};

\end{tikzpicture}

\centering
\caption{Randkontrollvolumen im eindimensionalen Fall}
\label{fig:kv1d_rand}
\end{figure}
Das Vorgehen soll beispielhaft für die Ableitung $\phi'_w$ für das in Abbildung~\ref{fig:kv1d_rand} markierte Kontrollvolumen
vorgestellt werden.
Über die Randbedingung ist der Wert $\phi^w$ bekannt. Nun wird $\phi_P$ als Taylorreihe mit Entwicklungspunkt
$x_w$ dargestellt.
\begin{equation*}
  \phi_P = \phi^w + \phi'_w (x_P-x_w) + \frac{1}{2} \phi''_w (x_P-x_w)^2 + HOT
\end{equation*}
Nach Umstellen ergibt sich für $\phi'_w$ der folgende Wert:
\begin{equation}
  \phi'_w = \frac{\phi_P-\phi^w}{x_p-x_w} -\frac{1}{2} \phi''_w(x_p-x_w) + HOT
\end{equation}
Hier lässt sich ebenfalls der Abbruchfehler ablesen. Mit der Approximation der Ableitung
ergibt sich:
\begin{equation}
  TE_{diff, konv, Rand} = -\frac{1}{2} \left({\frac{\phi_e-\phi^w}{x_e-x_w}-
  \frac{\phi_P-\phi^w}{x_p-x_w} }\right)
\end{equation}
Wird die Ableitung nicht direkt auf dem Rand betrachtet, kann der Zentraldifferenzenquotient
aus Gleichung~\ref{eq:te_dif_e} genutzt werden. Hier müssen jedoch ebenfalls die verwendeten
Approximation der höheren Ableitungen angepasst werden, um nicht vorhandene Werte nicht zu verwenden.
Soll beispielsweise die Ableitung $\phi'_e$ für das in Abbildung~\ref{fig:kv1d_rand} markierte Kontrollvolumen
berechnet werden, so müssen die Ableitungen $\phi''_e$ und $\phi'''_e$ diskretisiert werden ohne den Wert
$\phi_W$ zu verwenden. Die sich ergebenden Approximationen lauten damit:
\begin{align*}
  \phi''_e &= \frac{1}{(x_E-x_P)}\left({
\frac{\phi_{EE}-\phi_P}{x_{EE}-x_P}-\frac{\phi_E-\phi^w}{x_E-x_w}}\right)\\
\phi'''_e &= \frac{1}{(x_E-x_P)}\left({
\frac{1}{(x_{ee}-x_e)}
\left({\frac{\phi_{EE}-\phi_E}{x_{EE}-x_E}-\frac{\phi_E-\phi_P}{x_E-x_P} }\right)
-\frac{1}{(x_e-x_w)}
\left({\frac{\phi_E-\phi_P}{x_E-x_P} - \frac{\phi_P-\phi^w}{x_P-x_w}  }\right)
}\right)
\end{align*}
Die oben vorgestellte Behandlung des westlichen Randes kann äquivalent auf die anderen Ränder übertragen werden.
Beim zweidimensionalen Fall ist zudem besonders auf die Ecken zu achten.


%\paragraph{Westlicher Rand}

%Es ergeben sich für gegebenes $\phi_w$:


%\begin{equation*}
  %\phi''_{w, W-Rand} = \frac{1}{(x_P-x_w)}\left({
%\frac{\phi_{e}-\phi_w}{x_{e}-x_w}-\frac{\phi_P-\phi_w}{x_P-x_w}}\right)
%\end{equation*}

%Da bei der dritten Ableitung zwei linke Kontrollvolumen genutzt werden, müssen hier
%vom Kontrollvolumen am Rand sowohl der westliche also auch der östliche Rand
%betrachtet werden.

%\begin{align*}
  %\phi'''_{w, W-Rand} &= \frac{1}{(x_p-x_w)} \left({
  %\frac{1}{(x_e-x_w)} \left({
    %\frac{\phi_E-\phi_P}{x_E-x_P} - \frac{\phi_P-\phi_w}{x_P-x_w}
    %}\right) -
  %\frac{1}{x_P-x_w} \left({
    %\frac{\phi_e-\phi_w}{x_e-x_w} - \frac{\phi_P-\phi_w}{x_P-x_w}
    %}\right)
  %}\right)
  %\\
  %\phi'''_{e, W-Rand} &= \frac{1}{(x_E-x_P)} \left({
  %\frac{1}{(x_{ee}-x_e)} \left({
      %\frac{\phi_{EE}-\phi_E}{x_{EE}-x_E} - \frac{\phi_E-\phi_P}{x_E-x_P}
    %}\right) -
  %\frac{1}{x_e-x_w} \left({
    %\frac{\phi_E-\phi_P}{x_E-x_P} - \frac{\phi_P-\phi_w}{x_P-x_w}
    %}\right)
  %}\right)
%\end{align*}

%Hier wird noch der Wert $\phi_e$ benutzt, der aber nicht bekannt ist. Er wird deshalb
%durch eine lineare Interpolation von $\phi_E$ und $\phi_P$ bestimmt.

%\begin{equation}
  %\phi_e = \phi_P \frac{x_E-x_e}{x_E-x_P} + \phi_E \frac{x_e-x_P}{x_E-x_P}
%\end{equation}

%Damit ergibt sich für $\phi''_{w,Rand}$und $\phi'''_{w, Rand}$:

%\begin{align}
  %\phi''_{w, W-Rand} &= \frac{1}{(x_P-x_w)}\left({
%\frac{
  %\left({\phi_P \frac{x_E-x_e}{x_E-x_P} + \phi_E \frac{x_e-x_P}{x_E-x_P}
%}\right)
%-\phi_w}{x_{e}-x_w}-\frac{\phi_P-\phi_w}{x_P-x_w}}\right)\\
  %\phi'''_{w, W-Rand} &= \frac{1}{(x_p-x_w)} \left({
  %\frac{1}{(x_e-x_w)} \left({
    %\frac{\phi_E-\phi_P}{x_E-x_P} - \frac{\phi_P-\phi_w}{x_P-x_w}
    %}\right) -
  %\frac{1}{x_P-x_w} \left({
    %\frac{ \phi_P \frac{x_E-x_e}{x_E-x_P} + \phi_E \frac{x_e-x_P}{x_E-x_P}
%-\phi_w}{x_e-x_w} - \frac{\phi_P-\phi_w}{x_P-x_w}
    %}\right)
  %}\right)
%\end{align}


%\paragraph{Östlicher Rand}
%Äquivalent ergibt sich $\phi''_{e, E-Rand}$ bei gegebenem $\phi_e$:

%\begin{align*}
  %\phi''_{e, E-Rand} &= \frac{1}{(x_e-x_P)}\left({
%\frac{\phi_{e}-\phi_P}{x_{e}-x_P}-\frac{\phi_e-\phi_w}{x_e-x_w}}\right)
%\\
  %\phi'''_{e, E-Rand} &= \frac{1}{(x_e-x_P)} \left({
  %\frac{1}{(x_e-x_P)} \left({
    %\frac{\phi_e-\phi_P}{x_e-x_P} - \frac{\phi_e-\phi_w}{x_e-x_w}
    %}\right) -
  %\frac{1}{x_e-x_w} \left({
    %\frac{\phi_e-\phi_P}{x_e-x_P} - \frac{\phi_P-\phi_W}{x_P-x_W}
    %}\right)
  %}\right)
  %\\
  %\phi'''_{w, E-Rand} &= \frac{1}{(x_P-x_W)} \left({
  %\frac{1}{(x_e-x_w)} \left({
      %\frac{\phi_e-\phi_P}{x_e-x_P} - \frac{\phi_P-\phi_W}{x_P-x_W}
    %}\right) -
    %\frac{1}{x_w-x_{ww}} \left({
        %\frac{\phi_P-\phi_W}{x_P-x_W} - \frac{\phi_W-\phi_{WW}}{x_W-x_{WW}}
    %}\right)
  %}\right)
%\end{align*}

%$\phi_w$ wird wie folgt linear interpoliert:

%\begin{equation*}
  %\phi_w = \phi_W \frac{x_P-x_w}{x_P-x_W} + \phi_P \frac{x_w-x_W}{x_P-x_W}
%\end{equation*}

%Damit ergibt sich für $\phi''_{e,Rand}$:

%\begin{align}
  %\phi''_{e,Rand} &= \frac{1}{(x_e-x_P)}\left({
%\frac{\phi_{e}-\phi_P}{x_{e}-x_P}-\frac{\phi_e-
  %\left({
   %\phi_W \frac{x_P-x_w}{x_P-x_W} + \phi_P \frac{x_w-x_W}{x_P-x_W}
  %}\right)
%}{x_e-x_w}}\right)
%\\
  %\phi'''_{e, E-Rand} &= \frac{1}{(x_e-x_P)} \left({
  %\frac{1}{(x_e-x_P)} \left({
    %\frac{\phi_e-\phi_P}{x_e-x_P} - \frac{\phi_e- \phi_W \frac{x_P-x_w}{x_P-x_W} + \phi_P \frac{x_w-x_W}{x_P-x_W}
%}{x_e-x_w}
    %}\right) -
  %\frac{1}{x_e-x_w} \left({
    %\frac{\phi_e-\phi_P}{x_e-x_P} - \frac{\phi_P-\phi_W}{x_P-x_W}
    %}\right)
  %}\right)
%\end{align}


%\subsubsection{Äquidistante Gitter}

%\paragraph{Westlicher Rand}

%\begin{align}
  %\phi''_{w, W-Rand} &= \frac{1}{\Delta x^2} (\phi_E-3\phi_P+2\phi_w)\\
  %\phi'''_{w, W-Rand} &= 0\\
  %\phi'''_{e, W-Rand} &= \frac{1}{\Delta x^3} (\phi_{EE} -3\phi_E + 4\phi_P -2\phi_w)
%\end{align}

%\paragraph{Östlicher Rand}

%\begin{align}
  %\phi''_{e, E-Rand} &= \frac{1}{\Delta x^2} (\phi_W-3\phi_P+2\phi_e)\\
  %\phi'''_{e, E-Rand} &= 0\\
  %\phi'''_{w, E-Rand} &= \frac{1}{\Delta x^3} (-\phi_{WW} +3\phi_W - 4\phi_P +2\phi_e)
%\end{align}

\subsection{Zweidimensionale Probleme}

Äquivalent zu den Herleitungen des Abbruchfehlersim Osten und Westen ergeben sich
sich für die Ableitungen $\phi'_n$ und $\phi'_s$ die folgenden Reihenentwicklungen:

\begin{equation}
  \phi'_n = \frac{\phi_N-\phi_P}{y_N-y_P}+\frac{1}{2}\phi''_n
\left({\frac{{(y_P-y_n)}^2-{(y_N-y_n)}^2}{y_N-y_P}}\right)+
\frac{1}{6} \phi'''_n \left({\frac{{(y_P-y_n)}^3-{(y_N-y_n)}^3}{y_N-y_P}}\right)+HOT
\end{equation}


\begin{equation}
  \phi'_s = \frac{\phi_P-\phi_S}{y_P-y_S}+\frac{1}{2}\phi''_s
\left({\frac{{(y_S-y_s)}^2-{(y_P-y_s)}^2}{y_P-y_S}}\right)+
\frac{1}{6} \phi'''_s \left({\frac{{(y_S-y_s)}^3-{(y_P-y_s)}^3}{y_P-y_S}}\right)+HOT
\end{equation}
Nach der Diskretisierung der auftretenden, unbekannten Ableitungen ergeben sich die folgenden
Terme für den Abbruchfehler:

\begin{equation}
  \begin{IEEEeqnarraybox}[][c]{rCl}
    {TE}_n &=& \frac{1}{2\,(y_N-y_P)}\left({
\frac{\phi_{NN}-\phi_P}{y_{NN}-y_P}-\frac{\phi_N-\phi_S}{y_N-y_S}}\right) \left({\frac{{(y_P-y_n)}^2-{(y_N-y_n)}^2}{y_N-y_P}}\right)\\
&&+
\left({
\frac{1}{(y_{nn}-y_n)}
\left({\frac{\phi_{NN}-\phi_N}{y_{NN}-y_N}-\frac{\phi_N-\phi_P}{y_N-y_P} }\right)
-\frac{1}{(y_n-y_s)}
\left({\frac{\phi_N-\phi_P}{y_N-y_P} - \frac{\phi_P-\phi_S}{y_P-y_S}  }\right)
}\right)\\
&&\frac{1}{6\,(y_N-y_P)}\left({\frac{{(y_P-y_n)}^3-{(y_N-y_n)}^3}{y_N-y_P}}\right)
+HOT
\end{IEEEeqnarraybox}
\end{equation}


\begin{equation}
\begin{IEEEeqnarraybox}[][c]{rCl}
  TE_s &=& \frac{1}{2\,(y_P-y_S)} \left({
\frac{\phi_{N}-\phi_S}{y_{N}-y_S}-\frac{\phi_P-\phi_{SS}}{y_P-y_{SS}}}\right)
  \left({\frac{{(y_S-y_s)}^2-{(y_P-y_s)}^2}{y_P-y_S}}\right)\\
  &&+
\left({
\frac{1}{(y_n-y_s)}
\left({\frac{\phi_N-\phi_P}{y_N-y_P}-\frac{\phi_P-\phi_S}{y_P-y_S} }\right)
-\frac{1}{(y_s-y_{ss})}
\left({\frac{\phi_P-\phi_S}{y_P-y_S} - \frac{\phi_S-\phi_{SS}}{y_S-y_{SS}}  }\right)
}\right)\\
&&\frac{1}{6\,(y_P-y_S)}\left({\frac{{(y_S-y_s)}^3-{(y_P-y_s)}^3}{y_P-y_S}}\right)
  +HOT
\end{IEEEeqnarraybox}
\end{equation}
Die Behandlung der Randwerte erfolgt ebenfalls wir in Abschnitt~\ref{sec:te_dif_rand} erläutert und wird hier
nicht wiederholt.

\subsection{Äquidistante Gitter}

\begin{align*}
  {TE}_n &= \left({
\frac{1}{6\Delta y^2}
\left({\frac{\phi_{NN}-\phi_N}{\Delta y}-\frac{\phi_N-\phi_P}{\Delta y} }\right)
-\frac{1}{6\Delta y^2}
\left({\frac{\phi_N-\phi_P}{\Delta y} - \frac{\phi_P-\phi_S}{\Delta y}  }\right)
}\right)\left({-\frac{\Delta y^2}{4} }\right)+HOT\\
&= -\frac{1}{24\Delta y}\left({
\phi_{NN}-3\phi_N+3\phi_P-\phi_S}\right)+HOT
\end{align*}

\begin{align*}
  TE_s &=\left({
\frac{1}{6 \Delta y^2}
\left({\frac{\phi_N-\phi_P}{\Delta y}-\frac{\phi_P-\phi_S}{\Delta y} }\right)
-\frac{1}{6\Delta y^2}
\left({\frac{\phi_P-\phi_S}{\Delta y} - \frac{\phi_S-\phi_{SS}}{\Delta y}  }\right)
}\right)
\left({-\frac{\Delta y^2}{4} }\right)+HOT\\
&= -\frac{1}{24 \Delta y}\left({
\phi_N-3\phi_P+3\phi_S-\phi_{SS}}\right)+HOT
\end{align*}
