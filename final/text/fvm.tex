\subsection{Finite Volumen Methode}

Die Finite-Volumen Methoden sind ein numerisches Verfahren zur Lösung von (partiellen)
Differential\-gleich\-ungen. Sie sind heute das Standardverfahren zu Lösung von Strömungsproblemen
wie zum Beispiel den Euler- oder Navier-Stokes-Gleichungen aber keineswegs nur darauf beschränkt.
Die charakteristische Eigenschaft der Finite-Volumen Methoden ist ihre Konservativität, also
die Erfüllung der den mathematischen Modellen zugrunde liegenden Erhaltungsprinzipien in den
diskreten Gleichungen.

\subsubsection{Ableitung der Finite Volumen diskretisierten Transportgleichung}

Ausgangspukt ist die in Abschnitt~\ref{sec:num_gitter} beschriebene Zerlegung des Problemgebiets
in diskrete Teilgebiete. Diese werden bei der Finite-Volumen Methode Kontrollvolumen
genannt (im zweidimensionalen Fall steht Volumen stellvertretend für Fläche,
im Eindimensionalen für Länge).
Für jedes dieser Kontrollvolumen werden die Erhaltungsgleichungen in Integralform formuliert
beziehungsweise durch Integration aus den entsprchenden Diffenrentialgleichungen gewonnen.

Beispielhaft soll das Vorgehen hier anhand der in Abschnitt~\ref{sec:transportgl}
vorgestellten stationären Transportgleichung im zweidimensionalen Fall
($i=1$, $2$) vorgestellt werden.
\begin{equation}
  \frac{\partial}{\partial x_i} \left({\rho v_i \phi
- \alpha \frac{\partial \phi}{\partial x_i} }\right) = f
\label{eq:transp_fvm}
\end{equation}
Durch Integration von Gleichung~\eqref{eq:transp_fvm} über ein Kontrollvolumen $V$ (Gl.~\ref{eq:fvm1})
erhält man bei anschließender Anwendung des Gaußschen Integralsatzes die in Gleichung~\eqref{eq:fvm2} beschriebene Beziehung.
\begin{align}
  \int_V \frac{\partial}{\partial x_i} \left({\rho v_i \phi
- \alpha \frac{\partial \phi}{\partial x_i} }\right) dV &= \int_V f dV \label{eq:fvm1}\\
  \int_S  \left({\rho v_i \phi
- \alpha \frac{\partial \phi}{\partial x_i} }\right) n_i dS&= \int_V f dV \label{eq:fvm2}
\end{align}
Hierbei beschreibt $S$ die Oberfläche des Kontrollvolumens $V$, $dS$ ein Oberflächenelement
und $n_i$ die Komponenten des Einheitsnormalenvektors auf der Oberfläche.

Geht man nun von beliebig geformten Kontrollvolumen zu viereckigen Kontrollvolumen über,
so lässt sich das Integral über die Oberfläche $S$ in Gleichung~\eqref{eq:fvm2} durch eine
Summation über die Integrale der vier Seitenflächen umformen. Hierbei beschreibt $c$ die
einzelnen Seitenflächen und ist bei viereckigen Kontrollvolumen als $c=(e,w,n,s)$ definiert.
\begin{equation}
  \sum_c \int_{S_c} \left(\rho v_i \phi - \alpha \frac{\partial \phi}{\partial x_i}
\right) n_{ci} dS_c = \int_V f dV\label{eq:fvm3}
\end{equation}
Gleichung~\eqref{eq:fvm3} stellt nun die Bilanzgleichung für die konvektiven Flüsse
$F_c^C$ sowie die diffusiven Flüsse $F_c^D$ durch die Seiten des Kontrollvolumens $V$ dar.
\begin{align}
  F_c^C &=  \int_{S_c} \left(\rho v_i \phi \right) n_{ci} dS_c \\
  F_c^D &=  -\int_{S_c} \left(\alpha \frac{\partial \phi}{\partial x_i}\right) n_{ci} dS_c 
\end{align}

\subsubsection{Integralapproximation}

In dieser Arbeit wird von einer zellenorientierten Variablenanordnung (s. \cite{num_maschbau})
ausgegangen. Damit liegt die gesuchte diskrete Größe im Mittelpunkt jedes Kontrollvolumens.
Die Integrale der konvektiven und diffusiven Flüsse müssen durch diese Variablen
approximiert werden. Dabei ist es zweckmäßig zuerst die Oberflächenintegrale durch Variablen
auf der jeweilen Seite des Kontrollvolumens zu beschrieben und anschließend diese durch die Werte
im Zentrum des Kontrollvolumens zu beschreiben.

Beispielhaft soll hier das Oberflächenintegral
\begin{equation*}
  \int_{S_w} p_i n_{wi}\,dS_w
\end{equation*}
der Westseite $S_w$ betrachtet werden. Hierbei bezeichnet $p_i$ die Komponenten einer allgemeinen
Funktion $\mathbf{p} = (p_1(\mathbf{x}), p_2(\mathbf{x}))$. Die einfachste
Approximationsmöglichkeit besteht darin, den Funktionswert von $\mathbf{p}$ im Mittelpunkt
der Seite zu nutzen.
\begin{equation}
  \int_{S_w} p_i n_{wi}\,dS_w \approx q_w \delta S_w
\end{equation}
Die Normalkomponenten von $\mathbf{p}$ an der Stelle $w$ werden hierbei mit $q_w = p_{wi} n_{wi}$ bezeichnet.
Aufgrund der Nutzung des Funktionswertes im Seitenmittelpunkt wird diese Approximation auch Mittelpunktsregel genannt.
Sie hat eine Ordnung von zwei bezüglich der Seitenlänge $\delta S_w$.

Weitere gängige Approximationen umfassen die Trapezregel oder die Simpsonregel, auf die jedoch
in der vorliegenden Arbeit nicht näher eingegangen wird.

Für die Flüsse $F_c^C$ und $F_c^D$ ergeben sich mit der Mittelpunktsregel folgende Approximationen:
\begin{align}
  F_c^C &\approx \underbrace{\rho v_i n_{ci} \delta S_c}_{\dot{m_c}}  \phi_c \label{eq:disk_konv}\\
  F_c^D &\approx  -\alpha  n_{ci} \delta S_c \left(\frac{\partial \phi}{\partial x_i}\right)_c\label{eq:disk_dif}
\end{align}
wobei $\dot{m_c}$ den Massenstrom durch die Seite $S_c$ bezeichnet.

Im nächsten Schritt müssen nun die Variablenwert $\phi_c$ durch die Variablenwerte in den
Mittelpunkten der Kontrollvolumen ausgedrückt werden.
Zunächst soll jedoch die rechte Seite von Gleichung~\eqref{eq:fvm3} betrachtet werden.
Da der Quellterm $f$ oft nicht analytisch integrierbar ist erfolgt im Allgemeinen eine Approximation
mittels numerischer Integration. Oftmals ist hierbei die Nutzung der Mittelpunktsregel mit der Ordnung zwei
ausreichend. Sie geht von der Annahme aus, das der Wert $f_P$ im Mittelpunkt des Kontrollvolumens
einen Mittelwert über die Werte von $f$ im gesamten Kontrollvolumen darstellt.

Die Auswertung des Volumenintegrals ergibt sich damit im zweidimensionalen Fall zu
\begin{equation}
  \int_V f\,dV \approx f_p\,\delta V
\end{equation}
wobei $\delta V$ das Volumen des Kontrollvolumens beschreibt und im Falle eines orthogonalen Gitters
mittels
\begin{equation}
  \delta V = (x_e - x_w)(y_n-y_s) = \Delta x \Delta y
\end{equation}
berechnet werden kann. Auf den bei dieser Appoximation entstehenden Abbruchfehler wird in
Abschnitt~\ref{sec:Quellterm} genauer eingegegangen.


\subsubsection{Ableitung konvektiver Flüsse}
\label{sec:konv_fluss}

Die durch die Anwendung der Mittelpunkteregel gewonnene Approximation (Gleichung~\ref{eq:disk_konv}) für konvektive
Flüsse beinhaltet noch den Variablenwert des Seitenmittelpunkts, der jedoch bei
zellenorientierter Variablenanordnung nicht gesucht ist. Deshalb muss dieser Wert $\phi_c$ im
nächsten Schritt durch Variablenwerte in den Kontrollvolumenmittelpunkten ausgedrückt werden.
Hierbei ist es sinnvoll möglichst lokale Approximationen zu verwenden, also beispielsweise
bei der Approximation von $\phi_e$ die Variablenwerte $\phi_P$ und $\phi_E$ zu benutzen.

Es werden nun einige gängige Approximationen für die Nutzung bei konvektiven Flüssen vorgestellt.
Es handelt sich hierbei genau genommen um Interpolationen aus den umliegenden Mittelpunkten der Kontrollvolumen.
Die Betrachtungen finden dabei aus Gründen der Übersichtlichkeit im eindimensionalen statt,
können jedoch problemlos auf höhere Dimensionen übertragen werden.

\paragraph{Upwind-Verfahren}
Bei Upwind-Verfahren (auch Upwind Differencing Scheme, UDS) wird der Verlauf von $\phi$ durch eine Treppenfunktion approximiert, die formuliert
beziehungsweise durch Integration aus den entsprechenden Diffenrentialgleichungen 
von der Geschwindigkeit $v$ abhängt.

\begin{equation*}
\phi_e=\left\{\begin{array}{cl} \phi_P, & \mbox{falls }v>0\\
\phi_E, & \mbox{falls } v<0\end{array}\right.
\end{equation*}

Das UDS-Verfahren besitzt einen formalen Interpolationsfehler erster Ordnung, liefert jedoch bei
Transport\-richt\-ung\-en, die senkrecht zu den Kontrollvolumenseiten liegen vergleichsweise gute Approximationen.
Außerdem liefert es uneingeschränkt beschränkte Lösungsalgorithmen, was ein großer Vorteil ist.




\paragraph{Zentraldifferenzen-Verfahren}
Das Zentraldifferenzen-Verfahren (auch Central Differencing Scheme, CDS) wird der
Wert $\phi_e$ durch eine lineare Interpolation aus $\phi_P$ und $\phi_E$ gewonnen.

\begin{equation*}
  \phi_e \approx \phi_E\gamma_e + \phi_P (1-\gamma_e)
\end{equation*}
Der Interpolationsfaktor $\gamma_e$ ist hierbei gegeben als:
\begin{equation}
  \gamma_e = \frac{x_e-x_P}{x_E-x_P}
  \label{eq:cds_faktor}
\end{equation}
Das CDS-Verfahren besitzt einen formalen Interpolationsfehler zweiter Ordnung. Es neigt jedoch zu
numerischen Oszillationen.

\paragraph{``Flux-Blending''-Verfahren}

Beim Flux-Blending versucht man Vor- und Nachteile von Verfahren höherer und niedrigerer
Ordnung zu verbinden. So hat das UDS-Verfahren einen Interpolationsfehler erster Ordnung,
während das CDS-Verfahren zu numerischen Oszillationen neigt.

Beim Flux-Blending werden nun beide Verfahren gewichtet zusammengeführt. Der Gewichtungsfaktor
wird hierbei $\beta$ genannt. Gilt $\beta = 1$, so ergibt sich das CDS-Verfahren, für
$\beta = 0$ das UDS-Verfahren.

\begin{equation}
  \phi_e \approx (1-\beta)\phi_e^{UDS} + \beta \phi_e^{CDS}
  \label{eq:flux_blending}
\end{equation}



\paragraph{Weitere Verfahren}
Neben UDS und CDS gibt es weitere Verfahren zur Approximation der Randvariablen durch
die Variablen im Zentrum eines Kontrollvolumens.

So wird beim QUICK-Verfahren (Quadratic Upwind Interpolation for Convective
Kinematics) ein quadratisches Polynom, welches auf die Flussrichtung abgestimmt ist,
zur Approximation genutzt.



\subsubsection{Ableitung diffusiver Flüsse}
\label{sec:dif_fluss}

Um den konvektiven Fluss (Gl. \ref{eq:disk_dif}) weiter zu approximieren, ist es nötig die Ableitung von
$\phi$ durch die Variablenwerte in den Kontrollvolumenmittelpunkten auszudrücken.
Beispielhaft soll hier wieder der eindimensionale, kartesische Fall betrachtet werden.
Für die östliche Seite muss somit die Ableitung $(\partial \phi /\partial x)_e$ approximiert werden.
Hierfür werden im Allgemeinen Differenzenformeln verwendet, wie sie auch bei der
Finite-Differenzen-Methode eingesetzt werden. Zentraldifferenzen werden hierbei
gegenüber Vorwärts- oder Rückwärtsdifferenzen bevorzugt eingesetzt, da sie die größte
Genauigkeit für die gleiche Menge von in die Approximation einbezogenen Anzahl von Gitterpunkten
besitzen.

Bei Verwendung einer Zentraldifferenz für die Ostseite des Kontrollvolumens ergibt sich folgende
Approximation:
\begin{equation}
  \left({\pder[\phi]{x}}\right)_e \approx \frac{\phi_E -\phi_P}{x_e-x_P}
\end{equation}
Zentraldifferenzen können nur genutzt werden solange auch alle benötigten Variablenwerte
verfügbar sind. Dies ist an der Rändern des betrachteten Gebiets meist nicht der Fall.
Hier muss dann selektiv auf Vorwärts- oder Rückwärtsdifferenzenformeln zurückgegriffen werden.
