\section{Finite Volumen Methode}

Die Finite-Volumen Methoden sind ein numerisches Verfahren zur Lösung von (partiellen)
Differentialgleichungen. Sie sind heute das Standardverfahren zu Lösung von Strömungsproblemen
wie zum Beispiel den Euler- oder Navier-Stokes-Gleichungen aber keineswegs nur darauf beschränkt.
Die charakteristische Eigenschaft der Finite-Volumen Methoden ist ihre Konservativität, also
die Erfüllung der den mathematischen Modellen zugrunde liegenden Erhaltungsprinzipien in den
diskreten Gleichungen.

\subsection{Ableitung der Finite Volumen diskretisierten Transportgleichung}

Ausgangspukt ist die in Abschnitt~\ref{sec:num_gitter} beschriebene Zerlegung des Problemgebiets
in diskrete Teilgebiete. Diese werden bei der Finite-Volumen Methode Kontrollvolumen
genannt (im zweidimensionalen Fall steht Volumen stellvertretend für Fläche).
Für jedes dieser Kontrollvolumen werden die Erhaltungsgleichungen in Integralform formuliert
beziehungsweise durch Integration aus den entsprchenden Diffenrentialgleichungen gewonnen.

Beispielhaft soll das Vorgehen hier anhand der in Abschnitt~\ref{sec:transportgl}
vorgestellten stationären Transportgleichung im zweidimensionalen Fall
($i=1$, $2$) vorgestellt werden.
\begin{equation}
  \frac{\partial}{\partial x_i} \left({\rho v_i \phi
- \alpha \frac{\partial \phi}{\partial x_i} }\right) = f
\label{eq:transp_fvm}
\end{equation}
Durch Integration von Gleichung~\eqref{eq:transp_fvm} über ein Kontrollvolumen $V$ (Gl.~\ref{eq:fvm1})
erhält man bei anschließender Anwendung des Gaußschen Integralsatzes die in Gleichung~\eqref{eq:fvm2} beschriebene Beziehung.
\begin{align}
  \int_V \frac{\partial}{\partial x_i} \left({\rho v_i \phi
- \alpha \frac{\partial \phi}{\partial x_i} }\right) dV &= \int_V f dV \label{eq:fvm1}\\
  \int_S  \left({\rho v_i \phi
- \alpha \frac{\partial \phi}{\partial x_i} }\right) n_i dS&= \int_V f dV \label{eq:fvm2}
\end{align}
Hierbei beschreibt $S$ die Oberfläche des Kontrollvolumens $V$, $dS$ ein Oberflächenelement
und $n_i$ die Komponenten des Einheitsnormalenvektors auf der Oberfläche.

Geht man nun von beliebig geformten Kontrollvolumen zu viereckigen Kontrollvolumen über,
so lässt sich das Integral über die Oberfläche $S$ in Gleichung~\eqref{eq:fvm2} durch eine
Summation über die Integrale der vier Seitenflächen umformen. Hierbei beschreibt $c$ die
einzelnen Seitenflächen und ist bei viereckigen Kontrollvolumen als $c=(e,w,n,s)$ definiert.
\begin{equation}
  \sum_c \int_{S_c} \left(\rho v_i \phi - \alpha \frac{\partial \phi}{\partial x_i}
\right) n_{ci} dS_c = \int_V f dV\label{eq:fvm3}
\end{equation}
Gleichung~\eqref{eq:fvm3} stellt nun die Bilanzgleichung für die konvektiven Flüsse
$F_c^C$ sowie die diffusiven Flüsse $F_c^D$ durch die Seiten des Kontrollvolumens $V$ dar.
\begin{align}
  F_c^C &=  \int_{S_c} \left(\rho v_i \phi \right) n_{ci} dS_c \\
  F_c^D &=  -\int_{S_c} \left(\alpha \frac{\partial \phi}{\partial x_i}\right) n_{ci} dS_c 
\end{align}

\subsection{Integralapproximation}

In dieser Arbeit wird von einer zellenorientierten Variablenanordnung (s. \cite{num_maschbau})
ausgegangen. Damit liegen die gesuchte diskrete Größe im Mittelpunkt jedes Kontrollvolumens.
Die Integrale der konvektiven und diffusiven Flüsse müssen durch diese Variablen
approximiert werden. Dabei ist es zweckmäßig zuerst die Oberflächenintegrale durch Variablen
auf der jeweilen Seite des Kontrollvolumens zu beschrieben und anschließend diese durch die Werte
im Zentrum des Kontrollvolumens zu beschreiben.

Beispielhaft soll hier das Oberflächenintegral
\begin{equation*}
  \int_{S_w} p_i n_{wi}\,dS_w
\end{equation*}
der Westseite $S_w$ betrachtet werden. Hierbei bezeichnet $p_i$ die Komponenten einer allgemeinen
Funktion $\mathbf{p} = (p_1(\mathbf{x}), p_2(\mathbf{x}))$. Die einfachste
Approximationsmöglichkeit besteht darin, den Funktionswert von $\mathbf{p}$ im Mittelpunkt
der Seite zu nutzen.
\begin{equation}
  \int_{S_w} p_i n_{wi}\,dS_w \approx q_w \delta S_w
\end{equation}
Die Normalkomponenten von $\mathbf{p}$ an der Stelle $w$ werden hierbei mit $q_w = p_{wi} n_{wi}$ bezeichnet.
Aufgrund der Nutzung des Funktionswertes im Seitenmittelpunkt wird diese Approximation auch Mittelpunktsregel genannt.
Sie hat eine Ordnung von zwei bezüglich der Seitenlänge $\delta S_w$.

Weitere gängige Approximationen umfassen die Trapezregel oder die Simpsonregel, auf die jedoch
in der vorliegenden Arbeit nicht näher eingegangen wird.

Für die Flüsse $F_c^C$ und $F_c^D$ ergeben sich mit der Mittelpunktsregel folgende Approximationen:
\begin{align*}
  F_c^C &\approx \underbrace{\rho v_i n_{ci} \delta S_c}_{\dot{m_c}}  \phi_c\\
  F_c^D &\approx  -\alpha  n_{ci} \delta S_c \left(\frac{\partial \phi}{\partial x_i}\right)_c
\end{align*}
wobei $\dot{m_c}$ den Massenstrom durch die Seite $S_c$ bezeichnet.

Im nächsten Schritt müssen nun die Variablenwert $\phi_c$ durch die Variablenwerte in den
Mittelpunkten der Kontrollvolumen ausgedrückt werden.
Zunächst soll jedoch die rechte Seite von Gleichung~\eqref{eq:fvm3} betrachtet werden.
Da der Quellterm $f$ oft nicht analytisch integrierbar ist erfolgt im Allgemeinen eine Approximation
mittels numerischer Integration. Oftmals ist hierbei die Nutzung der Mittelpunktsregel mit der Ordnung zwei
ausreichend. Sie geht von der Annahme aus, das der Wert $f_P$ im Mittelpunkt des Kontrollvolumens
einen Mittelwert über die Werte von $f$ im gesamten Kontrollvolumen darstellt.

Die Auswertung des Volumenintegrals ergibt sich damit im zweidimensionalen Fall zu
\begin{equation}
  \int_V f\,dV \approx f_p\,\delta V
\end{equation}
wobei $\delta V$ das Volumen des Kontrollvolumens beschreibt und im Falle eines orthogonalen Gitters
mittels
\begin{equation}
  \delta V = (x_e - x_w)(y_n-y_s) = \Delta x \Delta y
\end{equation}
berechnet werden kann. Auf den bei dieser Appoximation entstehenden Abbruchfehler wird in
Abschnitt~\ref{sec:Quellterm} genauer eingegegangen.
\subsection{Ableitung diffusiver Fluss}
\subsection{Ableitung konvektiver Fluss}
