\subsection{Numerische Gitter}
\label{sec:num_gitter}

Sind das mathematische Modell mit seinen Differentialgleichungen sowie den nötigen
Rand- sowie gegebenenfalls Anfangsbedingungen festgelegt, besteht der nächste Schritt
bei der Anwendung eines numerischen Berechnungsverfahren darin, das kontinuierliche
Gebiet von Raum und ggf. Zeit durch eine endliche Menge von Teilgebieten zu approximieren,
in denen dann der Wert des gesuchten Größe berechnet wird.

Diese Teilgebiete werden im Allgemeinen in Form eines Gitters über den zu untersuchenden
Bereich $V$ gelegt, weshalb dieser Arbeitsschritt auch oft Gittergenerierung genannt wird.

Ein wichtiges Unterscheidungskriterium ist die Anordnung der Gitterpunkte.
Man unterscheidet grundsätzlich zwischen strukturierten und unstrukturierten Gittern.

Bei strukturierten Gittern ist die Lage der Gitterpunkte zueinander festgelegt. Diese
Lagebeziehungen müssen deshalb nicht aufwendig gespeichert werden und die Gittergenerierung
ist schnell und ohne großen Rechenaufwand nötig. Zwar ist es möglich bestimmte Gebiete
aus der Gittergenerierung auszuschließen, eine Anpassung an die Problemgeometrie ist
jedoch nur innerhalb der vorgegebenen Lagebeziehungen möglich.

Bei unstrukturierten Gittern hingegen gibt es keine Vorschriften zu den Lagebeziehungen
der einzelnen Gitterpunkte. Diese müssen deshalb aufwendig gespeichert werden, was
Gittergeneration sowie Lösung komplizierter macht. Andererseits ist es Problemlos möglich
das Gitter an der Problemgeometrie auszurichten oder in bestimmten Gebieten selektiv zu verfeinern.

\subsubsection{Physikalisches und logisches Gebiet}

Oftmals lässt die Geometrie des zu untersuchenden Problems keine Anwendung eines
strukturierten Gitters zu. Nicht in jeden Fall ist es jedoch nötig daraufhin unstrukturierte
Gitter zu verwenden. Vielmehr ist es möglich das physikalische Gebiet des Problems
mittels einer geeigneten mathematischen Transformation auf ein sogenanntes
logisches Gebiet abzubilden, auf welchem dann die Anwendung eines strukturierten Gitters
möglich ist.

Die Abbildung von physikalischem auf logisches Gebiet muss dabei umkehrbar sein, das heißt
die Abbildungsfunktion muss bijektiv sein. Die Achsen des physikalischen Gebiets
werden im Allgemeinen mit lateinischen Buchstaben bezeichnet ($x$, $y$, $z$), während
für das logische Gebiet griechische Buchstaben verwendet werden ($\xi$, $\eta$, $\theta$).
Beispielhaft soll hier die formale, mathematische Beschreibung anhand eines ebenen
Beispiels gezeigt werden:
\begin{equation}
  f=f(x, y)\qquad \text{mit} \qquad x=x(\xi, \eta),\quad y=y(\xi, \eta)
\end{equation}


