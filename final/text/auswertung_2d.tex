\section{Zweidimensionale Testfälle}

\paragraph{Testfall 3}

\begin{figure}[h]
\centering
\begin{subfigure}[b]{.5\linewidth}
\centering
\begin{tikzpicture}
\begin{axis}[
%view={30+180}{30},
xlabel=$x$,
ylabel=$y$,
%zlabel={$f(x,y) = \sin(\frac{\pi}{2} x) \cos(\frac{\pi}{2} y)$},
domain=0:1,
height=7cm,
width=\textwidth
]
\addplot3[surf, mesh/ordering=y varies, faceted color=black] file{data/3/ERR_data.txt};
\end{axis}
\end{tikzpicture}
%\subcaption{Oberflächen}\label{fig:1a}
\end{subfigure}%
\begin{subfigure}[b]{.5\linewidth}
\centering
\begin{tikzpicture}
\begin{axis}[
view={0}{90},
xlabel=$x$,
ylabel=$y$,
zlabel={$f(x,y) = \sin(\frac{\pi}{2} x) \sin(\frac{\pi}{2} y)$},
domain=0:1,
height=7cm
]
\addplot3 +[contour prepared,very thick,mark=none, contour prepared format=matlab]
file {data/3/ERR_contour.txt};
\end{axis}
\end{tikzpicture}
%\subcaption{Another subfigure}\label{fig:1b}
\end{subfigure}
\caption{Absoluter Fehler (Testfall 3)}
\end{figure}


\begin{figure}[h]
\centering
\begin{subfigure}[b]{.5\linewidth}
\centering
\begin{tikzpicture}
\begin{axis}[
%view={30+180}{30},
xlabel=$x$,
ylabel=$y$,
height=6cm,
width=\textwidth
]
\addplot3[surf, mesh/ordering=y varies, faceted color=black] file{data/3/RES_data.txt};
\end{axis}
\end{tikzpicture}
\subcaption{Residuum}\label{fig:1a}
\end{subfigure}%
\begin{subfigure}[b]{.5\linewidth}
\centering
\begin{tikzpicture}
\begin{axis}[
%view={30+180}{30},
xlabel=$x$,
ylabel=$y$,
height=6cm,
width=\textwidth
]
\addplot3[surf, mesh/ordering=y varies, faceted color=black] file{data/3/TE_data.txt};
\end{axis}
\end{tikzpicture}
\subcaption{Abbruchfehler}\label{fig:1b}
\end{subfigure}
\caption{Residuum und Abbruchfehler mit Randsprüngen (Testfall 3)}
\end{figure}



\begin{figure}[h]
\centering
\begin{subfigure}[b]{.5\linewidth}
\centering
\begin{tikzpicture}
\begin{axis}[
view={150}{30},
xlabel=$x$,
ylabel=$y$,
height=6cm,
width=\textwidth
]
\addplot3[surf, mesh/ordering=y varies, faceted color=black] file{data/3/RES2_data.txt};
\end{axis}
\end{tikzpicture}
\subcaption{Residuum}\label{fig:1a}
\end{subfigure}%
\begin{subfigure}[b]{.5\linewidth}
\centering
\begin{tikzpicture}
\begin{axis}[
view={150}{30},
xlabel=$x$,
ylabel=$y$,
height=6cm,
width=\textwidth
]
\addplot3[surf, mesh/ordering=y varies, faceted color=black] file{data/3/TE2_data.txt};
\end{axis}
\end{tikzpicture}
\subcaption{Abbruchfehler}\label{fig:1b}
\end{subfigure}
\caption{Residuum und Abbruchfehler ohne Randsprünge (Testfall 3)}
\end{figure}






\begin{figure}[h]
\centering
\begin{subfigure}[b]{.5\linewidth}
\centering
\begin{tikzpicture}
\begin{axis}[
%view={170}{30},
xlabel=$x$,
ylabel=$y$,
height=6cm,
width=\textwidth
]
\addplot3[surf, mesh/ordering=y varies, faceted color=black] file{data/3/RESTE_data.txt};
\end{axis}
\end{tikzpicture}
\subcaption{Mit Randsprüngen}\label{fig:1a}
\end{subfigure}%
\begin{subfigure}[b]{.5\linewidth}
\centering
\begin{tikzpicture}
\begin{axis}[
view={180+20}{35},
xlabel=$x$,
ylabel=$y$,
height=6cm,
width=\textwidth
]
\addplot3[surf, mesh/ordering=y varies, faceted color=black] file{data/3/RESTE2_data.txt};
\end{axis}
\end{tikzpicture}
\subcaption{Ohne Randsprünge}\label{fig:1b}
\end{subfigure}
\caption{Differenz Residuum und Abbruchfehler (Testfall 3)}
\end{figure}
