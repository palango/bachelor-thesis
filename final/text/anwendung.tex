\chapter{Anwendung und Vergleich des Abbruchfehlerschätzers}

\section{Algorithmus zur Gitteradaption}

Zur Adaption des numerischen Gitters stehen verschiedene Verfahren zu Verfügung~\cite{roy2}.
Zum Einen können die vorhandenen Gitterpunkte verschoben werden. Dieses Verfahren wird r-Adaption
genannt. Fügt man weitere Gitterpunkte zum Gitter hinzu, so handelt es sich um eine h-Adaption.
Zuletzt kann lokal die Ordnung der Approximationen erhört werden, was p-Adaption genannt wird.
Weiterhin existieren zudem Mischformen der oben genannten Verfahren.

Während bei unstrukturierten Gittern heute hauptsächlich h-Adaption verwendet wird, ist bei
strukturierten Gittern die r-Adaption am gebräuchlichsten. Diese soll auch in der vorliegenden
Arbeit angewandt werden.

Der Algorithmus zur Gitteranpassung basiert hierbei auf der Analogie zu einem System
linearer Federn.
Zuerst wird eine Lösungsanpassungsfunktion $\theta_i$ erstellt, die auf den gewünschten
Lösungseigenschaften wie zum Beispiel Ableitung, Krümmung oder Abbruchfehler basiert. Hierbei
bezeichnet $i$ den betrachteten Gitterpunkt. Ist die gewählte Lösungseigenschaft nur in den
Mittelpunkten der Kontrollvolumen bekannt, kann sie beispielsweise durch lineare Interpolation
auf die Gitterpunkte überführt werden.

Anschließend wird aus der Anpassungsfunktion $\theta_i$ eine Gewichtungsfunktion
$W_i$ erzeugt.
Der frei wählbare Exponent $q$ wird in dieser Arbeit auf eins gesetzt.
\begin{equation}
  W_i = \vert\theta_i \vert^q
\end{equation}
Die Gewichtungsfunktion wird zur Steuerung des Adaptionsprozesses genutzt, wobei größere
Werte eine Gitterverfeinerung und kleinere Werte eine Weitung des Gitters bewirken. Um
weiche Änderungen der Größen von Nachbarzellen zu erreichen, wird die Gewichtungsfunktion
anschließend geglättet. Dies geschieht iterativ über folgende Formel:
\begin{equation}
  W_i = \frac{W_{i-1} + 4W_i + W_{i+1}}{6}
\end{equation}
Aus den geglätteten Gewichten $W_i$ können nun die Federkonstanten berechnet werden, wobei
die Federkonstante $k_{i+1/2}$ die Feder beschreibt, die die Punkte $x_{i}$ und $x_{i+1}$ verbindet.
\begin{equation}
  k_{i+1/2} = \frac{W_i + W_{i+1}}{2}
\end{equation}
Hohe Werte der Gewichtungsfunktion führen demnach zu einer hohen Federkonstante, was die
Gitterverfeinerung bewirkt. Die neuen Gitterkoordinaten $x_i^{m+1}$ können nun über folgende
Gleichung berechnet werden, wobei $x_i^m$ für die Gitterpunkte des alten, nicht adaptierten
Gitters steht.
\begin{equation}
  x_i^{m+1} = \frac{k_{i-1/2} x_{i-1}^m + k_{i+1/2} x_{i+1}^m}{k_{i-1/2} + k_{i+1/2}}
\end{equation}
Im Folgenden sollen nun verschiedene Lösungseigenschaften zur Gitteradaption untersucht werden.
Diese sind:
\begin{enumerate}
  \item Lösungsgradient: $\displaystyle\theta_i = \frac{\partial \phi}{\partial x}\bigg \vert_i$
  \item Abbruchfehler: $\theta_i = TE_i$
  \item Diskretisierungsfehler
\end{enumerate}
