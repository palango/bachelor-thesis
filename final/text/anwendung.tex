\chapter{Anwendung und Vergleich des Abbruchfehlerschätzers}

Zur Adaption des numerischen Gitters stehen verschiedene Verfahren zu Verfügung~\cite{roy2}.
Zum Einen können die vorhandenen Gitterpunkte verschoben werden. Dieses Verfahren wird r-Adaption
genannt. Fügt man weitere Gitterpunkte zum Gitter hinzu, so handelt es sich um eine h-Adaption.
Zuletzt kann lokal die Ordnung der Approximationen erhört werden, was p-Adaption genannt wird.
Es existieren zudem Mischformen der oben genannten Verfahren.

Während bei unstrukturierten Gittern heute hauptsächlich h-Adaption verwendet wird, ist bei
strukturierten Gittern die r-Adaption am gebräuchlichsten. Diese soll auch in der vorliegenden
Arbeit angewandt werden.

Im Folgenden sollen nun verschiedene Lösungseigenschaften zur Gitteradaption verglichen werden.
Ausgangspunkt ist dabei immer ein Gitter mit zwanzig Kontrollvolumen, dass mit dem
Expansionsfaktor $\alpha=0.5$ erzeugt wurde. Die Anzahl der Iterationen ist auf 350 festgelegt.
Die betrachteten Verfahren sind:
\begin{itemize}
  \item Lösungsgradient
  \item Abbruchfehler
  \item Richardson-Extrapolations-Indikator
  \item Zweigitter-Fehlerschätzer
\end{itemize}




\section{Algorithmus zur Gitteradaption}

Der Algorithmus zur Gitteranpassung basiert hierbei auf der Analogie zu einem System
linearer Federn.
Zuerst wird eine Lösungsanpassungsfunktion $\theta_i$ erstellt, die auf den gewünschten
Lösungseigenschaften wie zum Beispiel Ableitung, Krümmung oder Abbruchfehler basiert. Hierbei
bezeichnet $i$ den betrachteten Gitterpunkt. Ist die gewählte Lösungseigenschaft nur in den
Mittelpunkten der Kontrollvolumen bekannt, kann sie beispielsweise durch lineare Interpolation
auf die Gitterpunkte überführt werden.

Anschließend wird aus der Anpassungsfunktion $\theta_i$ eine Gewichtungsfunktion
$W_i$ erzeugt.
Der frei wählbare Exponent $q$ wird in dieser Arbeit auf eins gesetzt.
\begin{equation}
  W_i = \vert\theta_i \vert^q
\end{equation}
Die Gewichtungsfunktion wird zur Steuerung des Adaptionsprozesses genutzt, wobei größere
Werte eine Gitterverfeinerung und kleinere Werte eine Weitung des Gitters bewirken. Um
weiche Änderungen der Größen von Nachbarzellen zu erreichen, wird die Gewichtungsfunktion
anschließend geglättet. Dies geschieht iterativ über folgende Formel:
\begin{equation}
  W_i = \frac{W_{i-1} + 4W_i + W_{i+1}}{6}
\end{equation}
Aus den geglätteten Gewichten $W_i$ können nun die Federkonstanten berechnet werden, wobei
die Federkonstante $k_{i+1/2}$ die Feder beschreibt, die die Punkte $x_{i}$ und $x_{i+1}$ verbindet.
\begin{equation}
  k_{i+1/2} = \frac{W_i + W_{i+1}}{2}
\end{equation}
Hohe Werte der Gewichtungsfunktion führen demnach zu einer hohen Federkonstante, was die
Gitterverfeinerung bewirkt. Die neuen Gitterkoordinaten $x_i^{m+1}$ können nun über folgende
Gleichung berechnet werden, wobei $x_i^m$ für die Gitterpunkte des alten, nicht adaptierten
Gitters steht.
\begin{equation}
  x_i^{m+1} = \frac{k_{i-1/2} x_{i-1}^m + k_{i+1/2} x_{i+1}^m}{k_{i-1/2} + k_{i+1/2}}
\end{equation}


\section{Abbruchfehler}

Hier wird der berechnete Abbruchfehler als Indikator für die Güte der Lösung verwendet.
Damit wird $\theta_i = TE_i$ gesetzt.
\begin{figure}[h]
\centering
\begin{tikzpicture}
  \begin{axis}[width=0.7\textwidth, scaled y ticks=false,height=6cm,
  xlabel=$x$,
  ylabel=Absoluter Fehler]
    \addplot[tud2d, mark=*, very thick] file {anwendung/te/err1_data.txt};
    \addplot[tud9d, mark=*, very thick] file {anwendung/te/err2_data.txt};
    \legend{Startwert,Optimalwert}
  \end{axis}
\end{tikzpicture}
\caption{Vergleich absoluter Fehler von Start- und Optimalwert}
\end{figure}


\begin{figure}[h]
\centering
   \begin{subfigure}{0.49\linewidth} \centering
  \begin{tikzpicture}
    \begin{semilogyaxis}[width=\textwidth,height=5cm]
    %xlabel=$x$,
    %ylabel=$y$]
      \addplot[tud2d, very thick] file {anwendung/te/te_serr_data.txt};
    \end{semilogyaxis}
  \end{tikzpicture}
  \caption{Abbruchfehler}
\end{subfigure}
   \begin{subfigure}{0.49\linewidth} \centering
  \begin{tikzpicture}
    \begin{semilogyaxis}[width=\textwidth,height=5cm]
    %xlabel=$x$,
    %ylabel=$y$]
      \addplot[tud2d, very thick] file {anwendung/te/serr_data.txt};
    \end{semilogyaxis}
  \end{tikzpicture}
  \caption{Absoluter Fehler}
\end{subfigure}
\caption{Verlauf der Gitteradaption}
\end{figure}


\begin{figure}[h]
\centering
\begin{tikzpicture}
  \begin{axis}[width=0.7\textwidth, scaled y ticks=false,
%  xlabel=$x$,
  ylabel=Abbruchfehler,height=5cm]
    \addplot[tud2d, mark=*, very thick] file {anwendung/te/te1_data.txt};
    \addplot[tud9d, mark=*, very thick] file {anwendung/te/te2_data.txt};
    \legend{Startwert,Optimalwert}
  \end{axis}
\end{tikzpicture}
\caption{Vergleich Abbruchfehler von Start- und Optimalwert} \label{fig:twofigs}
\end{figure}

\clearpage






\section{Gradient}

Hier wird der berechnete Abbruchfehler als Indikator für die Güte der Lösung verwendet.
Damit wird $\theta_i = \frac{\partial \phi}{\partial x} \big\vert_i$ gesetzt.
\begin{figure}[h]
\centering
\begin{tikzpicture}
  \begin{axis}[width=0.7\textwidth, scaled y ticks=false,
  xlabel=$x$,
  ylabel=Absoluter Fehler]
    \addplot[tud2d, mark=*, very thick] file {anwendung/grad/err1_data.txt};
    \addplot[tud9d, mark=*, very thick] file {anwendung/grad/err2_data.txt};
    \legend{Startwert,Optimalwert}
  \end{axis}
\end{tikzpicture}
\caption{Vergleich absoluter Fehler von Start- und Optimalwert} \label{fig:twofigs}
\end{figure}


\begin{figure}[h]
\centering
   \begin{subfigure}{0.49\linewidth} \centering
  \begin{tikzpicture}
    \begin{semilogyaxis}[width=\textwidth]
    %xlabel=$x$,
    %ylabel=$y$]
      \addplot[tud2d, very thick] file {anwendung/grad/grad_serr_data.txt};
    \end{semilogyaxis}
  \end{tikzpicture}
  \caption{Summierter Gradient}
\end{subfigure}
   \begin{subfigure}{0.49\linewidth} \centering
  \begin{tikzpicture}
    \begin{semilogyaxis}[width=\textwidth]
    %xlabel=$x$,
    %ylabel=$y$]
      \addplot[tud2d, very thick] file {anwendung/grad/serr_data.txt};
    \end{semilogyaxis}
  \end{tikzpicture}
  \caption{Absoluter Fehler}
\end{subfigure}
\caption{Verlauf der Gitteradaption}
\end{figure}
\clearpage




\section{Richardson-Extrapolations-Indikator}

Bei der Richardson-Extrapolation ($RE$) \cite{roy} wird die Lösungs des Problems auf einem groben und einem
verfeinerten Gitter berechnet. Aus den beiden Lösungen lasst sich dann eine verbesserte
Lösung extrapolieren. Um diese als Indikator verwenden zu können, wird die Differenz der berechneten Lösung ($T$)
und der Richardson-Extrapolation berechnet.
\begin{equation}
  REI_i = RE_i - T_i
\end{equation}
\begin{figure}[h]
\centering
\begin{tikzpicture}
  \begin{axis}[width=0.7\textwidth, scaled y ticks=false,
  xlabel=$x$,
ylabel=Absoluter Fehler]
    \addplot[tud2d, mark=*, very thick] file {anwendung/rich/err1_data.txt};
    \addplot[tud9d, mark=*, very thick] file {anwendung/rich/err2_data.txt};
    \legend{Startwert,Optimalwert}
  \end{axis}
\end{tikzpicture}
\caption{Vergleich absoluter Fehler von Start- und Optimalwert} \label{fig:twofigs}
\end{figure}


\begin{figure}[h]
\centering
   \begin{subfigure}{0.49\linewidth} \centering
  \begin{tikzpicture}
    \begin{semilogyaxis}[width=\textwidth,height=5cm]
    %xlabel=$x$,
    %ylabel=$y$]
      \addplot[tud2d, very thick] file {anwendung/rich/rich_serr_data.txt};
    \end{semilogyaxis}
  \end{tikzpicture}
  \caption{Summierte Differenz zur Richardson-Extrapolation}
\end{subfigure}
   \begin{subfigure}{0.49\linewidth} \centering
  \begin{tikzpicture}
    \begin{semilogyaxis}[width=\textwidth,height=5cm]
    %xlabel=$x$,
    %ylabel=$y$]
      \addplot[tud2d, very thick] file {anwendung/rich/serr_data.txt};
    \end{semilogyaxis}
  \end{tikzpicture}
  \caption{Absoluter Fehler}
\end{subfigure}
\caption{Verlauf der Gitteradaption} \label{fig:twofigs}
\end{figure}


\begin{figure}[h]
\centering
\begin{tikzpicture}
  \begin{axis}[width=0.7\textwidth, scaled y ticks=false,height=5cm,
  xlabel=$x$,
  ylabel=REI,legend style={anchor=south east, at={(0.98,0.02)}}]
    \addplot[tud2d, mark=*, very thick] file {anwendung/rich/rich1_data.txt};
    \addplot[tud9d, mark=*, very thick] file {anwendung/rich/rich2_data.txt};
    \legend{Startwert,Optimalwert}
  \end{axis}
\end{tikzpicture}
\caption{Vergleich Richardson-Extrapolations-Indikator von Start- und Optimalwert} \label{fig:twofigs}
\end{figure}








\section{Zweigitter-Fehlerschätzer}
Hier wird ebenfalls die Lösungs des Problems auf einem groben ($T_c$) und einem
verfeinerten Gitter ($T_f$) berechnet. Allerdings wird dann keine neue Lösung extrapoliert sondern
beide Lösungen verglichen.
\begin{equation}
  ZG = T_g\vert_i - T_{f}\vert_{i1} - T_{f}\vert_{i2}
\end{equation}
\begin{figure}[h]
\centering
\begin{tikzpicture}
  \begin{axis}[width=0.7\textwidth, scaled y ticks=false,
  xlabel=$x$,
ylabel=Absoluter Fehler]
    \addplot[tud2d, mark=*, very thick] file {anwendung/zg/err1_data.txt};
    \addplot[tud9d, mark=*, very thick] file {anwendung/zg/err2_data.txt};
    \legend{Startwert,Optimalwert}
  \end{axis}
\end{tikzpicture}
\caption{Vergleich absoluter Fehler von Start- und Optimalwert} \label{fig:twofigs}
\end{figure}


\begin{figure}[h]
\centering
   \begin{subfigure}{0.49\linewidth} \centering
  \begin{tikzpicture}
    \begin{semilogyaxis}[width=\textwidth,height=5cm]
    %xlabel=$x$,
    %ylabel=$y$]
      \addplot[tud2d, very thick] file {anwendung/zg/zg_serr_data.txt};
    \end{semilogyaxis}
  \end{tikzpicture}
  \caption{Summierter Zweigitter-Fehler}
\end{subfigure}
   \begin{subfigure}{0.49\linewidth} \centering
  \begin{tikzpicture}
    \begin{semilogyaxis}[width=\textwidth,height=5cm]
    %xlabel=$x$,
    %ylabel=$y$]
      \addplot[tud2d, very thick] file {anwendung/zg/serr_data.txt};
    \end{semilogyaxis}
  \end{tikzpicture}
  \caption{Absoluter Fehler}
\end{subfigure}
\caption{Verlauf der Gitteradaption} \label{fig:twofigs}
\end{figure}


\begin{figure}[h]
\centering
\begin{tikzpicture}
  \begin{axis}[width=0.7\textwidth, xlabel=$x$,height=5cm,
  ylabel=Abbruchfehler,legend style={anchor=south east, at={(0.98,0.02)}}]
    \addplot[tud2d, mark=*, very thick] file {anwendung/zg/zg1_data.txt};
    \addplot[tud9d, mark=*, very thick] file {anwendung/zg/zg2_data.txt};
    \legend{Startwert,Optimalwert}
  \end{axis}
\end{tikzpicture}
\caption{Vergleich Zweigitter-Fehler von Start- und Optimalwert} \label{fig:twofigs}
\end{figure}
\clearpage







\section{Vergleich der Indikatoren}

Im folgenden Diagramm sind die Startkonfiguration sowie die adaptierte Version
jedes Indikators eingetragen. Man sieht das der Zweigitter-Fehlerschätzer hinter
den anderen Indikatoren zurückfällt. Der Richardson- und der Gradienten-Indikator weisen
leichte Schwächen an den Rändern des Problemgebiets auf.
\begin{figure}[h]
\centering
\begin{tikzpicture}
  \begin{axis}[width=0.9\textwidth, height=12cm,scaled y ticks=false,
  xlabel=$x$,
%ylabel=Absoluter Fehler]
]
    \addplot[tud2d, mark=*, very thick] file {anwendung/zg/err1_data.txt};
    \addplot[tud4d, mark=*, very thick] file {anwendung/te/err2_data.txt};
    \addplot[tud9d, mark=*, very thick] file {anwendung/grad/err2_data.txt};
    \addplot[tud6d, mark=*, very thick] file {anwendung/rich/err2_data.txt};
    \addplot[tud1d, mark=*, very thick] file {anwendung/zg/err2_data.txt};
    \legend{Startwert,Abbruchfehler, Gradient, Richardson, Zweigitter}
  \end{axis}
\end{tikzpicture}
\caption{Vergleich absoluter Fehler von Start- und Optimalwert}
\end{figure}

Im folgenden Diagramm sind die relativen Verbesserungen des absoluten Fehlers für jeden
Indikator aufgetragen. Die Gewinne sind beim Abbruchfehler mit 97,7\% am größten.
\begin{figure}[h]
\centering
\begin{tikzpicture}
\begin{axis}[width=0.9\textwidth,
    symbolic x coords={Abbruchfehler, Gradient, Richardson, Zweigitter},
    xtick=data]
    \addplot[ybar,fill=tud2d] coordinates {
      (Abbruchfehler, 97.77)
        (Gradient, 89.63)
        (Richardson, 90.82)
        (Zweigitter, 65.94)
    };
\end{axis}
\end{tikzpicture}
\caption{Relative Verbesserung des absoluten Fehlers in Prozent}
\end{figure}
