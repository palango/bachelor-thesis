\section{Verifikation von Ergebnissen}
\label{sec:verifik_allg}

Die Verifikation von numerischen Simulationen spielt eine wichtige Rolle
für deren weiter Verwendung zur Produktentwicklung und -verbesserung.
Man unterscheidet zwei Arten von Verifikationen: die Code-Verifikation und die Lösungsverifikation
\cite{veluri}.
Bei der Code-Verifikation wird nachgewiesen, dass keine Fehler oder Inkonsistenzen im numerischen
Algorithmus enthalten sind. In der Lösungsverifikation werden die drei Typen von numerischen Fehlern
abgeschätzt: der Rundungsfehler, der Iterationsfehler sowie der Diskretisierungsfehler.

\subsection{Konvergenzordnung}

Die Konvergenzordnung beschreibt den Zusammenhang zwischen Verringerung des Diskretisierungsfehlers
und der Verfeinerung des numerischen Gitters.
Die Überprüfung der Konvergenzordnung ist ein wichtiger Nachweis, da sie nicht nur Aussagen über
die Konvergenz der Lösung treffen lassen sondern auch geprüft werden kann ob der Diskretisierungsfehler
sich mit der erwarteten Rate verringert.

\subsubsection{Formale Konvergenzordnung}

Die Formale Konvergenzordnung ist die Rate mit der sich die diskretisierten Gleichungen
den ursprünglichen (partiellen) Differentialgleichungen annähern.
Sie wird durch eine Analyse des Abbruchfehlers
der diskretisierten Gleichung gewonnen.
\begin{equation}
  f''_i = \frac{f_{i+1}-2f_i +f_{i-1}}{\Delta x^2} -\frac{1}{12} f^{IV}_i \Delta x^2 + HOT
\end{equation}
Bei der hier diskretsierten zweiten Ableitung einer Funktion $f$ sieht man das der führende
Term des Abbruchfehlers quadratisch von $\Delta x$ abhängt. Die formale Konvergenzordnung im Bezug
auf $x$ ist somit zwei.

Wichtig ist anzumerken, das die Konvergenzordnungen verschiedener Dimensionen
eines Problems unterschiedlich sein können. So kann die Konvergenzordnung im Beztug auf die Zeit eins sein,
während sie im Raum zwei ist.


\subsubsection{Beobachtete Konvergenzordnung}

Die beobachtetet Konvergenzordnung wird direkt aus den Ergebnissen der Berechnung mit systematisch
verfeinerten Gittern gewonnen. Sollte die berechnete Konvergenzordnung nicht mit der formalen
Konvergenzordnung übereinstimmen ist von Fehlern im Programm oder fehlerhaften numerischen Algorithmen auszugehen.

Um die Konvergenzordnung eines Programms zu berechnen ist es zwingend notwendig die exakte Lösung der
Differentialgleichungen zu kennen. Eine Lösungsmöglichkeit für dieses Problem wird in Abschnitt~\ref{sec:man_sol}
vorgestellt.

Der Diskretisierungsfehler $DE$ ist definiert als Differenz zwischen der exakten Lösung $f_{exakt}$
der Differentialgleichungen und der exakten Lösung $f(\Delta)$ der diskreten Gleichungen.
Da jedoch die exakte Lösung der diskreten Gleichungen, die für verschiedene Gitter unterschiedlich ist,
nicht bekannt ist, wird stattdessen die numerische Lösung verwendet. Hierbei beachtet man dementsprechend
Iterationsfehler und Rundungsfehler nicht. Für eine Gitter $k$ lässt sich somit der Diskretisierungsfehler
über folgende Gleichung beschreiben:
\begin{equation*}
  DE_k=f(\Delta_k) - f_{exakt} = g_p h_k^p + HOT
\end{equation*}
wobei $g_p$ den Koeffizienten des führenden Terms des Abbruchfehlers darstellt.
Bei Vernachlässigung der Terme höherer Ordnung können nun die Diskretisierungsfehler bei
zwei verschiedenen Gittern betrachtet werden.
\begin{align*}
  DE_1 &= f(\Delta_1) - f_{exakt} = g_p h_1^{\tilde{p}}\\
  DE_2 &= f(\Delta_2) - f_{exakt} = g_p h_2^{\tilde{p}}
\end{align*}
Da die exakte Lösung der Differentalgleichung bekannt ist kann die beobachtete Konvergenzordnung
$\tilde{p}$ aus den beiden Gleichungen berechnet werden.

\begin{equation}
  \tilde{p}=\frac{\ln \left(\frac{DE_2}{DE_1}\right)}{\ln \left(\frac{h_2}{h_1}\right)}
\end{equation}


\subsection{Konstruierte Lösungen}
\label{sec:man_sol}

Zur Überprüfung der Konvergenzordnung ist es nötig die exakte Lösung der Differentialgleichung zu
kennen. Traditionell sind solche Lösungen nur für einfache Probleme oder bestimmte Randbedingungen
und Vereinfachungen bekannt. Für komplexe Gleichungen mit komplexen Modellen, komplizierten Geometrien
oder Nichtlinearitäten sind jedoch selten Lösungen bekannt. Mit Hilfe der Methode der konstruierten
Lösungen (engl. Manufactured Solution) lassen sich exakte Lösungen für Verifikationszwecke erstellen.

Die Methode der konstruierten Lösungen bietet viele Vorteile. So kann sie neben der Finite-Volumen Methode auch
für Finite-Elemente- oder Finite-Differenzen-Verfahren genutzt werden. Außerdem ist sie auch bei nichtlinearen
oder mehreren Gleichungen (z.B. Navier-Stokes-Gleichungen) anwendbar. Weiterhin wurde nachgewiesen~\cite{roache_book} das sich
sehr empfindliche Fehler erkennen lassen.

Das Vorgehen bei der Konstruktion einer Lösung soll anhand einer stationären, eindimensionalen Transportgleichung
mit Diffusions- und Konvektionstermem demonstriert werden.
\begin{equation*}
  \frac{\partial}{\partial x} \left({\rho v \phi
- \alpha \frac{\partial \phi}{\partial x} }\right) = f
\end{equation*}
Zuerst wird eine Lösung für $\phi$ gewählt und deren Ableitungen berechnet.
\begin{align*}
  \phi(x) &= A + \sin(Bx)\\
  \phi'(x) &= B \cos(Bx)\\
  \phi''(x) &= -B^2\sin(Bx)
\end{align*}
Nun führen wir den zusätzlichen Quellterm $f_{ad}$ ein und setzen $\phi(x)$ so wie die Ableitungen ein.
\begin{align*}
  \rho v \pder[\phi]{x} - \alpha\frac{\partial^2 \phi}{\partial x^2}-f&=f_{ad}\\
  \rho v B \cos(Bx) + \alpha B^2 \sin(Bx) - f &= f_{ad}
\end{align*}
Wird nun $f_{ad}$ in die Differentialgleichung zurück substituiert ergibt sich die konstruierte Lösung.
\begin{equation}
   \frac{\partial}{\partial x} \left({\rho v \phi
- \alpha \frac{\partial \phi}{\partial x} }\right)= \rho v B \cos(Bx) + \alpha B^2 \sin(Bx)
\end{equation}

\paragraph{Randbedingungen}

Um die Konstanten $A$ und $B$ in der konstruierten Lösung bestimmen zu können,
müssen konkrete Randbedingungen für das Problem festgelegt werden. Beispielhaft
soll das hier mit den Randwerten $\phi(0) = 1$ und $\phi(1) = 1$ geschehen.
Damit ergeben sich die Konstanten zu $A = 1$ sowie $B = \pi$. Die 
konstruierte Lösung $\phi(x)$ hat damit die folgende Form und der Quellterm $f$
ergibt sich zu:
\begin{equation*}
  \phi(x) = 1 + \sin(\pi x)
\end{equation*}
\begin{equation*}
  f=\rho v \cos(\pi x) + \alpha \pi^2 \sin(\pi x)
  \label{eq:quellterm}
\end{equation*}


\paragraph{Eigenschaften guter konstruierter Lösungen}

Bei der Wahl der Lösung sollten einige Kriterien eingehalten werden. So sollte die gewählte Lösung
aus glatten, analytischen Funktionen mit glatten Ableitungen bestehen. Die Bedingung der glatten Ableitungen
erlaubt es auch auf relativ groben Gittern die Konvergenzordnung nachzuweisen.
Aufgrund der genannten Kriterien werden für konstruierte Lösungen oft trigonometrische und Exponentialfunktionen benutzt.
Die haben den Vorteil unendlich viele glatte Ableitungen zu haben. Trigonometrische Funktionen lassen
sich zudem leicht in ihrem Wertebereich anpassen.

Weiterhin sollte, auch wenn von der konstruierten Lösung keine physikalische Realität erwartet wird, möglichst
erreichbare physikalische Zustände dargestellt werden. Werden beispielsweise Schallgeschwindigkeiten aus
aus Temperaturen berechnet, so setzt diese Berechnung positive Temperaturen voraus. Die konstruierte Lösung sollte dies wiederspiegeln
und positive Temperaturen zurückgeben.

Die konstruierte Lösung sollte außerdem so gewählt sein, dass alle Terme die gleiche Größenordnung haben
und kein Term von einem anderen dominiert wird. Beispielsweise sollten diffusive als auch konvektive Terme
möglichst gleich stark in die Lösung einfließen, sodass beide getestet werden.
