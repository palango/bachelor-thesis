\section{Verifikation von Ergebnissen}

Die Verifikation von numerischen Simulationen spielt eine wichtige Rolle
für deren weiter Verwendung zur Produktentwicklung und -verbesserung.
Man unterscheidet zwei Arten von Verifikationen: die Code-Verifikation und die Lösungsverifikation
\cite{veluri}.
Bei der Code-Verifikation wird nachgewiesen, dass keine Fehler oder Inkonsistenzen im numerischen
Algorithmus enthalten sind. In der Lösungsverifikation werden die drei Typen von numerischen Fehlern
abgeschätzt: der Rundungsfehler, der Iterationsfehler sowie der Diskretisierungsfehler.

\subsection{Konvergenzordnung}

Die Konvergenzordnung beschreibt den Zusammenhang zwischen Verringerung des Diskretisierungsfehlers
und der Verfeinerung des numerischen Gitters.
Die Überprüfung der Konvergenzordnung ist ein wichtiger Nachweis, da sie nicht nur Aussagen über
die Konvergenz der Lösung treffen lassen sondern auch geprüft werden kann ob der Diskretisierungsfehler
sich mit der erwarteten Geschwindigkeit verringert.

\subsubsection{Formale Konvergenzordnung}

Die Formale Konvergenzordnung ist die Geschwindigkeit mit der sich die diskretisierten Gleichungen
den ursprünglichen (partiellen) Differentialgleichungen annähern. Sie wird durch eine Analyse des Abbruchfehlers
der diskretisierten Gleichung gewonnen.
\begin{equation}
  f''_i = \frac{f_{i+1}-2f_i +f_{i-1}}{\Delta x^2} -\frac{1}{12} f^{IV}_i \Delta x^2 + HOT
\end{equation}
Bei der hier diskretsierten zweiten Ableitung einer Funktion $f$ sieht man das der führende
Term des Abbruchfehler quadratiusch von $\Delta x$ abhängt. Die formale Konvergenzordnung im Bezug
auf $x$ ist somit 2.

Wichtig ist anzumerken, das die Konvergenzordnungen verschiedener Dimensionen
eines Problems unterschiedlich sein können. So kann die Konvergenzordnung im Beztug auf die Zeit eins sein,
während sie im Raum zwei ist.


\subsubsection{Beobachtete Konvergenzordnung}

Die beobachtetet Konvergenzordnung wird direkt aus den Ergebnissen der Berechnung mit systematisch
verfeinerten Gittern gewonnen. Sollte die berechnete Konvergenzordnung nicht mit der formalen
Konvergenzordnung übereinstimmen ist von Fehlern im Programm oder fehlerhaften numerischen Algorithmen auszugehen.

Um die Konvergenzordnung eines Programms zu berechnen ist es zwingend notwendig die exakte Lösung der
Differentialgleichungen zu kennen. Eine Lösungsmöglichkeit für dieses Problem wird in Abschnitt~\ref{sec:man_sol}
vorgestellt.

Der Diskretisierungsfehler $DE$ ist definiert als Differenz zwischen der exakten Lösung $f_{exakt}$
der Differentialgleichungen und der exakten Lösung $f(\Delta)$ der diskreten Gleichungen 
Da jedoch die exakte Lösung der diskreten Gleichungen, die für verschiedene Gitter unterschiedlich ist,
nicht bekannt ist, wird stattdessen die numerische Lösung verwendet. Hierbei beachtet man dementsprechend
Iterationsfehler und Rundungsfehler nicht. Für eine Gitter $k$ lässt sich somit der Diskretisierungsfehler
über folgende Gleichung beschreiben:
\begin{equation*}
  DE_k=f(\Delta_k) - f_{exakt} = g_p h_k^p + HOT
\end{equation*}
wobei $g_p$ den Koeffizienten des führenden Terms des Abbruchfehlers darstellt.
Bei Vernachlässigung der Terme höherer Ordnung können nun die Diskretisierungsfehler bei
zwei verschiedenen Gittern betrachtet werden.
\begin{align*}
  DE_1 &= f(\Delta_1) - f_{exakt} = g_p h_1^{\tilde{p}}\\
  DE_2 &= f(\Delta_2) - f_{exakt} = g_p h_2^{\tilde{p}}
\end{align*}
Da die exakte Lösung des Differentalgleichung bekannt ist kann die beobachtete Konvergenzordnung
$\tilde{p}$ aus den beiden Gleichungen berechnet werden.

\begin{equation}
  \tilde{p}=\frac{\ln \left(\frac{DE_2}{DE_1}\right)}{\ln \left(\frac{h_2}{h_1}\right)}
\end{equation}


\subsection{Konstruierte Lösung}
\label{sec:man_sol}

Wir wählen die folgende Lösung für $\phi$ und berechnen die Ableitungen.

\begin{align*}
  \phi(x) &= A + \sin(Bx)\\
  \phi'(x) &= B \cos(Bx)\\
  \phi''(x) &= -B^2\sin(Bx)
\end{align*}

Nun führen wir den zusätzlichen Quellterm $f_{ad}$ ein und setzen $\phi(x)$ ein.

\begin{equation*}
  \alpha\frac{\partial^2 \phi}{\partial x^2}-f=f_{ad}
\end{equation*}

so erhält man:

\begin{equation*}
  f_{ad} = -\alpha B^2\sin(BX)-f
\end{equation*}

Damit ergibt sich als Manufactured Solution:
\begin{align}
  \alpha\frac{\partial^2 \phi}{\partial x^2} &= f -\alpha B^2\sin(Bx)-f\\
                                             &= -\alpha B^2\sin(Bx)
\end{align}

\subsection{Randbedingungen}
\label{sec:Randbedingungen}

Um die Konstanten $A$ und $B$ in der Manufactured Solution bestimmen zu können,
müssen konkrete Randbedingungen für das Problem festgelegt werden. Beispielhaft
soll das hier mit den Randwerten $\phi(0) = 1$ und $\phi(1) = 1$ geschehen.
Damit ergeben sich die Konstanten zu $A = 1$ sowie $B = \pi$. Die 
konstruierte Lösung $\phi(x)$ hat damit die folgende Form und der Quellterm $f$
ergibt sich zu:
\begin{equation*}
  \phi(x) = 1 + \sin(\pi x)
\end{equation*}
\begin{equation}
  f=-\pi^2 \sin(\pi x)
  \label{eq:quellterm}
\end{equation}

