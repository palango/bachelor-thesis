\section{Abbruchfehler von Konvektionstermenn}

\subsection{CDS-Verfahren}

\paragraph{Osten}
Entwickelt man die Taylorreihe von $\phi$ um den Punkt $x_P$ und wertet sie anschließend
an den Punkten $x_e$ und $x_E$ aus, so erhält man folgende Gleichungen.

\begin{align}
  \phi_e &= \phi_P + \phi'_P(x_e-x_P)+\frac{1}{2}\phi''_P(x_e-x_P)^2
  +\frac{1}{6}\phi'''_P(x_e-x_P)^3+HOT
  \label{eq:taylor_konv_eP}\\
  \phi_E &= \phi_P + \phi'_P(x_E-x_P)+\frac{1}{2}\phi''_P(x_E-x_P)^2
  +\frac{1}{6}\phi'''_P(x_E-x_P)^3+HOT
  \label{eq:taylor_konv_eE}
\end{align}

Werden die Gleichungen~\eqref{eq:taylor_konv_eE} und \eqref{eq:taylor_konv_eP} nun
voneinander subtrahiert und nach $\phi_e$ umgestellt, so ergibt sich:

\begin{align*}
  \frac{\phi_e}{x_e-x_P} &= \frac{\phi_E}{x_E-x_P} + \frac{\phi_P}{x_e-x_P} -
  \frac{\phi_P}{x_E-x_P} + \frac{1}{2} \phi''_P \left({(x_e-x_P)-(x_E-x_P)}\right)\\
  &+ \frac{1}{6} \phi'''_P \left({(x_e-x_P)^2-(x_E-x_P)^2}\right)\\
  \phi_e &= \phi_E \frac{x_e-x_P}{x_E-x_P} + \phi_P \left({1-\frac{x_e-x_P}{x_E-x_P} }\right)
  + \frac{1}{2} \phi''_P (x_e-x_E)(x_e-x_P)\\
  &+ \frac{1}{6} \phi'''_P \left({(x_e-x_P)^2-(x_E-x_P)^2}\right)(x_e-x_P)\\
  \phi_e &= \phi_E \gamma_e + \phi_P (1-\gamma_e)+ \frac{1}{2} \phi''_P (x_e-x_E)(x_e-x_P)\\
         &+ \frac{1}{6} \phi'''_P \left({(x_e-x_P)^2-(x_E-x_P)^2}\right)(x_e-x_P)
\end{align*}

Der Truncation Error lässt sich damit ablesen zu:

\begin{equation*}
  TE_{e, CDS} =  \frac{1}{2} \phi''_P (x_e-x_E)(x_e-x_P)+ \frac{1}{6} \phi'''_P \left({(x_e-x_P)^2-(x_E-x_P)^2}\right)(x_e-x_P)
\end{equation*}


Hier müssen nun wiederum die auftretenden Ableitungen diskretisiert werden.

\begin{align}
  \phi'_P &= \frac{\phi_E-\phi_W}{x_E-x_W}\\
  \phi''_P &= \frac{1}{(x_e-x_w)} \left({\frac{\phi_E-\phi_P}{x_E-x_P}
  - \frac{\phi_P-\phi_W}{x_P-x_W} }\right) \label{eq:ddphip}\\
  \phi'''_P &= \frac{1}{(x_e-x_w)} \left({
  \frac{1}{(x_E-x_P)} \left({\frac{\phi_{EE}-\phi_P}{x_{EE}-x_P}- \frac{\phi_E-\phi_W}{x_E-x_W} }\right)-
  \frac{1}{(x_P-x_W)} \left({\frac{\phi_E-\phi_W}{x_E-x_W} - \frac{\phi_P-\phi_{WW}}{x_P-x_{WW}} }\right)
  }\right) \label{eq:dddphip}
\end{align}

Der diskretisierte Truncation Error ergibt sich damit zu:

\begin{align}
  TE_{e, CDS} &=  \frac{1}{2} \frac{1}{(x_e-x_w)} \left({\frac{\phi_E-\phi_P}{x_E-x_P}
  - \frac{\phi_P-\phi_W}{x_P-x_W} }\right) (x_e-x_E) (x_e-x_P) \nonumber \\
  &+
 \frac{1}{(x_e-x_w)} \left({
  \frac{1}{(x_E-x_P)} \left({\frac{\phi_{EE}-\phi_P}{x_{EE}-x_P}- \frac{\phi_E-\phi_W}{x_E-x_W} }\right)-
  \frac{1}{(x_P-x_W)} \left({\frac{\phi_E-\phi_W}{x_E-x_W} - \frac{\phi_P-\phi_{WW}}{x_P-x_{WW}} }\right)
  }\right) \nonumber \\
  &\frac{1}{6} \left({(x_e-x_P)^2-(x_E-x_P)^2}\right)(x_e-x_P)
\end{align}

\paragraph{Westen}

Man entwickelt hier wiederum die Taylorreihendarstellungen von $\phi$ in $x_w$ und $x_W$.
Anschließend werden beide Reihen so von einander subtrahiert, dass alle Vorkommen
der ersten Ableitung von $\phi$ verschwinden. Damit ergeben sich folgende Terme:

\begin{align*}
  \phi_w &= \left({1-\frac{x_w-x_P}{x_W-x_P}}\right)\phi_P + \left({\frac{x_w-x_P}{x_W-x_P} }\right) \phi_W
  + \frac{1}{2} \phi''_P \left({(x_w-x_P)-(x_W-x_P)}\right)(x_w-x_P)\\
  &+ \frac{1}{6} \phi'''_P \left({(x_w-x_P)^2-(x_W-x_P)^2}\right)(x_w-x_P)\\
  \phi_w &= \left({1-\frac{x_P-x_w}{x_P-x_W}}\right)\phi_P + \left({\frac{x_P-x_w}{x_P-x_W} }\right) \phi_W
  + \frac{1}{2} \phi''_P \left({x_w-x_W}\right)(x_w-x_P)\\
  &+ \frac{1}{6} \phi'''_P \left({(x_w-x_P)^2-(x_W-x_P)^2}\right)(x_w-x_P)
\end{align*}

Der Abbruchfehler lässt sich nach der Diskretisierung der unbekannten Ableitungen nach Gleichung~\eqref{eq:ddphip}
und \eqref{eq:dddphip} leicht ablesen und ergibt sich zu:

\begin{align}
  TE_{w, CDS} &=  \frac{1}{2} \frac{1}{(x_e-x_w)} \left({\frac{\phi_E-\phi_P}{x_E-x_P}
  - \frac{\phi_P-\phi_W}{x_P-x_W} }\right) \left({x_w-x_W}\right)(x_w-x_P)  \nonumber\\
  &+ \frac{1}{(x_e-x_w)} \left({
  \frac{1}{(x_E-x_P)} \left({\frac{\phi_{EE}-\phi_P}{x_{EE}-x_P}- \frac{\phi_E-\phi_W}{x_E-x_W} }\right)-
  \frac{1}{(x_P-x_W)} \left({\frac{\phi_E-\phi_W}{x_E-x_W} - \frac{\phi_P-\phi_{WW}}{x_P-x_{WW}} }\right)
  }\right) \nonumber\\
  &\frac{1}{6}  \left({(x_w-x_P)^2-(x_W-x_P)^2}\right)(x_w-x_P)
\end{align}

\subsubsection{Äquidistante Gitter}

\paragraph{Osten}

\begin{equation}
  TE_{e, CDS} = -\frac{1}{8} (\phi_E-2\phi_P+\phi_W) - \frac{1}{32}
  (\phi_{EE} - 2\phi_E + 2\phi_W - \phi_{WW})
\end{equation}

\paragraph{Westen}

\begin{equation}
  TE_{w, CDS} = -\frac{1}{8} (\phi_E-2\phi_P+\phi_W) + \frac{1}{32}
  (\phi_{EE} - 2\phi_E + 2\phi_W - \phi_{WW})
\end{equation}

\subsubsection{Randwerte}

Bei Kontrollvolumen am Rand müssen die verwendeten Approximationen angepasst werden.
Am östlichen Rand mit dem Randwert $\phi^{ee}$ entsteht dabei folgender Term:

\begin{equation*}
  \phi'''_P = \frac{1}{(x_e-x_w)} \left({
      \frac{1}{(x_E-x_P)} \left({\frac{\phi^{ee}-\phi_e}{x_{ee}-x_e}- \frac{\phi_E-\phi_W}{x_E-x_W} }\right)-
  \frac{1}{(x_P-x_W)} \left({\frac{\phi_E-\phi_W}{x_E-x_W} - \frac{\phi_P-\phi_{WW}}{x_P-x_{WW}} }\right)
  }\right)
\end{equation*}
Hier muss abschließend $\phi_e$ aus $\phi_E$ und $\phi_P$ interpoliert werden. Letztendlich ergibt
sich für äquidistante Gitter folgender Term:

\begin{equation*}
  TE_{e, CDS} = -\frac{1}{8} (\phi_E-2\phi_P+\phi_W) - \frac{1}{32}
  (2\phi^{ee} - 3\phi_E + 2\phi_W - \phi_{WW})
\end{equation*}

Auf gleiche Art und Weise kann auch $TE_{w, CDS}$ berechnet werden.

\subsection{UDS-Verfahren}

\paragraph{Osten}

Wertet man die Taylorreihe von $\phi$ mit dem Entwicklungspunkt $x_P$ im Punkte $\phi_e$
aus, so ergibt sich:

\begin{equation*}
  \phi_e = \phi_P +(x_e-x_P) \phi'_P + \frac{1}{2} (x_e-x_P)^2 \phi''_P+HOT
\end{equation*}

Da im Falle einer positiven Geschwindigkeit das UDS-Verfahren $\phi_e$ mit $\phi_P$
gleichsetzt, lässt sich der Truncation Error hier direkt ablesen.

\begin{equation*}
  TE_{e, UDS} = (x_e-x_P) \phi'_P + \frac{1}{2} (x_e-x_P)^2 \phi''_P
\end{equation*}

Mit den diskretisierten Ableitungen von oben ergibt sich damit:

\begin{equation}
  TE_{e, UDS} = (x_e-x_P) \frac{\phi_E-\phi_W}{x_E-x_W}+
  \frac{1}{2} \frac{(x_e-x_P)^2}{(x_e-x_w)} \left({\frac{\phi_E-\phi_P}{x_E-x_P}
  - \frac{\phi_P-\phi_W}{x_P-x_W} }\right)
\end{equation}

Für äquidistante Gitter ergibt sich damit:

\begin{equation}
  TE_{e, UDS} = \frac{3}{8} \phi_E-\frac{1}{4} \phi_P - \frac{1}{8} \phi_W
\end{equation}


\paragraph{Westen}

Wertet man die Taylorreihe von $\phi$ mit dem Entwicklungspunkt $x_P$ im Punkte $\phi_w$
aus, so ergibt sich:

\begin{equation*}
  \phi_w = \phi_W +(x_w-x_W) \phi'_W + \frac{1}{2} (x_w-x_W)^2 \phi''_W+HOT
\end{equation*}

Da im Falle einer positiven Geschwindigkeit das UDS-Verfahren $\phi_w$ mit $\phi_W$
gleichsetzt, lässt sich der Truncation Error hier direkt ablesen.

\begin{equation*}
  TE_{w, UDS} = (x_w-x_W) \phi'_W + \frac{1}{2} (x_w-x_W)^2 \phi''_W
\end{equation*}

Hier müssen wiederum die Ableitungen $\phi'_W$ und $\phi''_W$ diskretisiert werden.

\begin{equation}
  TE_{w, UDS} = (x_w-x_W) \frac{\phi_P-\phi_{WW}}{x_P-x_{WW}}+
  \frac{1}{2} \frac{(x_w-x_W)^2}{(x_w-x_{ww})} \left({\frac{\phi_P-\phi_W}{x_P-x_W}
  - \frac{\phi_W-\phi_{WW}}{x_W-x_{WW}} }\right)
\end{equation}

Für äquidistante Gitter ergibt sich damit:

\begin{equation}
  TE_{w, UDS} = \frac{3}{8} \phi_P-\frac{1}{4} \phi_W - \frac{1}{8} \phi_{WW}
\end{equation}


\subsubsection{Randwerte}

Hier ist bei positivem $v$ insbesondere der westliche Rand interessant. Die
Randbedingung sei mit $\phi^w$ bezeichnet. Mit möglichst lokalen Differenzenquotienten
ergeben sich folgende Approximationen der Ableitungen.

\begin{align*}
  \phi'_W &= \frac{\phi_w-\phi^w}{x_w-x_{ww}}\\
  \phi''_W &= \frac{1}{x_W-x_{ww}} \left({\frac{\phi_w-\phi^w}{x_w-x_{ww}}
- \frac{\phi_W-\phi^w}{x_W-x_{ww}} }\right)
\end{align*}

$\phi_w$ wird aus der linearen Interpolation von $\phi_P$ und $\phi_W$ berechnet:

\begin{equation*}
  \phi_w = \phi_W \frac{x_P-x_w}{x_P-x_W} + \phi_P \frac{x_w-x_W}{x_P-x_W}
\end{equation*}

Damit ergeben sich folgende Terme:

\begin{align}
  \phi'_W &= \frac{\phi_W \frac{x_P-x_w}{x_P-x_W} + \phi_P \frac{x_w-x_W}{x_P-x_W}
-\phi^w}{x_w-x_{ww}}\\
  \phi''_W &= \frac{1}{x_W-x_{ww}} \left({\frac{\phi_W \frac{x_P-x_w}{x_P-x_W} + \phi_P \frac{x_w-x_W}{x_P-x_W}
-\phi^w}{x_w-x_{ww}}
- \frac{\phi_W-\phi^w}{x_W-x_{ww}} }\right)
\end{align}


\paragraph{Äquidistante Gitter}

\begin{align*}
  \phi'_W  &= \frac{1}{2\Delta x} \phi_P + \frac{1}{2\Delta x} \phi_W - \phi^w\\
  \phi''_W &= \frac{1}{\Delta x^2} \phi_P -\frac{1}{\Delta x^2} \phi_W
\end{align*}


\subsection{``Flux-Blending''-Verfahren}

Das Flux-Blending-Verfahren setzt sich aus UDS- und CDS-Verfahren zusammen. Beide
Verfahren werden dabei über den Faktor $\beta$ gewichtet.

\begin{equation*}
\phi_e \approx (1-\beta)\phi_e^{UDS} + \beta \phi_e^{CDS} 
\end{equation*}

Aufgrund dessen ist es möglich den Truncation Error des Flux-Blending-Verfahrens aus
den vorangegenagen Betrachtungen zu UDS- und CDS-Verfahren herzuleiten.
Dabei werden Die Abbruchfehler der einzelnen Verfahren ebenso über $\beta$ gewichtet.

\begin{equation}
  TE_{e, Flux} = (1-\beta) TE_{e, UDS} + \beta TE_{e, CDS}
\end{equation}



\section{Truncation Error eines Kontrollvolumens}
\label{sec:Truncation Error eines Kontrollvolumens}

Der Truncation Error für ein Kontrollvolumen setzt sich nun aus den Fehlern von Quell-
und Diffusionstermen zusammen:

\begin{equation*}
  TE = \frac{TE_{source} - TE_e - TE_w}{\Delta x}
\end{equation*}

Für den äquidistanten Fall ergibt sich damit für zentrale Kontrollvolumen der folgende
Truncation Error. Wichtig ist es, hier die durch den Gauß'schen Integralsatz 
entstehenden Vorzeichen mit zu beachten. Weiterhin muss durch $\Delta x$ geteilt werden,
da die gleiche Transformation beim Aufstellen des Gesamtgleichungsystem angewendet wird.

\begin{align}
  TE &= \frac{\frac{\Delta x}{24} \left({f_E-2f_P+f_W}\right)
   +\frac{1}{24\Delta x}\left({
\phi_{EE}-3\phi_E+3\phi_P-\phi_W}\right)
  -\frac{1}{24 \Delta x}\left({
\phi_E-3\phi_P+3\phi_W-\phi_{WW}}\right)}{\Delta x}
\end{align}
