\section{Verifikation der Lösungsprogramme}

Zur Berechnung des Abbruchfehlers wird die Lösung des mit finiten Volumen
diskretisierten Problems benötigt. Dafür wurden Lösungsprogramme
für ein- und zweidimensionale Probleme in Matlab implementiert.
Diese arbeiten mit orthogonalen Gittern und können reine Diffusionsprobleme
oder kombinierte Diffusions- und Konvektionsprobleme lösen.

Vor der Verifizierung des abgeleiteten Indikators für den Abbruchfehler muss daher die
korrekte Funktion der Programme nachgewiesen werden. Dies geschieht wie in Abschnitt
\ref{sec:verifik_allg} beschrieben über den Vergleich der formalen mit der beobachteten
Konvergenzordnung.

Bei der Berechnung der beobachteten Konvergenzordnung muss beachtet werden, dass mit dem
feinsten Gitter begonnen wird und dieses schrittweise ausgedünnt wird. Andernfalls
wird die Konvergenzordnung falsch bestimmt.

Die beobachtete Konvergenzordnung lässt sich neben den berechneten Zahlenwerten auch gut
an der Steigung in einem doppelt-logarithmischen Diagramm ablesen. Entsteht dort eine
Gerade, so bedeutet das, dass die Lösung mit einen festen Koeffizienten konvergiert.

\paragraph{1D Diffussion und CDS, nicht äquidistant}

\begin{table}[h]
  \begin{tabular}{r r l}
  \toprule
  N & Summierter Fehler & Beobachtete Konvergenzordnung \\
  \midrule
  5  & $1,27\cdot10^{-1}$ & \multirow{2}{*}{2,73}\\
  10 & $1.91\cdot10^{-2}$ & \multirow{2}{*}{2,22}\\
  20 & $4.07\cdot10^{-3}$ & \multirow{2}{*}{2,02}\\
  40 & $9.96\cdot10^{-4}$ & \\
  \bottomrule
\end{tabular}
\caption{Daten}
\end{table}

\begin{figure}[h]
  \begin{tikzpicture}
  \begin{loglogaxis}[%xlabel=Anzahl der Kontrollvolumen
    %$N$,ylabel=Gemittelter Fehler,
  width=0.55\textwidth]
  \addplot[color=tud2d,mark=*, very thick] coordinates {
    (5, 1.2662511400e-01)
    (10,1.9068695077e-02)
    (20,4.0741696509e-03)
    (40,9.9564249986e-04)
  };
  \end{loglogaxis}
\end{tikzpicture}
\caption{Fehler}
\end{figure}

%\begin{figure}[ht]
%\begin{floatrow}

%\capbtabbox{%
%\begin{tabular}{r r r}
  %\toprule
  %N & Summierter Fehler & Konvergenzordnung \\
  %\midrule
  %5  & $1,27*10^{-1}$ & \multirow{2}{*}{333}\\
  %10 & $1.91*10^{-1}$ & \multirow{2}{*}{333}\\
  %20 & $4.07*10^{-1}$ & \multirow{2}{*}{333}\\
  %40 & $9.96*10^{-1}$ & \\
  %\bottomrule
%\end{tabular}
%}{%
  %\caption{A table}%
%}
%\ffigbox{%
%\begin{tikzpicture}
%\begin{loglogaxis}[%xlabel=Anzahl der Kontrollvolumen
  %%$N$,ylabel=Gemittelter Fehler,
%width=0.55\textwidth]
%\addplot[color=tud2d,mark=*, very thick] coordinates {
  %(5, 1.2662511400e-01)
  %(10,1.9068695077e-02)
  %(20,4.0741696509e-03)
  %(40,9.9564249986e-04)
%};
%\end{loglogaxis}
%\end{tikzpicture}
%}{%
  %\caption{A figure}%
%}
%\end{floatrow}
%\end{figure}

