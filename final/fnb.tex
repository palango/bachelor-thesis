\documentclass[colorback,longdoc,blackrule,bigchapter,accentcolor=tud2d,11pt]{tudreport}

\usepackage{etex}
\usepackage[ngerman]{babel}
\usepackage[utf8]{inputenc}
\usepackage[T1]{fontenc}


%Seitenränder einstellen
\usepackage{geometry}
\geometry{a4paper,left=25mm,right=20mm, top=1.5cm, bottom=1.5cm}

\usepackage{mathdesign}
%\usepackage[]{subfigure}
\usepackage[]{graphicx}
%\usepackage[]{natbib}
\usepackage[]{amsmath}
%\usepackage{dsfont}
\usepackage{color}
%Zur Darstellung von Code
%\usepackage[ruled,chapter]{algorithm}
%\usepackage[]{algorithmic}
\usepackage{subcaption}
\usepackage{microtype}
\usepackage[ngerman,pdfview=FitH,pdfstartview=FitV]{hyperref}

\setcounter{secnumdepth}{3}
\setcounter{tocdepth}{3}

\usepackage{pgfplots}
\pgfplotsset{compat=newest,
             height=6cm,
             grid=major}

\usepackage{tikz}
\usetikzlibrary{patterns}
\usetikzlibrary{calc,intersections,through,backgrounds}
\usepackage{IEEEtrantools}

\usepackage{multirow}
\usepackage{booktabs}
\usepackage{floatrow}
% Table float box with bottom caption, box width adjusted to content
\newfloatcommand{capbtabbox}{table}[][\FBwidth]

\newcommand{\pder}[2][]{\frac{\partial#1}{\partial#2}}
\newcommand{\pderf}[1]{\frac{\partial f}{\partial#1}}
\newcommand{\pderfs}[1]{\frac{\partial^2 f}{\partial#1}}
% Private Declarations
%\usepackage{psfrag}
%\usepackage{multirow}
%\usepackage{latexsym}
%\usepackage{pstricks,pst-node}
%\usepackage{pst-blur}
%\usepackage{pstricks-add}
%Colors
%\definecolor{Pink}{rgb}{1.,0.75,0.8}
%\definecolor{fond}{RGB}{0,240,240}

\title{Ableitung und Untersuchung des Abbruchfehlerschätzers für mittels
Finite Volumen diskretisierten Navier-Stokes Gleichungen\linebreak[1]}
\subtitle{Derivation and examination of the truncation error for finite volume discretised Navier-Stokes equations\linebreak Bachelor-Thesis | Paul Lange}
\subsubtitle{Betreuer: Dipl.-Ing. U. Falk | Prof. Dr. rer. nat. M. Schäfer}

%Settitlepicture auskommentieren wenn keines gewünscht wird
%\settitlepicture{titlepic}
%\printpicturesize
%Hier wird entweder ein Text für das Institut angewählt oder ein Logo
%\institution{Fachgebiet für \\Numerische Berechnungsverfahren \\im Maschinenbau}
\setinstitutionlogo[height]{bilder/fnb_logo_schriftzug}


% %%%%%%%%%%%%%%%%%%% Hier beginnt das Dokument
\begin{document}
\maketitle

% Verzeichnisse mit römischen Seitenzahlen
\cleardoublepage
\setcounter{page}{1}
\pagenumbering{roman}

\chapter*{Erklärung zur Abschlussarbeit gemäß § 22 Abs. 7 ABP der TU Darmstadt}
Hiermit versichere ich, Paul Lange, die vorliegende Bachelor-Thesis ohne
Hilfe Dritter und nur mit den angegebenen Quellen und Hilfsmitteln angefertigt
zu haben. Alle Stellen, die Quellen entnommen wurden, sind als solche kenntlich gemacht worden.
Diese Arbeit hat in gleicher oder ähnlicher Form noch keiner Prüfungsbehörde vorgelegen.
In der abgegebenen Thesis stimmen die schriftliche und elektronische Fassung überein.
\vspace{3cm}

Darmstadt, \today



\tableofcontents
\listoffigures
%\listoftables

% Eigentlicher Text mit arabischen Seitenzahlen
\cleardoublepage
\pagenumbering{arabic}
\setcounter{page}{1}

\chapter{Einleitung}
\section{Motivation}
Durch steigende Computerleistung sowie Verbesserungen in den numerischen Methoden konnten
sich die simulationsbasierten Produktentwicklungsprozesse einen festen Platz neben
Experimenten erarbeiten. Neben Kostenvorteilen ist es vor allem die Möglichkeit
schnell Aussagen über Anpassungen und Modifikationen an Produkten treffen zu können, die
simulationsbasierten Methoden zum Wachstum verhelfen.

Für die ingenieurmäßige Berechnung von Strömungsproblemen hat sich die Methode der
finiten Volumen auf breitem Gebiet durchgesetzt. Ihr Fokus liegt hier auf der Simulation
der Navier-Stokes Gleichungen, oft auch unter Vernachlässigung kompressibler Strömungseffekte und
anderen zulässigen Vereinfachungen.

Um die Genauigkeit der Ergebnisse zu verbessern werden immer feinere Gebiete aufgelöst,
bei denen auch heutige Computersysteme an ihre Grenzen stoßen. Es ist deshalb notwendig
auf adaptive Methoden zurückzugreifen, die die Auflösung an interessanten Stellen wie
zum Beispiel Grenzschichten oder besonders interessanten Strömungsgebieten selektiv erhöhen können.
Weiterhin sollen Fehler, die durch die numerische Behandlung der kontinuierlichen Strömungsgebiete
entstehen, minimiert werden. Dafür ist es notwendig, Indikatoren für den Fehler zu definieren.

Heutige Methoden schätzen diesen Fehler ab, negieren aber seine Ursachen bei der Diskretisierung
und damit dem Übergang von der kontinuierlichen Differentialgleichung zur diskreten Gleichung.
Hier liegt der Ansatzpunkt der Fehlerbetrachtung über den Abbruchfehler. Dieser wird in der
vorliegenden Arbeit für Transportprobleme hergeleitet und seine Genauigkeit an Beispielen
verifiziert. Weiterhin wird er zur Gitteradaption eingesetzt.


\section{Stand der Forschung}

Der Abbruchfehler als Indikator für die Gitteradaption ist in der Literatur bereits bekannt.
Roy~\cite{roy2} stellt für die eindimensionale Burger-Gleichung mittels
Finiten-Differenzen her und vergleicht anschließend den abbruchfehlerbasierten Fehlerindikator
mit lösungsbasierten Indikatoren wie Gradient oder Krümmung. Weiterhin wird gezeigt,
dass für lineare Probleme der Abbruchfehler
als Quelle des Diskretisierungsfehlers interpretiert werden kann und
eine Gitterverfeinerung basierend auf dem Abbruchfehler
somit die Auflösung an der Quelle des Diskretisierungsfehler vergrößert.
Damit verkleinert sich dann der gesamte Diskretisierungsfehler.

Vor der Gitteradaption muss eine Lösung auf dem Problemgebiet berechnet werden. Diese
Lösung beziehungsweise das Lösungsprogramm muss verifiziert werden. Veluri~\cite{veluri} demonstriert
Methoden zur Verifikation von Programm über den Vergleich von formaler und beobachteter
Konvergenzordnung und stellt außerdem die Methode der konstruierten Lösung vor.
Auch Roache~\cite{roache} stellt diese vor und demonstriert ihre Nützlichkeit bei der Verifikation von
Lösungsprogrammen.

Der Einfluss nicht-orthogonaler Gitter auf die Lösungsgenauigkeit gehen Huang und Prosperetti~\cite{grid_ortho}
ein. Lee und Tsuei~\cite{lee} zeigen die Ableitung des Abbruchfehlers bei Finite-Differenzen Verfahren
auf nicht-orthogonalen Gittern. Die Transformation von Kontrollgebieten der Finite-Elemente Methode
wird von Fath~\cite{fath} gezeigt.

\section{Zielsetzung}

Im ersten Teil der Arbeit werden grundlegende Konzepte und Begriffe der Arbeit erläutert.
Anschließend wird der Abbruchfehler für alle Terme von skalaren Transportgleichungen
abgeleitet. Dabei werden ein- und zweidimensionale Probleme sowie kartesische, orthogonale
und nicht-orthogonale Gitter betrachtet.

Anschließend wird der abgeleitete Abbruchfehlerindikator an konkreten Testfällen verifiziert.
Die Funktionsweise und Genauigkeit der dazu nötigen Lösungsprogramme wird ebenfalls
untersucht. Abschließend wird dann der Abbruchfehler zur lokalen Gitteradaption genutzt
und hierbei mit anderen bekannten Verfahren verglichen.



\cleardoublepage

\cleardoublepage

\chapter{Grundlagen}
\section{Skalare Probleme}

Eine Vielzahl praktisch relevanter Probleme lässt sich durch skalare (partielle)
Differential\-glei\-chungen beschreiben. Anwendungsgebiete reichen hier von Wärmetransportproblemen
über strukturmechanische Probleme (zum BEispiel die Auslenkung von Stäben und Membranen bei elastischem
Werkstoffverhalten) bis hin zu strömungsmechanischen Problemen (zum Beispiel das Geschwindigkeitspotential
wirbelfreier Strömungsfelder).

\subsection{Einfache Feldprobleme}

Einige kontinuumsmechanische Problemstellungen können durch eine Differentialgleichung
der Form

\begin{equation}
  -\frac{\partial}{\partial x_i}\left({\alpha \frac{\partial \phi}{\partial x_i} }\right)=g
  \label{eq:feldproblem_stat}
\end{equation}
 beschrieben werden. Diese muss auf dem gesamten betrachteten Gebiet $V$ gelten, das
vom Rand $S$ begrenzt wird. Bei $\phi=\phi(\mathbf{x})$ handelt es sich um die gesuchte,
skalare Feldgröße. $g=g(\mathbf{x})$ und $\alpha=\alpha(\mathbf{x})$ sind vorgegeben.


Weiterhin müssen auf dem gesamten Rand $S$ Randbedingungen
gegeben sein. Üblich sind hierbei folgende Typen von Randbedingungen:

\begin{itemize}
  \item Dirichletsche Randbedingung: $\quad \phi=\phi_S$
  \item Neumannsche Randbedingung: $\quad \alpha \frac{\partial \phi}{\partial x_i} n_i = b_s $
  \item Cauchysche Randbedingung: $\quad c_S \phi + \alpha \frac{\partial \phi}{\partial x_i} n_i = b_s$
\end{itemize}

Hierbei sind $\phi_S$, $b_S$ und $c_S$ vorgegebene Funktionen auf dem Rand $S$. Die
Komponenten des nach außen gerichteten Normaleneinheitsvektors an $S$ werden mit $n_i$ bezeichnet.

Treten mehrere Randbedingungstypen in einem Problem auf spricht man von gemischten Rand\-wert\-problemen.

Die mit Gleichung~\ref{eq:feldproblem_stat} beschriebenen Probleme beinhalten keine Zeitabhängigkeit
und werden deshalb als stationär bezeichnet.
Im instationären, also zeitabhängigem Fall, erhalten alle Größen neben der Ortsabhängigkeit
eine Zeitabhängigkeit. Die entsprechende Differentialgleichung lautet damit
\begin{equation}
  \frac{\partial \phi}{\partial t}
  -\frac{\partial}{\partial x_i}\left({\alpha \frac{\partial \phi}{\partial x_i} }\right)=g
  \label{eq:feldproblem_instat}
\end{equation}
für die unbekannte Größe $\phi=\phi(\mathbf{x}, t)$.
Für instationäre Probleme muss neben den Rand\-beding\-ungen auch eine Anfangsbedingung
$\phi(\mathbf{x}, t_0) = \phi_0(\mathbf{x})$ gegeben werden.

Im weiteren Verlauf der Arbeit werden nur stationäre Probleme betrachtet. Alle Ergebnisse
lassen sich aber auf instationäre Probleme übertragen.

\subsection{Allgemeine Transportgleichung}
\label{sec:transportgl}

Eine wichtige Problemklasse innerhalb des Maschinenbaus stellen Transportprobleme in
Festkörpern oder Fluiden dar. Bei der transportierten Größe kann es sich dabei beispielsweise
um joulsche Wärme (Wärmetransportprobleme) oder Stoffmengen (Stofftransportprobleme) handeln.

Stationäre Transportprobleme können in Differentialform durch die Gleichung
\begin{equation}
  \frac{\partial}{\partial x_i} \left({\rho v_i \phi
- \alpha \frac{\partial \phi}{\partial x_i} }\right) = f
\label{eq:transportgl}
\end{equation}
beschrieben werden. Die auftretenden Terme werden wie folgt benannt:
\begin{itemize}
  \item Quellterm $\quad f$
  \item Konvektionsterm $\quad\frac{\partial}{\partial x_i} \rho v_i \phi$
  \item Diffusionsterm $\quad\alpha \frac{\partial^2 \phi}{\partial x_i \partial x_i}$
\end{itemize}
Für spezielle Probleme müssen die entsprechenden Parameter angepasst werden
(für nähere Information siehe \cite{num_maschbau}).

Alle Betrachtungen in der vorliegenden Arbeit bauen auf Problemen auf, die durch
Transport\-gleich\-ungen beschrieben werden können.

\section{Navier-Stokes Gleichungen}

Für viele reale Probleme ist es nötig das Verhalten von Fluiden vorherzusagen.
Teilweise sind auch gleichzeitige Stoff- und Wärmetransportvorgänge interessant.
Die in praktisch relevanten Problemen am häufigsten auftretenden Fluide werden Newtonsche Fluide genannt.
Das bedeutet, dass sie linear-viskos und isotrop sind und damit durch den Cauchyschen
Spannungstensor $T$ beschrieben werden können, wobei $p$ für den Druck und $\mu$ für die dynamische Viskosität stehen.
\begin{equation}
  T_{ij} = \mu\left({\frac{\partial v_i}{\partial x_j}
  + \frac{\partial v_j}{\partial x_i}
-\frac{2}{3} \frac{\partial v_k}{\partial x_k} \delta_{ij}}\right)
-p\delta_{ij}
\end{equation}
Die
Erhaltungssätze für Masse, Impuls und innere Energie lauten mit diesem Materialgesetz dann:
\begin{align}
  \frac{\partial p}{\partial t} + \frac{\partial (\rho v_i)}{\partial x_i} &= 0\\
  \frac{\partial (\rho v_i)}{\partial t} + \frac{\partial (\rho v_i v_j)}{\partial x_j} &=
  \frac{\partial}{\partial x_j} \left[{\mu
  \left({\frac{\partial v_i}{\partial x_j}
  +\frac{\partial v_j}{\partial x_i}
  - \frac{2}{3} \frac{\partial v_k}{\partial x_k}\delta_{ij}}\right)}\right]
  -\frac{\partial p}{\partial x_i} + \rho f_i\\
  \frac{\partial (\rho e)}{\partial t} + \frac{\partial (\rho v_i e)}{\partial x_i} &=
  \mu \left[{\frac{\partial v_i}{\partial x_j}
  \left({\frac{\partial v_i}{\partial x_j}
  +\frac{\partial v_j}{\partial x_i}}\right)
  - \frac{2}{3} \left({\frac{\partial v_i}{\partial x_i}}\right)^2}\right]
  -p\frac{\partial v_i}{\partial x_i} +\frac{\partial}{\partial x_i}
  \left({\kappa \frac{\partial T}{\partial x_i}}\right)
  + \rho q
\end{align}
Neben diesen Erhaltungsgleichungen benötigt man weiterhin thermische und kalorische Zustands\-gleich\-ungen.
Diese definieren die thermodynamischen Eigenschaften des Fluids.

Bei
vielen realen Problemen müssen die obigen Gleichungen nicht in der allgemeinen Form gelöst werden, sondern
können durch zusätzliche Annahmen vereinfacht werden.
So wird beispielsweise bei vielen praktischen Anwendungen das Fluid nicht entscheidend komprimiert
und kann deshalb als inkompressibel angesehen werden. Als Kriterium gilt weithin,
das die Machzahl ($Ma = v/a$) kleiner~als~0,3 ist. Beispiele für inkompressible Probleme
sind Strömungen von Flüssigkeiten wie Rohrströmungen oder die Umströmung von Bootsrümpfen.
Auch die Strömung an langsam fliegenden Flugzeugen, zum Beispiel Segelflugzeugen, kann als
inkompressibel modelliert werden.

Ist die Bedingung für Inkompressibilität erfüllt,
vereinfachen sich die obigen Gleichungen für kompressible Fluide und es ergeben sich folgende Gleichungen:
\begin{align}
  \frac{\partial v_i}{\partial x_i} &= 0\\
  \frac{\partial (\rho v_i)}{\partial t} + \frac{\partial (\rho v_i v_j)}{\partial x_j} &=
  \frac{\partial}{\partial x_j} \left[{\mu
  \left({\frac{\partial v_i}{\partial x_j}
  +\frac{\partial v_j}{\partial x_i}}\right)}\right]
  -\frac{\partial p}{\partial x_i} + \rho f_i\\
  \frac{\partial (\rho e)}{\partial t} + \frac{\partial (\rho v_i e)}{\partial x_i} &=
  \mu \frac{\partial v_i}{\partial x_j}
  \left({\frac{\partial v_i}{\partial x_j}
  +\frac{\partial v_j}{\partial x_i}}\right)
  +\frac{\partial}{\partial x_i}
  \left({\kappa \frac{\partial T}{\partial x_i}}\right)
  + \rho q
\end{align}
\clearpage

\section{Numerische Methoden}

Zum Erkenntnisgewinn in den Ingenieurwissenschaften stehen drei grundsätzlichen
Wege zur Verfügung \cite{num_maschbau}. Diese lauten:
\begin{itemize}
  \item Theoretische Methoden
  \item Experimentelle Versuche
  \item Numerische Simulationen
\end{itemize}

Theoretische Methoden, also insbesondere analytische Betrachtungen der das Problem
beschreibenden Gleichungen sind im allgemeinen nur sehr beschränkt möglich. Das heißt das
die Gleichungen, die zur Beschreibung realer Prozesse genutzt werden meist nur für
bestimmte Randbedingungen und unter bestimmten Vereinfachungen überhaupt analytisch lösbar
sind. Da jedoch diese Vereinfachungen in realen Prozessen nicht vernachlässigbar klein sind
sowie die geforderten Randbedingungen nicht erfüllbar oder präzise einhaltbar sind, ist es
für komplexere Problemstellungen unmöglich analytische Lösungen zu finden.

Die Intention bei experimentellen Untersuchungen ist über Versuche an Modellen oder
realen Bauteilen an die benötigten Systemgrößen heranzukommen. Diese Vorgehensweise
bereitet jedoch in vielen Fällen Probleme:
\begin{itemize}
  \item Messungen bestimmter Systemgrößen sind oft schwierig bis unmöglich. Gründe
    dafür können die Dimensionen der Objekte (z.B. Molekulare Prozesse, Weltmeere),
    die Geschwindigkeit der Zustandsänderung (z.B. Explosionen) oder moralische Gründe
    (z.B. Versuche an Mensch und Tieren, Versuche mit Gefahrenstoffen) sein.
  \item Experimente an Modellen lassen nur begrenzte Rückschlüsse auf das reale Objekt
    zu. So lassen sich beispielsweise Erkenntnisse aus dem Windkanal nur teilweise auf
    das reale Automobil oder Flugzeug übernehmen.
  \item Experimente sind oft teuer und zeitraubend. So muss beispielsweise um die Auswirkungen
    einer Änderung zu testen ein neues Modell bzw. Objekt hergestellt werden. In anderen Fällen
    ist das Modell nach dem Versuch zerstört (z.B. Crashtest). Generell ist auch der Betrieb
    von Messeinrichtungen teuer.
\end{itemize}

Aufgrund der Nachteile der oben genannten Verfahren setzt sich im Maschinenbau und der
Naturwissenschaft im allgemeinen die numerische Simulation immer weiter durch. Vorteile
gegenüber den anderen Verfahren liegen auf der Hand:

\begin{itemize}
  \item Simulationen sind meist schneller und kosteneffizienter zu erstellen als experimentelle
    Versuche.
  \item Änderungen am Objekt sowie Parameterstudien zur Optimierung lassen sich leicht
    erstellen.
  \item Die Ergebnisse der Simulation enthalten meist alle problemrelevanten Größen,
    deren Messung im Versuch mit viel Aufwand verbunden wäre.
\end{itemize}

Grundlegend für die Nutzung dieser Vorteile ist jedoch die Zuverlässigkeit der
berechneten Ergebnisse. Deshalb gehen mit der Verbesserung der numerischen Simulationsmethoden
immer auch experimentelle Validierung von Ergebnissen und verwendeten Modellen einher.

\subsection{Numerische Gitter}

Sind das mathematische Modell mit seinen Differentialgleichungen sowie den nötigen
Rand- sowie gegebenenfalls Anfangsbedingungen festgelegt, besteht der nächste Schritt
bei der Anwendung eines numerischen Berechnungsverfahren darin, das kontinuierliche
Gebiet von Raum und ggf. Zeit durch eine endliche Menge von Teilgebieten zu approximieren,
in denen dann der Wert des gesuchten Größe berechnet wird.

Diese Teilgebiete werden im Allgemeinen in Form eines Gitters über den zu untersuchenden
Bereich $V$ gelegt, weshalb dieser Arbeitsschritt auch oft Gittergenerierung genannt wird.

Ein wichtiges Unterscheidungskriterium ist die Anordnung der Gitterpunkte.
Man unterscheidet grundsätzlich zwischen strukturierten und unstrukturierten Gittern.

Bei strukturierten Gittern ist die Lage der Gitterpunkte zueinander festgelegt. Diese
Lagebeziehungen müssen deshalb nicht aufwendig gespeichert werden und die Gittergenerierung
ist schnell und ohne großen Rechenaufwand nötig. Zwar ist es möglich bestimmte Gebiete
aus der Gittergenerierung auszuschließen, eine Anpassung an die Problemgeometrie ist
jedoch nur innerhalb der vorgegebenen Lagebeziehungen möglich.

Bei unstrukturierten Gittern hingegen gibt es keine Vorschriften zu den Lagebeziehungen
der einzelnen Gitterpunkte. Diese müssen deshalb aufwendig gespeichert werden, was
Gittergeneration sowie Lösung komplizierter macht. Andererseits ist es Problemlos möglich
das Gitter an der Problemgeometrie auszurichten oder in bestimmten Gebieten selektiv zu verfeinern.

\subsubsection{Physikalisches und logisches Gebiet}

Oftmals lässt die Geometrie des zu untersuchenden Problems keine Anwendung eines
strukturierten Gitters zu. Nicht in jeden Fall ist es jedoch nötig daraufhin unstrukturierte
Gitter zu verwenden. Vielmehr ist es möglich das physikalische Gebiet des Problems
mittels einer geeigneten mathematischen Transformation auf ein sogenanntes
logisches Gebiet abzubilden, auf welchem dann die Anwendung eines strukturierten Gitters
möglich ist.

Die Abbildung von physikalischem auf logisches Gebiet muss dabei umkehrbar sein, das heißt
die Abbildungsfunktion muss EINEINDEUTIG??? sein. Die Achsen des physikalischen Gebiets
werden im Allgemeinen mit lateinischen Buchstaben bezeichnet ($x$, $y$, $z$), während
für das logische Gebiet griechische Buchstaben verwendet werden ($\xi$, $\eta$, $\theta$).
Beispielhaft soll hier die formale, mathematische Beschreibung anhand eines ebenen
Beispiels gezeigt werden:
\begin{equation}
  f=f(x, y)\qquad \text{mit} \qquad x=x(\xi, \eta),\quad y=y(\xi, \eta)
\end{equation}



\section{Finite Volumen Methode}

Die Finite-Volumen Methoden sind ein numerisches Verfahren zur Lösung von (partiellen)
Differentialgleichungen. Sie sind heute das Standardverfahren zu Lösung von Strömungsproblemen
wie zum Beispiel den Euler- oder Navier-Stokes-Gleichungen aber keineswegs nur darauf beschränkt.
Die charakteristische Eigenschaft der Finite-Volumen Methoden ist ihre Konservativität, also
die Erfüllung der den mathematischen Modellen zugrunde liegenden Erhaltungsprinzipien in den
diskreten Gleichungen.

\subsection{Ableitung der Finite Volumen diskretisierten Transportgleichung}

Ausgangspukt ist die in Abschnitt~\ref{sec:num_gitter} beschriebene Zerlegung des Problemgebiets
in diskrete Teilgebiete. Diese werden bei der Finite-Volumen Methode Kontrollvolumen
genannt (im zweidimensionalen Fall steht Volumen stellvertretend für Fläche,
im Eindimensionalen für Länge).
Für jedes dieser Kontrollvolumen werden die Erhaltungsgleichungen in Integralform formuliert
beziehungsweise durch Integration aus den entsprchenden Diffenrentialgleichungen gewonnen.

Beispielhaft soll das Vorgehen hier anhand der in Abschnitt~\ref{sec:transportgl}
vorgestellten stationären Transportgleichung im zweidimensionalen Fall
($i=1$, $2$) vorgestellt werden.
\begin{equation}
  \frac{\partial}{\partial x_i} \left({\rho v_i \phi
- \alpha \frac{\partial \phi}{\partial x_i} }\right) = f
\label{eq:transp_fvm}
\end{equation}
Durch Integration von Gleichung~\eqref{eq:transp_fvm} über ein Kontrollvolumen $V$ (Gl.~\ref{eq:fvm1})
erhält man bei anschließender Anwendung des Gaußschen Integralsatzes die in Gleichung~\eqref{eq:fvm2} beschriebene Beziehung.
\begin{align}
  \int_V \frac{\partial}{\partial x_i} \left({\rho v_i \phi
- \alpha \frac{\partial \phi}{\partial x_i} }\right) dV &= \int_V f dV \label{eq:fvm1}\\
  \int_S  \left({\rho v_i \phi
- \alpha \frac{\partial \phi}{\partial x_i} }\right) n_i dS&= \int_V f dV \label{eq:fvm2}
\end{align}
Hierbei beschreibt $S$ die Oberfläche des Kontrollvolumens $V$, $dS$ ein Oberflächenelement
und $n_i$ die Komponenten des Einheitsnormalenvektors auf der Oberfläche.

Geht man nun von beliebig geformten Kontrollvolumen zu viereckigen Kontrollvolumen über,
so lässt sich das Integral über die Oberfläche $S$ in Gleichung~\eqref{eq:fvm2} durch eine
Summation über die Integrale der vier Seitenflächen umformen. Hierbei beschreibt $c$ die
einzelnen Seitenflächen und ist bei viereckigen Kontrollvolumen als $c=(e,w,n,s)$ definiert.
\begin{equation}
  \sum_c \int_{S_c} \left(\rho v_i \phi - \alpha \frac{\partial \phi}{\partial x_i}
\right) n_{ci} dS_c = \int_V f dV\label{eq:fvm3}
\end{equation}
Gleichung~\eqref{eq:fvm3} stellt nun die Bilanzgleichung für die konvektiven Flüsse
$F_c^C$ sowie die diffusiven Flüsse $F_c^D$ durch die Seiten des Kontrollvolumens $V$ dar.
\begin{align}
  F_c^C &=  \int_{S_c} \left(\rho v_i \phi \right) n_{ci} dS_c \\
  F_c^D &=  -\int_{S_c} \left(\alpha \frac{\partial \phi}{\partial x_i}\right) n_{ci} dS_c 
\end{align}

\subsection{Integralapproximation}

In dieser Arbeit wird von einer zellenorientierten Variablenanordnung (s. \cite{num_maschbau})
ausgegangen. Damit liegen die gesuchte diskrete Größe im Mittelpunkt jedes Kontrollvolumens.
Die Integrale der konvektiven und diffusiven Flüsse müssen durch diese Variablen
approximiert werden. Dabei ist es zweckmäßig zuerst die Oberflächenintegrale durch Variablen
auf der jeweilen Seite des Kontrollvolumens zu beschrieben und anschließend diese durch die Werte
im Zentrum des Kontrollvolumens zu beschreiben.

Beispielhaft soll hier das Oberflächenintegral
\begin{equation*}
  \int_{S_w} p_i n_{wi}\,dS_w
\end{equation*}
der Westseite $S_w$ betrachtet werden. Hierbei bezeichnet $p_i$ die Komponenten einer allgemeinen
Funktion $\mathbf{p} = (p_1(\mathbf{x}), p_2(\mathbf{x}))$. Die einfachste
Approximationsmöglichkeit besteht darin, den Funktionswert von $\mathbf{p}$ im Mittelpunkt
der Seite zu nutzen.
\begin{equation}
  \int_{S_w} p_i n_{wi}\,dS_w \approx q_w \delta S_w
\end{equation}
Die Normalkomponenten von $\mathbf{p}$ an der Stelle $w$ werden hierbei mit $q_w = p_{wi} n_{wi}$ bezeichnet.
Aufgrund der Nutzung des Funktionswertes im Seitenmittelpunkt wird diese Approximation auch Mittelpunktsregel genannt.
Sie hat eine Ordnung von zwei bezüglich der Seitenlänge $\delta S_w$.

Weitere gängige Approximationen umfassen die Trapezregel oder die Simpsonregel, auf die jedoch
in der vorliegenden Arbeit nicht näher eingegangen wird.

Für die Flüsse $F_c^C$ und $F_c^D$ ergeben sich mit der Mittelpunktsregel folgende Approximationen:
\begin{align}
  F_c^C &\approx \underbrace{\rho v_i n_{ci} \delta S_c}_{\dot{m_c}}  \phi_c \label{eq:disk_konv}\\
  F_c^D &\approx  -\alpha  n_{ci} \delta S_c \left(\frac{\partial \phi}{\partial x_i}\right)_c
\end{align}
wobei $\dot{m_c}$ den Massenstrom durch die Seite $S_c$ bezeichnet.

Im nächsten Schritt müssen nun die Variablenwert $\phi_c$ durch die Variablenwerte in den
Mittelpunkten der Kontrollvolumen ausgedrückt werden.
Zunächst soll jedoch die rechte Seite von Gleichung~\eqref{eq:fvm3} betrachtet werden.
Da der Quellterm $f$ oft nicht analytisch integrierbar ist erfolgt im Allgemeinen eine Approximation
mittels numerischer Integration. Oftmals ist hierbei die Nutzung der Mittelpunktsregel mit der Ordnung zwei
ausreichend. Sie geht von der Annahme aus, das der Wert $f_P$ im Mittelpunkt des Kontrollvolumens
einen Mittelwert über die Werte von $f$ im gesamten Kontrollvolumen darstellt.

Die Auswertung des Volumenintegrals ergibt sich damit im zweidimensionalen Fall zu
\begin{equation}
  \int_V f\,dV \approx f_p\,\delta V
\end{equation}
wobei $\delta V$ das Volumen des Kontrollvolumens beschreibt und im Falle eines orthogonalen Gitters
mittels
\begin{equation}
  \delta V = (x_e - x_w)(y_n-y_s) = \Delta x \Delta y
\end{equation}
berechnet werden kann. Auf den bei dieser Appoximation entstehenden Abbruchfehler wird in
Abschnitt~\ref{sec:Quellterm} genauer eingegegangen.


\subsection{Ableitung konvektiver Fluss}

Die durch die Anwendung der Mittelpunkteregel gewonnene Approximation (Gleichung~\ref{eq:disk_konv}) für konvektive
Flüsse beinhaltet noch den Variablenwert des Seitenmittelpunkts, der jedoch bei
zellenorientierter Variablenanordnung nicht gesucht ist. Deshalb muss dieser Wert $\phi_c$ im
nächsten Schritt durch variablenwerte in den Kontrollvolumenmittelpunkten ausgedrückt werden.
Hierbei ist es immer ratsam möglichst lokale Approximationen zu nutzen, aslo beispielsweise
bei der Approximation von $\phi_e$ die Variablenwerte $\phi_P$ und $\phi_E$ zu benutzen.

Es werden nun einige gängige Approximationen für die Nutzung bei konvektiven Flüssen vorgestellt.
Es handelt sich hierbei genaugenommen um Interpolationen aus den umliegenden Mittelpunkten der Kontrollvolumen.
Die Betrachtungen finden dabei aus Gründen der Übersichtlichkeit im eindimensionalen statt,
können jedoch problemlos in höhere Dimensionen übertragen werden.

\paragraph{Upwind-Verfahren}
Bei Upwind-Verfahren (auch Upwind Differencing Scheme, UDS) wird der Verlauf von $\phi$ durch eine Treppenfunktion approximiert, die
von der Geschwindigkeit $v$ abhängt.

\begin{equation*}
\phi_e=\left\{\begin{array}{cl} \phi_P, & \mbox{falls }v>0\\
\phi_E, & \mbox{falls } v<0\end{array}\right.
\end{equation*}

Das UDS-Verfahren besitzt einen Interpolationsfehler erster Ordnung, liefert jedoch bei
Transportrichtungen, die senkrecht zu den Kontrollvolumenseiten liegen vergleichsweise gute Approximationen.
Außerdem liefert es uneingeschränkt beschränkte Lösungsalgorithmen, was ein großer Vorteil ist.




\paragraph{Zentraldifferenzen-Verfahren}
Das Zentraldifferenzen-Verfahren (auch Central Differencing Scheme, CDS) wird der
Wert $\phi_e$ durch eine lineare Interpolation aus $\phi_P$ und $\phi_E$ gewonnen.

\begin{equation*}
  \phi_e \approx \phi_E\gamma_e + \phi_P (1-\gamma_e)
\end{equation*}
Der Interpolationsfaktor $\gamma_e$ ist hierbei gegeben als:
\begin{equation}
  \gamma_e = \frac{x_e-x_P}{x_E-x_P}
  \label{eq:cds_faktor}
\end{equation}
Das CDS-verfahren besitzt einen Interpolationsfehler zweiter Ordnung. Es neigt jedoch zu
numerischen Oszillationen.

\paragraph{``Flux-Blending''-Verfahren}

Beim Flux-Blending versucht man Vor- und Nachteile von Verfahren höherer und niedrigerer
Ordnung zu verbinden. So hat das UDS-Verfahren einen Interpolationsfehler erster Ordnung,
während das CDS-Verfahren zu numerischen Oszillationen neigt.

Beim Flux-Blending werden nun beide Verfahren gewichtet zusammengeführt. Der Gewichtungsfaktor
wird hierbei $\beta$ genannt. Gilt $\beta = 1$, so ergibt sich das CDS-Verfahren, für
$\beta = 0$ das UDS-Verfahren.

\begin{equation}
  \phi_e \approx (1-\beta)\phi_e^{UDS} + \beta \phi_e^{CDS}
  \label{eq:flux_blending}
\end{equation}



\paragraph{Weitere Verfahren}
Neben UDS und CDS gibt es weitere Verfahren zur Approximation der Randvariablen durch
die Variablen im Zentrum eines Kontrollvolumens.

So wird beim QUICK-verfahren (Quadratic Upwind Interpolation for Convective
Kinematics) ein quadratisches Polynom, welches auf die Flussrichtung abgestimmt ist,
zur Approximation genutzt.



\subsection{Ableitung diffusiver Fluss}

\section{Abbruchfehler}

Der Abbruchfehler ist der Fehler der beim Abschneiden einer unendlichen Summe
und deren Approximation durch eine endliche Summe entsteht.

In der Praxis werden Approximationen, wie die in Abschnitt~\ref{sec:konv_fluss} oder
\ref{sec:dif_fluss} beschriebenen, aus Taylorreihenentwicklungen um die betrachteten
Punkte gewonnen\footnote{Die Herleitungen sind in Kapitel~\ref{chap:herleitung} ausführlich ausgeführt.}. 
Da die Approximation jedoch zwangsläufig endlich sein muss, werden
dabei Glieder der unendlichen Taylorreihenentwicklung abgeschnitten. Der dadruch entstehende
Fehler wird Abbruchfehler (engl. Truncation Error) genannt.
Die restlichen, für die Approximation nicht genutzten Terme können zur Fehlerabschätzung
bei numerischen Berechnungen genutzt werden. Die grundlegenden Zusammenhänge sollen hier an einem einfachen Beispiel gezeigt werden.

Wir nehmen an, die Funktion $f=\sin(x)$ soll zur einfacheren Berechenbarkeit durch
ein Polynom approximiert werden. Anschließend soll ein Indikator für den dabei
entstehenden Fehler gefunden werden ohne dabei auf $f$
selbst zurückzugreifen.
Zuerst wird die Taylorreihe von $\sin(x)$ um den Entwicklungspunkt $x_0 =0$ berechnet:

\begin{equation}
  \sin(x) = \sum_{n=0}^{\infty}(-1)^n \frac{x^{2n+1}}{(2n+1)!} = 
  x-\frac{x^3}{3!} +\frac{x^5}{5!} -\frac{x^7}{7!} +\frac{x^9}{9!}\cdots
 \label{eq:taylor_example}
\end{equation}

Nun wird die Approximation für $\sin(x)$ festgelegt. Diese soll in diesem Beispiel
$x-\frac{x^3}{6} $ sein. In Abbildung~\ref{fig:taylor_example} sieht man, wie sich die Güte der
Approximation mit zunehmenden Abstand von $0$ verschlechtert.

\begin{figure}[h]
\begin{tikzpicture}
\begin{axis}[width=0.8*\textwidth, height=150pt, grid=major]
  \addplot[domain=-2*pi:2*pi, samples=100, color=tud2b, very thick]{sin(deg(x))};
  \addplot[domain=-3:3, samples=100, color=tud9b, very thick]{x - x*x*x/6};
  \legend{$\sin(x)$,$x-\frac{x^3}{6} $}
\end{axis}
\end{tikzpicture}
\centering
\caption{Vergleich von Funktion und Approximation}
 \label{fig:taylor_example}
\end{figure}

Der absolute Fehler kann nun einfach über die Formel $\sin(x) - x + \frac{x^3}{6}$
berechnet werden und entspricht den übrigen Gliedern der Taylorreihe~\eqref{eq:taylor_example}.
Der absolute Fehler entspricht also in diesem einfachen Beispiel dem Abbruchfehler.
Dieser wird nach seinem englischen Namen \textit{Truncation Error} mit $TE$ bezeichnet.
\begin{equation}
  TE = \frac{x^5}{5!} -\frac{x^7}{7!} +\frac{x^9}{9!}\cdots
 \label{eq:taylor_example_te}
\end{equation}
Da der Abbruchfehler selbst wieder eine unendliche Summe darstellt muss zur Auswertung
wieder eine endliche Anzahl von Termen abgetrennt werden. Dies stellt jedoch im allgemeinen
kein Problem dar, da der Einfluss der Terme mit steigendem Exponenten stark abnimmt.

Zur Abschätzund des Fehlers sollen im Beispiel zwei Glieder genutzt werden. Aus
Gleichung~\eqref{eq:taylor_example_te} ergibt sich damit der Fehlerindikator zu
$\frac{x^5}{5!} -\frac{x^7}{7!} $.
\begin{figure}[h]
\begin{tikzpicture}
\begin{axis}[width=0.8*\textwidth, height=150pt, grid=major, legend pos=north west]
  \addplot[domain=-6:6, samples=100, color=tud2b, very thick]{sin(deg(x))-x+x*x*x/6};
  \addplot[domain=-6:6, samples=100, color=tud9b, very thick]{x^5/120-x^7/(120*6*7)};
  \legend{Absoluter Fehler, Fehlerapproximation}
\end{axis}
\end{tikzpicture}
\centering
\caption{Vergleich des absoluten Fehlers mit der Fehlerschätzung}
 \label{fig:taylor_example_te}
\end{figure}
In Abbildung~\ref{fig:taylor_example_te} sieht man die gute Übereinstimmung von absolutem Fehler und Fehlerapproximation um den Entwicklungspunkt. Die Genauigkeit kann durch Hinzunahme weiterer Terme aus der Reihenentwicklung des Abbruchfehlers beliebig verbessert werden.

\section{Verifikation von Ergebnissen}

Die Verifikation von numerischen Simulationen spielt eine wichtige Rolle
für deren weiter Verwendung zur Produktentwicklung und -verbesserung.
Man unterscheidet zwei Arten von Verifikationen: die Code-Verifikation und die Lösungsverifikation
\cite{veluri}.
Bei der Code-Verifikation wird nachgewiesen, dass keine Fehler oder Inkonsistenzen im numerischen
Algorithmus enthalten sind. In der Lösungsverifikation werden die drei Typen von numerischen Fehlern
abgeschätzt: der Rundungsfehler, der Iterationsfehler sowie der Diskretisierungsfehler.

\subsection{Konvergenzordnung}

Die Konvergenzordnung beschreibt den Zusammenhang zwischen Verringerung des Diskretisierungsfehlers
und der Verfeinerung des numerischen Gitters.
Die Überprüfung der Konvergenzordnung ist ein wichtiger Nachweis, da sie nicht nur Aussagen über
die Konvergenz der Lösung treffen lassen sondern auch geprüft werden kann ob der Diskretisierungsfehler
sich mit der erwarteten Geschwindigkeit verringert.

\subsubsection{Formale Konvergenzordnung}

Die Formale Konvergenzordnung ist die Geschwindigkeit mit der sich die diskretisierten Gleichungen
den ursprünglichen (partiellen) Differentialgleichungen annähern. Sie wird durch eine Analyse des Abbruchfehlers
der diskretisierten Gleichung gewonnen.
\begin{equation}
  f''_i = \frac{f_{i+1}-2f_i +f_{i-1}}{\Delta x^2} -\frac{1}{12} f^{IV}_i \Delta x^2 + HOT
\end{equation}
Bei der hier diskretsierten zweiten Ableitung einer Funktion $f$ sieht man das der führende
Term des Abbruchfehlers quadratisch von $\Delta x$ abhängt. Die formale Konvergenzordnung im Bezug
auf $x$ ist somit zwei.

Wichtig ist anzumerken, das die Konvergenzordnungen verschiedener Dimensionen
eines Problems unterschiedlich sein können. So kann die Konvergenzordnung im Beztug auf die Zeit eins sein,
während sie im Raum zwei ist.


\subsubsection{Beobachtete Konvergenzordnung}

Die beobachtetet Konvergenzordnung wird direkt aus den Ergebnissen der Berechnung mit systematisch
verfeinerten Gittern gewonnen. Sollte die berechnete Konvergenzordnung nicht mit der formalen
Konvergenzordnung übereinstimmen ist von Fehlern im Programm oder fehlerhaften numerischen Algorithmen auszugehen.

Um die Konvergenzordnung eines Programms zu berechnen ist es zwingend notwendig die exakte Lösung der
Differentialgleichungen zu kennen. Eine Lösungsmöglichkeit für dieses Problem wird in Abschnitt~\ref{sec:man_sol}
vorgestellt.

Der Diskretisierungsfehler $DE$ ist definiert als Differenz zwischen der exakten Lösung $f_{exakt}$
der Differentialgleichungen und der exakten Lösung $f(\Delta)$ der diskreten Gleichungen 
Da jedoch die exakte Lösung der diskreten Gleichungen, die für verschiedene Gitter unterschiedlich ist,
nicht bekannt ist, wird stattdessen die numerische Lösung verwendet. Hierbei beachtet man dementsprechend
Iterationsfehler und Rundungsfehler nicht. Für eine Gitter $k$ lässt sich somit der Diskretisierungsfehler
über folgende Gleichung beschreiben:
\begin{equation*}
  DE_k=f(\Delta_k) - f_{exakt} = g_p h_k^p + HOT
\end{equation*}
wobei $g_p$ den Koeffizienten des führenden Terms des Abbruchfehlers darstellt.
Bei Vernachlässigung der Terme höherer Ordnung können nun die Diskretisierungsfehler bei
zwei verschiedenen Gittern betrachtet werden.
\begin{align*}
  DE_1 &= f(\Delta_1) - f_{exakt} = g_p h_1^{\tilde{p}}\\
  DE_2 &= f(\Delta_2) - f_{exakt} = g_p h_2^{\tilde{p}}
\end{align*}
Da die exakte Lösung der Differentalgleichung bekannt ist kann die beobachtete Konvergenzordnung
$\tilde{p}$ aus den beiden Gleichungen berechnet werden.

\begin{equation}
  \tilde{p}=\frac{\ln \left(\frac{DE_2}{DE_1}\right)}{\ln \left(\frac{h_2}{h_1}\right)}
\end{equation}


\subsection{Konstruierte Lösungen}
\label{sec:man_sol}

Zur Überprüfung der Konvergenzordnung ist es nötig die exakte Lösung der Differentialgleichung zu
kennen. Traditionell sind solche Lösungen nur für einfache Probleme oder bestimmte Randbedingungen
und Vereinfachungen bekannt. Für komplexe Gleichungen mit komplexen Modellen, komplizierten Geometrien
oder Nichtlinearitäten sind jedoch selten Lösungen bekannt. Mit Hilfe der Methode der konstruierten
Lösungen (engl. Manufactured Solution) lassen sich exakte Lösungen für Verifikationszwecke erstellen.

Die Methode der konstruierten Lösungen bietet viele Vorteile. So kann sie neben der Finite-Volumen Methode auch
für Finite-Elemente- oder Finite-Differenzen-Verfahren genutzt werden. Außerdem ist sie auch bei nichtlinearen
oder mehreren Gleichungen (z.B. Navier-Stokes-Gleichungen) anwendbar. Weiterhin wurde nachgewiesen~\cite{roache_book} das sich
sehr empfindliche Fehler erkennen lassen.

Das Vorgehen bei der Konstruktion einer Lösung soll anhand einer stationären, eindimensionalen Transportgleichung
mit Diffusions- und Konvektionstermem demonstriert werden.
\begin{equation*}
  \frac{\partial}{\partial x} \left({\rho v \phi
- \alpha \frac{\partial \phi}{\partial x} }\right) = f
\end{equation*}
Zuerst wird eine Lösung für $\phi$ gewählt und deren Ableitungen berechnet.
\begin{align*}
  \phi(x) &= A + \sin(Bx)\\
  \phi'(x) &= B \cos(Bx)\\
  \phi''(x) &= -B^2\sin(Bx)
\end{align*}
Nun führen wir den zusätzlichen Quellterm $f_{ad}$ ein und setzen $\phi(x)$ so wie die Ableitungen ein.
\begin{align*}
  \rho v \pder[\phi]{x} - \alpha\frac{\partial^2 \phi}{\partial x^2}-f&=f_{ad}\\
  \rho v B \cos(Bx) + \alpha B^2 \sin(Bx) - f &= f_{ad}
\end{align*}
Wird nun $f_{ad}$ in die Differentialgleichung zurück substituiert ergibt sich die konstruierte Lösung.
\begin{equation}
   \frac{\partial}{\partial x} \left({\rho v \phi
- \alpha \frac{\partial \phi}{\partial x} }\right)= \rho v B \cos(Bx) + \alpha B^2 \sin(Bx)
\end{equation}

\paragraph{Randbedingungen}

Um die Konstanten $A$ und $B$ in der konstruierten Lösung bestimmen zu können,
müssen konkrete Randbedingungen für das Problem festgelegt werden. Beispielhaft
soll das hier mit den Randwerten $\phi(0) = 1$ und $\phi(1) = 1$ geschehen.
Damit ergeben sich die Konstanten zu $A = 1$ sowie $B = \pi$. Die 
konstruierte Lösung $\phi(x)$ hat damit die folgende Form und der Quellterm $f$
ergibt sich zu:
\begin{equation*}
  \phi(x) = 1 + \sin(\pi x)
\end{equation*}
\begin{equation*}
  f=\rho v \cos(\pi x) + \alpha \pi^2 \sin(\pi x)
  \label{eq:quellterm}
\end{equation*}


\paragraph{Eigenschaften guter konstruierter Lösungen}

Bei der Wahl der Lösung sollten einige Kriterien eingehalten werden. So sollte die gewählte Lösung
aus glatten, analytischen Funktionen mit glatten Ableitungen bestehen. Die Bedingung der glatten Ableitungen
erlaubt es auch auf relativ groben Gittern die Konvergenzordnung nachzuweisen.
Aufgrund der genannten Kriterien werden für konstruierte Lösungen oft trigonometrische und Exponentialfunktionen benutzt.
Die haben den Vorteil unendlich viele glatte Ableitungen zu haben. Trigonometrische Funktionen lassen
sich zudem leicht in ihrem Wertebereich anpassen.

Weiterhin sollte, auch wenn von der konstruierten Lösung keine physikalische Realität erwartet wird, möglichst
erreichbare physikalische Zustände dargestellt werden. Werden beispielsweise Schallgeschwindigkeiten aus
aus Temperaturen berechnet, so setzt diese Berechnung positive Temperaturen voraus. Die konstruierte Lösung sollte dies wiederspiegeln
und positive Temperaturen zurückgeben.

Die konstruierte Lösung sollte außerdem so gewählt sein, dass alle Terme die gleiche Größenordnung haben
und kein Term von einem anderen dominiert wird. Beispielsweise sollten diffusive als auch konvektive Terme
möglichst gleich stark in die Lösung einfließen, sodass beide getestet werden.

\cleardoublepage

\chapter{Ableitung des Abbruchfehlers}
\label{chap:herleitung}
Um den Abbruchfehler eines Kontrollvolumens zu bestimmen, werden die
Abbruchfehler aller zu betrachtenden Terme addiert. Nach Gleichung~\eqref{eq:fvm3}
müssen demnach die Abbruchfehler für diffusive und konvektive Flüsse sowie
für Quellterme bestimmt werden.


\section{Abbruchfehler für Quellterme}
\label{sec:Quellterm}

\subsection{Eindimensionale Probleme}
\label{sec:source1d}
Die Taylorreihenentwicklung einer Funktion $f(x)$ um den Punkt $x_0$
liefert:
\begin{equation*}
  f(x) = f(x_0) + f'(x_0)(x-x_0) + \frac{1}{2} f''(x_0)(x-x_0)^2 + HOT
\end{equation*}
Nach Integration im Intervall $[a, b]$ ergibt sich:
\begin{align*}
  \int_a^b f(x) dx &= \int_a^b f(x_0) dx + \int_a^b f'(x_0)(x-x_0) dx
+ \int_a^b \frac{1}{2} f''(x_0)(x-x_0)^2 dx + HOT\\
&= f(x_0) (b-a) + f'(x_0) \frac{(b-x_0)^2-(a-x_0)^2}{2}
+ \frac{1}{2} f''(x_0) \frac{(b-x_0)^3-(a-x_0)^3}{3} +HOT
\end{align*}
Die Anwendung auf ein eindimensionales Kontrollvolumen (siehe Abbildung~\ref{fig:kv1d}) mit den Randpunkten $x_e$ und
$x_w$ sowie dem Mittelpunkt $x_P$ ergibt sich damit die folgende Formel:
\begin{equation*}
  \int_{x_w}^{x_e} f(x)dx = f_P(x_e-x_w)
  + \frac{1}{2} f'_P \underbrace{\left[{(x_e-x_P)^2-(x_w-x_P)^2}\right]}_{=0}
+ \frac{1}{6} f''_P \left[{{(x_e-x_P)}^3-{(x_w-x_P)}^3}\right] + HOT
\end{equation*}
Der sich der ersten Ableitung von $f$ anschließende Term wird immer Null, da nach
Definition der Finite-Volumen Methode der Mittelpunkt eines Kontrollvolumens immer
zwischen seinen Grenzen liegt.
Die verbleibende zweite Ableitung wird nun durch einen Differenzenquotienten
beschrieben.
\begin{equation}
  \label{eq:diskretisierung_f''P}
  f''_P = \frac{1}{x_e-x_w}\left(\frac{f_E-f_P}{x_E-x_P}-\frac{f_P-f_W}{x_P-x_W}\right)
\end{equation}
Der Truncation Error der Diskretisierung des Quelltermes lässt sich damit über die
folgende Gleichung beschreiben.
\begin{equation}
  TE_{source,1D} =
\frac{1}{6(x_e-x_w)}\left(\frac{f_E-f_P}{x_E-x_P}-\frac{f_P-f_W}{x_P-x_W}\right)
\left[{{(x_e-x_P)}^3-{(x_w-x_P)}^3}\right] + HOT\label{eq:te_source1}
\end{equation}

\begin{figure}[bt]

\begin{tikzpicture}[scale=0.8]
  %\draw[->] (6,0) -- (6.5,0) node[right] {$x$} coordinate(x axis);
  %\draw[->] (0,6) -- (0,6.5) node[above] {$y$} coordinate(y axis);
  \fill[tud0a] (2,2) rectangle +(2,2);
  \draw[step=2cm, thick] (0, 2) grid (6, 4);

  \fill (3, 3)  circle[radius=1.5pt];
  \node (N) at (3,3) [label=above:$P$]{};

  \fill (5, 3)  circle[radius=1.5pt];
  \node (N) at (5,3) [label=above:$E$]{};

  \fill (1, 3)  circle[radius=1.5pt];
  \node (N) at (1,3) [label=above:$W$]{};

  %\fill (7, 3)  circle[radius=1.5pt];
  %\node (N) at (7,3) [label=right:$EE$]{};

  %\fill (-1, 3)  circle[radius=1.5pt];
  %\node (N) at (-1,3) [label=left:$WW$]{};

  \fill (4, 3)  circle[radius=1.5pt];
  \node (N) at (3.84,3) [label={right=0pt}:$e$]{};

  \fill (2, 3)  circle[radius=1.5pt];
  \node (N) at (2.16,3) [label={left=0pt}:$w$]{};

  %\fill (6, 3)  circle[radius=1.5pt];
  %\node (N) at (6,3) [label=right:$ee$]{};

  %\fill (0, 3)  circle[radius=1.5pt];
  %\node (N) at (0,3) [label=left:$ww$]{};
\end{tikzpicture}

\centering
\caption{Allgemeines Kontrollvolumen im eindimensionalen Fall}
\label{fig:kv1d}
\end{figure}

\subsubsection{Kontrollvolumen am Rand}

Bei Kontrollvolumen am Rand sind nicht mehr alle in Gleichung~\eqref{eq:te_source1}
genutzten Werte gegeben. Der genutzte Differenzenquotient muss demnach angepasst werden.
Da man sich dabei auch von der Form des Zentraldifferenzenquotienten entfernt,
sinkt gleichzeitig die Genauigkeit in den Randgebieten.

Die Anpassung soll beispielhaft am westlichen Rand gezeigt werden. Der Differenzenquotient
ergibt sich bei Benutzung des mit $f^w$ bezeichneten Randwerts, der beispielsweise als
dirichletsche Randbedingung gegeben sein kann, zu:
\begin{equation}
  f''_P = \frac{1}{x_e-x_w}\left(\frac{f_E-f_P}{x_E-x_P}-\frac{f_P-f^w}{x_P-x_w}\right)
\end{equation}
Eingesetzt ergibt sich damit folgende Formel für den Abbruchfehler des Quellterms
für Kontrollvolumen am Rand.
\begin{equation}
  TE_{source, W-Rand} =
\frac{1}{6\,(x_e-x_w)}\left(\frac{f_E-f_P}{x_E-x_P}-\frac{f_P-f^w}{x_P-x_w}\right)
  \left[{{(x_e-x_P)}^3-{(x_w-x_P)}^3}\right] + HOT
\end{equation}
Mit der gleichen Vorgehensweise kann auch der der östliche Rand behandelt werden. Darauf
sei jedoch hier nicht weiter eingegangen.



\subsection{Zweidimensionale Probleme}


Bei gleichem Vorgehen wie in Abschnitt~\ref{sec:source1d} beschrieben, ergibt sich für
eine zweidimensionale Funktion $f(x, y)$ in einem in Abbildung~\ref{fig:kv2d} 
dargestelltem Kontrollvolumen der folgende Ausdruck:
\begin{equation}
  \begin{IEEEeqnarraybox}[][c]{rCl}
  \int_{y_s}^{y_n}\int_{x_w}^{x_e} f dx dy &=& f_P (x_e-x_w)(y_n-y_s)\\
  & &+ \pderf{x} \underbrace{\frac{(x_e-x_P)^2 - (x_w-x_P)^2}{2}}_{=0} (y_n-y_s)\\
  && + \pderf{y} \underbrace{\frac{(y_n-y_P)^2-(y_s-y_P)^2}{2}}_{=0} (x_e-x_w) \\
  & &+ \frac{1}{2} \pderfs{x^2}\frac{(x_e-x_P)^3 - (x_w-x_P)^3}{3} (y_n-y_s)\\
  &&+ \frac{1}{2} \pderfs{y^2} \frac{(y_n-y_P)^3-(y_s-y_P)^3}{3} (x_e-x_w) \\
  & &+ \pderfs{x\partial y} \underbrace{\frac{(x_e-x_P)^2 - (x_w-x_P)^2}{2}}_{=0} \cdot
  \underbrace{\frac{(y_n-y_P)^2-(y_s-y_P)^2}{2}}_{=0} + HOT
\end{IEEEeqnarraybox}
\end{equation}
Die sich den ersten Ableitungen anschließenden Terme ergeben bei Auswertung wiederum Null
und müssen deshalb nicht weiter beachtet werden.
Der Wert von $f$ ist im Mittelpunkt des Kontrollvolumens bekannt, nicht aber die auftretenden
Ableitungen von $f$. Diese müssen deshalb diskretisiert werden. Man kann hierbei auf
Gleichung~\eqref{eq:diskretisierung_f''P} zurückgreifen und ebenfalls für die $y$-Richtung anpassen.
Der Abbruchfehler ergibt sich damit zu:
\begin{align}
  \begin{split}
  TE_{source, 2D} &=                \frac{1}{2}
  \left[{\frac{f_E-f_P}{(x_E-x_P)(x_e-x_w)}-\frac{f_P-f_W}{(x_P-x_W)(x_e-x_w)}  }\right]
  \frac{(x_e-x_P)^3 - (x_w-x_P)^3}{3} (y_n-y_s)\\
  &+ \frac{1}{2}
  \left[{\frac{f_N-f_P}{(y_N-y_P)(y_n-y_s)}-\frac{f_P-f_S}{(y_P-y_S)(y_n-y_s)}  }\right]
  \frac{(y_n-y_P)^3-(y_s-y_P)^3}{3} (x_e-x_w)+HOT \\
  \end{split}
\end{align}
Die Betrachtung von Randkontrollvolumen erfolgt äquivalent zum Eindimensionalen. Es müssen
jedoch auch die Fälle für nördliche und südliche Ränder, als auch die Ecken betrachtet werden.
\begin{figure}[hbt]
\begin{tikzpicture}[scale=1.1]
  %\draw[->] (6,0) -- (6.5,0) node[right] {$x$} coordinate(x axis);
  %\draw[->] (0,6) -- (0,6.5) node[above] {$y$} coordinate(y axis);
  \fill[tud0a] (2,2) rectangle +(2,2);
  \draw[step=2cm, thick] (-2.00001, 0) grid (8, 6);

  \fill (3, 3)  circle[radius=1.5pt];
  \node (N) at (3,3) [label=above:$P$]{};

  \fill (3, 5)  circle[radius=1.5pt];
  \node (N) at (3,5) [label=above:$N$]{};

  \fill (3, 1)  circle[radius=1.5pt];
  \node (N) at (3,1) [label=below:$S$]{};

  \fill (5, 3)  circle[radius=1.5pt];
  \node (N) at (5,3) [label=right:$E$]{};

  \fill (1, 3)  circle[radius=1.5pt];
  \node (N) at (1,3) [label=left:$W$]{};

  \fill (7, 3)  circle[radius=1.5pt];
  \node (N) at (7,3) [label=right:$EE$]{};

  \fill (-1, 3)  circle[radius=1.5pt];
  \node (N) at (-1,3) [label=left:$WW$]{};

  \fill (5, 5)  circle[radius=1.5pt];
  \node (N) at (5,5) [label={above right}:$NE$]{};
  \fill (5, 1)  circle[radius=1.5pt];
  \node (N) at (5,1) [label={below right}:$SE$]{};
  \fill (1, 1)  circle[radius=1.5pt];
  \node (N) at (1,1) [label={below left}:$SW$]{};
  \fill (1, 5)  circle[radius=1.5pt];
  \node (N) at (1,5) [label={above left}:$NW$]{};

  \fill (4, 3)  circle[radius=1.5pt];
  \node (N) at (4,3) [label={right=0pt}:$e$]{};

  \fill (3, 2)  circle[radius=1.5pt];
  \node (N) at (3,2) [label={below=0pt}:$s$]{};

  \fill (2, 3)  circle[radius=1.5pt];
  \node (N) at (2,3) [label={left=0pt}:$w$]{};

  \fill (3, 4)  circle[radius=1.5pt];
  \node (N) at (3,4) [label={above=0pt}:$n$]{};

  \fill (6, 3)  circle[radius=1.5pt];
  \node (N) at (6,3) [label=right:$ee$]{};

  \fill (0, 3)  circle[radius=1.5pt];
  \node (N) at (0,3) [label=left:$ww$]{};


\end{tikzpicture}
\centering
\caption{Allgemeines Kontrollvolumen im zweidimensionalen Fall}
\label{fig:kv2d}
\end{figure}


\subsection{Äquidistante Gitter}

Für äquidistante Gitter vereinfacht sich der Terme des Abbruchfehlers und es ergeben sich
für ein- und zweidimensionale Fälle die folgenden Gleichungen.

\begin{align*}
  TE_{source,\ 1D} &= \frac{\Delta x}{24} \left({f_E-2f_P+f_W}\right) + HOT\\
  TE_{source,\ 2D} &= \frac{1}{24} \Delta x \Delta y \left[{f_E+f_W+f_N+f_S - 4f_P} \right] + HOT
\end{align*}

\newpage

\section{Abbruchfehler von Diffusionstermen}
\label{sec:Diffusionsterme}

\subsection{Eindimensionale Probleme}

Um den Abbruchfehler von Diffusionstermen zu bestimmen werden die Differenzenquotienten
für die Ableitungen erster Ordnung hergeleitet. Dazu werden die Taylorreihendarstellungen der bekannten
Werte in der unmittelbaren Umgebung genutzt. Beispielhaft soll das anhand des östlichen
Randes des Kontrollvolumens geschehen.

Zuerst entwickeln wir die Taylordarstellungen vom Punkt $x_e$ aus in Richtung der anliegenden
Kontrollvolumenmittelpunkte $\phi_E$ und $\phi_P$.
\begin{align}
  \phi_E &= \phi_e + \phi'_e(x_E-x_e)+\frac{1}{2}\phi''_e(x_E-x_e)^2
  +\frac{1}{6}\phi'''_e(x_E-x_e)^3+HOT
  \label{eq:taylor_eE}\\
  \phi_P &= \phi_e + \phi'_e(x_P-x_e)+\frac{1}{2}\phi''_e(x_P-x_e)^2
  +\frac{1}{6}\phi'''_e(x_P-x_e)^3+HOT
  \label{eq:taylor_eP}
\end{align}
Wird nun Gleichung~\eqref{eq:taylor_eP} von Gleichung~\eqref{eq:taylor_eE} subtrahiert. Es ergibt sich:
\begin{equation*}
  \phi_E-\phi_P=\phi'_e(x_E-x_P)+
  \frac{1}{2}\phi''_e\left[{{(x_E-x_e)}^2-{(x_P-x_e)}^2}\right]+
  \frac{1}{6}\phi'''_e\left[{{(x_E-x_e)}^3-{(x_P-x_e)}^3}\right]+HOT
\end{equation*}
Nach Umstellen ergibt sich daraus für die Ableitung $\phi'_e$ der folgende Term, aus dem bereits
der Abbruchfehler abgelesen werden kann.
\begin{equation}
  \phi'_e = \frac{\phi_E-\phi_P}{x_E-x_P}+\frac{1}{2}\phi''_e
\left({\frac{{(x_P-x_e)}^2-{(x_E-x_e)}^2}{x_E-x_P}}\right)+
\frac{1}{6} \phi'''_e \left({\frac{{(x_P-x_e)}^3-{(x_E-x_e)}^3}{(x_E-x_P)}}\right)+HOT \label{eq:te_dif_e}
\end{equation}
Die hier auftretenden Ableitungen $\phi''_e$ und $\phi'''_e$ sind nicht bekannt und
müssen diskretisiert werden. Bei der Nutzung von
möglichst lokalen Differenzenquotienten ergeben sich die folgenden Approximationen der Ableitungen.
Die genutzten Punkte sind in Abbildung~\ref{fig:kv1d} ersichtlich.
\begin{align*}
  \phi''_e &= \frac{1}{(x_E-x_P)}\left({
\frac{\phi_{EE}-\phi_P}{x_{EE}-x_P}-\frac{\phi_E-\phi_W}{x_E-x_W}}\right)\\
\phi'''_e &= \frac{1}{(x_E-x_P)}\left({
\frac{1}{(x_{ee}-x_e)}
\left({\frac{\phi_{EE}-\phi_E}{x_{EE}-x_E}-\frac{\phi_E-\phi_P}{x_E-x_P} }\right)
-\frac{1}{(x_e-x_w)}
\left({\frac{\phi_E-\phi_P}{x_E-x_P} - \frac{\phi_P-\phi_W}{x_P-x_W}  }\right)
}\right)
\end{align*}
Der Abbruchfehler des Diffusionsterms an der östlichen Grenze eines Kontrollvolumens
lässt sich damit über folgende Gleichung beschreiben.
\begin{align}
\begin{split}
    \begin{IEEEeqnarraybox}[][c]{rCl}
      {TE}_e &=& \frac{1}{2 (x_E-x_P)}\left({
\frac{\phi_{EE}-\phi_P}{x_{EE}-x_P}-\frac{\phi_E-\phi_W}{x_E-x_W}}\right) \left({\frac{{(x_P-x_e)}^2-{(x_E-x_e)}^2}{x_E-x_P}}\right)\\
&&+
\left({
\frac{1}{(x_{ee}-x_e)}
\left({\frac{\phi_{EE}-\phi_E}{x_{EE}-x_E}-\frac{\phi_E-\phi_P}{x_E-x_P} }\right)
-\frac{1}{(x_e-x_w)}
\left({\frac{\phi_E-\phi_P}{x_E-x_P} - \frac{\phi_P-\phi_W}{x_P-x_W}  }\right)
}\right)\\
&&\frac{1}{6(x_E-x_P)}\left({\frac{{(x_P-x_e)}^3-{(x_E-x_e)}^3}{(x_E-x_P)}}\right)
+HOT
    \end{IEEEeqnarraybox}
\end{split}
\end{align}
Der Abbruchfehler für Diffusionsterme im Westen erfolgt äquivalent zur oben
gezeigten Herleitung im Osten. Es ergibt sich:
\begin{align}
  \begin{split}
    \begin{IEEEeqnarraybox}[][c]{rCl}
      TE_w &=& \frac{1}{2 (x_P-x_W)} \left({
\frac{\phi_{E}-\phi_W}{x_{E}-x_W}-\frac{\phi_P-\phi_{WW}}{x_P-x_{WW}}}\right)
  \left({\frac{{(x_W-x_w)}^2-{(x_P-x_w)}^2}{x_P-x_W}}\right)\\
  &&+
\left({
\frac{1}{(x_e-x_w)}
\left({\frac{\phi_E-\phi_P}{x_E-x_P}-\frac{\phi_P-\phi_W}{x_P-x_W} }\right)
-\frac{1}{(x_w-x_{ww})}
\left({\frac{\phi_P-\phi_W}{x_P-x_W} - \frac{\phi_W-\phi_{WW}}{x_W-x_{WW}}  }\right)
}\right)\\
&&\frac{1}{6(x_P-x_W)}\left({\frac{{(x_W-x_w)}^3-{(x_P-x_w)}^3}{(x_P-x_W)}}\right)
  +HOT
    \end{IEEEeqnarraybox}
\end{split}
\end{align}

%\subsection{Äquidistante Gitter}

%Für äquidistante Gitter vereinfache sich die Terme des Truncation Error. So löschen
%sich beispielsweise die quadratischen Terme gegenseitig aus. Es ergeben sich die
%folgenden Fehler:

%\begin{align*}
  %{TE}_e &= \left({
%\frac{1}{6\Delta x^2}
%\left({\frac{\phi_{EE}-\phi_E}{\Delta x}-\frac{\phi_E-\phi_P}{\Delta x} }\right)
%-\frac{1}{6\Delta x^2}
%\left({\frac{\phi_E-\phi_P}{\Delta x} - \frac{\phi_P-\phi_W}{\Delta x}  }\right)
%}\right)\left({-\frac{\Delta x^2}{4} }\right)+HOT\\
%&= -\frac{1}{24\Delta x}\left({
%\phi_{EE}-3\phi_E+3\phi_P-\phi_W}\right)+HOT
%\end{align*}

%\begin{align*}
  %TE_w &=\left({
%\frac{1}{6 \Delta x^2}
%\left({\frac{\phi_E-\phi_P}{\Delta x}-\frac{\phi_P-\phi_W}{\Delta x} }\right)
%-\frac{1}{6\Delta x^2}
%\left({\frac{\phi_P-\phi_W}{\Delta x} - \frac{\phi_W-\phi_{WW}}{\Delta x}  }\right)
%}\right)
%\left({-\frac{\Delta x^2}{4} }\right)+HOT\\
%&= -\frac{1}{24 \Delta x}\left({
%\phi_E-3\phi_P+3\phi_W-\phi_{WW}}\right)+HOT
%\end{align*}


\subsubsection{Kontrollvolumen am Rand}
\label{sec:te_dif_rand}

Kontrollvolumen, die in der Nähe des Randes des Problemgebietes liegen müssen wie schon beim
Abbruchfehler von Quelltermen besonders behandelt werden. Hierbei muss unterschieden werden ob es sich
bei der zu approximierenden Ableitung um die Ableitung genau am Rand oder weiter im Inneren des
Problemgebiets handelt.

Muss die Ableitung genau am Rand approximiert werden, so muss der in Gleichung~\ref{eq:te_dif_e} hergeleitete
Zentraldifferenzenquotient durch einen einseitigen Differenzenquotient ersetzt werden. Je nachdem ob das
betrachtete Volumen am Ost- oder Westrand liegt, müssen Vorwärts- oder Rückwärtsdifferenzenquotienten genutzt werden.
\begin{figure}[hb]
\begin{tikzpicture}[scale=1.1]
  %\draw[->] (6,0) -- (6.5,0) node[right] {$x$} coordinate(x axis);
  %\draw[->] (0,6) -- (0,6.5) node[above] {$y$} coordinate(y axis);
  \fill[tud0a] (2,2) rectangle +(2,2);
  \draw[step=2cm, thick] (2, 2) grid (8, 4);

  \filldraw[pattern=north east lines, thick] (1,2) rectangle +(1,2);

  \fill (3, 3)  circle[radius=1.5pt];
  \node (N) at (3,3) [label=above:$P$]{};

  \fill (5, 3)  circle[radius=1.5pt];
  \node (N) at (5,3) [label=right:$E$]{};


  \fill (7, 3)  circle[radius=1.5pt];
  \node (N) at (7,3) [label=right:$EE$]{};


  \fill (4, 3)  circle[radius=1.5pt];
  \node (N) at (4,3) [label={right=0pt}:$e$]{};

  \fill (2, 3)  circle[radius=1.5pt];
  \node (N) at (2,3) [label={right=0pt}:$w$]{};

  \fill (6, 3)  circle[radius=1.5pt];
  \node (N) at (6,3) [label=right:$ee$]{};

\end{tikzpicture}

\centering
\caption{Randkontrollvolumen im eindimensionalen Fall}
\label{fig:kv1d_rand}
\end{figure}
Das Vorgehen soll beispielhaft für die Ableitung $\phi'_w$ für das in Abbildung~\ref{fig:kv1d_rand} markierte Kontrollvolumen
vorgestellt werden.
Über die Randbedingung ist der Wert $\phi^w$ bekannt. Nun wird $\phi_P$ als Taylorreihe mit Entwicklungspunkt
$x_w$ dargestellt.
\begin{equation*}
  \phi_P = \phi^w + \phi'_w (x_P-x_w) + \frac{1}{2} \phi''_w (x_P-x_w)^2 + HOT
\end{equation*}
Nach Umstellen ergibt sich für $\phi'_w$ der folgende Wert:
\begin{equation}
  \phi'_w = \frac{\phi_P-\phi^w}{x_p-x_w} -\frac{1}{2} \phi''_w(x_p-x_w) + HOT
\end{equation}
Hier lässt sich ebenfalls der Abbruchfehler ablesen. Mit der Approximation der Ableitung
ergibt sich:
\begin{equation}
  TE_{diff, konv, Rand} = -\frac{1}{2} \left({\frac{\phi_e-\phi^w}{x_e-x_w}-
  \frac{\phi_P-\phi^w}{x_p-x_w} }\right)
\end{equation}
Wird die Ableitung nicht direkt auf dem Rand betrachtet, kann der Zentraldifferenzenquotient
aus Gleichung~\ref{eq:te_dif_e} genutzt werden. Hier müssen jedoch ebenfalls die verwendeten
Approximation der höheren Ableitungen angepasst werden, um nicht vorhandene Werte nicht zu verwenden.
Soll beispielsweise die Ableitung $\phi'_e$ für das in Abbildung~\ref{fig:kv1d_rand} markierte Kontrollvolumen
berechnet werden, so müssen die Ableitungen $\phi''_e$ und $\phi'''_e$ diskretisiert werden ohne den Wert
$\phi_W$ zu verwenden. Die sich ergebenden Approximationen lauten damit:
\begin{align*}
  \phi''_e &= \frac{1}{(x_E-x_P)}\left({
\frac{\phi_{EE}-\phi_P}{x_{EE}-x_P}-\frac{\phi_E-\phi^w}{x_E-x_w}}\right)\\
\phi'''_e &= \frac{1}{(x_E-x_P)}\left({
\frac{1}{(x_{ee}-x_e)}
\left({\frac{\phi_{EE}-\phi_E}{x_{EE}-x_E}-\frac{\phi_E-\phi_P}{x_E-x_P} }\right)
-\frac{1}{(x_e-x_w)}
\left({\frac{\phi_E-\phi_P}{x_E-x_P} - \frac{\phi_P-\phi^w}{x_P-x_w}  }\right)
}\right)
\end{align*}
Die oben vorgestellte Behandlung des westlichen Randes kann äquivalent auf die anderen Ränder übertragen werden.
Beim zweidimensionalen Fall ist zudem besonders auf die Ecken zu achten.


%\paragraph{Westlicher Rand}

%Es ergeben sich für gegebenes $\phi_w$:


%\begin{equation*}
  %\phi''_{w, W-Rand} = \frac{1}{(x_P-x_w)}\left({
%\frac{\phi_{e}-\phi_w}{x_{e}-x_w}-\frac{\phi_P-\phi_w}{x_P-x_w}}\right)
%\end{equation*}

%Da bei der dritten Ableitung zwei linke Kontrollvolumen genutzt werden, müssen hier
%vom Kontrollvolumen am Rand sowohl der westliche also auch der östliche Rand
%betrachtet werden.

%\begin{align*}
  %\phi'''_{w, W-Rand} &= \frac{1}{(x_p-x_w)} \left({
  %\frac{1}{(x_e-x_w)} \left({
    %\frac{\phi_E-\phi_P}{x_E-x_P} - \frac{\phi_P-\phi_w}{x_P-x_w}
    %}\right) -
  %\frac{1}{x_P-x_w} \left({
    %\frac{\phi_e-\phi_w}{x_e-x_w} - \frac{\phi_P-\phi_w}{x_P-x_w}
    %}\right)
  %}\right)
  %\\
  %\phi'''_{e, W-Rand} &= \frac{1}{(x_E-x_P)} \left({
  %\frac{1}{(x_{ee}-x_e)} \left({
      %\frac{\phi_{EE}-\phi_E}{x_{EE}-x_E} - \frac{\phi_E-\phi_P}{x_E-x_P}
    %}\right) -
  %\frac{1}{x_e-x_w} \left({
    %\frac{\phi_E-\phi_P}{x_E-x_P} - \frac{\phi_P-\phi_w}{x_P-x_w}
    %}\right)
  %}\right)
%\end{align*}

%Hier wird noch der Wert $\phi_e$ benutzt, der aber nicht bekannt ist. Er wird deshalb
%durch eine lineare Interpolation von $\phi_E$ und $\phi_P$ bestimmt.

%\begin{equation}
  %\phi_e = \phi_P \frac{x_E-x_e}{x_E-x_P} + \phi_E \frac{x_e-x_P}{x_E-x_P}
%\end{equation}

%Damit ergibt sich für $\phi''_{w,Rand}$und $\phi'''_{w, Rand}$:

%\begin{align}
  %\phi''_{w, W-Rand} &= \frac{1}{(x_P-x_w)}\left({
%\frac{
  %\left({\phi_P \frac{x_E-x_e}{x_E-x_P} + \phi_E \frac{x_e-x_P}{x_E-x_P}
%}\right)
%-\phi_w}{x_{e}-x_w}-\frac{\phi_P-\phi_w}{x_P-x_w}}\right)\\
  %\phi'''_{w, W-Rand} &= \frac{1}{(x_p-x_w)} \left({
  %\frac{1}{(x_e-x_w)} \left({
    %\frac{\phi_E-\phi_P}{x_E-x_P} - \frac{\phi_P-\phi_w}{x_P-x_w}
    %}\right) -
  %\frac{1}{x_P-x_w} \left({
    %\frac{ \phi_P \frac{x_E-x_e}{x_E-x_P} + \phi_E \frac{x_e-x_P}{x_E-x_P}
%-\phi_w}{x_e-x_w} - \frac{\phi_P-\phi_w}{x_P-x_w}
    %}\right)
  %}\right)
%\end{align}


%\paragraph{Östlicher Rand}
%Äquivalent ergibt sich $\phi''_{e, E-Rand}$ bei gegebenem $\phi_e$:

%\begin{align*}
  %\phi''_{e, E-Rand} &= \frac{1}{(x_e-x_P)}\left({
%\frac{\phi_{e}-\phi_P}{x_{e}-x_P}-\frac{\phi_e-\phi_w}{x_e-x_w}}\right)
%\\
  %\phi'''_{e, E-Rand} &= \frac{1}{(x_e-x_P)} \left({
  %\frac{1}{(x_e-x_P)} \left({
    %\frac{\phi_e-\phi_P}{x_e-x_P} - \frac{\phi_e-\phi_w}{x_e-x_w}
    %}\right) -
  %\frac{1}{x_e-x_w} \left({
    %\frac{\phi_e-\phi_P}{x_e-x_P} - \frac{\phi_P-\phi_W}{x_P-x_W}
    %}\right)
  %}\right)
  %\\
  %\phi'''_{w, E-Rand} &= \frac{1}{(x_P-x_W)} \left({
  %\frac{1}{(x_e-x_w)} \left({
      %\frac{\phi_e-\phi_P}{x_e-x_P} - \frac{\phi_P-\phi_W}{x_P-x_W}
    %}\right) -
    %\frac{1}{x_w-x_{ww}} \left({
        %\frac{\phi_P-\phi_W}{x_P-x_W} - \frac{\phi_W-\phi_{WW}}{x_W-x_{WW}}
    %}\right)
  %}\right)
%\end{align*}

%$\phi_w$ wird wie folgt linear interpoliert:

%\begin{equation*}
  %\phi_w = \phi_W \frac{x_P-x_w}{x_P-x_W} + \phi_P \frac{x_w-x_W}{x_P-x_W}
%\end{equation*}

%Damit ergibt sich für $\phi''_{e,Rand}$:

%\begin{align}
  %\phi''_{e,Rand} &= \frac{1}{(x_e-x_P)}\left({
%\frac{\phi_{e}-\phi_P}{x_{e}-x_P}-\frac{\phi_e-
  %\left({
   %\phi_W \frac{x_P-x_w}{x_P-x_W} + \phi_P \frac{x_w-x_W}{x_P-x_W}
  %}\right)
%}{x_e-x_w}}\right)
%\\
  %\phi'''_{e, E-Rand} &= \frac{1}{(x_e-x_P)} \left({
  %\frac{1}{(x_e-x_P)} \left({
    %\frac{\phi_e-\phi_P}{x_e-x_P} - \frac{\phi_e- \phi_W \frac{x_P-x_w}{x_P-x_W} + \phi_P \frac{x_w-x_W}{x_P-x_W}
%}{x_e-x_w}
    %}\right) -
  %\frac{1}{x_e-x_w} \left({
    %\frac{\phi_e-\phi_P}{x_e-x_P} - \frac{\phi_P-\phi_W}{x_P-x_W}
    %}\right)
  %}\right)
%\end{align}


%\subsubsection{Äquidistante Gitter}

%\paragraph{Westlicher Rand}

%\begin{align}
  %\phi''_{w, W-Rand} &= \frac{1}{\Delta x^2} (\phi_E-3\phi_P+2\phi_w)\\
  %\phi'''_{w, W-Rand} &= 0\\
  %\phi'''_{e, W-Rand} &= \frac{1}{\Delta x^3} (\phi_{EE} -3\phi_E + 4\phi_P -2\phi_w)
%\end{align}

%\paragraph{Östlicher Rand}

%\begin{align}
  %\phi''_{e, E-Rand} &= \frac{1}{\Delta x^2} (\phi_W-3\phi_P+2\phi_e)\\
  %\phi'''_{e, E-Rand} &= 0\\
  %\phi'''_{w, E-Rand} &= \frac{1}{\Delta x^3} (-\phi_{WW} +3\phi_W - 4\phi_P +2\phi_e)
%\end{align}

\subsection{Zweidimensionale Probleme}

Äquivalent zu den Herleitungen des Abbruchfehlersim Osten und Westen ergeben sich
sich für die Ableitungen $\phi'_n$ und $\phi'_s$ die folgenden Reihenentwicklungen:

\begin{equation}
  \phi'_n = \frac{\phi_N-\phi_P}{y_N-y_P}+\frac{1}{2}\phi''_n
\left({\frac{{(y_P-y_n)}^2-{(y_N-y_n)}^2}{y_N-y_P}}\right)+
\frac{1}{6} \phi'''_n \left({\frac{{(y_P-y_n)}^3-{(y_N-y_n)}^3}{y_N-y_P}}\right)+HOT
\end{equation}


\begin{equation}
  \phi'_s = \frac{\phi_P-\phi_S}{y_P-y_S}+\frac{1}{2}\phi''_s
\left({\frac{{(y_S-y_s)}^2-{(y_P-y_s)}^2}{y_P-y_S}}\right)+
\frac{1}{6} \phi'''_s \left({\frac{{(y_S-y_s)}^3-{(y_P-y_s)}^3}{y_P-y_S}}\right)+HOT
\end{equation}
Nach der Diskretisierung der auftretenden, unbekannten Ableitungen ergeben sich die folgenden
Terme für den Abbruchfehler:

\begin{equation}
  \begin{IEEEeqnarraybox}[][c]{rCl}
    {TE}_n &=& \frac{1}{2\,(y_N-y_P)}\left({
\frac{\phi_{NN}-\phi_P}{y_{NN}-y_P}-\frac{\phi_N-\phi_S}{y_N-y_S}}\right) \left({\frac{{(y_P-y_n)}^2-{(y_N-y_n)}^2}{y_N-y_P}}\right)\\
&&+
\left({
\frac{1}{(y_{nn}-y_n)}
\left({\frac{\phi_{NN}-\phi_N}{y_{NN}-y_N}-\frac{\phi_N-\phi_P}{y_N-y_P} }\right)
-\frac{1}{(y_n-y_s)}
\left({\frac{\phi_N-\phi_P}{y_N-y_P} - \frac{\phi_P-\phi_S}{y_P-y_S}  }\right)
}\right)\\
&&\frac{1}{6\,(y_N-y_P)}\left({\frac{{(y_P-y_n)}^3-{(y_N-y_n)}^3}{y_N-y_P}}\right)
+HOT
\end{IEEEeqnarraybox}
\end{equation}


\begin{equation}
\begin{IEEEeqnarraybox}[][c]{rCl}
  TE_s &=& \frac{1}{2\,(y_P-y_S)} \left({
\frac{\phi_{N}-\phi_S}{y_{N}-y_S}-\frac{\phi_P-\phi_{SS}}{y_P-y_{SS}}}\right)
  \left({\frac{{(y_S-y_s)}^2-{(y_P-y_s)}^2}{y_P-y_S}}\right)\\
  &&+
\left({
\frac{1}{(y_n-y_s)}
\left({\frac{\phi_N-\phi_P}{y_N-y_P}-\frac{\phi_P-\phi_S}{y_P-y_S} }\right)
-\frac{1}{(y_s-y_{ss})}
\left({\frac{\phi_P-\phi_S}{y_P-y_S} - \frac{\phi_S-\phi_{SS}}{y_S-y_{SS}}  }\right)
}\right)\\
&&\frac{1}{6\,(y_P-y_S)}\left({\frac{{(y_S-y_s)}^3-{(y_P-y_s)}^3}{y_P-y_S}}\right)
  +HOT
\end{IEEEeqnarraybox}
\end{equation}
Die Behandlung der Randwerte erfolgt ebenfalls wir in Abschnitt~\ref{sec:te_dif_rand} erläutert und wird hier
nicht wiederholt.

\subsection{Äquidistante Gitter}

\begin{align*}
  {TE}_n &= \left({
\frac{1}{6\Delta y^2}
\left({\frac{\phi_{NN}-\phi_N}{\Delta y}-\frac{\phi_N-\phi_P}{\Delta y} }\right)
-\frac{1}{6\Delta y^2}
\left({\frac{\phi_N-\phi_P}{\Delta y} - \frac{\phi_P-\phi_S}{\Delta y}  }\right)
}\right)\left({-\frac{\Delta y^2}{4} }\right)+HOT\\
&= -\frac{1}{24\Delta y}\left({
\phi_{NN}-3\phi_N+3\phi_P-\phi_S}\right)+HOT
\end{align*}

\begin{align*}
  TE_s &=\left({
\frac{1}{6 \Delta y^2}
\left({\frac{\phi_N-\phi_P}{\Delta y}-\frac{\phi_P-\phi_S}{\Delta y} }\right)
-\frac{1}{6\Delta y^2}
\left({\frac{\phi_P-\phi_S}{\Delta y} - \frac{\phi_S-\phi_{SS}}{\Delta y}  }\right)
}\right)
\left({-\frac{\Delta y^2}{4} }\right)+HOT\\
&= -\frac{1}{24 \Delta y}\left({
\phi_N-3\phi_P+3\phi_S-\phi_{SS}}\right)+HOT
\end{align*}

\section{Abbruchfehler von Konvektionstermen}

Wie in Abschnitt~\ref{sec:konv_fluss} beschrieben, existieren verschiedene Verfahren
um konvektive Flüsse abzuschätzen. Der Abbruchfehler ist vom genutzten Verfahren
abhängig und müss für jedes Verfahren hergeleitet werden.

Zu diesem Fehler kommt ab der zweiten Dimension wiederum der Abbruchfehler aus
der Integralapproximation, der bereits in Abschnitt~\ref{sec:source1d} hergeleitet wurde.

\subsection{CDS-Verfahren}
\label{sec:te_cds}

Für die Herleitung des Abbruchfehlers für das Zentraldifferenzen-Verfahren sei hier am Beispiel des östlichen Randes dargestellt.
Man entwickelt zuerst die Taylorreihe von $\phi$ um den Entwicklunspunktunkt $x_P$ und wertet sie anschließend
an den Punkten $x_e$ und $x_E$ aus. Es ergeben sich die folgenden Reihendarstellungen.
\begin{align}
  \phi_e &= \phi_P + \phi'_P(x_e-x_P)+\frac{1}{2}\phi''_P(x_e-x_P)^2
  +\frac{1}{6}\phi'''_P(x_e-x_P)^3+HOT
  \label{eq:taylor_konv_eP}\\
  \phi_E &= \phi_P + \phi'_P(x_E-x_P)+\frac{1}{2}\phi''_P(x_E-x_P)^2
  +\frac{1}{6}\phi'''_P(x_E-x_P)^3+HOT
  \label{eq:taylor_konv_eE}
\end{align}
Werden die Gleichungen~\eqref{eq:taylor_konv_eE} und \eqref{eq:taylor_konv_eP} nun
voneinander subtrahiert und nach $\phi_e$ umgestellt, so ergibt sich:

\begin{equation*}
  \begin{IEEEeqnarraybox}[][c]{rCl}
    \frac{\phi_e}{x_e-x_P} &=& \frac{\phi_E}{x_E-x_P} + \frac{\phi_P}{x_e-x_P} -
  \frac{\phi_P}{x_E-x_P} + \frac{1}{2} \phi''_P \left({(x_e-x_P)-(x_E-x_P)}\right)\\
  &&+ \frac{1}{6} \phi'''_P \left({(x_e-x_P)^2-(x_E-x_P)^2}\right)\\
  \phi_e &=& \phi_E \frac{x_e-x_P}{x_E-x_P} + \phi_P \left({1-\frac{x_e-x_P}{x_E-x_P} }\right)+HOT
  + \frac{1}{2} \phi''_P (x_e-x_E)(x_e-x_P)+HOT\\
  &&+ \frac{1}{6} \phi'''_P \left({(x_e-x_P)^2-(x_E-x_P)^2}\right)(x_e-x_P)\\
   &=& \phi_E \gamma_e + \phi_P (1-\gamma_e)+ \frac{1}{2} \phi''_P (x_e-x_E)(x_e-x_P)\\
         &&+ \frac{1}{6} \phi'''_P \left({(x_e-x_P)^2-(x_E-x_P)^2}\right)(x_e-x_P)+HOT
  \end{IEEEeqnarraybox}
\end{equation*}
Der Abbruchfehler lässt sich hier leicht ablesen. Er lautet:
\begin{equation*}
  TE_{e, CDS} =  \frac{1}{2} \phi''_P (x_e-x_E)(x_e-x_P)+ \frac{1}{6}
  \phi'''_P \left({(x_e-x_P)^2-(x_E-x_P)^2}\right)(x_e-x_P)+HOT
\end{equation*}
Die hier genutzten analytischen Ableitungen werden im nächsten Schritt durch Approximationen ersetzt.
\begin{align*}
  %\phi'_P &= \frac{\phi_E-\phi_W}{x_E-x_W}\\
  \phi''_P &= \frac{1}{(x_e-x_w)} \left({\frac{\phi_E-\phi_P}{x_E-x_P}
  - \frac{\phi_P-\phi_W}{x_P-x_W} }\right)\\
  %\label{eq:ddphip}\\
  \phi'''_P &= \frac{1}{(x_e-x_w)} \left({
  \frac{1}{(x_E-x_P)} \left({\frac{\phi_{EE}-\phi_P}{x_{EE}-x_P}- \frac{\phi_E-\phi_W}{x_E-x_W} }\right)-
  \frac{1}{(x_P-x_W)} \left({\frac{\phi_E-\phi_W}{x_E-x_W} - \frac{\phi_P-\phi_{WW}}{x_P-x_{WW}} }\right)
  }\right) \label{eq:dddphip}
\end{align*}
Der Abbruchfehler ergibt sich damit zu folgendem Term.
\begin{equation}
  \begin{IEEEeqnarraybox}{rCl}
    TE_{e, CDS} &=&  \frac{1}{2} \frac{1}{(x_e-x_w)} \left({\frac{\phi_E-\phi_P}{x_E-x_P}
  - \frac{\phi_P-\phi_W}{x_P-x_W} }\right) (x_e-x_E) (x_e-x_P) \nonumber \\
  &&+
 \frac{1}{(x_e-x_w)} \left({
  \frac{1}{(x_E-x_P)} \left({\frac{\phi_{EE}-\phi_P}{x_{EE}-x_P}- \frac{\phi_E-\phi_W}{x_E-x_W} }\right)-
  \frac{1}{(x_P-x_W)} \left({\frac{\phi_E-\phi_W}{x_E-x_W} - \frac{\phi_P-\phi_{WW}}{x_P-x_{WW}} }\right)
  }\right) \nonumber \\
  &&\frac{1}{6} \left({(x_e-x_P)^2-(x_E-x_P)^2}\right)(x_e-x_P)+HOT
  \end{IEEEeqnarraybox}
\end{equation}
Im Westen wird die Ableitung äquivalent zum Osten durchgeführt. Für den Abbruchfehler ergibt sich letztendlich
folgender Term:
%\begin{align*}
  %\begin{IEEEeqnarraybox}{rCl}
  %\phi_w &= \left({1-\frac{x_w-x_P}{x_W-x_P}}\right)\phi_P + \left({\frac{x_w-x_P}{x_W-x_P} }\right) \phi_W
  %+ \frac{1}{2} \phi''_P \left({(x_w-x_P)-(x_W-x_P)}\right)(x_w-x_P)\\
  %&+ \frac{1}{6} \phi'''_P \left({(x_w-x_P)^2-(x_W-x_P)^2}\right)(x_w-x_P)\\
  %\phi_w &= \left({1-\frac{x_P-x_w}{x_P-x_W}}\right)\phi_P + \left({\frac{x_P-x_w}{x_P-x_W} }\right) \phi_W
  %+ \frac{1}{2} \phi''_P \left({x_w-x_W}\right)(x_w-x_P)\\
  %&+ \frac{1}{6} \phi'''_P \left({(x_w-x_P)^2-(x_W-x_P)^2}\right)(x_w-x_P)
  %\end{IEEEeqnarraybox}
%\end{align*}

%Der Abbruchfehler lässt sich nach der Diskretisierung der unbekannten Ableitungen nach Gleichung~\eqref{eq:ddphip}
%und \eqref{eq:dddphip} leicht ablesen und ergibt sich zu:

\begin{equation}
  \begin{IEEEeqnarraybox}[][b]{rCl}
    TE_{w, CDS} &=&  \frac{1}{2} \frac{1}{(x_e-x_w)} \left({\frac{\phi_E-\phi_P}{x_E-x_P}
  - \frac{\phi_P-\phi_W}{x_P-x_W} }\right) \left({x_w-x_W}\right)(x_w-x_P)  \nonumber\\
  &&+ \frac{1}{(x_e-x_w)} \left({
  \frac{1}{(x_E-x_P)} \left({\frac{\phi_{EE}-\phi_P}{x_{EE}-x_P}- \frac{\phi_E-\phi_W}{x_E-x_W} }\right)-
  \frac{1}{(x_P-x_W)} \left({\frac{\phi_E-\phi_W}{x_E-x_W} - \frac{\phi_P-\phi_{WW}}{x_P-x_{WW}} }\right)
  }\right) \nonumber\\
  &&\frac{1}{6}  \left({(x_w-x_P)^2-(x_W-x_P)^2}\right)(x_w-x_P)
  \end{IEEEeqnarraybox}
\end{equation}

\subsubsection{Randwerte}
\label{sec:te_cds_rand}

Bei Kontrollvolumen direkt am Rand, wie zum Beispiel das in Abbildung~\ref{fig:kv1d_rand}
hervorgehobene Kontrollvolumen, müssen einige Besonderheiten beachtet werden.
So ist der konvektive Fluss, der direkt auf dem Rand verläuft, durch die Randbedingung ($\phi^w$) gegeben und
muss dementsprechend nicht angepasst werden. Am östlichen Rand muss hingegen die Approximation
der Ableitungen angepasst werden. Sie lauten dann:
\begin{align*}
  \phi''_P &= \frac{1}{(x_e-x_w)} \left({\frac{\phi_E-\phi_P}{x_E-x_P}
  - \frac{\phi_P-\phi^w}{x_P-x_w} }\right)\\
  \phi'''_P &= \frac{1}{(x_e-x_w)} \left({
  \frac{1}{(x_E-x_P)} \left({\frac{\phi_{EE}-\phi_P}{x_{EE}-x_P}- \frac{\phi_E-\phi^w}{x_E-x_w} }\right)-
  \frac{1}{(x_P-x_w)} \left({\frac{\phi_E-\phi^w}{x_E-x_w} - \frac{\phi_P-\phi^{w}}{x_P-x_{w}} }\right)
  }\right)
\end{align*}
Hat das betrachtete Kontrollvolumen  einen Abstand von eins zum Rand, so muss nur die Approximation der dritten
Ableitung angepasst werden.
\begin{equation*}
  \phi'''_P = \frac{1}{(x_e-x_w)} \left({
  \frac{1}{(x_E-x_P)} \left({\frac{\phi_{EE}-\phi_P}{x_{EE}-x_P}- \frac{\phi_E-\phi^w}{x_E-x_w} }\right)-
  \frac{1}{(x_P-x_W)} \left({\frac{\phi_E-\phi_W}{x_E-x_W} - \frac{\phi_P-\phi^{w}}{x_P-x_{w}} }\right)
  }\right)
\end{equation*}
Auf gleiche Art und Weise kann auch der östliche Rand behandelt werden.
%Hier muss abschließend $\phi_e$ aus $\phi_E$ und $\phi_P$ interpoliert werden. Letztendlich ergibt
%sich für äquidistante Gitter folgender Term:

%\begin{equation*}
  %TE_{e, CDS} = -\frac{1}{8} (\phi_E-2\phi_P+\phi_W) - \frac{1}{32}
  %(2\phi^{ee} - 3\phi_E + 2\phi_W - \phi_{WW})
%\end{equation*}

%Auf gleiche Art und Weise kann auch $TE_{w, CDS}$ berechnet werden.
\subsubsection{Äquidistante Gitter}

Für äquidistante Gitter ergeben sich die folgenden vereinfachten Formeln zur Berechnung der konvektiven Flüsse
mit dem CDS-Verfahren.

\begin{equation*}
  TE_{e, CDS} = -\frac{1}{8} (\phi_E-2\phi_P+\phi_W) - \frac{1}{32}
  (\phi_{EE} - 2\phi_E + 2\phi_W - \phi_{WW})
\end{equation*}
\begin{equation*}
  TE_{w, CDS} = -\frac{1}{8} (\phi_E-2\phi_P+\phi_W) + \frac{1}{32}
  (\phi_{EE} - 2\phi_E + 2\phi_W - \phi_{WW})
\end{equation*}








\subsection{UDS-Verfahren}

Um der Abbruchfehler des Upwind-Verfahrens herzuleiten, wertet man die
Taylorreihe von $\phi$ mit dem Entwicklungspunkt $x_P$ im Punkte $\phi_e$
aus.

\begin{equation*}
  \phi_e = \phi_P +(x_e-x_P) \phi'_P + \frac{1}{2} (x_e-x_P)^2 \phi''_P+HOT
\end{equation*}
Da im Falle einer positiven Geschwindigkeit das UDS-Verfahren $\phi_e$ mit $\phi_P$
gleichsetzt, lässt sich der Abbruchfehler hier direkt ablesen.
\begin{equation*}
  TE_{e, UDS} = (x_e-x_P) \phi'_P + \frac{1}{2} (x_e-x_P)^2 \phi''_P+HOT
\end{equation*}
Setzt man nun die diskretisierten Ableitungen aus Abschnitt~\ref{sec:te_cds} ein,
ergibt sich die Formel für den Abbruchfehler.
\begin{equation}
  TE_{e, UDS} = (x_e-x_P) \frac{\phi_E-\phi_W}{x_E-x_W}+
  \frac{1}{2} \frac{(x_e-x_P)^2}{(x_e-x_w)} \left({\frac{\phi_E-\phi_P}{x_E-x_P}
  - \frac{\phi_P-\phi_W}{x_P-x_W} }\right)+HOT
\end{equation}
Betrachtet man den westlichen Rand und geht von einer positiven Geschwindigkeit aus,
so ergibt sich der folgende Abbruchfehler:
\begin{equation*}
  TE_{w, UDS} = (x_w-x_W) \phi'_W + \frac{1}{2} (x_w-x_W)^2 \phi''_W + HOT
\end{equation*}
Hier werden nun wiederum die Ableitungen $\phi'_W$ und $\phi''_W$ diskretisiert. Der
Abbruchfehler lautet damit:
\begin{equation}
  TE_{w, UDS} = (x_w-x_W) \frac{\phi_P-\phi_{WW}}{x_P-x_{WW}}+
  \frac{1}{2} \frac{(x_w-x_W)^2}{(x_w-x_{ww})} \left({\frac{\phi_P-\phi_W}{x_P-x_W}
  - \frac{\phi_W-\phi_{WW}}{x_W-x_{WW}} }\right)+HOT
\end{equation}



\subsubsection{Randwerte}

Auch beim UDS-Verfahren müssen am Rand die Approximationen der verwendeten Ableitungen
so angepasst werden, das nur vorhandene Werte benutzt werden.
Für den östlichen Rand der Kontrollvolumens können dabei die in Abschnitt~\ref{sec:te_cds_rand}
gezeigten Differenzenquotienten genutzt werden.

Für das Kontrollvolumen direkt am Rand ist der konvektive Fluss gegeben. Es gibt somit keinen
Abbruchfehler. Für das daneben liegende Kontrollvolumen müssen die Differenzenquotienten wie folgt angepasst werden:
\begin{align*}
  \phi'_W &= \frac{\phi_w-\phi^w}{x_w-x_{ww}}\\
  \phi''_W &= \frac{1}{x_W-x_{ww}} \left({\frac{\phi_w-\phi^w}{x_w-x_{ww}}
- \frac{\phi_W-\phi^w}{x_W-x_{ww}} }\right)+HOT
\end{align*}
$\phi_w$ wird hier beispielsweise aus der linearen Interpolation von $\phi_P$ und $\phi_W$ berechnet.
\begin{equation*}
  \phi_w = \phi_W \frac{x_P-x_w}{x_P-x_W} + \phi_P \frac{x_w-x_W}{x_P-x_W}
\end{equation*}
%etzt ergeben sich damit die folgenden Terme.
%\begin{align}
  %\phi'_W &= \frac{\phi_W \frac{x_P-x_w}{x_P-x_W} + \phi_P \frac{x_w-x_W}{x_P-x_W}
%-\phi^w}{x_w-x_{ww}}\\
  %\phi''_W &= \frac{1}{x_W-x_{ww}} \left({\frac{\phi_W \frac{x_P-x_w}{x_P-x_W} + \phi_P \frac{x_w-x_W}{x_P-x_W}
%-\phi^w}{x_w-x_{ww}}
%- \frac{\phi_W-\phi^w}{x_W-x_{ww}} }\right)+HOT
%\end{align}


\subsubsection{Äquidistante Gitter}

%\begin{align*}
  %\phi'_W  &= \frac{1}{2\Delta x} \phi_P + \frac{1}{2\Delta x} \phi_W - \phi^w\\
  %\phi''_W &= \frac{1}{\Delta x^2} \phi_P -\frac{1}{\Delta x^2} \phi_W
%\end{align*}
Für äquidistante Gitter ereben sich folgende Gleichungen für den Abbruchfehler.

\begin{equation*}
  TE_{e, UDS} = \frac{3}{8} \phi_E-\frac{1}{4} \phi_P - \frac{1}{8} \phi_W
\end{equation*}
\begin{equation*}
  TE_{w, UDS} = \frac{3}{8} \phi_P-\frac{1}{4} \phi_W - \frac{1}{8} \phi_{WW}
\end{equation*}





\subsection{``Flux-Blending''-Verfahren}

Das Flux-Blending-Verfahren setzt sich aus UDS- und CDS-Verfahren zusammen. Beide
Verfahren werden dabei über den Faktor $\beta$ gewichtet. Dementsprechend müssen auch die
Abbruchfehler über $\beta$ gewichtet werden um den Abbruchfehler für das ``Flux-Blending''-Verfahren zu berechnen.
Mit Gleichung~\ref{eq:flux_blending} ergibt sich damit:

\begin{equation}
  TE_{e, Flux} = (1-\beta) TE_{e, UDS} + \beta TE_{e, CDS}
\end{equation}



%\section{Truncation Error eines Kontrollvolumens}
%\label{sec:Truncation Error eines Kontrollvolumens}

%Der Truncation Error für ein Kontrollvolumen setzt sich nun aus den Fehlern von Quell-
%und Diffusionstermen zusammen:

%\begin{equation*}
  %TE = \frac{TE_{source} - TE_e - TE_w}{\Delta x}
%\end{equation*}
%Für den äquidistanten Fall ergibt sich damit für zentrale Kontrollvolumen der folgende
%Truncation Error. Wichtig ist es, hier die durch den Gauß'schen Integralsatz 
%entstehenden Vorzeichen mit zu beachten. Weiterhin muss durch $\Delta x$ geteilt werden,
%da die gleiche Transformation beim Aufstellen des Gesamtgleichungsystem angewendet wird.

%\begin{align}
  %TE &= \frac{\frac{\Delta x}{24} \left({f_E-2f_P+f_W}\right)
   %+\frac{1}{24\Delta x}\left({
%\phi_{EE}-3\phi_E+3\phi_P-\phi_W}\right)
  %-\frac{1}{24 \Delta x}\left({
%\phi_E-3\phi_P+3\phi_W-\phi_{WW}}\right)}{\Delta x}
%\end{align}

\subsection{Zweidimensionale Probleme}
Die gezeigten Herleitungen lassen sich problemlos ins Zweidimensionale übertragen. Es muss
jedoch darauf geachtet werdenn den Fehler aus der Integralapproximation nciht zu vergessen.
Dieser lautet für die Ostseite:
\begin{equation}
  TE_{e,int} = \frac{1}{6} \frac{\partial^2\phi}{\partial x^2}\bigg\vert_e
    \left[{{(y_n-y_P)}^3-{(y_s-y_P)}^3}\right] + HOT
\end{equation}
Er lässt sich leicht auf die anderen Seiten des Kontrollvolumens übertragen.
\clearpage

%\newcommand{\pder}[2][]{\frac{\partial#1}{\partial#2}}
%\newcommand{\pderf}[1]{\frac{\partial f}{\partial#1}}
%\newcommand{\pderfs}[1]{\frac{\partial^2 f}{\partial#1}}

\newcommand{\fxi}{f_{\xi}}
\newcommand{\fxxi}{f_{\xi\xi}}
\newcommand{\fxxxi}{f_{\xi\xi\xi}}
\newcommand{\fxxxxi}{f_{\xi\xi\xi\xi}}
\newcommand{\xxi}{x_{\xi}}
\newcommand{\xxxi}{x_{\xi\xi}}
\newcommand{\xxxxi}{x_{\xi\xi\xi}}
\newcommand{\xxxxxi}{x_{\xi\xi\xi\xi}}
\newcommand{\yxi}{y_{\xi}}
\newcommand{\yxxi}{y_{\xi\xi}}
\newcommand{\yxxxi}{y_{\xi\xi\xi}}
\newcommand{\yxxxxi}{y_{\xi\xi\xi\xi}}

\newcommand{\feta}{f_{\eta}}
\newcommand{\feeta}{f_{\eta\eta}}
\newcommand{\feeeta}{f_{\eta\eta\eta}}
\newcommand{\feeeeta}{f_{\eta\eta\eta\eta}}
\newcommand{\xeta}{x_{\eta}}
\newcommand{\xeeta}{x_{\eta\eta}}
\newcommand{\xeeeta}{x_{\eta\eta\eta}}
\newcommand{\xeeeeta}{x_{\eta\eta\eta\eta}}
\newcommand{\yeta}{y_{\eta}}
\newcommand{\yeeta}{y_{\eta\eta}}
\newcommand{\yeeeta}{y_{\eta\eta\eta}}
\newcommand{\yeeeeta}{y_{\eta\eta\eta\eta}}


\newcommand{\etai}{\eta_{e1}}
\newcommand{\etaii}{\eta_{e2}}
\newcommand{\etilde}{\tilde{e}}


%\chapter{Transformation}

%Die Transformation vom physikalischen auf den logischen Bereich mit
%den physikalischen Koordinaten $(x, y)$ sowie den logischen
%Koordinaten $(\xi, \eta)$ kann wie folgt definiert werden:
%\begin{equation}
  %x=x(\xi,\eta),\quad y=y(\xi, \eta)
%\end{equation}
%\begin{equation}
  %\pder[x]{\xi}=\xi_x=\frac{y_{\eta}}{J}
%\end{equation}
%Damit folgt für für die bei Konvektion und Diffusion auftretende
%erste und zweite Ableitung:

%\begin{align*}
  %\pderf{x}&=\pder[\xi]{x}\pderf{\xi}+\pder[\eta]{x}\pderf{\eta}
  %=\frac{y_{\eta}}{J} f_{\xi} - \frac{y_{\xi}}{J}f_{\eta}\\
  %\pderf{y}&=\pder[\xi]{y}\pderf{\xi}+\pder[\eta]{y}\pderf{\eta}
  %=\frac{x_{\xi}}{J}f_{\eta}-\frac{x_{\eta}}{J} f_{\xi} \\
  %\pderfs{x^2} &= \pder{x}\left({\frac{y_{\eta}}{J} f_{\xi} - \frac{y_{\xi}}{J}f_{\eta}}\right)\\
               %&= \frac{\yeta^2}{J^2} \fxxi + \frac{\yxi^2}{J^2} \feeta-2 \frac{\yxi \yeta}{J^2} f_{\xi\eta}\\
               %&+ \frac{1}{J^3} \Big[\fxi
  %\Big(-2\xxi\yxi\yeta\yeeta + 2\xxi\yeta^2y_{\xi\eta} + \xxxi \yeta^3 - 2x_{\xi \eta}\yxi\yeta^2
  %-\xeta\yeta^2\yxxi + \xeta\yxi^2\yeeta + \yeeta\yxi^2\yeta\Big)\\
  %&+ \feta\Big(
  %2\xeta\yxi\yeta\yxxi - 2\xeta\yxi^2y_{\xi\eta} -\xeeta\yxi^3+2x_{\xi\eta}\yeta\yxi^2
  %+\xxi\yxi^2\yeeta-\xxi\yeta^2\yxxi-\xxxi\yeta^2\yxi
%\Big)\Big]
%\end{align*}


%Im logischen Gebiet können nun die vorhandenen Ableitungen von $f$
%($f_{\xi}$, $f_{\eta}$, $\dots$) diskretisiert werden.
%Es werden hierbei Zentraldifferenzen zweiter Ordnung gewählt.

%\begin{equation}
  %f_{\xi}=\frac{f_{i+1,j}-f_{i-1,j}}{2\Delta \xi} - \frac{f_{\xi\xi\xi}}{6}\Delta \xi^2 + HOT
%\end{equation}
%\begin{equation}
  %f_{\eta}=\frac{f_{i,j+1}-f_{i,j-1}}{2 \Delta \eta} - \frac{f_{\eta\eta\eta}}{6}\Delta \eta^2 + HOT
%\end{equation}
%\begin{equation}
  %\fxxi=\frac{f_{i+1,j}-2f_{i,j}+f_{i-1,j}}{\Delta \xi^2} -\frac{\fxxxxi}{12} \Delta \xi^2 + HOT
%\end{equation}
%\begin{equation}
  %\feeta=\frac{f_{i,j+1}-2f_{i,j}+f_{i,j-1}}{\Delta \eta^2} -\frac{\feeeeta}{12} \Delta \eta^2 + HOT
%\end{equation}
%\begin{equation}
  %f_{\xi\eta}=\frac{f_{i+1,j+1}-2f_{i,j}+f_{i-1,j-1}}{\Delta \xi\Delta \eta} -\frac{\fxxi+\feeta}{4} \Delta \eta \Delta \xi + HOT
%\end{equation}

%Werden diese nun eingesetzt ergibt sich für $\Delta \xi = 1$ folgende Form:

%\begin{equation}
  %\pderf{x}=\frac{y_{\eta}}{J}\frac{f_{i+1,j}-f_{i-1,j}}{2}
  %- \frac{y_{\xi}}{J}\frac{f_{i,j+1}-f_{i,j-1}}{2}
%\end{equation}

%sowie der entstehende Abbruchfehler:

%\begin{equation}
  %TE_x = -\frac{y_{\eta}}{J}\frac{f_{\xi\xi\xi}}{6}
  %+ \frac{y_{\xi}}{J}\frac{f_{\eta\eta\eta}}{6}+HOT
%\end{equation}

%Anschließend können die berechneten Werte für $f_{\xi},\dots$
%eingesetzt werden.

%\begin{align*}
  %\fxi &= \xxi f_x + \yxi f_y\\
  %\fxxi &= \xxxi f_x + \xxi^2f_{xx}+2\xxi \yxi f_{xy}
  %+\yxi^2 f_{yy} + \yxxi f_y\\
  %\fxxxi &= \xxxxi f_x + \yxxxi f_y + 3 \xxxi\xxi f_{xx}+
  %3 \yxxi\yxi f_{yy} + \left({3\xxxi\yxi+3\yxxi\xxi}\right)f_{xy}\\
  %&+ \xxi^3f_{xxx}+\yxi^3f_{yyy}+3\xxi^2\yxi f_{xxy}+3\xxi\yxi^2f_{xyy}\\
 %\feeeta &= \xeeeta f_x + \yeeeta f_y + 3 \xeeta\xeta f_{xx}+
  %3 \yeeta\yeta f_{yy} + \left({3\xeeta\yeta+3\yeeta\xeta}\right)f_{xy}\\
  %&+ \xeta^3f_{xxx}+\yeta^3f_{yyy}+3\xeta^2\yeta f_{xxy}+3\xeta\yeta^2f_{xyy}
%\end{align*}

%Es ergibt sich:

%\begin{align*}
  %TE_x&= -\frac{y_{\eta}}{J} \frac{1}{6}\Big[\xxxxi f_x + \yxxxi f_y + 3 \xxxi\xxi f_{xx}+
  %3 \yxxi\yxi f_{yy} + \left({3\xxxi\yxi+3\yxxi\xxi}\right)f_{xy}\\
%&+ \xxi^3f_{xxx}+\yxi^3f_{yyy}+3\xxi^2\yxi f_{xxy}+3\xxi\yxi^2f_{xyy}\Big]\\
%&+  \frac{y_{\xi}}{J}\frac{1}{6} \Big[\xeeeta f_x + \yeeeta f_y + 3 \xeeta\xeta f_{xx}+
  %3 \yeeta\yeta f_{yy} + \left({3\xeeta\yeta+3\yeeta\xeta}\right)f_{xy}\\
  %&+ \xeta^3f_{xxx}+\yeta^3f_{yyy}+3\xeta^2\yeta f_{xxy}+3\xeta\yeta^2f_{xyy}
%\Big]\\
  %&= TE_{x1} + TE_{x2} + TE_{x3}
%\end{align*}

%wobei $TE_{x1}$die ersten Ableitungen von $f$ enthält, usw.:

%\begin{align*}
  %TE_{x1} &= \frac{1}{6\ J}\left[{
  %\left({-\yeta\xxxxi + \yxi\xeeeta}\right) f_x +
  %\left({-\yeta\yxxxi + \yxi\yeeeta}\right) f_y
  %}\right]\\
  %TE_{x2} &= \frac{1}{2\ J} \Big[
  %\left({-\yeta\xxxi\xxi + \yxi\xeeta\xeta}\right) f_{xx}+
  %\left({-\yeta\yxxi\yxi + \yxi\yeeta\yeta}\right) f_{yy}\\&+
  %\left({-\yeta \left({\xxxi\yxi+\yxxi\xxi}\right) +
  %\yxi \left({\xeeta\yeta+\yeeta\xeta}\right)}\right) f_{xy}
  %\Big]\\
  %TE_{x3}&=\frac{1}{6\ J} \Big[
  %\left({-\yeta\xxi^3+\yxi\xeta^3}\right) f_{xxx}+
  %\yxi \yeta\left({-\yxi^2+\yeta^2}\right) f_{yyy}\\&+
  %3 \yxi \yeta \left({-\xxi^2+\xeta^2}\right) f_{xxy}+
  %3 \yxi \yeta \left({-\xxi\yxi+\xeta\yeta}\right) f_{xyy}
  %\Big]
%\end{align*}




\section{Abbruchfehler für nicht-orthogonale Gitter}

Bei der Verwendung von orthogonalen Gittern vereinfachen sich viele Terme oder
heben sich mit der Gegenseite auf. Diese müssen aber im Falle eines nicht-orthogonalen
numerischen Gitters beachtet werden, soll die Berechnung realistische Werte liefern.

Da viele der zusätzlichen Terme auf der Verzerrtheit des Gitters beruhen und eine
Betrachtung im physikalischen Bereich verkomplizieren, wird eine Transformation auf
den logischen Bereich durchgeführt. Der Vorteil besteht darin, dass auf dem dortigen kartesischen
Gitter die Anwendung von Differenzenschemata einfach ist und man die zusätzlichen Fehler des
verzerrten Gitters vermeiden kann. Der Nachteil ist die zusätzliche Komplexität durch
die Transformation. Die Transformation eines Kontrollvolumens vom physikalischen in
den logischen Bereich ist in Abbildung~\ref{fig:no-trans} dargestellt.
\begin{figure}[ht]
  \begin{tikzpicture}[scale=1.3]
  \draw[->, thick] (-1,-0.5) -- (-0.5,-0.5) node[right] {$x$} coordinate(x axis);
  \draw[->, thick] (-1,-0.5) -- (-1,0) node[above] {$y$} coordinate(y axis);
  \fill[tud0a] (0,0) -- (2,0) -- (3,2) -- (-1, 3) --cycle;
  \draw[thick] (0,0) -- (2,0) -- (3,2) -- (-1, 3) --cycle;

  \fill (1,1.25) circle[radius=1.5pt];
  \node (x) at (1,1.25) [label=above:$P$] {};
  \fill (2,0) circle[radius=1.5pt];
  \node (x) at (2,0) [label=below right:$\vec x_{e1}$] {};
  \fill (3,2) circle[radius=1.5pt];
  \node (x) at (3,2) [label=right:$\vec x_{e2}$] {};

  \fill (-1,3) circle[radius=1.5pt];
  \fill (0,0) circle[radius=1.5pt];

  \fill (1,0) circle[radius=1pt];
  \node (x) at (1,0) [label=below:$s$] {};
  \fill (2.5,1) circle[radius=1.5pt];
  \node (x) at (2.5,1) [label=left:$e$] {};
  \fill (1,2.5) circle[radius=1pt];
  \node (x) at (1,2.5) [label=above:$n$] {};
  \fill (-0.5,1.5) circle[radius=1pt];
  \node (x) at (-0.5,1.5) [label=left:$w$] {};


  \draw[->, thick] (2.5,1) -> (3.5,0.5);
  \node (x) at (3.5,0.5) [label=above:$\vec n_e$] {};



  % xi-eta
  \draw[->, thick] (4.5,-0.5) -- (5,-0.5) node[right] {$\xi$} coordinate(x axis);
  \draw[->, thick] (4.5,-0.5) -- (4.5,0) node[above] {$\eta$} coordinate(y axis);
  \fill[tud0a] (5, 0) rectangle (7, 2);
  \draw[thick] (5, 0) rectangle (7, 2);
  \fill (7, 1)  circle[radius=1.5pt];
  \node (x) at (7,1) [label={right}:{$\tilde{e}$}] {};

  \fill (5, 0)  circle[radius=1.5pt];
  \fill (5, 2)  circle[radius=1.5pt];
  \fill (5, 1)  circle[radius=1pt];
  \fill (6, 2)  circle[radius=1pt];
  \fill (6, 0)  circle[radius=1pt];

  \fill (7, 0)  circle[radius=1.5pt];
  \node (x) at (7,0) [label=below:{$(\xi_{e1},\eta_{e1})$}] {};
  \fill (7, 2)  circle[radius=1.5pt];
  \node (x) at (7,2) [label=above:{$(\xi_{e2}, \eta_{e2})$}] {};

  \fill (6, 1)  circle[radius=1.5pt];
  \node (x) at (6,1) [label=above:{$\tilde{P}$}] {};
\end{tikzpicture}

\centering
\caption{Transformation eines Kontrollvolumens}
\label{fig:no-trans}
\end{figure}

Die Ableitung des Abbruchfehler soll hier beispielhaft an der östlichen Seite des Kontrollvolumens
gezeigt werden. Bei einer Transportgleichung mit konvektivem und diffusivem Term entsteht damit folgendes
Integral:
\begin{equation}
  \int_{\mathbf{x}_{e1}}^{\mathbf{x}_{e2}}\left({\rho v_i\phi+\alpha \pder[\phi]{x_i}}\right)n_{ei}ds
  \label{eq:int_trans}
\end{equation}
Hierbei bezeichnen $\mathbf{x}_{e1}$ und $\mathbf{x}_{e2}$ Anfangs- und Endpunkt, sowie $\mathbf{n}_e$
den Einheitsnormalenvektor der Ostseite wie in Abbildung~\ref{fig:no-trans} gezeigt.
Statt der direkten Auswertung erfolgt nun die Transformation ins logische Gebiet. Dazu wird die
Transformationsregel eines Wegintegrals genutzt.
\begin{equation}
  \int_{\gamma}f\ ds=\int_a^bf(\gamma (t)) \lVert \dot{\gamma}(t)\rVert_2 dt
\end{equation}
Bei der Koordinatentransformation ergibt sich, dass der Wert von $\xi$ auf der Ostseite konstant sein muss.
Die Transformation von Gleichung~\eqref{eq:int_trans} ergibt:
\begin{align*}
  \underbrace{
    \int_{\eta_{e1}}^{\eta_{e2}} \left({\rho v_1 \phi n_{e1}}\right)
  \left\lVert \frac{\partial x(\xi=e, \eta)}{\partial \eta} \right\rVert_2 d\eta
  }_I
  &+ \underbrace{
  \int_{\eta_{e1}}^{\eta_{e2}} \left({\alpha \pder[\phi]{x} n_{e1}}\right)
  \left\lVert \frac{\partial x(\xi=e, \eta)}{\partial \eta} \right\rVert_2 d\eta
}_{II}\\
  + \underbrace{
  \int_{\eta_{e1}}^{\eta_{e2}} (\rho v_2 \phi n_{e2})
  \left\lVert \frac{\partial x(\xi=e, \eta)}{\partial \eta} \right\rVert_2 d\eta
  }_{III}
  &+ \underbrace{
  \int_{\eta_{e1}}^{\eta_{e2}} \left({\alpha \pder[\phi]{x} n_{e2}}\right)
  \left\lVert \frac{\partial x(\xi=e, \eta)}{\partial \eta} \right\rVert_2 d\eta
  }_{IV}
\end{align*}
Hierbei beschreiben die Terme $I$ und $III$ die konvektiven Flüsse, die Terme $II$ und $IV$ sind
diffusive Flüsse. Bei $III$ und $IV$ handelt es sich um Terme, die auf orthogonalen Gittern wegfallen würden,
da hier die zweite Komponente des Einheitsnormalenvektors Null wäre.
Im Folgenden werden nun wieder Abbruchfehler der einzelnen Terme abgeleitet. Bei den Flüssen muss dabei
einerseits der Abbruchfehler der Integralapproximation über die Mittelpunktsregel ermittelt werden.
Andererseits muss der Abbruchfehler aus der Fluss- bzw. Ableitungsapproximation ermittelt werden.
\paragraph{Konvektive Terme}
Bei Integration mit der Mittelpunktsregel folgt für Term $I$:
\begin{equation*}
  C=(\rho v_1 \phi n_{e1})
  \left\lVert \frac{\partial x(\xi=e, \eta)}{\partial \eta} \right\rVert_2
\end{equation*}

\begin{align*}
  I&= C\big\vert_{\tilde{e}} \underbrace{(\etaii-\etai)}_{=1} + \frac{1}{2} \pder[C]{\eta}\bigg\vert_{\tilde{e}}
  \underbrace{\left({(\etaii-\etilde)^2-(\etai-\etilde)^2}\right)}_{=0}
  +\frac{1}{6} \frac{\partial^2 C}{\partial \eta^2}\bigg\vert_{\tilde{e}}
  \underbrace{\left({(\etaii-\etilde)^3-(\etai-\etilde)^3}\right)}_{=\frac{1}{4}} + HOT\\
  &= C \big\vert_{\tilde{e}}+ \underbrace{\frac{1}{24}  \frac{\partial^2 C}{\partial \eta^2}
\bigg\vert_{\tilde{e}}+HOT}_{TE_{int,konv}}
\end{align*}
Hierbei wird die Abkürzung $C$ zur Übersichtlichkeit eingeführt und $\tilde{e}$ bezeichnet den Mittelpunkt
der transformierten Ostseite (siehe Abbildung~\ref{fig:no-trans}).  Nach dem allgemeinen Vorgehen bei der
Methode der finiten Volumen muss nun $C\vert_{\tilde{e}}$ durch die umgebenden Zellenmittelpunkte ausgedrückt werden.
Dies soll hier mit dem Zentraldifferenzenschema gezeigt werden.
\begin{align*}
   C\big\vert_{\tilde{e}}
   =& C\big\vert_{\tilde{E}}
   \left({\frac{\xi_{\tilde{e}}-\xi_{\tilde{P}}}{\xi_{\tilde{E}}-\xi_{\tilde{P}}}}\right)
   + C\big\vert_{\tilde{P}} \left({1-\frac{\xi_{\tilde{e}}-\xi_{\tilde{P}}}{\xi_{\tilde{E}}-\xi_{\tilde{P}}} }\right)
   + \frac{1}{2} \frac{\partial^2 C}{\partial \xi^2}\bigg\vert_{\tilde{P}}
   (\xi_{\tilde{e}}-\xi_{\tilde{E}})(\xi_{\tilde{e}}-\xi_{\tilde{P}})\\
   &+ \frac{1}{6}  \frac{\partial^3 C}{\partial \xi^3}\bigg\vert_{\tilde{P}}
   \left({(\xi_{\tilde{e}}-\xi_{\tilde{P}})^2-(\xi_{\tilde{E}}-\xi_{\tilde{P}})^2}\right)
   (\xi_{\tilde{e}}-\xi_{\tilde{P}}) + HOT\\
   =&\frac{1}{2} C \big\vert_{\tilde{E}} + \frac{1}{2} C \big\vert_{\tilde{P}}
   \underbrace{- \frac{1}{8} \frac{\partial^2 C}{\partial \xi^2}\bigg\vert_{\tilde{P}}
   - \frac{1}{16} \frac{\partial^3 C}{\partial \xi^3}\bigg\vert_{\tilde{P}} + HOT}_{TE_{konv}}
\end{align*}



\paragraph{Diffusive Terme}
Bei Term $II$ wird ebenfalls die Mittelpunktsregel angewandt. Mit der Abkürzung $D$
ergibt sich hier:
\begin{equation*}
  D=  \left({\alpha \pder[\phi]{x} n_{e1}}\right)
  \left\lVert \frac{\partial x(\xi=e, \eta)}{\partial \eta} \right\rVert_2 d\eta
\end{equation*}

\begin{align*}
  II&= D\big\vert_{\tilde{e}} \underbrace{(\etaii-\etai)}_{=1} + \frac{1}{2}
  \frac{\partial^2 D}{\partial \eta^2} \bigg\vert_{\tilde{e}}
  \underbrace{\left({(\etaii-\etilde)^2-(\etai-\etilde)^2}\right)}_{=0}
  +\frac{1}{6} \frac{\partial^3 D}{\partial \eta^3}\bigg\vert_{\tilde{e}}
  \underbrace{\left({(\etaii-\etilde)^3-(\etai-\etilde)^3}\right)}_{=\frac{1}{4}} + HOT\\
  &= D \big\vert_{\tilde{e}}+ \underbrace{\frac{1}{24}  \frac{\partial^3 D}{\partial \eta^3}
\bigg\vert_{\tilde{e}}+HOT}_{TE_{int, dif}}
\end{align*}
Hier muss nun beachtet werden, dass $D$ selbst noch die Ableitung $\pder[\phi]{x}$ im physikalischen Gebiet enthält.
Diese muss wie das Kontrollvolumen ins logische Gebiet transformiert werden. Die Transformation ergibt
mit der Jakobimatrix aus Gleichung~\eqref{eq:detj} für
die Ableitung $\pder[\phi]{x}$:% und $\pder[\phi]{y}$:
\begin{align}
  \pder[\phi]{x}&=\pder[\xi]{x}\pder[\phi]{\xi}+\pder[\eta]{x}\pder[\phi]{\eta}
  = \frac{\yeta}{det(J)} \phi_{\xi} - \frac{\yxi}{det(J)} \phi_{\eta}
\end{align}
  %\pder[\phi]{y}&=\pder[\xi]{y}\pderf{\xi}+\pder[\eta]{y}\pderf{\eta}
  %=\frac{x_{\xi}}{det(J)}\phi_{\eta}-\frac{x_{\eta}}{det(J)} \phi_{\xi}
%\end{align}

Bei der anschließenden Approximation von $D_{\tilde{e}}$ durch die Mittelpunkte der Nachbarkontrollvolumen
muss ebenfalls die $\frac{\partial \phi}{\partial x}$ transformiert werden. Damit ergibt sich folgende Gleichung:



%\begin{equation}
  %D =\left({\frac{\alpha\ n_{e1}}{det\ J}}\right)
  %\left\lVert \frac{\partial x(\xi=e, \eta)}{\partial \eta} \right\rVert_2 
%\left({\yeta(\phi_{\tilde{E}}-\phi_{\tilde{P}}) - \yxi(\phi_{\tilde{ne}}-\phi_{\tilde{se}})}\right)
%\end{equation}

\begin{align*}
  D\big\vert_{\etilde} &=
  \left({\frac{\alpha\ n_{e1}}{det\ J}}\right)
  \left\lVert \frac{\partial x(\xi=e, \eta)}{\partial \eta} \right\rVert_2 
  \Bigg(\yeta(\phi_{\tilde{E}}-\phi_{\tilde{P}}) - \yxi(\phi_{\tilde{ne}}-\phi_{\tilde{se}})
      \underbrace{ -\frac{\yeta}{6} \frac{\partial^3 \phi}{\partial \xi^3} \bigg\vert_{\etilde}
      +\frac{\yeta}{6} \frac{\partial^3 \phi}{\partial \xi^3} \bigg\vert_{\etilde}
+ HOT}_{TE_{konv}}\Bigg)
\end{align*}
Auch $\phi_{\tilde{ne}}$ und $\phi_{\tilde{se}}$ müssen durch die umliegenden Zellenwerte ausgedrückt werden.
Dies kann beispielsweise durch lineare Interpolation erfolgen.
\begin{equation}
\phi_{\tilde{ne}} = \frac{\phi_{\tilde{P}} +\phi_{\tilde{E}} + \phi_{\tilde{NE}}+ \phi_{\tilde{N}}}{4} 
\end{equation}

In der oben stehenden Gleichung treten auch $\yxi$ und $\yeta$ auf. Diese werden Metriken der Transformation genannt und
sind nur vom Gitter, nicht aber von der Lösungsfunktion abhängig. Ihre Approximation
wird in Abschnitt~\ref{sec:verz-metrik} gezeigt.





\paragraph{Quellterm}

Zur Berechnung des Abbruchfehlers des Quellterm von nicht-orthogonalen Gittern wird der Transformationssatz benötigt.
Dieser lautet im zweidimensionalen Fall:
\begin{equation}
  \int \int_A f(x,y)\ dx\,dy = \int \int_P f\left({x(\xi,\,\eta),\ y(\xi,\,\eta)}\right)\ det(J)\ d\xi\,d\eta
\end{equation}
Angewandt auf den Quellterm und der Abkürzung $Q$ ergibt sich für den Quellterm:
\begin{equation}
  \begin{IEEEeqnarraybox}[][c]{rCl}
    \int\int_{P} \underbrace{\Pi_{\phi}(\xi, \eta) det(J) }_{Q} d\xi d\eta
    &=& Q\big\vert_{\tilde{P}} (\xi_e-\xi_w)(\eta_n-\eta_s)\\
  & &+ Q_{\xi}\big\vert_{\tilde{P}} \frac{(\xi_e-\xi_P)^2 - (\xi_w-\xi_P)^2}{2} (\eta_n-\eta_s)\\
  && + Q_{\eta}\big\vert_{\tilde{P}} \frac{(\eta_n-\eta_P)^2-(\eta_s-\eta_P)^2}{2} (\xi_e-\xi_w) \\
  & &+ \frac{1}{2} Q_{\xi\xi}\big\vert_{\tilde{P}}\frac{(\xi_e-\xi_P)^3 - (\xi_w-\xi_P)^3}{3} (\eta_n-\eta_s)\\
  &&+ \frac{1}{2} Q_{\eta\eta}\big\vert_{\tilde{P}} \frac{(\eta_n-\eta_P)^3-(\eta_s-\eta_P)^3}{3} (\xi_e-\xi_w) \\
  & &+ Q_{\xi\eta}\big\vert_{\tilde{P}} \frac{(\xi_e-\xi_P)^2 - (\xi_w-\xi_P)^2}{2} \cdot
  \frac{(\eta_n-\eta_P)^2-(\eta_s-\eta_P)^2}{2} + HOT\\
  &=&Q\big\vert_{\tilde{P}} + \underbrace{\frac{1}{24} \left({Q_{\xi\xi}\big\vert_{\tilde{P}}+Q_{\eta\eta}\big\vert_{\tilde{P}}}\right)
+ HOT}_{TE_{source}}
\end{IEEEeqnarraybox}
\end{equation}
Die gezeigten Ableitungen müssen nun für die anderen Seiten des Kontrollvolumens wiederholt werden.
Anschließend werden sie summiert und ergeben dann den Abbruchfehler eines Kontrollvolumens auf
einem nicht-orthogonalem Gitter.





\subsection{Diskretisierung der Metriken}
\label{sec:verz-metrik}

In den oben gezeigten Ableitungen treten mehrfach Größen auf, die nur vom numerischen Gitter, nicht aber
von der Lösungsfunktion abhängen. Diese werden Metriken genannt.
Meistens handelt es sich hierbei um physikalische Variablen, die nach logischen Koordinaten abgeleitet werden.
Beispiele sind $x_{\xi}$ oder $y_{\eta\eta}$.

Auch die Metriken sind analytische Größen und müssen im Rahmen der Berechnung diskretisiert werden.
Treten die Metriken hierbei bei einer Approximation über die Grenzen des Kontrollvolumens hinweg auf,
ist die Diskretisierung verschieden zur Approximation innerhalb der Grenzen eines Kontrollvolumens.

\paragraph{Diskretisierung innerhalb eines Kontrollvolumens}
\begin{figure}[ht]
  \begin{tikzpicture}[scale=1.2]
  \draw[->, thick] (-2,0) -- (-1.5,0) node[right] {$x$} coordinate(x axis);
  \draw[->, thick] (-2,0) -- (-2,0.5) node[above] {$y$} coordinate(y axis);

  \fill[tud0a] (0,0) -- (3,0.5) -- (3.5,3.5) -- (0.5, 3) --cycle;

  \draw[thick] (-0.5,-0.08333) -- (4,0.66666);
  \draw[thick] (-0.08333,-0.5) -- (0.66666,4);
  \draw[thick] (-0.5,-0.16666+3) -- (4.5,3.66666);
  \draw[thick] (-.08333+3,0) -- (0.5833+3,4);

  \draw[thick,<->] (15mm,0mm) arc (0:9.5:15mm);
  \draw[thick,<->] (0mm,15mm) arc (90:80.5:15mm);
  \draw[thick,<->] (9mm,1.5mm) arc (9.5:81:9mm);

  \node (x) at (1.5,0.05) [label=below:$\gamma$]{};
  \node (x) at (0.05,1.5) [label=left:$\beta$]{};
  \node (x) at (0.5,0.5) [label=above right:$\theta$]{};

  \node (x) at (2.5,0.3) [label=above left:$a$]{};
  \node (x) at (0.3,2.5) [label=below right:$b$]{};

  %\node (x) at (4,0.66666) [label=right:{$\eta=j-\frac{1}{2}$}]{};
  %\node (x) at (4.5,3.66666) [label=right:{$\eta=j+\frac{1}{2}$}]{};
  %\node (x) at (0.66666,4) [label=left:{$\xi=i-\frac{1}{2}$}]{};
  %\node (x) at (0.66666+3,4) [label=left:{$\xi=i+\frac{1}{2}$}]{};

  \draw[thick] (0,0) -- (2,0);
  \draw[thick] (0,0) -- (0,2);
\end{tikzpicture}

\centering
\caption{Größen für die Berechnung der Metriken innerhalb des Kontrollvolumens}
\end{figure}
Die Metriken innerhalb eines Kontrollvolumens sind beispielsweise für die Transformation
des Quellterms nötig. Zur Diskretisierung der Metriken werden
Differenzenquotienten als Approximationen genutzt\cite{lee}.
Beispielsweise ergibt sich auf einem zweidimensionalen nicht-orthogonalen
Gitter mit $\Delta\xi = \Delta \eta = 1$:
\begin{equation}
  x_{\xi} = \frac{x_{i+1, j} - x_{i-1,j}}{\Delta \xi} = x_{i+1, j} - x_{i-1,j} = a \cos \gamma
\end{equation}
Für die anderen Metriken erster Ordnung ergeben sich:
\begin{align*}
  x_{\eta} &= b\sin \beta\\
  y_{\xi} &= a \sin \gamma\\
  y_{\eta} &= b \cos \beta
\end{align*}
Eine weitere benötige Größe stellt die Determinante der Jakobimatrix
dar. Sie ist für jedes Kon\-troll\-volumen unterschiedlich. Werden
die diskretisierten Metriken eingesetzt ergibt sich folgende Gleichung:
\begin{align}
  det(J) &= x_{\xi}y_{\eta}-x_{\eta}y_{\xi}\nonumber\\
    &= a \cos \gamma \, b \cos \beta - 
       a \sin \beta \, b \sin \gamma\nonumber\\
       &= ab(\cos\gamma\cos\beta-\sin\beta\sin\gamma)
\end{align}
Hier kann nun das folgende Additionstheorem
angewendet werden.
\begin{equation*}
\cos(\beta+\gamma)=\cos\gamma\cos\beta-\sin\beta\sin\gamma
\end{equation*}
Da weiterhin bekannt ist, dass $\beta+\gamma
+\theta = 90^{\circ}$ ist, ergibt sich mit der Beziehung
$\sin(x)=\cos(\frac{\pi}{2} -x)$ letztendlich aus
Gleichung~\eqref{eq:detj} der folgende Term für $J$. Er stellt den Betrag des
Kreuzprodukts der beiden
Vektoren dar,
die das Kontrollvolumen aufspannen.
Es handelt sich somit um eine Approximation des Flächeninhalts
des Kontrollvolumens.
\begin{equation}
  det(J) = a b \sin \theta
\end{equation}

\paragraph{Diskretisierung über Kontrollvolumengrenzen}

Bei der Approximation der diffusiven und konvektiven Flüsse werden meist
auch die Grenzen eines Kontrollvolumens überschritten. Damit kommt auch
die Geometrie des Nachbarkontrollvolumens zum tragen und die Diskretisierung der
Metriken muss angepasst werden.
\begin{figure}[ht]
  \begin{tikzpicture}[scale=0.8]
  \draw[->, thick] (-1.5,0) -- (-1,0) node[right] {$x$} coordinate(x axis);
  \draw[->, thick] (-1.5,0) -- (-1.5,0.5) node[above] {$y$} coordinate(y axis);

  \fill[tud0a] (0,1) -- (2,0.5) -- (2.5,2) -- (0.5, 3) --cycle;
  \draw[thick] (0,1) -- (2,0.5) -- (2.5,2) -- (0.5, 3) --cycle;
  \draw[thick] (4,-0.5) -- (2,0.5) -- (2.5,2) -- (5, 2.5) --cycle;

  %\fill (2,0.5,0) circle[radius=1.5pt];
  %\node (x) at (2,0.5) [label=below left:$se$] {};
  %\fill (2.5,2) circle[radius=1.5pt];
  %\node (x) at (2.5,2) [label=below right:$ne$] {};
  \fill (1.25,1.625) circle[radius=1.5pt];
  \node (x) at (1.25,1.625) [label=above:$P$] {};
  %\node (x) at (1.25,1.625) [label=below:{$(\xi_{i,j},\ \eta_{i,j})$}] {};
  \fill (3.25,1.125) circle[radius=1.5pt];
  \node (x) at (3.25,1.125) [label=above:$E$] {};
  %\node (x) at (3.25,1.125) [label=below:{$(\xi_{i+1,j},\ \eta_{i+1,j})$}] {};
  \fill (2.25,1.25) circle[radius=1.5pt];
  \node (x) at (2.25,1.25) [label=above left:$e$] {};

  %\draw[thick,->] (-0.75,2.125) -- (5.25,0.625);
  %\node (x) at (5.25,0.625) [label=right:$\xi$] {};

  %\draw[thick,->] (1.75,-0.25) -- (2.75,2.75);
  %\node (x) at (2.75,2.75) [label=right:$\eta$] {};
\end{tikzpicture}

\centering
\caption{Berechnung der Metriken über Grenzen von Kontrollvolumen}
\end{figure}

Die Ableitungen nach $\xi$ werden nach folgendem Schema approximiert, wobei
$\vert \mathbf{x}_E -\mathbf{x}_P \vert$ den Abstand der Punkte $E$ und $P$ beschreibt.
\begin{equation}
  \pder[x]{\xi} \approx \frac{x_E-x_P}{\vert \mathbf{x}_E -\mathbf{x}_P \vert} \qquad
  \text{und} \qquad \pder[y]{\xi} \approx \frac{y_E-y_P}{\vert \mathbf{x}_E -\mathbf{x}_P \vert}
\end{equation}
Die Ableitungen nach $\eta$ ergeben sich mit der Seitenlänge $\delta S_e$ der Ostseite
zu:
\begin{equation}
  \pder[x]{\eta} \approx \frac{x_{ne}-x_{se}}{\delta S_e} \qquad
  \text{und} \qquad \pder[y]{\eta} \approx \frac{y_{ne}-y_{se}}{\delta S_e}
\end{equation}

\cleardoublepage

\chapter{Verifizierung des Abbruchfehlerindikators}
%\section{Testfälle}
\subsection{Eindimensionale Testfälle}

Die Testfälle im eindimensionalen Raum setzen als Problemgebiet das
Intervall $[0,1]$ fest. Eine Transformation von diesem Intervall auf jedes 
andere ist möglich.

Der erste eindimensionale Testfall stellt eine symmetrische Sinusfunktion
dar. Es wird eine halbe Periode genutzt. Um auch die Randbedingungen zu testen,
wird sie eine Einheit nach oben geschoben.
\begin{equation}
f(x) = \sin(\pi x)+1
\end{equation}
\begin{figure}[h]
\begin{tikzpicture}
\begin{axis}[
xlabel=$x$,
ylabel={$f(x)$},
domain=0:1,
height=5cm,
width=0.6*\textwidth
]
% use TeX as calculator:
\addplot +[mark=none, samples=100,very thick]{sin(pi*deg(x))};
\end{axis}
\end{tikzpicture}
\centering
\caption{Testfunktion 1 im Intervall [0,1]}
\label{fig:testfn1}
\end{figure}


Der zweite Testfall stellt eine unsymmetrische Funktion dar. Es handelt sich um
die halbe Periode der Kosinusfunktion, welche um eine Einheit nach unten verschoben wurde.
\begin{equation}
f(x) = \cos(\pi x) -1
\end{equation}
\begin{figure}[h]
\begin{tikzpicture}
\begin{axis}[
xlabel=$x$,
ylabel={$f(x)$},
domain=0:1,
height=5cm,
width=0.6*\textwidth
]
\addplot +[mark=none,samples=100, very thick]{cos(pi*deg(x))-1};
\end{axis}
\end{tikzpicture}
\centering
\caption{Testfunktion 2 im Intervall [0,1]}
\label{fig:testfn2}
\end{figure}

\clearpage
\subsection{Zweidimensionale Testfälle}

Auch die zweidimensionalen Testfälle legen als Problemgebiet das Intervall
$[0,1]\times[0,1]$ fest. Wie im Eindimensionalen kann dieses durch Transformation
in jedes andere überführt werden.

Der erste zweidimensionale Testfall ist wiederrum symmetrisch und testet alle
Randbedingungen.
\begin{equation}
f(x,y) = \sin(\pi x) \sin(\pi y) + 1
\end{equation}

\begin{figure}[h]
\centering
\begin{subfigure}[b]{.5\linewidth}
\centering
\begin{tikzpicture}
\begin{axis}[
xlabel=$x$,
ylabel=$y$,
%zlabel={$f(x,y) = \sin(\pi x) \sin(\pi y) + 1$},
domain=0:1,
height=7cm
]
\addplot3 +[surf, mark=none, samples=20]{sin(pi*deg(x))*sin(pi*deg(y))+1};
\end{axis}
\end{tikzpicture}
%\subcaption{Oberflächen}\label{fig:1a}
\end{subfigure}%
\begin{subfigure}[b]{.5\linewidth}
\centering
\begin{tikzpicture}
\begin{axis}[
view={0}{90},
xlabel=$x$,
ylabel=$y$,
zlabel={$f(x,y) = \sin(\frac{\pi}{2} x) \cos(\frac{\pi}{2} y)$},
domain=0:1,
height=7cm
]
\addplot3 +[contour prepared,very thick,mark=none, contour prepared format=matlab]
file {data/3_contour.txt};
\end{axis}
\end{tikzpicture}
%\subcaption{Another subfigure}\label{fig:1b}
\end{subfigure}
\caption{Testfunktion 3 im Intervall [0,1]$\times$[0,1]}\label{fig:1}
\end{figure}

Die zweite zweidimensionale Testfunktion ist unsymmetrisch. Ihre Gleichung lautet:
\begin{equation}
f(x,y) = \sin(\frac{\pi}{2} x) \sin(\frac{\pi}{2} y)
\end{equation}
\begin{figure}[h]
\centering
\begin{subfigure}[b]{.5\linewidth}
\centering
\begin{tikzpicture}
\begin{axis}[
%view={30+180}{30},
xlabel=$x$,
ylabel=$y$,
%zlabel={$f(x,y) = \sin(\frac{\pi}{2} x) \cos(\frac{\pi}{2} y)$},
domain=0:1,
height=7cm,
]
\addplot3 +[surf, mark=none, samples=20]{sin(pi/2*deg(x))*sin(pi/2*deg(y))};
\end{axis}
\end{tikzpicture}
%\subcaption{Oberflächen}\label{fig:1a}
\end{subfigure}%
\begin{subfigure}[b]{.5\linewidth}
\centering
\begin{tikzpicture}
\begin{axis}[
view={0}{90},
xlabel=$x$,
ylabel=$y$,
zlabel={$f(x,y) = \sin(\frac{\pi}{2} x) \sin(\frac{\pi}{2} y)$},
domain=0:1,
height=7cm
]
\addplot3 +[contour prepared,very thick,mark=none, contour prepared format=matlab]
file {data/4_contour.txt};
\end{axis}
\end{tikzpicture}
%\subcaption{Another subfigure}\label{fig:1b}
\end{subfigure}
\caption{Testfunktion 4 im Intervall [0,1]$\times$[0,1]}\label{fig:1}
\end{figure}

\clearpage

%\section{Verifikation der Lösungsprogramme}

Zur Berechnung des Abbruchfehlers wird die Lösung des mit finiten Volumen
diskretisierten Problems benötigt. Dafür wurden Lösungsprogramme
für ein- und zweidimensionale Probleme in Matlab implementiert.
Diese arbeiten mit orthogonalen Gittern und können reine Diffusionsprobleme
oder kombinierte Diffusions- und Konvektionsprobleme lösen.

Vor der Verifizierung des abgeleiteten Indikators für den Abbruchfehler muss daher die
korrekte Funktion der Programme nachgewiesen werden. Dies geschieht wie in Abschnitt
\ref{sec:verifik_allg} beschrieben über den Vergleich der formalen mit der beobachteten
Konvergenzordnung.

Bei der Berechnung der beobachteten Konvergenzordnung muss beachtet werden, dass mit dem
feinsten Gitter begonnen wird und dieses schrittweise ausgedünnt wird. Andernfalls
wird die Konvergenzordnung falsch bestimmt.

Die beobachtete Konvergenzordnung lässt sich neben den berechneten Zahlenwerten auch gut
an der Steigung in einem doppelt-logarithmischen Diagramm ablesen. Entsteht dort eine
Gerade, so bedeutet das, dass die Lösung mit einen festen Koeffizienten konvergiert.

\paragraph{1D Diffussion und CDS, nicht äquidistant}

\begin{table}[h]
  \begin{tabular}{r r l}
  \toprule
  N & Summierter Fehler & Beobachtete Konvergenzordnung \\
  \midrule
  5  & $1,27\cdot10^{-1}$ & \multirow{2}{*}{2,73}\\
  10 & $1.91\cdot10^{-2}$ & \multirow{2}{*}{2,22}\\
  20 & $4.07\cdot10^{-3}$ & \multirow{2}{*}{2,02}\\
  40 & $9.96\cdot10^{-4}$ & \\
  \bottomrule
\end{tabular}
\caption{Daten}
\end{table}

\begin{figure}[h]
  \begin{tikzpicture}
  \begin{loglogaxis}[%xlabel=Anzahl der Kontrollvolumen
    %$N$,ylabel=Gemittelter Fehler,
  width=0.55\textwidth]
  \addplot[color=tud2d,mark=*, very thick] coordinates {
    (5, 1.2662511400e-01)
    (10,1.9068695077e-02)
    (20,4.0741696509e-03)
    (40,9.9564249986e-04)
  };
  \end{loglogaxis}
\end{tikzpicture}
\caption{Fehler}
\end{figure}

%\begin{figure}[ht]
%\begin{floatrow}

%\capbtabbox{%
%\begin{tabular}{r r r}
  %\toprule
  %N & Summierter Fehler & Konvergenzordnung \\
  %\midrule
  %5  & $1,27*10^{-1}$ & \multirow{2}{*}{333}\\
  %10 & $1.91*10^{-1}$ & \multirow{2}{*}{333}\\
  %20 & $4.07*10^{-1}$ & \multirow{2}{*}{333}\\
  %40 & $9.96*10^{-1}$ & \\
  %\bottomrule
%\end{tabular}
%}{%
  %\caption{A table}%
%}
%\ffigbox{%
%\begin{tikzpicture}
%\begin{loglogaxis}[%xlabel=Anzahl der Kontrollvolumen
  %%$N$,ylabel=Gemittelter Fehler,
%width=0.55\textwidth]
%\addplot[color=tud2d,mark=*, very thick] coordinates {
  %(5, 1.2662511400e-01)
  %(10,1.9068695077e-02)
  %(20,4.0741696509e-03)
  %(40,9.9564249986e-04)
%};
%\end{loglogaxis}
%\end{tikzpicture}
%}{%
  %\caption{A figure}%
%}
%\end{floatrow}
%\end{figure}


%\section{Eindimensionale Testfälle}

\paragraph{Testfall 1}
\noindent
Der erste Testfall ist ein reines Diffusionsproblem auf kartesischem
Gitter. Die Testfunktion ist Nummer~2 (siehe Abbildung~\ref{fig:testfn2}).
Die Anzahl der Kontrollvolumen beträgt $N=20$ und der Diffusionskoeffizient
ist $D = 1$.
\begin{figure}[ht]
\centering
   \begin{subfigure}{0.49\linewidth} \centering
  \begin{tikzpicture}
    \begin{axis}[width=\textwidth, scaled y ticks=true,
    xlabel=$x$,
    ylabel=Absoluter Fehler]
      \addplot[tud2d, mark=*, very thick] file {data/1/1_cos_aqui_err.txt};
    \end{axis}
  \end{tikzpicture}
     \caption{Absoluter Fehler}
   \end{subfigure}
   \begin{subfigure}{0.49\linewidth} \centering
     \vspace{15pt}
  \begin{tikzpicture}
    \begin{axis}[width=\textwidth, scaled y ticks=true,
    xlabel=$x$,
    ylabel=Residuum/Abbruchfehler]
      \addplot[tud9c, mark=*, very thick] file {data/1/1_cos_aqui_te.txt};
      \addplot[tud2d, mark=*, very thick] file {data/1/1_cos_aqui_res.txt};
      \legend{Abbruchfehler, Residuum}
    \end{axis}
  \end{tikzpicture}
  \caption{Residuum der Lösung und Abbruchfehler}
   \end{subfigure}
   \caption{Lösungseigenschaften (Testfall 1)}
\end{figure}
Man sieht gut das der absolute Lösungsfehler schon bei zwanzig Kontrollvolumen klein ist.
Außerdem hat das Residuum am Rand Sprünge. Diese werden auch vom Abbruchfehlerindikator abgebildet,
durch die Verwendung einseitiger Differenzenquotienten jedoch mit verringerter Genauigkeit.
Die Sprünge werden in den folgenden Diagrammen teilweise ausgeblendet um
Residuum und Abbruchfehler besser vergleichen zu können.

\begin{figure}[ht]
\centering
   \begin{subfigure}{0.49\linewidth} \centering
  \begin{tikzpicture}
    \begin{axis}[width=\textwidth,
    xlabel=$x$,
    ylabel=RES--TE]
      \addplot[tud2d, mark=*, very thick] file {data/1/1_cos_aqui_teres.txt};
    \end{axis}
  \end{tikzpicture}
  \caption{Mit Randsprüngen}
   \end{subfigure}
   \begin{subfigure}{0.49\linewidth} \centering
  \begin{tikzpicture}
    \begin{axis}[width=\textwidth,
    xlabel=$x$,
    ylabel=RES--TE]
      \addplot[tud2d, mark=*, very thick] file {data/1/1_cos_aqui_teres2.txt};
    \end{axis}
  \end{tikzpicture}
  \caption{Ohne Randsprünge}
   \end{subfigure}
   \caption{Differenz Residuum und Abbruchfehler (Testfall 1)}
\end{figure}
\noindent
Auf feineren Gittern erwartet man einen kleineren Abbruchfehler. Dies
wird durch die folgenden Diagramme bestätigt, in denen der Abbruchfehler auf
Gittern mit $N=10$ und $N=40$ mit dem Abbruchfehler des Testfalles verglichen werden.
Die Verbesserung beträgt ungefähr eine halbe Ordnung pro Verdoppelung der Schrittweite.
\begin{figure}[ht]
\centering
\begin{tikzpicture}
  \begin{axis}[width=0.7\textwidth, scaled y ticks=false,
  xlabel=$x$,
  ylabel=Abbruchfehler]
    \addplot[tud2d, mark=*, very thick] file {data/1/1_cos_aqui_te2.txt};
    %\addplot[tud2d, mark=+, thick] file {data/1/1_cos_aqui_res2.txt};
  \end{axis}
\end{tikzpicture}
\caption{Abbruchfehler ohne Randsprünge (Testfall 1)}
\end{figure}
\begin{figure}[ht]
\centering
   \begin{subfigure}{0.49\linewidth} \centering
  \begin{tikzpicture}
    \begin{axis}[width=\textwidth,
    xlabel=$x$,
    ylabel=Abbruchfehler]
      \addplot[tud2d, mark=*, very thick] file {data/5/cos_kart_10.txt};
    \end{axis}
  \end{tikzpicture}
  \caption{N=10}\label{fig:figA}
   \end{subfigure}
   \begin{subfigure}{0.49\linewidth} \centering
  \begin{tikzpicture}
    \begin{axis}[width=\textwidth,
    xlabel=$x$,
    ylabel=Abbruchfehler]
      \addplot[tud2d, mark=*, very thick] file {data/5/cos_kart_40.txt};
    \end{axis}
  \end{tikzpicture}
  \caption{N=40}\label{fig:figB}
   \end{subfigure}
   \caption{Abbruchfehler ohne Randsprünge für veränderte Gitter (Testfall 1)}
\end{figure}

\vspace{2cm}


\paragraph{Testfall 2}
\noindent
Der zweite Testfall verwendet Testfunktion 1 auf einem orthogonalen Gitter mit
Expansionsfaktor $\alpha=0,9$. Es gibt wieder zwanzig Kontrollvolumen.

\begin{figure}[ht]
\centering
   \begin{subfigure}{0.49\linewidth} \centering
  \begin{tikzpicture}
    \begin{axis}[width=\textwidth,
    xlabel=$x$,
    ylabel=Absoluter Fehler]
      \addplot[tud2d, mark=*, very thick] file {data/2/2_sin_var_err.txt};
    \end{axis}
  \end{tikzpicture}
     \caption{Absoluter Fehler}
   \end{subfigure}
   \begin{subfigure}{0.49\linewidth} \centering
  \begin{tikzpicture}
    \begin{axis}[width=\textwidth,
    xlabel=$x$,
    ylabel=Residuum/Abbruchfehler]
      \addplot[tud9c, mark=*, very thick] file {data/2/2_sin_var_te.txt};
      \addplot[tud2d, mark=*, very thick] file {data/2/2_sin_var_res.txt};
      \legend{Abbruchfehler, Residuum}
    \end{axis}
  \end{tikzpicture}
  \caption{Residuum der Lösung und Abbruchfehler}
   \end{subfigure}
   \caption{Lösungseigenschaften (Testfall 2)}
\end{figure}
\noindent
Die Abhängigkeit des absoluten Fehlers vom Gitter ist gut zu erkennen. Im groben
Bereich steigt der Fehler, während er im fein aufgelösten Bereich sinkt.
Auch Abbruchfehler und Residuum stimmen gut überein. Die verminderte Genauigkeit
der einseitigen Differenzenquotienten am Rand bewirkt wieder Abweichungen an den Rändern.
%\begin{figure}[ht]
%\centering
%\begin{tikzpicture}
  %\begin{axis}[width=0.7\textwidth]
  %%xlabel=$x$,
  %%ylabel=$y$]
    %\addplot[tud2d, mark=*, very thick] file {data/2/2_sin_var_te2.txt};
    %%\addplot[tud2d, mark=+, thick] file {data/2/2_sin_var_res2.txt};
  %\end{axis}
%\end{tikzpicture}
%\caption{Abbruchfehler ohne Randsprünge} \label{fig:twofigs}
%\end{figure}
\begin{figure}[ht]
\centering
   \begin{subfigure}{0.49\linewidth} \centering
  \begin{tikzpicture}
    \begin{axis}[width=\textwidth,
    xlabel=$x$,
    ylabel=RES--TE]
      \addplot[tud2d, mark=*, very thick] file {data/2/2_sin_var_teres.txt};
    \end{axis}
  \end{tikzpicture}
  \caption{Mit Randsprüngen}
   \end{subfigure}
   \begin{subfigure}{0.49\linewidth} \centering
  \begin{tikzpicture}
    \begin{axis}[width=\textwidth,
    xlabel=$x$,
    ylabel=RES--TE]
      \addplot[tud2d, mark=*, very thick] file {data/2/2_sin_var_teres2.txt};
    \end{axis}
  \end{tikzpicture}
  \caption{Ohne Randsprünge}
   \end{subfigure}
   \caption{Differenz Residuum und Abbruchfehler (Testfall 2)}
\end{figure}

Eine weitere Bedingung ist, dass der Abbruchfehler unabhängig von
Diffusions- und Konvektionskoeffizient ist. Dazu wird jeweils einer der beiden Koeffizienten
um eine Ordnung vergrößert. Die Ergebnisse sind in den folgenden Diagrammen zu sehen.
Man erkennt, das die Genauigkeit im Vergleich zu Koeffizienten gleicher Größe
um etwa eine Größenordnung abnimmt.
\begin{figure}[ht]
\centering
   \begin{subfigure}{0.49\linewidth} \centering
  \begin{tikzpicture}
    \begin{axis}[width=\textwidth,
    xlabel=$x$,
    ylabel=RES--TE]
      \addplot[tud2d, mark=*, very thick] file {data/5/sin_orth_DIF10.txt};
    \end{axis}
  \end{tikzpicture}
  \caption{DIV=10, KONV=1}
   \end{subfigure}
   \begin{subfigure}{0.49\linewidth} \centering
  \begin{tikzpicture}
    \begin{axis}[width=\textwidth,
    xlabel=$x$,
    ylabel=RES--TE]
      \addplot[tud2d, mark=*, very thick] file {data/5/sin_orth_KONV10.txt};
    \end{axis}
  \end{tikzpicture}
  \caption{DIF=1, KONV=10}
   \end{subfigure}
   \caption{Differenz Residuum und Abbruchfehler mit angepassten Koeffizienten (Testfall 2)}
\end{figure}
\clearpage

%\section{Zweidimensionale Testfälle}

\paragraph{Testfall 3}

\begin{figure}[h]
\centering
\begin{subfigure}[b]{.5\linewidth}
\centering
\begin{tikzpicture}
\begin{axis}[
%view={30+180}{30},
xlabel=$x$,
ylabel=$y$,
%zlabel={$f(x,y) = \sin(\frac{\pi}{2} x) \cos(\frac{\pi}{2} y)$},
domain=0:1,
height=7cm,
width=\textwidth
]
\addplot3[surf, mesh/ordering=y varies, faceted color=black] file{data/3/ERR_data.txt};
\end{axis}
\end{tikzpicture}
%\subcaption{Oberflächen}\label{fig:1a}
\end{subfigure}%
\begin{subfigure}[b]{.5\linewidth}
\centering
\begin{tikzpicture}
\begin{axis}[
view={0}{90},
xlabel=$x$,
ylabel=$y$,
zlabel={$f(x,y) = \sin(\frac{\pi}{2} x) \sin(\frac{\pi}{2} y)$},
domain=0:1,
height=7cm
]
\addplot3 +[contour prepared,very thick,mark=none, contour prepared format=matlab]
file {data/3/ERR_contour.txt};
\end{axis}
\end{tikzpicture}
%\subcaption{Another subfigure}\label{fig:1b}
\end{subfigure}
\caption{Absoluter Fehler (Testfall 3)}
\end{figure}


\begin{figure}[h]
\centering
\begin{subfigure}[b]{.5\linewidth}
\centering
\begin{tikzpicture}
\begin{axis}[
%view={30+180}{30},
xlabel=$x$,
ylabel=$y$,
height=6cm,
width=\textwidth
]
\addplot3[surf, mesh/ordering=y varies, faceted color=black] file{data/3/RES_data.txt};
\end{axis}
\end{tikzpicture}
\subcaption{Residuum}\label{fig:1a}
\end{subfigure}%
\begin{subfigure}[b]{.5\linewidth}
\centering
\begin{tikzpicture}
\begin{axis}[
%view={30+180}{30},
xlabel=$x$,
ylabel=$y$,
height=6cm,
width=\textwidth
]
\addplot3[surf, mesh/ordering=y varies, faceted color=black] file{data/3/TE_data.txt};
\end{axis}
\end{tikzpicture}
\subcaption{Abbruchfehler}\label{fig:1b}
\end{subfigure}
\caption{Residuum und Abbruchfehler mit Randsprüngen (Testfall 3)}
\end{figure}



\begin{figure}[h]
\centering
\begin{subfigure}[b]{.5\linewidth}
\centering
\begin{tikzpicture}
\begin{axis}[
view={150}{30},
xlabel=$x$,
ylabel=$y$,
height=6cm,
width=\textwidth
]
\addplot3[surf, mesh/ordering=y varies, faceted color=black] file{data/3/RES2_data.txt};
\end{axis}
\end{tikzpicture}
\subcaption{Residuum}\label{fig:1a}
\end{subfigure}%
\begin{subfigure}[b]{.5\linewidth}
\centering
\begin{tikzpicture}
\begin{axis}[
view={150}{30},
xlabel=$x$,
ylabel=$y$,
height=6cm,
width=\textwidth
]
\addplot3[surf, mesh/ordering=y varies, faceted color=black] file{data/3/TE2_data.txt};
\end{axis}
\end{tikzpicture}
\subcaption{Abbruchfehler}\label{fig:1b}
\end{subfigure}
\caption{Residuum und Abbruchfehler ohne Randsprünge (Testfall 3)}
\end{figure}






\begin{figure}[h]
\centering
\begin{subfigure}[b]{.5\linewidth}
\centering
\begin{tikzpicture}
\begin{axis}[
%view={170}{30},
xlabel=$x$,
ylabel=$y$,
height=6cm,
width=\textwidth
]
\addplot3[surf, mesh/ordering=y varies, faceted color=black] file{data/3/RESTE_data.txt};
\end{axis}
\end{tikzpicture}
\subcaption{Mit Randsprüngen}\label{fig:1a}
\end{subfigure}%
\begin{subfigure}[b]{.5\linewidth}
\centering
\begin{tikzpicture}
\begin{axis}[
view={180+20}{35},
xlabel=$x$,
ylabel=$y$,
height=6cm,
width=\textwidth
]
\addplot3[surf, mesh/ordering=y varies, faceted color=black] file{data/3/RESTE2_data.txt};
\end{axis}
\end{tikzpicture}
\subcaption{Ohne Randsprünge}\label{fig:1b}
\end{subfigure}
\caption{Differenz Residuum und Abbruchfehler (Testfall 3)}
\end{figure}



\cleardoublepage

\chapter{Anwendung und Vergleich des Abbruchfehlerschätzers}

Zur Adaption des numerischen Gitters stehen verschiedene Verfahren zu Verfügung~\cite{roy2}.
Zum Einen können die vorhandenen Gitterpunkte verschoben werden. Dieses Verfahren wird r-Adaption
genannt. Fügt man weitere Gitterpunkte zum Gitter hinzu, so handelt es sich um eine h-Adaption.
Zuletzt kann lokal die Ordnung der Approximationen erhört werden, was p-Adaption genannt wird.
Es existieren zudem Mischformen der oben genannten Verfahren.

Heutzutage wird Gitterverfeinerung hauptsächlich mittels h-Adaption durchgeführt. Im Gegensatz dazu
soll hier eine r-Adaption verwendet werden.

Im Folgenden sollen nun verschiedene Lösungseigenschaften zur Gitteradaption verglichen werden.
Ausgangspunkt ist dabei immer ein Gitter mit zwanzig Kontrollvolumen, dass mit dem
Expansionsfaktor $\alpha=0.5$ erzeugt wurde. Die Anzahl der Iterationen ist auf 350 festgelegt.
Als Lösung wurde Testfunktion 2 gewählt. Damit ergeben sich Randwerte von $x(0)=0$ und $x(1)=-2$.
Die betrachteten Verfahren sind:
\begin{itemize}
  \item Lösungsgradient
  \item Abbruchfehler
  \item Richardson-Extrapolations-Indikator
  \item Zweigitter-Fehlerschätzer
\end{itemize}
Weitere Informationen zu den Indikatoren können in \cite{celik} gefunden werden.




\section{Algorithmus zur Gitteradaption}

Der Algorithmus zur Gitteranpassung basiert hierbei auf der Analogie zu einem System
linearer Federn.
Zuerst wird eine Lösungsanpassungsfunktion $\theta_i$ erstellt, die auf den gewünschten
Lösungseigenschaften wie zum Beispiel Ableitung, Krümmung oder Abbruchfehler basiert. Hierbei
bezeichnet $i$ den betrachteten Gitterpunkt. Ist die gewählte Lösungseigenschaft nur in den
Mittelpunkten der Kontrollvolumen bekannt, kann sie beispielsweise durch lineare Interpolation
auf die Gitterpunkte überführt werden.

Anschließend wird aus der Anpassungsfunktion $\theta_i$ eine Gewichtungsfunktion
$W_i$ erzeugt.
Der frei wählbare Exponent $q$ wird in dieser Arbeit auf eins gesetzt.
\begin{equation}
  W_i = \vert\theta_i \vert^q
\end{equation}
Die Gewichtungsfunktion wird zur Steuerung des Adaptionsprozesses genutzt, wobei größere
Werte eine Gitterverfeinerung und kleinere Werte eine Weitung des Gitters bewirken. Um
weiche Änderungen der Größen von Nachbarzellen zu erreichen, wird die Gewichtungsfunktion
anschließend geglättet. Dies geschieht iterativ über folgende Formel:
\begin{equation}
  W_i = \frac{W_{i-1} + 4W_i + W_{i+1}}{6}
\end{equation}
Aus den geglätteten Gewichten $W_i$ können nun die Federkonstanten berechnet werden, wobei
die Federkonstante $k_{i+1/2}$ die Feder beschreibt, die die Punkte $x_{i}$ und $x_{i+1}$ verbindet.
\begin{equation}
  k_{i+1/2} = \frac{W_i + W_{i+1}}{2}
\end{equation}
Hohe Werte der Gewichtungsfunktion führen demnach zu einer hohen Federkonstante, was die
Gitterverfeinerung bewirkt. Die neuen Gitterkoordinaten $x_i^{m+1}$ können nun über folgende
Gleichung berechnet werden, wobei $x_i^m$ für die Gitterpunkte des alten, nicht adaptierten
Gitters steht.
\begin{equation}
  x_i^{m+1} = \frac{k_{i-1/2} x_{i-1}^m + k_{i+1/2} x_{i+1}^m}{k_{i-1/2} + k_{i+1/2}}
\end{equation}


\section{Abbruchfehler}

Hier wird der berechnete Abbruchfehler als Indikator für die Güte der Lösung verwendet.
Damit wird $\theta_i = TE_i$ gesetzt. Man erkennt da sich das Gitter in Richtung
eines kartsischen Gitters entwickelt.
\begin{figure}[h]
\centering
\begin{tikzpicture}
  \begin{axis}[width=0.7\textwidth, scaled y ticks=false,height=8cm,
  xlabel=$x$,
  ylabel=Absoluter Fehler]
    \addplot[tud2d, mark=*, very thick] file {anwendung/te/err1_data.txt};
    \addplot[tud9d, mark=*, very thick] file {anwendung/te/err2_data.txt};
    \legend{Startwert,Kleinster Abbruchfehler}
  \end{axis}
\end{tikzpicture}
\caption{Vergleich absoluter Fehler}
\end{figure}
\noindent
Der gemittelter Abbruchfehler und der gemittelte absolute Fehler
oszillieren während der Optimierung und haben bei ungefähr 320 Iterationen
ihren kleinsten Wert erreicht.
\begin{figure}[h]
\centering
   \begin{subfigure}{0.45\linewidth} \centering
  \begin{tikzpicture}
    \begin{semilogyaxis}[width=\textwidth,height=7cm,
        ylabel=Gemittelter Abbruchfehler,
      xlabel=Iterationen]
    %ylabel=$y$]
      \addplot[tud2d, very thick] file {anwendung/te/te_serr_data.txt};
    \end{semilogyaxis}
  \end{tikzpicture}
  \caption{Abbruchfehler}
\end{subfigure}
\hspace{5mm}
   \begin{subfigure}{0.45\linewidth} \centering
  \begin{tikzpicture}
    \begin{semilogyaxis}[width=\textwidth,height=7cm,
        ylabel=Gemittelter absoluter Fehler,
      xlabel=Iterationen]
    %xlabel=$x$,
    %ylabel=$y$]
      \addplot[tud2d, very thick] file {anwendung/te/serr_data.txt};
    \end{semilogyaxis}
  \end{tikzpicture}
  \caption{Absoluter Fehler}
\end{subfigure}
\caption{Verlauf der Gitteradaption}
\end{figure}


\begin{figure}[h]
\centering
\begin{tikzpicture}
  \begin{axis}[width=0.7\textwidth, scaled y ticks=false,
  xlabel=$x$,
  ylabel=Abbruchfehler,height=5cm]
    \addplot[tud2d, mark=*, very thick] file {anwendung/te/te1_data.txt};
    \addplot[tud9d, mark=*, very thick] file {anwendung/te/te2_data.txt};
    \legend{Startwert,Kleinster Abbruchfehler}
  \end{axis}
\end{tikzpicture}
\caption{Vergleich Abbruchfehler}
\end{figure}







\section{Lösungsgradient}

Hier wird der berechnete Abbruchfehler als Indikator für die Güte der Lösung verwendet.
Damit wird $\theta_i = \frac{\partial \phi}{\partial x} \big\vert_i$ gesetzt.
\begin{figure}[h]
\centering
\begin{tikzpicture}
  \begin{axis}[width=0.7\textwidth, scaled y ticks=false,
  xlabel=$x$,
  ylabel=Absoluter Fehler]
    \addplot[tud2d, mark=*, very thick] file {anwendung/grad/err1_data.txt};
    \addplot[tud9d, mark=*, very thick] file {anwendung/grad/err2_data.txt};
    \legend{Startwert,Kleinster Gradient}
  \end{axis}
\end{tikzpicture}
\caption{Vergleich absoluter Fehler} 
\end{figure}
Die Iteration mti dem Lösungsgradient konvergiert schnell und verbessert sich
nach 100 Iterationen nur noch minimal.

\begin{figure}[h]
\centering
   \begin{subfigure}{0.45\linewidth} \centering
  \begin{tikzpicture}
    \begin{semilogyaxis}[width=\textwidth,height=5cm,
    xlabel=Iterationen,ylabel=Gemittelter Gradient]
    %xlabel=$x$,
    %ylabel=$y$]
      \addplot[tud2d, very thick] file {anwendung/grad/grad_serr_data.txt};
    \end{semilogyaxis}
  \end{tikzpicture}
  \caption{Summierter Gradient}
\end{subfigure}
\hspace{4mm}
   \begin{subfigure}{0.45\linewidth} \centering
  \begin{tikzpicture}
    \begin{semilogyaxis}[width=\textwidth,height=5cm,
    xlabel=Iterationen,ylabel=Gemittelter abs. Fehler]
    %xlabel=$x$,
    %ylabel=$y$]
      \addplot[tud2d, very thick] file {anwendung/grad/serr_data.txt};
    \end{semilogyaxis}
  \end{tikzpicture}
  \caption{Absoluter Fehler}
\end{subfigure}
\caption{Verlauf der Gitteradaption}
\end{figure}

\begin{figure}[h]
\centering
\begin{tikzpicture}
  \begin{axis}[width=0.7\textwidth, scaled y ticks=false,
  xlabel=$x$,
  ylabel=Lösungsgradient,height=7cm,legend style={anchor=south east, at={(0.98,0.02)}}]
    \addplot[tud2d, mark=*, very thick] file {anwendung/grad/grad1_data.txt};
    \addplot[tud9d, mark=*, very thick] file {anwendung/grad/grad2_data.txt};
    \legend{Startwert,Kleinster Gradient}
  \end{axis}
\end{tikzpicture}
\caption{Vergleich Lösungsgradient}
\end{figure}

\vspace{2cm}



\section{Richardson-Extrapolations-Indikator}

Bei der Richardson-Extrapolation ($RE$) \cite{roy} wird die Lösungs des Problems auf einem groben und einem
verfeinerten Gitter berechnet. Aus den beiden Lösungen lasst sich dann eine verbesserte
Lösung extrapolieren. Um diese als Indikator zur Gitteradaption
verwenden zu können, wird die Differenz der berechneten Lösung ($T$)
und der Richardson-Extrapolation berechnet.
\begin{equation}
  REI_i = RE_i - T_i
\end{equation}
\begin{figure}[h]
\centering
\begin{tikzpicture}
  \begin{axis}[width=0.7\textwidth, scaled y ticks=false,
  xlabel=$x$,
ylabel=Absoluter Fehler,height=8cm]
    \addplot[tud2d, mark=*, very thick] file {anwendung/rich/err1_data.txt};
    \addplot[tud9d, mark=*, very thick] file {anwendung/rich/err2_data.txt};
    \legend{Startwert,Kleinster REI}
  \end{axis}
\end{tikzpicture}
\caption{Vergleich absoluter Fehler} \label{fig:twofigs}
\end{figure}


\begin{figure}[h]
\centering
   \begin{subfigure}{0.45\linewidth} \centering
  \begin{tikzpicture}
    \begin{semilogyaxis}[width=\textwidth,height=5cm,
      xlabel=Iterationen,ylabel=Gemittelter REI]
    %xlabel=$x$,
    %ylabel=$y$]
      \addplot[tud2d, very thick] file {anwendung/rich/rich_serr_data.txt};
    \end{semilogyaxis}
  \end{tikzpicture}
  \caption{Summierte Differenz zur Richardson-Extrapolation}
\end{subfigure}
\hspace{4mm}
   \begin{subfigure}{0.45\linewidth} \centering
  \begin{tikzpicture}
    \begin{semilogyaxis}[width=\textwidth,height=5cm,
    xlabel=Iterationen,ylabel=Gemittelter abs. Fehler]
    %xlabel=$x$,
    %ylabel=$y$]
      \addplot[tud2d, very thick] file {anwendung/rich/serr_data.txt};
    \end{semilogyaxis}
  \end{tikzpicture}
  \caption{Absoluter Fehler}
\end{subfigure}
\caption{Verlauf der Gitteradaption} \label{fig:twofigs}
\end{figure}


\begin{figure}[h]
\centering
\begin{tikzpicture}
  \begin{axis}[width=0.7\textwidth, scaled y ticks=false,height=5cm,
  xlabel=$x$,
  ylabel=REI,legend style={anchor=south east, at={(0.98,0.02)}}]
    \addplot[tud2d, mark=*, very thick] file {anwendung/rich/rich1_data.txt};
    \addplot[tud9d, mark=*, very thick] file {anwendung/rich/rich2_data.txt};
    \legend{Startwert,Kleinster REI}
  \end{axis}
\end{tikzpicture}
\caption{Vergleich Richardson-Extrapolations-Indikator}
\end{figure}



\vspace{3cm}




\section{Zweigitter-Fehlerschätzer}
Hier wird ebenfalls die Lösungs des Problems auf einem groben ($T_c$) und einem
verfeinerten Gitter ($T_f$) berechnet. Allerdings wird dann keine neue Lösung extrapoliert sondern
die Differenz der beiden Lösungen als Fehlerindikator verwendet.
\begin{equation}
  ZGF = T_g\vert_i - T_{f}\vert_{i1} - T_{f}\vert_{i2}
\end{equation}
\begin{figure}[h]
\centering
\begin{tikzpicture}
  \begin{axis}[width=0.7\textwidth, scaled y ticks=false,
  xlabel=$x$,
ylabel=Absoluter Fehler]
    \addplot[tud2d, mark=*, very thick] file {anwendung/zg/err1_data.txt};
    \addplot[tud9d, mark=*, very thick] file {anwendung/zg/err2_data.txt};
    \legend{Startwert,Kleinster ZGF}
  \end{axis}
\end{tikzpicture}
\caption{Vergleich absoluter Fehler} \label{fig:twofigs}
\end{figure}


\begin{figure}[h]
\centering
   \begin{subfigure}{0.45\linewidth} \centering
  \begin{tikzpicture}
    \begin{semilogyaxis}[width=\textwidth,height=7cm,
      xlabel=Iterationen,ylabel=Gemittelter ZGF]
    %xlabel=$x$,
    %ylabel=$y$]
      \addplot[tud2d, very thick] file {anwendung/zg/zg_serr_data.txt};
    \end{semilogyaxis}
  \end{tikzpicture}
  \caption{Summierter Zweigitter-Fehler}
\end{subfigure}
\hspace{4mm}
   \begin{subfigure}{0.45\linewidth} \centering
  \begin{tikzpicture}
    \begin{semilogyaxis}[width=\textwidth,height=7cm,
      xlabel=Iterationen, ylabel=Gemittelter abs. Fehler]
    %xlabel=$x$,
    %ylabel=$y$]
      \addplot[tud2d, very thick] file {anwendung/zg/serr_data.txt};
    \end{semilogyaxis}
  \end{tikzpicture}
  \caption{Absoluter Fehler}
\end{subfigure}
\caption{Verlauf der Gitteradaption} \label{fig:twofigs}
\end{figure}


\begin{figure}[h]
\centering
\begin{tikzpicture}
  \begin{axis}[width=0.7\textwidth, xlabel=$x$,height=8cm,
  ylabel=Abbruchfehler,legend style={anchor=south east, at={(0.98,0.02)}}]
    \addplot[tud2d, mark=*, very thick] file {anwendung/zg/zg1_data.txt};
    \addplot[tud9d, mark=*, very thick] file {anwendung/zg/zg2_data.txt};
    \legend{Startwert,Kleinster ZGF}
  \end{axis}
\end{tikzpicture}
\caption{Vergleich Zweigitter-Fehler} \label{fig:twofigs}
\end{figure}
\clearpage







\section{Vergleich der Indikatoren}

Im folgenden Diagramm sind die Startkonfiguration sowie die adaptierte Version
jedes Indikators eingetragen. Man sieht das der Zweigitter-Fehlerschätzer hinter
den anderen Indikatoren zurückfällt. Der Richardson- und der Gradienten-Indikator weisen
leichte Schwächen an den Rändern des Problemgebiets auf.
\begin{figure}[h]
\centering
\begin{tikzpicture}
  \begin{axis}[width=0.9\textwidth, height=12cm,scaled y ticks=false,
  xlabel=$x$,
ylabel=Absoluter Fehler
]
    \addplot[tud2d, mark=*, very thick] file {anwendung/zg/err1_data.txt};
    \addplot[tud4d, mark=*, very thick] file {anwendung/te/err2_data.txt};
    \addplot[tud9d, mark=*, very thick] file {anwendung/grad/err2_data.txt};
    \addplot[tud6d, mark=*, very thick] file {anwendung/rich/err2_data.txt};
    \addplot[tud1d, mark=*, very thick] file {anwendung/zg/err2_data.txt};
    \legend{Startwert,Abbruchfehler, Gradient, Richardson, Zweigitter}
  \end{axis}
\end{tikzpicture}
\caption{Vergleich absoluter Fehler der untersuchten Indikatoren}
\end{figure}

\begin{figure}[h]
\centering
\begin{tikzpicture}
  \begin{axis}[width=0.9\textwidth, height=12cm,scaled y ticks=false,
  xlabel=Iteration,
ylabel=Gemittelter absoluter Fehler
]
    \addplot[tud4d, very thick] file {anwendung/te/serr_data.txt};
    \addplot[tud9d, very thick] file {anwendung/grad/serr_data.txt};
    \addplot[tud6d, very thick] file {anwendung/rich/serr_data.txt};
    \addplot[tud1d, very thick] file {anwendung/zg/serr_data.txt};
    \legend{Abbruchfehler, Gradient, Richardson, Zweigitter}
  \end{axis}
\end{tikzpicture}
\caption{Gemittelter absoluter Fehler der untersuchten Indikatoren während der Iteration}
\end{figure}
\noindent
In Tabelle~\ref{tab:verb} sind die relativen Verbesserungen des absoluten Fehlers für jeden
Indikator aufgetragen. Die Gewinne sind beim Abbruchfehler mit 97,7\% am größten.
Dicht dahinter folgen der Lösungsgradient und der Richardson-Extrapolations-Indikator.
Der Zweigitter-Fehlerschätzer sit mit 65,94\% Verbesserung weit abgeschlagen.

\begin{table}[h]
  \begin{tabular}{l r}
  \toprule
  Indikator & Relative Verbesserung in \% \\
  \midrule
      Abbruchfehler & 97.77\\
      Gradient & 93.56\\
      Richardson & 90.82\\
      Zweigitter & 65.94\\
  \bottomrule
\end{tabular}
\caption{Relative Verbesserung des absoluten Fehlers}
\label{tab:verb}
\end{table}



\cleardoublepage

\chapter{Fazit}
In der vorliegenden Arbeit wurde der Abbruchfehler als Indikator zur
Gitteradaption untersucht. Dazu wurde er für skalare Tranbsportgleichungen 
im Ein- und Zweidimensionalen hergeleitet. Dies geschah für kartesische,
orthogonale und nicht-orthogonale Gitter.
Anschließend wurden die Ergebnisse verifiziert. Die dafür nötigen Lösungsprogramme
wurden implementiert und nach gängigen Methoden auf ihre Korrektheit überprüft.
Anschließend wurde der Abbruchfehler berechnet und mit anderen fehlerindikatoren verglichen.
Hier stellte sich eine gute Übereinstimmung heraus. Schlussendlich
wurde der Abbruchfehler als indikator dur adaptiven Gitterverfeinerung mit dem r-Verfahren genutzt.
Auch hier konnten Vorteile gegenüber anderen gängigen Indikatoren nachgewiesen werden.

Der abgeleite Abbruchfehler verwendet zur Diskretisierung von Ableitungen
finite Differenzen. Dies führt gerade am Rand des problemgebietes zu ungenauen Vorhersagen.
Alternativ kann hier die Verwendung von Interpolationspolynomen 
zur Ableitungsdiskretisierung geprüft werden.
Desweiteren steht eine Implementation des Abbruchfehlers für nicht-orthogonale Gitter
noch aus. Dies ist gerade im Hinblick auf die Nutzung gängiger Programmpakete zur Gitteradaption
sinnvoll.

\cleardoublepage

%% Bibliographie
\bibliographystyle{abbrv}
\addcontentsline{toc}{chapter}{Literaturverzeichnis}
\bibliography{bib}
\nocite{*}

%Appendix
%\appendix
%\include{anhanga}

\end{document}
