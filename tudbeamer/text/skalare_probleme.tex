\section{Skalare Probleme}

Eine Vielzahl praktisch relevanter Probleme lässt sich durch skalare (partielle)
Differential\-glei\-chungen beschreiben. Anwendungsgebiete reichen hier von Wärmetransportproblemen
über strukturmechanische Probleme (zum BEispiel die Auslenkung von Stäben und Membranen bei elastischem
Werkstoffverhalten) bis hin zu strömungsmechanischen Problemen (zum Beispiel das Geschwindigkeitspotential
wirbelfreier Strömungsfelder).

\subsection{Einfache Feldprobleme}

Einige kontinuumsmechanische Problemstellungen können durch eine Differentialgleichung
der Form

\begin{equation}
  -\frac{\partial}{\partial x_i}\left({\alpha \frac{\partial \phi}{\partial x_i} }\right)=g
  \label{eq:feldproblem_stat}
\end{equation}
 beschrieben werden. Diese muss auf dem gesamten betrachteten Gebiet $V$ gelten, das
vom Rand $S$ begrenzt wird. Bei $\phi=\phi(\mathbf{x})$ handelt es sich um die gesuchte,
skalare Feldgröße. $g=g(\mathbf{x})$ und $\alpha=\alpha(\mathbf{x})$ sind vorgegeben.


Weiterhin müssen auf dem gesamten Rand $S$ Randbedingungen
gegeben sein. Üblich sind hierbei folgende Typen von Randbedingungen:

\begin{itemize}
  \item Dirichletsche Randbedingung: $\quad \phi=\phi_S$
  \item Neumannsche Randbedingung: $\quad \alpha \frac{\partial \phi}{\partial x_i} n_i = b_s $
  \item Cauchysche Randbedingung: $\quad c_S \phi + \alpha \frac{\partial \phi}{\partial x_i} n_i = b_s$
\end{itemize}

Hierbei sind $\phi_S$, $b_S$ und $c_S$ vorgegebene Funktionen auf dem Rand $S$. Die
Komponenten des nach außen gerichteten Normaleneinheitsvektors an $S$ werden mit $n_i$ bezeichnet.

Treten mehrere Randbedingungstypen in einem Problem auf spricht man von gemischten Rand\-wert\-problemen.

Die mit Gleichung~\ref{eq:feldproblem_stat} beschriebenen Probleme beinhalten keine Zeitabhängigkeit
und werden deshalb als stationär bezeichnet.
Im instationären, also zeitabhängigem Fall, erhalten alle Größen neben der Ortsabhängigkeit
eine Zeitabhängigkeit. Die entsprechende Differentialgleichung lautet damit
\begin{equation}
  \frac{\partial \phi}{\partial t}
  -\frac{\partial}{\partial x_i}\left({\alpha \frac{\partial \phi}{\partial x_i} }\right)=g
  \label{eq:feldproblem_instat}
\end{equation}
für die unbekannte Größe $\phi=\phi(\mathbf{x}, t)$.
Für instationäre Probleme muss neben den Rand\-beding\-ungen auch eine Anfangsbedingung
$\phi(\mathbf{x}, t_0) = \phi_0(\mathbf{x})$ gegeben werden.

Im weiteren Verlauf der Arbeit werden nur stationäre Probleme betrachtet. Alle Ergebnisse
lassen sich aber auf instationäre Probleme übertragen.

\subsection{Allgemeine Transportgleichung}
\label{sec:transportgl}

Eine wichtige Problemklasse innerhalb des Maschinenbaus stellen Transportprobleme in
Festkörpern oder Fluiden dar. Bei der transportierten Größe kann es sich dabei beispielsweise
um joulsche Wärme (Wärmetransportprobleme) oder Stoffmengen (Stofftransportprobleme) handeln.

Stationäre Transportprobleme können in Differentialform durch die Gleichung
\begin{equation}
  \frac{\partial}{\partial x_i} \left({\rho v_i \phi
- \alpha \frac{\partial \phi}{\partial x_i} }\right) = f
\label{eq:transportgl}
\end{equation}
beschrieben werden. Die auftretenden Terme werden wie folgt benannt:
\begin{itemize}
  \item Quellterm $\quad f$
  \item Konvektionsterm $\quad\frac{\partial}{\partial x_i} \rho v_i \phi$
  \item Diffusionsterm $\quad\alpha \frac{\partial^2 \phi}{\partial x_i \partial x_i}$
\end{itemize}
Für spezielle Probleme müssen die entsprechenden Parameter angepasst werden
(für nähere Information siehe \cite{num_maschbau}).

Alle Betrachtungen in der vorliegenden Arbeit bauen auf Problemen auf, die durch
Transport\-gleich\-ungen beschrieben werden können.

\section{Navier-Stokes Gleichungen}

Für viele reale Probleme ist es nötig das Verhalten von Fluiden vorherzusagen.
Teilweise sind auch gleichzeitige Stoff- und Wärmetransportvorgänge interessant.
Die in praktisch relevanten Problemen am häufigsten auftretenden Fluide werden Newtonsche Fluide genannt.
Das bedeutet, dass sie linear-viskos und isotrop sind und damit durch den Cauchyschen
Spannungstensor $T$ beschrieben werden können, wobei $p$ für den Druck und $\mu$ für die dynamische Viskosität stehen.
\begin{equation}
  T_{ij} = \mu\left({\frac{\partial v_i}{\partial x_j}
  + \frac{\partial v_j}{\partial x_i}
-\frac{2}{3} \frac{\partial v_k}{\partial x_k} \delta_{ij}}\right)
-p\delta_{ij}
\end{equation}
Die
Erhaltungssätze für Masse, Impuls und innere Energie lauten mit diesem Materialgesetz dann:
\begin{align}
  \frac{\partial p}{\partial t} + \frac{\partial (\rho v_i)}{\partial x_i} &= 0\\
  \frac{\partial (\rho v_i)}{\partial t} + \frac{\partial (\rho v_i v_j)}{\partial x_j} &=
  \frac{\partial}{\partial x_j} \left[{\mu
  \left({\frac{\partial v_i}{\partial x_j}
  +\frac{\partial v_j}{\partial x_i}
  - \frac{2}{3} \frac{\partial v_k}{\partial x_k}\delta_{ij}}\right)}\right]
  -\frac{\partial p}{\partial x_i} + \rho f_i\\
  \frac{\partial (\rho e)}{\partial t} + \frac{\partial (\rho v_i e)}{\partial x_i} &=
  \mu \left[{\frac{\partial v_i}{\partial x_j}
  \left({\frac{\partial v_i}{\partial x_j}
  +\frac{\partial v_j}{\partial x_i}}\right)
  - \frac{2}{3} \left({\frac{\partial v_i}{\partial x_i}}\right)^2}\right]
  -p\frac{\partial v_i}{\partial x_i} +\frac{\partial}{\partial x_i}
  \left({\kappa \frac{\partial T}{\partial x_i}}\right)
  + \rho q
\end{align}
Neben diesen Erhaltungsgleichungen benötigt man weiterhin thermische und kalorische Zustands\-gleich\-ungen.
Diese definieren die thermodynamischen Eigenschaften des Fluids.

Bei
vielen realen Problemen müssen die obigen Gleichungen nicht in der allgemeinen Form gelöst werden, sondern
können durch zusätzliche Annahmen vereinfacht werden.
So wird beispielsweise bei vielen praktischen Anwendungen das Fluid nicht entscheidend komprimiert
und kann deshalb als inkompressibel angesehen werden. Als Kriterium gilt weithin,
das die Machzahl ($Ma = v/a$) kleiner~als~0,3 ist. Beispiele für inkompressible Probleme
sind Strömungen von Flüssigkeiten wie Rohrströmungen oder die Umströmung von Bootsrümpfen.
Auch die Strömung an langsam fliegenden Flugzeugen, zum Beispiel Segelflugzeugen, kann als
inkompressibel modelliert werden.

Ist die Bedingung für Inkompressibilität erfüllt,
vereinfachen sich die obigen Gleichungen für kompressible Fluide und es ergeben sich folgende Gleichungen:
\begin{align}
  \frac{\partial v_i}{\partial x_i} &= 0\\
  \frac{\partial (\rho v_i)}{\partial t} + \frac{\partial (\rho v_i v_j)}{\partial x_j} &=
  \frac{\partial}{\partial x_j} \left[{\mu
  \left({\frac{\partial v_i}{\partial x_j}
  +\frac{\partial v_j}{\partial x_i}}\right)}\right]
  -\frac{\partial p}{\partial x_i} + \rho f_i\\
  \frac{\partial (\rho e)}{\partial t} + \frac{\partial (\rho v_i e)}{\partial x_i} &=
  \mu \frac{\partial v_i}{\partial x_j}
  \left({\frac{\partial v_i}{\partial x_j}
  +\frac{\partial v_j}{\partial x_i}}\right)
  +\frac{\partial}{\partial x_i}
  \left({\kappa \frac{\partial T}{\partial x_i}}\right)
  + \rho q
\end{align}
\clearpage
