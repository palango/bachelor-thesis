\section{Zweidimensionale Testfälle}

\paragraph{Testfall 3}

\noindent
Im dritten Testfall wird ein reines Diffusionsproblem im Zweidimensionalen
betrachtet. Das genutzte Gitter ist orthogonal und hat in beiden Dimensionen
einen Expansionsfaktor $\alpha=0,95$. Zum Test wurde Testfunktion 4
gewählt. Der Quellterm ergibt sich mit der Methode der konstruierten Lösungen zu:
\begin{equation}
  f=-\frac{\pi}{2} \sin\left(\frac{\pi}{2} x\right) \sin\left(\frac{\pi}{2} y\right)
\end{equation}
Der absolute Fehler ist in Abbildung~\ref{fig:t3_abs} dargestellt.
Die folgenden Abbildungen vergleichen das Residuum der Lösung mit dem
Abbruchfehler. Hier wird einmal das komplette Problemgebiet gezeigt, welches
am Rand Sprünge enthält (Abbildung~\ref{fig:t3_rand}), und im anderen die
Sprünge zur besseren Vergleichbarkeit ausgeblendet (Abbildung~\ref{fig:t3_krand}).

Abschließend zeigt Abbildung~\ref{fig:t3_teerr} die Differenz von Residuum und
Abbruchfehler. Man sieht hier die gute Übereinstimmung bei Differenzen
im Bereich von $10^{-6}$.

\begin{figure}[h]
\centering
\begin{subfigure}[b]{.5\linewidth}
\centering
\begin{tikzpicture}
\begin{axis}[
%view={30+180}{30},
xlabel=$x$,
ylabel=$y$,
%zlabel={$f(x,y) = \sin(\frac{\pi}{2} x) \cos(\frac{\pi}{2} y)$},
domain=0:1,
height=7cm,
width=\textwidth
]
\addplot3[surf, mesh/ordering=y varies, faceted color=black] file{data/3/ERR_data.txt};
\end{axis}
\end{tikzpicture}
%\subcaption{Oberflächen}\label{fig:1a}
\end{subfigure}%
\begin{subfigure}[b]{.5\linewidth}
\centering
\begin{tikzpicture}
\begin{axis}[
view={0}{90},
xlabel=$x$,
ylabel=$y$,
zlabel={$f(x,y) = \sin(\frac{\pi}{2} x) \sin(\frac{\pi}{2} y)$},
domain=0:1,
height=7cm
]
\addplot3 +[contour prepared,very thick,mark=none, contour prepared format=matlab]
file {data/3/ERR_contour.txt};
\end{axis}
\end{tikzpicture}
%\subcaption{Another subfigure}\label{fig:1b}
\end{subfigure}
\caption{Absoluter Fehler (Testfall 3)}
\label{fig:t3_abs}
\end{figure}


\begin{figure}[h]
\centering
\begin{subfigure}[b]{.5\linewidth}
\centering
\begin{tikzpicture}
\begin{axis}[
%view={30+180}{30},
xlabel=$x$,
ylabel=$y$,
height=6cm,
width=\textwidth
]
\addplot3[surf, mesh/ordering=y varies, faceted color=black] file{data/3/RES_data.txt};
\end{axis}
\end{tikzpicture}
\subcaption{Residuum}
\end{subfigure}%
\begin{subfigure}[b]{.5\linewidth}
\centering
\begin{tikzpicture}
\begin{axis}[
%view={30+180}{30},
xlabel=$x$,
ylabel=$y$,
height=6cm,
width=\textwidth
]
\addplot3[surf, mesh/ordering=y varies, faceted color=black] file{data/3/TE_data.txt};
\end{axis}
\end{tikzpicture}
\subcaption{Abbruchfehler}
\end{subfigure}
\caption{Residuum und Abbruchfehler mit Randsprüngen (Testfall 3)}
\label{fig:t3_rand}
\end{figure}



\begin{figure}[h]
\centering
\begin{subfigure}[b]{.5\linewidth}
\centering
\begin{tikzpicture}
\begin{axis}[
view={150}{30},
xlabel=$x$,
ylabel=$y$,
height=6cm,
width=\textwidth
]
\addplot3[surf, mesh/ordering=y varies, faceted color=black] file{data/3/RES2_data.txt};
\end{axis}
\end{tikzpicture}
\subcaption{Residuum}
\end{subfigure}%
\begin{subfigure}[b]{.5\linewidth}
\centering
\begin{tikzpicture}
\begin{axis}[
view={150}{30},
xlabel=$x$,
ylabel=$y$,
height=6cm,
width=\textwidth
]
\addplot3[surf, mesh/ordering=y varies, faceted color=black] file{data/3/TE2_data.txt};
\end{axis}
\end{tikzpicture}
\subcaption{Abbruchfehler}
\end{subfigure}
\caption{Residuum und Abbruchfehler ohne Randsprünge (Testfall 3)}
\label{fig:t3_krand}
\end{figure}






\begin{figure}[h]
\centering
\begin{subfigure}[b]{.5\linewidth}
\centering
\begin{tikzpicture}
\begin{axis}[
%view={170}{30},
xlabel=$x$,
ylabel=$y$,
height=6cm,
width=\textwidth
]
\addplot3[surf, mesh/ordering=y varies, faceted color=black] file{data/3/RESTE_data.txt};
\end{axis}
\end{tikzpicture}
\subcaption{Mit Randsprüngen}
\end{subfigure}%
\begin{subfigure}[b]{.5\linewidth}
\centering
\begin{tikzpicture}
\begin{axis}[
view={180+20}{35},
xlabel=$x$,
ylabel=$y$,
height=6cm,
width=\textwidth
]
\addplot3[surf, mesh/ordering=y varies, faceted color=black] file{data/3/RESTE2_data.txt};
\end{axis}
\end{tikzpicture}
\subcaption{Ohne Randsprünge}
\end{subfigure}
\caption{Differenz Residuum und Abbruchfehler (Testfall 3)}
\label{fig:t3_teerr}
\end{figure}
%\clearpage
\vspace{5cm}

\clearpage

\paragraph{Testfall 4}

\noindent
Im vierten Testfall wird ein kombiniertes Diffusions- und Konvektionsproblem im Zweidimensionalen
betrachtet. Das genutzte Gitter ist orthogonal und hat in beiden Dimensionen
einen Expansionsfaktor $\alpha=0,9$. Hier wird Testfunktion 3 verwendet. Der
Quellterm lautet damit:
\begin{equation}
  f=-2 \pi \sin(\pi x) \sin(\pi y) + \pi \cos(\pi x)\sin(\pi y) + \pi\sin(\pi x)\cos(\pi y)
\end{equation}
Der absolute Fehler ist in Abbildung~\ref{fig:t4_abs} dargestellt. Man erkennt
gut den größeren Fehler auf dem gröberen Bereich des Gitters und die kleinere
Abweichung im feiner aufgelösten Gebiet.

Die folgenden Abbildungen vergleichen das Residuum der Lösung mit dem
Abbruchfehler. Hier wird einmal das komplette Problemgebiet gezeigt, welches
beim Abbruchfehler am Rand Sprünge enthält (Abbildung~\ref{fig:t4_rand}).
Diese sind auf die niedrigere Ordnung der Diskretisierungsverfahren am Rand
zurückzuführen. In Abbildung~\ref{fig:t4_krand} sind diese Sprünge ausgeblendet
und können so besser mit dem Residuum verglichen werden.
Abschließend zeigt Abbildung~\ref{fig:t4_teerr} die Differenz von Residuum und
Abbruchfehler.


\begin{figure}[h]
\centering
\begin{subfigure}[b]{.5\linewidth}
\centering
\begin{tikzpicture}
\begin{axis}[
%view={30+180}{30},
xlabel=$x$,
ylabel=$y$,
%zlabel={$f(x,y) = \sin(\frac{\pi}{2} x) \cos(\frac{\pi}{2} y)$},
domain=0:1,
height=7cm,
width=\textwidth
]
\addplot3[surf, mesh/ordering=y varies, faceted color=black] file{data/4/ERR_data.txt};
\end{axis}
\end{tikzpicture}
%\subcaption{Oberflächen}\label{fig:1a}
\end{subfigure}%
\begin{subfigure}[b]{.5\linewidth}
\centering
\begin{tikzpicture}
\begin{axis}[
view={0}{90},
xlabel=$x$,
ylabel=$y$,
zlabel={$f(x,y) = \sin(\frac{\pi}{2} x) \sin(\frac{\pi}{2} y)$},
domain=0:1,
height=7cm
]
\addplot3 +[contour prepared,very thick,mark=none, contour prepared format=matlab]
file {data/4/ERR_contour.txt};
\end{axis}
\end{tikzpicture}
%\subcaption{Another subfigure}\label{fig:1b}
\end{subfigure}
\caption{Absoluter Fehler (Testfall 4)}
\label{fig:t4_abs}
\end{figure}


\begin{figure}[h]
\centering
\begin{subfigure}[b]{.5\linewidth}
\centering
\begin{tikzpicture}
\begin{axis}[
view={-30+180}{30},
xlabel=$x$,
ylabel=$y$,
height=6cm,
width=\textwidth
]
\addplot3[surf, mesh/ordering=y varies, faceted color=black] file{data/4/RES_data.txt};
\end{axis}
\end{tikzpicture}
\subcaption{Residuum}
\end{subfigure}%
\begin{subfigure}[b]{.5\linewidth}
\centering
\begin{tikzpicture}
\begin{axis}[
view={-30+180}{30},
xlabel=$x$,
ylabel=$y$,
height=6cm,
width=\textwidth
]
\addplot3[surf, mesh/ordering=y varies, faceted color=black] file{data/4/TE_data.txt};
\end{axis}
\end{tikzpicture}
\subcaption{Abbruchfehler}
\end{subfigure}
\caption{Residuum und Abbruchfehler mit Randsprüngen (Testfall 4)}
\label{fig:t4_rand}
\end{figure}



\begin{figure}[h]
\centering
\begin{subfigure}[b]{.5\linewidth}
\centering
\begin{tikzpicture}
\begin{axis}[
view={150}{30},
xlabel=$x$,
ylabel=$y$,
height=6cm,
width=\textwidth
]
\addplot3[surf, mesh/ordering=y varies, faceted color=black] file{data/4/RES2_data.txt};
\end{axis}
\end{tikzpicture}
\subcaption{Residuum}
\end{subfigure}%
\begin{subfigure}[b]{.5\linewidth}
\centering
\begin{tikzpicture}
\begin{axis}[
view={150}{30},
xlabel=$x$,
ylabel=$y$,
height=6cm,
width=\textwidth
]
\addplot3[surf, mesh/ordering=y varies, faceted color=black] file{data/4/TE2_data.txt};
\end{axis}
\end{tikzpicture}
\subcaption{Abbruchfehler}
\end{subfigure}
\caption{Residuum und Abbruchfehler ohne Randsprünge (Testfall 4)}
\label{fig:t4_krand}
\end{figure}

\vspace{2cm}





\begin{figure}[h]
\centering
\begin{subfigure}[b]{.5\linewidth}
\centering
\begin{tikzpicture}
\begin{axis}[
view={360-30}{30},
xlabel=$x$,
ylabel=$y$,
height=6cm,
width=\textwidth
]
\addplot3[surf, mesh/ordering=y varies, faceted color=black] file{data/4/RESTE_data.txt};
\end{axis}
\end{tikzpicture}
\subcaption{Mit Randsprüngen}
\end{subfigure}%
\begin{subfigure}[b]{.5\linewidth}
\centering
\begin{tikzpicture}
\begin{axis}[
view={180-30}{30},
xlabel=$x$,
ylabel=$y$,
height=6cm,
width=\textwidth
]
\addplot3[surf, mesh/ordering=y varies, faceted color=black] file{data/4/RESTE2_data.txt};
\end{axis}
\end{tikzpicture}
\subcaption{Ohne Randsprünge}
\end{subfigure}
\caption{Differenz Residuum und Abbruchfehler (Testfall 4)}
\label{fig:t4_teerr}
\end{figure}




\section{Zusammenfassung}
Nach der Vorstellung der Testfälle mit ihren Randbedingungen wurden
die Lösungsprogamme verifiziert. Hier konnte über den Vergleich von formaler
und beobachteter Konvergenzordnung die korrekte Funktionsweise nachgewiesen werden.

Der Abbruchfehlerindikator wurde daraufhin im Ein- und Zweidimensionalen untersucht.
Die Ergebnisse bescheinigen eine sehr gute Übereinstimmung mit dem Residuum
der Lösung. Lediglich am Rand des Problemgebiets kommt es, bedingt durch die dortige Verwendung von
einseitigen Differenzenquotienten, zu einer Verringerung der Genauigkeit.

Bei einer Verfeinerung erhöht sich erwartungsgemäß auch die Genauigkeit
des Abbruchfehlers. Die Erhöhung beträgt dabei ungefähr eine halbe Ordnung
bei einer Halbierung der Zellengröße. Sind Diffusions- und Konvektionskoeffizient
nicht von der gleichen Größenordnung verringert sich die Genauigkeit
des Abbruchfehlers um eine ungefähr eine Ordnung.



\clearpage
