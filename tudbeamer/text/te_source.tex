Um den Abbruchfehler eines Kontrollvolumens zu bestimmen, werden die
Abbruchfehler aller zu betrachtenden Terme addiert. Nach Gleichung~\ref{eq:fvm3}
müssen demnach die Abbruchfehler für diffusive und konvektive Flüsse sowie
für Quellterme bestimmt werden.

Das Vorgehen zur Ableitung des Abbruchfehlers ist bei allen Terme gleich.
Für alle in den Diskretisierungen vorkommenden Terme werden Taylorreihendarstellungen
um den zu betrachtenden Punkt entwickelt. Diese werden dann so umgeformt, dass die
verwendete Approximation entsteht. Die zusätzlichen Terme, die in der Approximation nicht
vorkommen, bilden den Abbruchfehler. Dieser enthält, bedingt durch die Taylorreihenentwicklungen,
noch analytische Ableitungen. Diese werden im letzten Schritt diskretisiert, was zum Beispiel
mittels finiter Differenzen erfolgen kann.

Die unendlichen Tayloreihenentwicklungen werden in den folgenden Herleitungen nach den benötigten
Termen abgebrochen. Die weitere Entwicklung wird durch die Abkürzung $HOT$ dargestellt.
Diese steht für \emph{Higher Order Terms}, zu deutsch: Terme höherer Ordnung.

\section{Abbruchfehler für Quellterme}
\label{sec:Quellterm}

\subsection{Eindimensionale Probleme}
\label{sec:source1d}
Die Taylorreihenentwicklung einer Funktion $f(x)$ um den Punkt $x_0$
liefert:
\begin{equation*}
  f(x) = f(x_0) + f'(x_0)(x-x_0) + \frac{1}{2} f''(x_0)(x-x_0)^2 + HOT
\end{equation*}
Nach Integration im Intervall $[a, b]$ ergibt sich:
\begin{align*}
  \int_a^b f(x) dx =& \int_a^b f(x_0) dx + \int_a^b f'(x_0)(x-x_0) dx
+ \int_a^b \frac{1}{2} f''(x_0)(x-x_0)^2 dx + HOT\\
=& f(x_0) (b-a) + f'(x_0) \frac{(b-x_0)^2-(a-x_0)^2}{2}\\
 &+ \frac{1}{2} f''(x_0) \frac{(b-x_0)^3-(a-x_0)^3}{3} +HOT
\end{align*}
Die Anwendung auf ein eindimensionales Kontrollvolumen (siehe Abbildung~\ref{fig:kv1d}) mit den Randpunkten $x_e$ und
$x_w$ sowie dem Mittelpunkt $x_P$ ergibt damit die folgende Formel:
\begin{align*}
  \int_{x_w}^{x_e} f(x)dx =& f_P(x_e-x_w)
  + \frac{1}{2} f'_P \underbrace{\left[{(x_e-x_P)^2-(x_w-x_P)^2}\right]}_{=0}\\
  &+ \frac{1}{6} f''_P \left[{{(x_e-x_P)}^3-{(x_w-x_P)}^3}\right] + HOT
\end{align*}
Der sich der ersten Ableitung von $f$ anschließende Term wird immer Null, da nach
Definition der Finite-Volumen Methode der Mittelpunkt eines Kontrollvolumens immer
zwischen seinen Grenzen liegt.
Die verbleibende zweite Ableitung wird nun durch einen Differenzenquotienten
beschrieben.
\begin{equation}
  \label{eq:diskretisierung_f''P}
  f''_P = \frac{1}{x_e-x_w}\left(\frac{f_E-f_P}{x_E-x_P}-\frac{f_P-f_W}{x_P-x_W}\right)
\end{equation}
Der Abbruchfehler der Diskretisierung des Quellterms lässt sich damit über die
folgende Gleichung beschreiben.
\begin{equation}
  TE_{Q,1D} =
\frac{1}{6(x_e-x_w)}\left(\frac{f_E-f_P}{x_E-x_P}-\frac{f_P-f_W}{x_P-x_W}\right)
\left[{{(x_e-x_P)}^3-{(x_w-x_P)}^3}\right] + HOT\label{eq:te_source1}
\end{equation}

\begin{figure}[bt]

\begin{tikzpicture}[scale=0.8]
  %\draw[->] (6,0) -- (6.5,0) node[right] {$x$} coordinate(x axis);
  %\draw[->] (0,6) -- (0,6.5) node[above] {$y$} coordinate(y axis);
  \fill[tud0a] (2,2) rectangle +(2,2);
  \draw[step=2cm, thick] (0, 2) grid (6, 4);

  \fill (3, 3)  circle[radius=1.5pt];
  \node (N) at (3,3) [label=above:$P$]{};

  \fill (5, 3)  circle[radius=1.5pt];
  \node (N) at (5,3) [label=above:$E$]{};

  \fill (1, 3)  circle[radius=1.5pt];
  \node (N) at (1,3) [label=above:$W$]{};

  %\fill (7, 3)  circle[radius=1.5pt];
  %\node (N) at (7,3) [label=right:$EE$]{};

  %\fill (-1, 3)  circle[radius=1.5pt];
  %\node (N) at (-1,3) [label=left:$WW$]{};

  \fill (4, 3)  circle[radius=1.5pt];
  \node (N) at (3.84,3) [label={right=0pt}:$e$]{};

  \fill (2, 3)  circle[radius=1.5pt];
  \node (N) at (2.16,3) [label={left=0pt}:$w$]{};

  %\fill (6, 3)  circle[radius=1.5pt];
  %\node (N) at (6,3) [label=right:$ee$]{};

  %\fill (0, 3)  circle[radius=1.5pt];
  %\node (N) at (0,3) [label=left:$ww$]{};
\end{tikzpicture}

\centering
\caption{Allgemeines Kontrollvolumen im eindimensionalen Fall}
\label{fig:kv1d}
\end{figure}

\subsubsection{Kontrollvolumen am Rand}

Bei Kontrollvolumen am Rand sind nicht mehr alle in Gleichung~\ref{eq:te_source1}
genutzten Werte gegeben. Der genutzte Differenzenquotient muss demnach angepasst werden.
Da man sich dabei auch von der Form des Zentraldifferenzenquotienten entfernt,
sinkt gleichzeitig die Genauigkeit in den Randgebieten.

Die Anpassung soll beispielhaft am westlichen Rand gezeigt werden. Der Differenzenquotient
ergibt sich bei Benutzung des mit $f^w$ bezeichneten Randwerts, der beispielsweise als
dirichletsche Randbedingung gegeben sein kann, zu:
\begin{equation}
  f''_P = \frac{1}{x_e-x_w}\left(\frac{f_E-f_P}{x_E-x_P}-\frac{f_P-f^w}{x_P-x_w}\right)
\end{equation}
Eingesetzt ergibt sich damit folgende Formel für den Abbruchfehler des Quellterms
für Kontrollvolumen am Rand.
\begin{equation}
  TE_{Q,\ W-Rand} =
\frac{1}{6\,(x_e-x_w)}\left(\frac{f_E-f_P}{x_E-x_P}-\frac{f_P-f^w}{x_P-x_w}\right)
  \left[{{(x_e-x_P)}^3-{(x_w-x_P)}^3}\right] + HOT
\end{equation}
Mit der gleichen Vorgehensweise kann auch der der östliche Rand behandelt werden. Darauf
sei jedoch hier nicht weiter eingegangen.



\subsection{Zweidimensionale Probleme}


Bei gleichem Vorgehen wie in Abschnitt~\ref{sec:source1d} beschrieben, ergibt sich für
eine zweidimensionale Funktion $f(x, y)$ in einem in Abbildung~\ref{fig:kv2d} 
dargestelltem Kontrollvolumen der folgende Ausdruck:
\begin{equation}
  \begin{IEEEeqnarraybox}[][c]{rCl}
  \int_{y_s}^{y_n}\int_{x_w}^{x_e} f dx dy &=& f_P (x_e-x_w)(y_n-y_s)\\
  & &+ \pderf{x}\bigg\vert_P {\frac{(x_e-x_P)^2 - (x_w-x_P)^2}{2}} (y_n-y_s)\\
  && + \pderf{y}\bigg\vert_P {\frac{(y_n-y_P)^2-(y_s-y_P)^2}{2}} (x_e-x_w) \\
  & &+ \frac{1}{2} \pderfs{x^2}\bigg\vert_P\frac{(x_e-x_P)^3 - (x_w-x_P)^3}{3} (y_n-y_s)\\
  &&+ \frac{1}{2} \pderfs{y^2}\bigg\vert_P \frac{(y_n-y_P)^3-(y_s-y_P)^3}{3} (x_e-x_w) \\
  & &+ \pderfs{x\partial y}\bigg\vert_P {\frac{(x_e-x_P)^2 - (x_w-x_P)^2}{2}} \cdot
  {\frac{(y_n-y_P)^2-(y_s-y_P)^2}{2}} + HOT
\end{IEEEeqnarraybox}
\end{equation}
Die sich den ersten Ableitungen anschließenden Terme ergeben bei Auswertung wiederum Null
und müssen deshalb nicht weiter beachtet werden.
Der Wert von $f$ ist im Mittelpunkt des Kontrollvolumens bekannt, nicht aber die auftretenden
Ableitungen von $f$. Diese müssen deshalb diskretisiert werden. Man kann hierbei auf
Gleichung~\ref{eq:diskretisierung_f''P} zurückgreifen und ebenfalls für die $y$-Richtung anpassen.
Der Abbruchfehler ergibt sich damit zu:
\begin{align}
  \begin{split}
    TE_{Q, 2D} =&                \frac{1}{2}
  \left[{\frac{f_E-f_P}{(x_E-x_P)(x_e-x_w)}-\frac{f_P-f_W}{(x_P-x_W)(x_e-x_w)}  }\right]
  \frac{(x_e-x_P)^3 - (x_w-x_P)^3}{3} (y_n-y_s)\\
  +& \frac{1}{2}
  \left[{\frac{f_N-f_P}{(y_N-y_P)(y_n-y_s)}-\frac{f_P-f_S}{(y_P-y_S)(y_n-y_s)}  }\right]
  \frac{(y_n-y_P)^3-(y_s-y_P)^3}{3} (x_e-x_w) \\ +&\ HOT
  \end{split}
\end{align}
Die Betrachtung von Randkontrollvolumen erfolgt äquivalent zum Eindimensionalen. Es müssen
jedoch auch die Fälle für nördliche und südliche Ränder, als auch die Ecken betrachtet werden.
\begin{figure}[hbt]
\begin{tikzpicture}[scale=1.1]
  %\draw[->] (6,0) -- (6.5,0) node[right] {$x$} coordinate(x axis);
  %\draw[->] (0,6) -- (0,6.5) node[above] {$y$} coordinate(y axis);
  \fill[tud0a] (2,2) rectangle +(2,2);
  \draw[step=2cm, thick] (-2.00001, 0) grid (8, 6);

  \fill (3, 3)  circle[radius=1.5pt];
  \node (N) at (3,3) [label=above:$P$]{};

  \fill (3, 5)  circle[radius=1.5pt];
  \node (N) at (3,5) [label=above:$N$]{};

  \fill (3, 1)  circle[radius=1.5pt];
  \node (N) at (3,1) [label=below:$S$]{};

  \fill (5, 3)  circle[radius=1.5pt];
  \node (N) at (5,3) [label=right:$E$]{};

  \fill (1, 3)  circle[radius=1.5pt];
  \node (N) at (1,3) [label=left:$W$]{};

  \fill (7, 3)  circle[radius=1.5pt];
  \node (N) at (7,3) [label=right:$EE$]{};

  \fill (-1, 3)  circle[radius=1.5pt];
  \node (N) at (-1,3) [label=left:$WW$]{};

  \fill (5, 5)  circle[radius=1.5pt];
  \node (N) at (5,5) [label={above right}:$NE$]{};
  \fill (5, 1)  circle[radius=1.5pt];
  \node (N) at (5,1) [label={below right}:$SE$]{};
  \fill (1, 1)  circle[radius=1.5pt];
  \node (N) at (1,1) [label={below left}:$SW$]{};
  \fill (1, 5)  circle[radius=1.5pt];
  \node (N) at (1,5) [label={above left}:$NW$]{};

  \fill (4, 3)  circle[radius=1.5pt];
  \node (N) at (4,3) [label={right=0pt}:$e$]{};

  \fill (3, 2)  circle[radius=1.5pt];
  \node (N) at (3,2) [label={below=0pt}:$s$]{};

  \fill (2, 3)  circle[radius=1.5pt];
  \node (N) at (2,3) [label={left=0pt}:$w$]{};

  \fill (3, 4)  circle[radius=1.5pt];
  \node (N) at (3,4) [label={above=0pt}:$n$]{};

  \fill (6, 3)  circle[radius=1.5pt];
  \node (N) at (6,3) [label=right:$ee$]{};

  \fill (0, 3)  circle[radius=1.5pt];
  \node (N) at (0,3) [label=left:$ww$]{};


\end{tikzpicture}
\centering
\caption{Allgemeines Kontrollvolumen im zweidimensionalen Fall}
\label{fig:kv2d}
\end{figure}


\subsection{Äquidistante Gitter}

Für äquidistante Gitter vereinfacht sich der Terme des Abbruchfehlers und es ergeben sich
für ein- und zweidimensionale Fälle die folgenden Gleichungen.

\begin{align*}
  TE_{Q,\ 1D} &= \frac{\Delta x}{24} \left({f_E-2f_P+f_W}\right) + HOT\\
  TE_{Q,\ 2D} &= \frac{1}{24} \Delta x \Delta y \left[{f_E+f_W+f_N+f_S - 4f_P} \right] + HOT
\end{align*}

\newpage
