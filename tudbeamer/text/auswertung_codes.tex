\section{Verifikation der Lösungsprogramme}

Zur Berechnung des Abbruchfehlers wird die Lösung des mit finiten Volumen
diskretisierten Problems benötigt. Dafür wurden Lösungsprogramme
für ein- und zweidimensionale Probleme in Matlab implementiert.
Diese arbeiten mit orthogonalen Gittern und können reine Diffusionsprobleme
oder kombinierte Diffusions- und Konvektionsprobleme lösen.

Vor der Verifizierung des abgeleiteten Indikators für den Abbruchfehler muss daher die
korrekte Funktion der Programme nachgewiesen werden. Dies geschieht wie in Abschnitt
\ref{sec:verifik_allg} beschrieben über den Vergleich der formalen mit der beobachteten
Konvergenzordnung. Die formale Konvergenzordnung ist hierbei abhängig vom
verwendeten Diskretisierungsverfahren. Die beobachtete Konvergenzordnung wird mit
Gleichung~\ref{eq:beobachtet} berechnet.

%Die beobachtete Konvergenzordnung lässt sich neben den berechneten Zahlenwerten auch gut
%an der Steigung in einem doppelt-logarithmischen Diagramm ablesen. Entsteht dort eine
%Gerade, so bedeutet das, dass die Lösung mit einen festen Koeffizienten konvergiert.

\paragraph{Eindimensionales Problemgebiet, Diffusion}
\noindent
Hier wurde ein kartesisches Gitter verwendet und sukzessive verfeinert. Die
formale Konvergenzordnung beträgt zwei.
\begin{table}[h]
  \begin{tabular}{r r r}
  \toprule
  N & Gemittelter Fehler & Beobachtete Konvergenzordnung \\
  \midrule
  5  & $3.33\cdot10^{-2}$ & \multirow{2}{*}{2,020}\\
  10 & $8.21\cdot10^{-3}$ & \multirow{2}{*}{2,005}\\
  20 & $2.05\cdot10^{-3}$ & \multirow{2}{*}{2,001}\\
  40 & $5.11\cdot10^{-4}$ & \multirow{2}{*}{2,000}\\
  80 & $1.28\cdot10^{-4}$ & \\
  \bottomrule
\end{tabular}
\caption{Konvergenzordnung eindimensionale Diffusion}
\end{table}




\paragraph{Eindimensionales Problemgebiet, Diffusion und Konvektion (CDS)}
\noindent
Hier wurde ein orthogonales Gitter mit dem Koeffizienten $\alpha=0,9$ erstellt.
Bei der Berechnung der beobachteten Konvergenzordnung muss beachtet werden, dass mit dem
feinsten Gitter begonnen wird und dieses schrittweise ausgedünnt wird. Andernfalls
wird die Konvergenzordnung falsch bestimmt.
Die formale Konvergenzordnung beträgt zwei.
\begin{table}[h]
  \begin{tabular}{r r r}
  \toprule
  N & Gemittelter Fehler & Beobachtete Konvergenzordnung \\
  \midrule
  5  & $1,27\cdot10^{-1}$ & \multirow{2}{*}{2,731}\\
  10 & $1.91\cdot10^{-2}$ & \multirow{2}{*}{2,227}\\
  20 & $4.07\cdot10^{-3}$ & \multirow{2}{*}{2,032}\\
  40 & $9.96\cdot10^{-4}$ & \\
  \bottomrule
\end{tabular}
\caption{Konvergenzordnung eindimensionale Diffusion und Konvektion (CDS)}
\end{table}

%\begin{figure}[htb]
  %\begin{tikzpicture}
  %\begin{loglogaxis}[%xlabel=Anzahl der Kontrollvolumen
    %%$N$,
%ylabel=Absoluter Fehler,
  %width=0.55\textwidth,height=4.2cm]
  %\addplot[color=tud2d,mark=*, very thick] coordinates {
    %(5, 1.2662511400e-01)
    %(10,1.9068695077e-02)
    %(20,4.0741696509e-03)
    %(40,9.9564249986e-04)
  %};
  %\end{loglogaxis}
%\end{tikzpicture}
%\caption{Fehlerentwicklung eindimensionale Diffusion und Konvektion}
%\end{figure}


\vspace{4cm}


\paragraph{Eindimensionales Problemgebiet, Diffusion und Konvektion (Flux-Blending)}
\noindent
Hier wurde ein kartesisches Gitter verwendet und sukzessive verfeinert. Der
Gewichtungsfaktor $\beta$ wurde auf den Wert $0,5$ gesetzt. Die formale
Konvergenzordnung liegt aufgrund des Flux-Blending Verfahrens bei eins.
\begin{table}[h]
  \begin{tabular}{r r r}
  \toprule
  N & Gemittelter Fehler & Beobachtete Konvergenzordnung \\
  \midrule
  5  & $4,73\cdot10^{-2}$ & \multirow{2}{*}{1,215}\\
  10 & $2,04\cdot10^{-2}$ & \multirow{2}{*}{1,113}\\
  20 & $9,41\cdot10^{-3}$ & \multirow{2}{*}{1,057}\\
  40 & $4,52\cdot10^{-3}$ & \multirow{2}{*}{1,029}\\
  80 & $2,22\cdot10^{-3}$ & \\
  \bottomrule
\end{tabular}
\caption{Konvergenzordnung eindimensionale Diffusion und Konvektion (Flux-Blending)}
\end{table}


\paragraph{Zweidimensionales Problemgebiet, Diffusion}
\noindent
Hier wird ein orthogonales Gitter mit $\alpha=0,95$ verwendet. 
Die formale Konvergenzordnung beträgt zwei. 
\begin{table}[h]
  \begin{tabular}{r r r}
  \toprule
  N & Gemittelter Fehler & Beobachtete Konvergenzordnung \\
  \midrule
  5  & $2,55\cdot10^{-2}$ & \multirow{2}{*}{2,272}\\
  10 & $5,28\cdot10^{-3}$ & \multirow{2}{*}{2,058}\\
  20 & $1.27\cdot10^{-3}$ & \multirow{2}{*}{2,012}\\
  40 & $3.15\cdot10^{-4}$ & \\
  \bottomrule
\end{tabular}
\caption{Konvergenzordnung zweidimensionale Diffusion}
\end{table}


\paragraph{Zweidimensionales Problemgebiet, Diffusion und Konvektion (CDS)}
\noindent
Hier wird ein orthogonales Gitter mit $\alpha=0,9$ verwendet. Die
formale Konvergenzordnung beträgt zwei.
\begin{table}[h]
  \begin{tabular}{r r r}
  \toprule
  N & Gemittelter Fehler & Beobachtete Konvergenzordnung \\
  \midrule
  5  & $5,78\cdot10^{-2}$ & \multirow{2}{*}{2,651}\\
  10 & $9,21\cdot10^{-3}$ & \multirow{2}{*}{2,160}\\
  20 & $2,06\cdot10^{-3}$ & \multirow{2}{*}{2,022}\\
  40 & $5,07\cdot10^{-4}$ & \\
  \bottomrule
\end{tabular}
\caption{Konvergenzordnung zweidimensionale Diffusion und Konvektion}
\end{table}


\paragraph{Zusammenfassung}
\noindent
Mit den verwendeten Testfällen werden kartesische und orthogonale Gitter getestet.
Die Testfunktionen überprüfen zudem alle Randbedingungen.
Bei allen Lösungsprogrammen stimmen formale und beobachtete Konvergenzordnung überein, was auf
eine korrekte Implementierung hindeutet.

\clearpage
