\section{Abbruchfehler von Konvektionstermen}

Wie in Abschnitt~\ref{sec:konv_fluss} beschrieben, existieren verschiedene Verfahren
um konvektive Flüsse abzuschätzen. Der Abbruchfehler ist vom genutzten Verfahren
abhängig und muss für jedes Verfahren hergeleitet werden.

Zu diesem Fehler kommt ab der zweiten Dimension wiederum der Abbruchfehler aus
der Integralapproximation, der bereits in Abschnitt~\ref{sec:source1d} hergeleitet wurde.

\subsection{CDS-Verfahren}
\label{sec:te_cds}

Die Herleitung des Abbruchfehlers für das Zentraldifferenzen-Verfahren sei hier am Beispiel des östlichen Randes dargestellt.
Man entwickelt zuerst die Taylorreihe von $\phi$ um den Entwicklungspunkt $x_P$ und wertet sie anschließend
an den Punkten $x_e$ und $x_E$ aus. Es ergeben sich die folgenden Reihendarstellungen:
\begin{align}
  \phi_e &= \phi_P + \phi'_P(x_e-x_P)+\frac{1}{2}\phi''_P(x_e-x_P)^2
  +\frac{1}{6}\phi'''_P(x_e-x_P)^3+HOT
  \label{eq:taylor_konv_eP}\\
  \phi_E &= \phi_P + \phi'_P(x_E-x_P)+\frac{1}{2}\phi''_P(x_E-x_P)^2
  +\frac{1}{6}\phi'''_P(x_E-x_P)^3+HOT
  \label{eq:taylor_konv_eE}
\end{align}
Werden die Gleichungen~\ref{eq:taylor_konv_eE} und \ref{eq:taylor_konv_eP} nun
voneinander subtrahiert und nach $\phi_e$ umgestellt, so ergibt sich mit dem
Interpolationsfaktor $\gamma_e$:

\begin{equation*}
  \begin{IEEEeqnarraybox}[][c]{rCl}
    \frac{\phi_e}{x_e-x_P} &=& \frac{\phi_E}{x_E-x_P} + \frac{\phi_P}{x_e-x_P} -
  \frac{\phi_P}{x_E-x_P} + \frac{1}{2} \phi''_P \left({(x_e-x_P)-(x_E-x_P)}\right)\\
  &&+ \frac{1}{6} \phi'''_P \left({(x_e-x_P)^2-(x_E-x_P)^2}\right)+HOT\\
  \phi_e &=& \phi_E \frac{x_e-x_P}{x_E-x_P} + \phi_P \left({1-\frac{x_e-x_P}{x_E-x_P} }\right)
  + \frac{1}{2} \phi''_P (x_e-x_E)(x_e-x_P)\\
  &&+ \frac{1}{6} \phi'''_P \left({(x_e-x_P)^2-(x_E-x_P)^2}\right)(x_e-x_P)+HOT\\
   &=& \phi_E \gamma_e + \phi_P (1-\gamma_e)+ \frac{1}{2} \phi''_P (x_e-x_E)(x_e-x_P)\\
         &&+ \frac{1}{6} \phi'''_P \left({(x_e-x_P)^2-(x_E-x_P)^2}\right)(x_e-x_P)+HOT
  \end{IEEEeqnarraybox}
\end{equation*}
Der Abbruchfehler lässt sich hier leicht ablesen. Er lautet:
\begin{equation*}
  TE_{e, CDS} =  \frac{1}{2} \phi''_P (x_e-x_E)(x_e-x_P)+ \frac{1}{6}
  \phi'''_P \left({(x_e-x_P)^2-(x_E-x_P)^2}\right)(x_e-x_P)+HOT
\end{equation*}
Die hier vorliegenden analytischen Ableitungen werden im nächsten Schritt durch Approximationen ersetzt.
\begin{align*}
  %\phi'_P &= \frac{\phi_E-\phi_W}{x_E-x_W}\\
  \phi''_P &= \frac{1}{(x_e-x_w)} \left({\frac{\phi_E-\phi_P}{x_E-x_P}
  - \frac{\phi_P-\phi_W}{x_P-x_W} }\right)\\
  %\label{eq:ddphip}\\
  \phi'''_P &= \frac{1}{(x_e-x_w)} \left({
  \frac{1}{(x_E-x_P)} \left({\frac{\phi_{EE}-\phi_P}{x_{EE}-x_P}- \frac{\phi_E-\phi_W}{x_E-x_W} }\right)-
  \frac{1}{(x_P-x_W)} \left({\frac{\phi_E-\phi_W}{x_E-x_W} - \frac{\phi_P-\phi_{WW}}{x_P-x_{WW}} }\right)
  }\right) \label{eq:dddphip}
\end{align*}
Der Abbruchfehler ergibt sich damit zu folgendem Term.
\begin{equation}
  \begin{IEEEeqnarraybox}{rCl}
    TE_{CDS,\,e} &=&  \frac{1}{2} \frac{1}{(x_e-x_w)} \left({\frac{\phi_E-\phi_P}{x_E-x_P}
  - \frac{\phi_P-\phi_W}{x_P-x_W} }\right) (x_e-x_E) (x_e-x_P) \nonumber \\
  &&+
 \frac{1}{(x_e-x_w)} \bigg(
  \frac{1}{(x_E-x_P)} \left({\frac{\phi_{EE}-\phi_P}{x_{EE}-x_P}- \frac{\phi_E-\phi_W}{x_E-x_W} }\right)
  \\&&-
  \frac{1}{(x_P-x_W)} \left({\frac{\phi_E-\phi_W}{x_E-x_W} - \frac{\phi_P-\phi_{WW}}{x_P-x_{WW}} }\right)
  \bigg) \nonumber \\
  &&\cdot \frac{1}{6} \left({(x_e-x_P)^2-(x_E-x_P)^2}\right)(x_e-x_P)+HOT
  \end{IEEEeqnarraybox}
\end{equation}
Im Westen wird die Ableitung äquivalent zum Osten durchgeführt. Für den Abbruchfehler ergibt sich letztendlich
folgender Term:
%\begin{align*}
  %\begin{IEEEeqnarraybox}{rCl}
  %\phi_w &= \left({1-\frac{x_w-x_P}{x_W-x_P}}\right)\phi_P + \left({\frac{x_w-x_P}{x_W-x_P} }\right) \phi_W
  %+ \frac{1}{2} \phi''_P \left({(x_w-x_P)-(x_W-x_P)}\right)(x_w-x_P)\\
  %&+ \frac{1}{6} \phi'''_P \left({(x_w-x_P)^2-(x_W-x_P)^2}\right)(x_w-x_P)\\
  %\phi_w &= \left({1-\frac{x_P-x_w}{x_P-x_W}}\right)\phi_P + \left({\frac{x_P-x_w}{x_P-x_W} }\right) \phi_W
  %+ \frac{1}{2} \phi''_P \left({x_w-x_W}\right)(x_w-x_P)\\
  %&+ \frac{1}{6} \phi'''_P \left({(x_w-x_P)^2-(x_W-x_P)^2}\right)(x_w-x_P)
  %\end{IEEEeqnarraybox}
%\end{align*}

%Der Abbruchfehler lässt sich nach der Diskretisierung der unbekannten Ableitungen nach Gleichung~\ref{eq:ddphip}
%und \ref{eq:dddphip} leicht ablesen und ergibt sich zu:

\begin{equation}
  \begin{IEEEeqnarraybox}[][b]{rCl}
    TE_{CDS,\,w} &=&  \frac{1}{2} \frac{1}{(x_e-x_w)} \left({\frac{\phi_E-\phi_P}{x_E-x_P}
  - \frac{\phi_P-\phi_W}{x_P-x_W} }\right) \left({x_w-x_W}\right)(x_w-x_P)  \nonumber\\
  &&+ \frac{1}{(x_e-x_w)} \bigg(
  \frac{1}{(x_E-x_P)} \left({\frac{\phi_{EE}-\phi_P}{x_{EE}-x_P}- \frac{\phi_E-\phi_W}{x_E-x_W} }\right)
  \\ && -
  \frac{1}{(x_P-x_W)} \left({\frac{\phi_E-\phi_W}{x_E-x_W} - \frac{\phi_P-\phi_{WW}}{x_P-x_{WW}} }\right)
  \bigg) \nonumber\\
  &&\frac{1}{6}  \left({(x_w-x_P)^2-(x_W-x_P)^2}\right)(x_w-x_P)
  \end{IEEEeqnarraybox}
\end{equation}

\subsubsection{Randwerte}
\label{sec:te_cds_rand}

Bei Kontrollvolumen direkt am Rand, wie zum Beispiel das in Abbildung~\ref{fig:kv1d_rand}
hervorgehobene Kontrollvolumen, müssen einige Besonderheiten beachtet werden.
So ist der konvektive Fluss, der direkt auf dem Rand verläuft, durch die Randbedingung ($\phi^w$) gegeben und
muss dementsprechend nicht angepasst werden. Am östlichen Rand muss hingegen die Approximation
der Ableitungen angepasst werden. Sie lauten dann:
\begin{align*}
  \phi''_P &= \frac{1}{(x_e-x_w)} \left({\frac{\phi_E-\phi_P}{x_E-x_P}
  - \frac{\phi_P-\phi^w}{x_P-x_w} }\right)\\
  \phi'''_P &= \frac{1}{(x_e-x_w)} \left({
  \frac{1}{(x_E-x_P)} \left({\frac{\phi_{EE}-\phi_P}{x_{EE}-x_P}- \frac{\phi_E-\phi^w}{x_E-x_w} }\right)-
  \frac{1}{(x_P-x_w)} \left({\frac{\phi_E-\phi^w}{x_E-x_w} - \frac{\phi_P-\phi^{w}}{x_P-x_{w}} }\right)
  }\right)
\end{align*}
Hat das betrachtete Kontrollvolumen  einen Abstand von eins zum Rand, so muss nur die Approximation der dritten
Ableitung angepasst werden.
\begin{equation*}
  \phi'''_P = \frac{1}{(x_e-x_w)} \left({
  \frac{1}{(x_E-x_P)} \left({\frac{\phi_{EE}-\phi_P}{x_{EE}-x_P}- \frac{\phi_E-\phi^w}{x_E-x_w} }\right)-
  \frac{1}{(x_P-x_W)} \left({\frac{\phi_E-\phi_W}{x_E-x_W} - \frac{\phi_P-\phi^{w}}{x_P-x_{w}} }\right)
  }\right)
\end{equation*}
Auf gleiche Art und Weise kann auch der östliche Rand behandelt werden.
%Hier muss abschließend $\phi_e$ aus $\phi_E$ und $\phi_P$ interpoliert werden. Letztendlich ergibt
%sich für äquidistante Gitter folgender Term:

%\begin{equation*}
  %TE_{e, CDS} = -\frac{1}{8} (\phi_E-2\phi_P+\phi_W) - \frac{1}{32}
  %(2\phi^{ee} - 3\phi_E + 2\phi_W - \phi_{WW})
%\end{equation*}

%Auf gleiche Art und Weise kann auch $TE_{w, CDS}$ berechnet werden.
\subsubsection{Äquidistante Gitter}

Für äquidistante Gitter ergeben sich die folgenden vereinfachten Formeln zur Berechnung der konvektiven Flüsse
mit dem CDS-Verfahren.

\begin{equation*}
  TE_{CDS,\,e} = -\frac{1}{8} (\phi_E-2\phi_P+\phi_W) - \frac{1}{32}
  (\phi_{EE} - 2\phi_E + 2\phi_W - \phi_{WW})
\end{equation*}
\begin{equation*}
  TE_{CDS,\,w} = -\frac{1}{8} (\phi_E-2\phi_P+\phi_W) + \frac{1}{32}
  (\phi_{EE} - 2\phi_E + 2\phi_W - \phi_{WW})
\end{equation*}








\subsection{UDS-Verfahren}

Um den Abbruchfehler des Upwind-Verfahrens herzuleiten, wertet man die
Taylorreihe von $\phi$ mit dem Entwicklungspunkt $x_P$ im Punkte $\phi_e$
aus.

\begin{equation*}
  \phi_e = \phi_P +(x_e-x_P) \phi'_P + \frac{1}{2} (x_e-x_P)^2 \phi''_P+HOT
\end{equation*}
Da im Falle einer positiven Geschwindigkeit das UDS-Verfahren $\phi_e$ mit $\phi_P$
gleichsetzt, lässt sich der Abbruchfehler hier direkt ablesen.
\begin{equation*}
  TE_{UDS,\,e} = (x_e-x_P) \phi'_P + \frac{1}{2} (x_e-x_P)^2 \phi''_P+HOT
\end{equation*}
Setzt man nun die diskretisierten Ableitungen aus Abschnitt~\ref{sec:te_cds} ein,
ergibt sich die Formel für den Abbruchfehler.
\begin{equation}
  TE_{UDS,\,e} = (x_e-x_P) \frac{\phi_E-\phi_W}{x_E-x_W}+
  \frac{1}{2} \frac{(x_e-x_P)^2}{(x_e-x_w)} \left({\frac{\phi_E-\phi_P}{x_E-x_P}
  - \frac{\phi_P-\phi_W}{x_P-x_W} }\right)+HOT
\end{equation}
Betrachtet man den westlichen Rand und geht von einer positiven Geschwindigkeit aus,
so ergibt sich der folgende Abbruchfehler:
\begin{equation*}
  TE_{UDS,\,w} = (x_w-x_W) \phi'_W + \frac{1}{2} (x_w-x_W)^2 \phi''_W + HOT
\end{equation*}
Hier werden nun wiederum die Ableitungen $\phi'_W$ und $\phi''_W$ diskretisiert. Der
Abbruchfehler lautet damit:
\begin{equation}
  TE_{UDS,\,w} = (x_w-x_W) \frac{\phi_P-\phi_{WW}}{x_P-x_{WW}}+
  \frac{1}{2} \frac{(x_w-x_W)^2}{(x_w-x_{ww})} \left({\frac{\phi_P-\phi_W}{x_P-x_W}
  - \frac{\phi_W-\phi_{WW}}{x_W-x_{WW}} }\right)+HOT
\end{equation}



\subsubsection{Randwerte}

Auch beim UDS-Verfahren müssen am Rand die Approximationen der verwendeten Ableitungen
so angepasst werden, dass nur vorhandene Werte benutzt werden.
Für den östlichen Rand des Kontrollvolumens können dabei die in Abschnitt~\ref{sec:te_cds_rand}
gezeigten Differenzenquotienten genutzt werden.

Für das Kontrollvolumen direkt am Rand ist der konvektive Fluss gegeben. Es gibt somit keinen
Abbruchfehler. Für das daneben liegende Kontrollvolumen müssen die Differenzenquotienten wie folgt angepasst werden:
\begin{align*}
  \phi'_W &= \frac{\phi_w-\phi^w}{x_w-x_{ww}}\\
  \phi''_W &= \frac{1}{x_W-x_{ww}} \left({\frac{\phi_w-\phi^w}{x_w-x_{ww}}
- \frac{\phi_W-\phi^w}{x_W-x_{ww}} }\right)+HOT
\end{align*}
$\phi_w$ wird hier beispielsweise aus der linearen Interpolation von $\phi_P$ und $\phi_W$ berechnet.
\begin{equation*}
  \phi_w = \phi_W \frac{x_P-x_w}{x_P-x_W} + \phi_P \frac{x_w-x_W}{x_P-x_W}
\end{equation*}
%etzt ergeben sich damit die folgenden Terme.
%\begin{align}
  %\phi'_W &= \frac{\phi_W \frac{x_P-x_w}{x_P-x_W} + \phi_P \frac{x_w-x_W}{x_P-x_W}
%-\phi^w}{x_w-x_{ww}}\\
  %\phi''_W &= \frac{1}{x_W-x_{ww}} \left({\frac{\phi_W \frac{x_P-x_w}{x_P-x_W} + \phi_P \frac{x_w-x_W}{x_P-x_W}
%-\phi^w}{x_w-x_{ww}}
%- \frac{\phi_W-\phi^w}{x_W-x_{ww}} }\right)+HOT
%\end{align}


\subsubsection{Äquidistante Gitter}

%\begin{align*}
  %\phi'_W  &= \frac{1}{2\Delta x} \phi_P + \frac{1}{2\Delta x} \phi_W - \phi^w\\
  %\phi''_W &= \frac{1}{\Delta x^2} \phi_P -\frac{1}{\Delta x^2} \phi_W
%\end{align*}
Für äquidistante Gitter ergeben sich folgende Gleichungen für den Abbruchfehler.

\begin{equation*}
  TE_{UDS,\,e} = \frac{3}{8} \phi_E-\frac{1}{4} \phi_P - \frac{1}{8} \phi_W
\end{equation*}
\begin{equation*}
  TE_{UDS,\,w} = \frac{3}{8} \phi_P-\frac{1}{4} \phi_W - \frac{1}{8} \phi_{WW}
\end{equation*}





\subsection{``Flux-Blending''-Verfahren}

Das Flux-Blending-Verfahren setzt sich aus UDS- und CDS-Verfahren zusammen. Beide
Verfahren werden dabei über den Faktor $\beta$ gewichtet. Dementsprechend müssen auch die
Abbruchfehler über $\beta$ gewichtet werden um den Abbruchfehler für das ``Flux-Blending''-Verfahren zu berechnen.
Mit Gleichung~\ref{eq:flux_blending} ergibt sich damit:

\begin{equation}
  TE_{e, Flux} = (1-\beta) TE_{e, UDS} + \beta TE_{e, CDS}
\end{equation}



%\section{Truncation Error eines Kontrollvolumens}
%\label{sec:Truncation Error eines Kontrollvolumens}

%Der Truncation Error für ein Kontrollvolumen setzt sich nun aus den Fehlern von Quell-
%und Diffusionstermen zusammen:

%\begin{equation*}
  %TE = \frac{TE_{source} - TE_e - TE_w}{\Delta x}
%\end{equation*}
%Für den äquidistanten Fall ergibt sich damit für zentrale Kontrollvolumen der folgende
%Truncation Error. Wichtig ist es, hier die durch den Gauß'schen Integralsatz 
%entstehenden Vorzeichen mit zu beachten. Weiterhin muss durch $\Delta x$ geteilt werden,
%da die gleiche Transformation beim Aufstellen des Gesamtgleichungsystem angewendet wird.

%\begin{align}
  %TE &= \frac{\frac{\Delta x}{24} \left({f_E-2f_P+f_W}\right)
   %+\frac{1}{24\Delta x}\left({
%\phi_{EE}-3\phi_E+3\phi_P-\phi_W}\right)
  %-\frac{1}{24 \Delta x}\left({
%\phi_E-3\phi_P+3\phi_W-\phi_{WW}}\right)}{\Delta x}
%\end{align}

\subsection{Zweidimensionale Probleme}
Die gezeigten Herleitungen lassen sich problemlos ins Zweidimensionale übertragen. Es muss
jedoch darauf geachtet werden, den Fehler aus der Integralapproximation nicht zu vergessen.
Dieser lautet für die Ostseite:
\begin{equation}
  TE_{C,\,int,\,e} = \frac{1}{6} \frac{\partial^2\phi}{\partial x^2}\bigg\vert_e
    \left[{{(y_n-y_P)}^3-{(y_s-y_P)}^3}\right] + HOT
\end{equation}
Er lässt sich leicht auf die anderen Seiten des Kontrollvolumens übertragen. Die Diskretisierung
der analytischen Ableitung wurde ebenfalls bereits gezeigt.
\clearpage
