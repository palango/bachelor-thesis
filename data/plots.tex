\documentclass[10pt, ngerman,colorback,accentcolor=tud2d]{tudreport}
\usepackage[utf8]{inputenc}
\usepackage[ngerman]{babel}
\usepackage{amsmath}

\usepackage{microtype}

\usepackage{pgfplots}
\pgfplotsset{compat=1.3}

\begin{document}

\chapter{Diffusion}
\section{Testfunktion 1}

$1 + \sin(\pi x)$

\begin{figure}[h]
\begin{tikzpicture}
    \begin{axis}[
        height=8cm,
        width=0.9*\textwidth,
        grid=major,
        xlabel=$x$,
        ylabel=$\phi$,
    ]
    \addplot[tud2d, mark=*] file {solution_20_aequi.txt};
    \end{axis}
\end{tikzpicture}
\caption{Lösung des Diffusionsproblems}
\end{figure}

\begin{figure}[h]
\begin{tikzpicture}
    \begin{axis}[
        height=8cm,
        width=0.9*\textwidth,
        grid=major,
        xlabel=$x$,
        ylabel=$\phi - \phi_{exakt}$,
%scaled y ticks = false,
    ]
    \addplot[tud2d, mark=*] file {error_20_aequi.txt};
    \end{axis}
\end{tikzpicture}
\caption{Fehlerverteilung der Lösung}
\end{figure}

\begin{figure}[h]
\begin{tikzpicture}
    \begin{axis}[
        height=8cm,
        width=0.9*\textwidth,
        grid=major,
        xlabel=$x$,
        ylabel=$A\phi_{exakt}-b$,
%scaled y ticks = false,
    ]
    \addplot[tud2d, mark=*] file {residuum_20_aequi.txt};
    \end{axis}
\end{tikzpicture}
\caption{Residuum der Lösung}
\end{figure}

\begin{figure}[h]
\begin{tikzpicture}
    \begin{axis}[
        height=8cm,
        width=0.9*\textwidth,
        grid=major,
        xlabel=$x$,
        ylabel=$A\phi_{exakt}-b$,
%scaled y ticks = false,
    ]
    \addplot[tud2d, mark=*] file {te_20_aequi.txt};
    \end{axis}
\end{tikzpicture}
\caption{Truncation Error Lösung}
\end{figure}






\newpage
\pagebreak
\pagebreak

\section{Testfunktion 2}

$\cos(\pi x)-1$



\begin{figure}[h]
\begin{tikzpicture}
    \begin{axis}[
        height=8cm,
        width=0.9*\textwidth,
        grid=major,
        xlabel=$x$,
        ylabel=$\phi$,
    ]
    \addplot[tud2d, mark=*] file {solution_20_aequi_cos.txt};
    \end{axis}
\end{tikzpicture}
\caption{Lösung des Diffusionsproblems}
\end{figure}

\begin{figure}[h]
\begin{tikzpicture}
    \begin{axis}[
        height=8cm,
        width=0.9*\textwidth,
        grid=major,
        xlabel=$x$,
        ylabel=$\phi - \phi_{exakt}$,
%scaled y ticks = false,
    ]
    \addplot[tud2d, mark=*] file {error_20_aequi_cos.txt};
    \end{axis}
\end{tikzpicture}
\caption{Fehlerverteilung der Lösung}
\end{figure}

\begin{figure}[h]
\begin{tikzpicture}
    \begin{axis}[
        height=8cm,
        width=0.9*\textwidth,
        grid=major,
        xlabel=$x$,
        ylabel=$A\phi_{exakt}-b$,
%scaled y ticks = false,
    ]
    \addplot[tud2d, mark=*] file {residuum_20_aequi_cos.txt};
    \end{axis}
\end{tikzpicture}
\caption{Residuum der Lösung}
\end{figure}

\begin{figure}[h]
\begin{tikzpicture}
    \begin{axis}[
        height=8cm,
        width=0.9*\textwidth,
        grid=major,
        xlabel=$x$,
        ylabel=$A\phi_{exakt}-b$,
%scaled y ticks = false,
    ]
    \addplot[tud2d, mark=*] file {residuum_20_aequi_cos_without_border.txt};
    \end{axis}
\end{tikzpicture}
\caption{Residuum der Lösung (ohne Sprünge am Rand)}
\end{figure}

\begin{figure}[h]
\begin{tikzpicture}
    \begin{axis}[
        height=8cm,
        width=0.9*\textwidth,
        grid=major,
        xlabel=$x$,
        ylabel=$A\phi_{exakt}-b$,
%scaled y ticks = false,
    ]
    \addplot[tud2d, mark=*] file {te_20_aequi_cos.txt};
    \end{axis}
\end{tikzpicture}
\caption{Truncation Error Lösung}
\end{figure}


\end{document}
