\documentclass[10pt, ngerman,colorback,accentcolor=tud2d]{tudreport}
\usepackage[utf8]{inputenc}
\usepackage[ngerman]{babel}
\usepackage{amsmath}

\usepackage{microtype}

\usepackage{pgfplots}
\pgfplotsset{compat=1.8}

\usepackage{listings}
\begin{document}

\chapter{2D-Diffusionsgleichung für orthogonale Gitter}
\label{chap:2D-Diffusionsgleichung}
\newcommand{\dx}{(x_e-x_w)}
\newcommand{\dy}{(y_n-y_s)}

Es soll nun ein zweidimensionales Diffusionsproblem behandelt werden, welches durch
Gleichung~\eqref{eq:gl_2d_konv} beschrieben werden kann.

\begin{equation}
  \alpha \frac{\partial^2\phi}{\partial x_i \partial x_i} = f
  \label{eq:gl_2d_konv}
\end{equation}

Zur Lösung soll das Finite Volumen Verfahren eingesetzt werden. Der erste Schritt besteht
deshalb in einer Integration über das rechteckige Kontrollvolumen $V$. Anschließend wird auf die
entstandene Gleichung der Satz von Gauß angewendet und das Oberflächenintegral in die
Summe der Integrale der Seitenflächen überführt.

\begin{align*}
  \alpha \frac{\partial^2\phi}{\partial x_i \partial x_i} &= f\\
  \int_V \alpha \frac{\partial^2\phi}{\partial x_i \partial x_i} dV &= \int_V f dV\\
  \int_S \alpha \frac{\partial \phi}{\partial x_i}n_i dS &= \int_V f dV\\
  \sum_c \int_{S_c} \alpha \frac{\partial \phi}{\partial x_i}n_{ci} dS_c &= \int_V f dV
\end{align*}

Durch die Anwendung der Mittelpunktsregel können die beiden auftretenden Integralausdrücke
approximiert werden, wobei $\delta S_c$ der Länge des betrachteten Abschnitts entspricht.

\begin{align*}
  \int_V f dV &\approx f_P \dx \dy\\
  \sum_c \int_{S_c} \alpha \frac{\partial \phi}{\partial x_i}n_{ci} dS_c &\approx
  \sum_c \alpha \left({\frac{\partial \phi}{\partial x_i}}\right)_c n_{ci} \delta S_c
\end{align*}

Werden nun die Approximationen genutzt und die Summe aufgelöst ergibt sich folgende
Gleichung. In dieser werden dann die auftretenden Ableitungen durch zentrale
Differenzenquotienten ersetzt.

\begin{equation*}
  \alpha \left({\frac{\partial \phi}{\partial x}}\right)_e \dy -
  \alpha \left({\frac{\partial \phi}{\partial x}}\right)_w \dy +
  \alpha \left({\frac{\partial \phi}{\partial y}}\right)_n \dx -
  \alpha \left({\frac{\partial \phi}{\partial y}}\right)_s \dx =
  f_P \dx \dy
\end{equation*}

\begin{equation*}
  \alpha \frac{\phi_E-\phi_P}{x_E-x_P} \dy -
  \alpha \frac{\phi_P-\phi_W}{x_P-x_W} \dy +
  \alpha \frac{\phi_N-\phi_P}{y_N-y_P} \dx -
  \alpha \frac{\phi_P-\phi_S}{y_P-y_S} \dx =
  f_P \dx \dy
\end{equation*}

Diese Gleichung kann nun in die Form
$a_P\phi_P +a_E\phi_E +a_W\phi_W +a_N\phi_N +a_S\phi_S =b_P$
gebracht werden, welche für das spätere Aufstellen des Gesamtgleichungssystems
hilfreich sein wird. Es ergeben sich dafür die folgenden Koeffizienten.

\begin{align*}
  a_P &= -\frac{\alpha}{\dx} \left({\frac{1}{x_E-x_P} + \frac{1}{x_P-x_W}}\right)
  -\frac{\alpha}{\dy} \left({\frac{1}{(y_N-y_P} + \frac{1}{y_P-y_S}}\right)\\
  &= -a_E-a_W-a_N-a_S\\
  a_E &= \frac{\alpha}{(x_E-x_P)\dx} \\
  a_W &= \frac{\alpha}{(x_P-x_W)\dx} \\
  a_N &= \frac{\alpha}{(y_N-y_P)\dy} \\
  a_S &= \frac{\alpha}{(y_P-y_S)\dy} \\
  b_P &=f_P
\end{align*}

Für äquidistante Gitter ergibt sich:

\begin{align*}
  a_P &= -\frac{2 \alpha}{\Delta x^2} -\frac{2 \alpha}{\Delta y^2}\\
  a_E &= \frac{\alpha}{\Delta x^2} \\
  a_W &= \frac{\alpha}{\Delta x^2} \\
  a_N &= \frac{\alpha}{\Delta y^2} \\
  a_S &= \frac{\alpha}{\Delta y^2} \\
  b_P &=f_P
\end{align*}

\section{Randbedingungen}
\label{sec:Randbedingungen}

Als Randbedingungen sollen wiederum Dirichlet-Randbedingungen gelten. Die Herleitung
soll hier beispielhaft für den östlichen Rand erfolgen.
Die anderen Seiten können auf gleiche Art und Weise hergeleitet werden.

Aus der Dirichlet-Randbedingung $\phi_e = \phi^0$ lässt sich nun die Ableitung
$\left({\frac{\partial \phi}{\partial x}}\right)_e$ schreiben als:

\begin{equation}
  \left({\frac{\partial \phi}{\partial x}}\right)_e =
  \frac{\phi_E-\phi_P}{x_E-x_P} = \frac{\phi^0-\phi_P}{x_e-x_P}
\end{equation}

Die Koeffizienten ergeben sich damit zu:

\begin{align*}
  a_P &= -\frac{\alpha}{\dx} \left({\frac{1}{x_e-x_P} + \frac{1}{x_P-x_W}}\right)
  -\frac{\alpha}{\dy} \left({\frac{1}{(y_N-y_P} + \frac{1}{y_P-y_S}}\right)\\
  a_E &= 0 \\
  a_W &= \frac{\alpha}{(x_P-x_W)\dx} \\
  a_N &= \frac{\alpha}{(y_N-y_P)\dy} \\
  a_S &= \frac{\alpha}{(y_P-y_S)\dy} \\
  b_P &=f_P- \frac{1}{(x_e-x_P)(x_e-x_w)} \phi^0
\end{align*}

Für äquidistante Gitter ergibt sich:

\begin{align*}
  a_P &= -\frac{3 \alpha}{\Delta x^2} -\frac{2 \alpha}{\Delta y^2}\\
  a_E &= 0 \\
  a_W &= \frac{\alpha}{\Delta x^2} \\
  a_N &= \frac{\alpha}{\Delta y^2} \\
  a_S &= \frac{\alpha}{\Delta y^2} \\
  b_P &=f_P-\frac{2}{\Delta x^2} \phi^0
\end{align*}


\chapter{2D-Konvektionsgleichung für orthogonale Gitter}

\end{document}
