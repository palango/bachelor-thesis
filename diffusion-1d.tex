\documentclass[10pt, ngerman,colorback,accentcolor=tud2d]{tudreport}
\usepackage[utf8]{inputenc}
\usepackage[ngerman]{babel}
\usepackage{amsmath}

\usepackage{microtype}

\usepackage{pgfplots}
\pgfplotsset{compat=1.8}

\begin{document}
%\tableofcontents
\chapter{1D-Diffusionsgleichung mit FVM}

\begin{equation} \label{base-formula}
\alpha \frac{\partial^2 \phi}{\partial x^2} = f
\end{equation}

Bei Integralbildung über den Bereich eines Kontrollvolumens von $x_w$ bis $x_e$
und der Anwendung des Gaußschen Integralsatzes kann das wie folgt umgeschrieben werden.

\begin{align*}
\int_{x_w}^{x_e}\alpha \frac{\partial^2 \phi}{\partial x^2} dx
&= \alpha \frac{\partial \phi}{\partial x}\Big|_{x_w}^{x_e}\\
&= \alpha \frac{\partial \phi}{\partial x}\Big|_{x_e}
 - \alpha \frac{\partial \phi}{\partial x}\Big|_{x_w}\\
&= \alpha (\phi'_e - \phi'_w)
\end{align*}

Nun müssen die auftretenden Ableitungen durch Variablenwerte in den Mittelpunkten
der Kontrollvolumen ausgedrückt werden.

\begin{equation*}
\int_{x_w}^{x_e}f dx \approx f_P (x_e - x_w)
\end{equation*}

\begin{equation*}
\phi'_e \approx \frac{\phi_E - \phi_P}{x_E - x_P}
\end{equation*}

\begin{equation*}
\phi'_w \approx \frac{\phi_P - \phi_W}{x_P - x_W}
\end{equation*}

Eingesetzt in Gleichung~\ref{base-formula} ergibt dies:

\begin{equation}
\alpha \frac{\phi_E - \phi_P}{x_E - x_P} + \alpha \frac{\phi_W - \phi_P}{x_W - x_P}
= f_P (x_e - x_w)
\end{equation}

Diese Gleichung kann in folgende Beziehung umgeformt werden.

\newcommand{\diffEP}{(x_E-x_P)}
\newcommand{\diffWP}{(x_W-x_P)}
\newcommand{\diffew}{(x_e-x_w)}

\begin{equation}
a_P \phi_P + a_E \phi_E + a_W \phi_W = b_P
\end{equation}

Die Koeffizienten $a_i$ und $b_P$ nehmen hierbei die folgenden Werte an.

\begin{align*}
  a_P &= -\left( \frac{\alpha}{\diffEP \diffew} + \frac{\alpha}{\diffWP \diffew} \right) \\
a_E &= \frac{\alpha}{\diffEP \diffew}\\
a_W &= \frac{\alpha}{\diffWP \diffew}\\
b_P &= f_P
\end{align*}

Für äquidistante Gitter ergeben sich vereinfachte Terme für die Koeffizienten $a$.

\begin{align*}
  a_W &= \frac{\alpha}{\Delta x^2} = a\\
  a_E &= \frac{\alpha}{\Delta x^2} = a\\
  a_P &= - \frac{2 \alpha}{\Delta x^2} = -2a
\end{align*}

Betrachtet man nun alle Kontrollvolumen, kann mit diesen Beziehungen ein 
Gleichungssystem der Form $A\phi = b$ aufgestellt werden.

\section{Randbedingungen}
\label{sec:Randbedingungen}

Es wird angenommen, das der Randwert $\phi_w= \phi^0$ gegeben ist. Damit kann die
Ableitung $\frac{\partial \phi}{\partial x}$ wie folgt approximiert werden:

\begin{equation*}
\left(\frac{\partial \phi}{\partial x}\right)_W = \frac{\phi_P - \phi_W}{x_P - x_W} =
  \frac{\phi_P - \phi^0}{x_P - x_w}
\end{equation*}

Die Koeffizienten für ein Randkontrollvolumen ergeben sich damit zu:

\begin{align*}
  a_W &= 0\\
  a_P &=  -\left( \frac{\alpha}{\diffEP \diffew} + \frac{\alpha}{\diffWP \diffew} \right) \\
  a_E &= \frac{\alpha}{\diffEP \diffew}\\
  b_P &= f_P - \frac{\alpha}{(x_P -x_w)\diffew}
\end{align*}

beziehungsweise für den äquidistanten Fall:

\begin{align*}
  a_W &= 0\\
  a_P &=  -\frac{3 \alpha}{\Delta x^2} = -3a\\
  a_E &= \frac{\alpha}{\Delta x^2} = a\\
  b_P &= f_P - \frac{\alpha}{\Delta x \cdot \frac{1}{2} \Delta x}\phi^0 = f_P - 
  \frac{2\alpha}{\Delta x^2} \phi^0 = f_P - 2a \phi^0
\end{align*}

\chapter{Manufactured Solution}

Wir wählen die folgende Lösung für $\phi$ und berechnen die Ableitungen.

\begin{align*}
  \phi(x) &= A + \sin(Bx)\\
  \phi'(x) &= B \cos(Bx)\\
  \phi''(x) &= -B^2\sin(Bx)
\end{align*}

Nun führen wir den zusätzlichen Quellterm $f_{ad}$ ein und setzen $\phi(x)$ ein.

\begin{equation*}
  \alpha\frac{\partial^2 \phi}{\partial x^2}-f=f_{ad}
\end{equation*}

so erhält man:

\begin{equation*}
  f_{ad} = -\alpha B^2\sin(BX)-f
\end{equation*}

Damit ergibt sich als Manufactured Solution:
\begin{align}
  \alpha\frac{\partial^2 \phi}{\partial x^2} &= f -\alpha B^2\sin(Bx)-f\\
                                             &= -\alpha B^2\sin(Bx)
\end{align}

\section{Randbedingungen}
\label{sec:Randbedingungen}

Um die Konstanten $A$ und $B$ in der Manufactured Solution bestimmen zu können,
müssen konkrete Randbedingungen für das Problem festgelegt werden. Beispielhaft
soll das hier mit den Randwerten $\phi(0) = 1$ und $\phi(1) = 1$ geschehen.
Damit ergeben sich die Konstanten zu $A = 1$ sowie $B = \pi$. Die 
konstruierte Lösung $\phi(x)$ hat damit die folgende Form und der Quellterm $f$
ergibt sich zu:
\begin{equation*}
  \phi(x) = 1 + \sin(\pi x)
\end{equation*}
\begin{equation}
  f=-\pi^2 \sin(\pi x)
  \label{eq:quellterm}
\end{equation}

\section{Konvergenzuntersuchung}
\label{sec:Konvergenzuntersuchung}

Der Fehler $E$ wird definiert als Differenz zwischen diskreter Lösung $f(\Delta)$ sowie
der exakten Lösung $f_{exakt}$.
\begin{equation*}
  E=f(\Delta) - f_{exakt}
\end{equation*}

Für finite Methoden der Ordnung $p$ sollte sich der Fehler $DE$ proportional zu
$h^p$ verhalten. Daraus folgt mit dem Proportionalitätskoeffizienten $C$:

\begin{equation*}
  DE=f(\Delta) - f_{exakt}=C h^p + HOT
\end{equation*}

Wird nun die Gitterweite $h$ systematisch verkleinert, so lässst sich der beobachtete
Genauigkeitsgrad $p$ wie folgt berechnen.

\begin{equation}
  p=\frac{\ln \left(\frac{DE_2}{DE_1}\right)}{\ln \left(\frac{h_2}{h_1}\right)}
\end{equation}

\begin{figure}
\begin{tikzpicture}
    \begin{axis}[
        height=7cm,
        width=0.9*\textwidth,
        grid=major,
        xlabel=$x$,
        ylabel=$\phi$,
    ]
    \addplot[tud2d, mark=*] file {solution_20_aequi.txt};
    \end{axis}
\end{tikzpicture}
\caption{Lösung des Diffusionsproblems mit Quellterm aus Gleichung \ref{eq:quellterm}}
\end{figure}

\begin{figure}
\begin{tikzpicture}
    \begin{axis}[
        height=7cm,
        width=0.9*\textwidth,
        grid=major,
        xlabel=$x$,
        ylabel=Fehler,
%scaled y ticks = false,
    ]
    \addplot[tud2d, mark=*] file {error_20_aequi.txt};
    \end{axis}
\end{tikzpicture}
\caption{Fehlerverteilung der Lösung}
\end{figure}

\begin{figure}
\begin{tikzpicture}
    \begin{axis}[
        height=7cm,
        width=0.9*\textwidth,
        grid=major,
        xlabel=$x$,
        ylabel=$A\phi_{exact}-b$,
%scaled y ticks = false,
    ]
    \addplot[tud2d, mark=*] file {residuum_20_aequi.txt};
    \end{axis}
\end{tikzpicture}
\caption{Residuum der Lösung}
\end{figure}

\begin{figure}
\begin{tikzpicture}
    \begin{axis}[
        height=7cm,
        width=0.9*\textwidth,
        grid=major,
        xlabel=$x$,
        ylabel=TE,
%scaled y ticks = false,
    ]
    \addplot[tud2d, mark=*] file {te_20_aequi.txt};
    \end{axis}
\end{tikzpicture}
\caption{Truncation Error}
\end{figure}

\end{document}
