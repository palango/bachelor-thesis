\documentclass[11pt, ngerman,colorback,accentcolor=tud2d]{tudreport}
\usepackage[utf8]{inputenc}
\usepackage[ngerman]{babel}
\usepackage{amsmath}

\usepackage{microtype}


\newcommand{\pder}[2][]{\frac{\partial#1}{\partial#2}}
\newcommand{\pderf}[1]{\frac{\partial f}{\partial#1}}
\newcommand{\pderfs}[1]{\frac{\partial^2 f}{\partial#1}}

\newcommand{\fxi}{f_{\xi}}
\newcommand{\fxxi}{f_{\xi\xi}}
\newcommand{\fxxxi}{f_{\xi\xi\xi}}
\newcommand{\fxxxxi}{f_{\xi\xi\xi\xi}}
\newcommand{\xxi}{x_{\xi}}
\newcommand{\xxxi}{x_{\xi\xi}}
\newcommand{\xxxxi}{x_{\xi\xi\xi}}
\newcommand{\xxxxxi}{x_{\xi\xi\xi\xi}}
\newcommand{\yxi}{y_{\xi}}
\newcommand{\yxxi}{y_{\xi\xi}}
\newcommand{\yxxxi}{y_{\xi\xi\xi}}
\newcommand{\yxxxxi}{y_{\xi\xi\xi\xi}}

\newcommand{\feta}{f_{\eta}}
\newcommand{\feeta}{f_{\eta\eta}}
\newcommand{\feeeta}{f_{\eta\eta\eta}}
\newcommand{\feeeeta}{f_{\eta\eta\eta\eta}}
\newcommand{\xeta}{x_{\eta}}
\newcommand{\xeeta}{x_{\eta\eta}}
\newcommand{\xeeeta}{x_{\eta\eta\eta}}
\newcommand{\xeeeeta}{x_{\eta\eta\eta\eta}}
\newcommand{\yeta}{y_{\eta}}
\newcommand{\yeeta}{y_{\eta\eta}}
\newcommand{\yeeeta}{y_{\eta\eta\eta}}
\newcommand{\yeeeeta}{y_{\eta\eta\eta\eta}}

\begin{document}

\chapter{Transformation}

Die Transformation vom physikalischen auf den logischen Bereich mit
den physikalischen Koordinaten $(x, y)$ sowie den logischen
Koordinaten $(\xi, \eta)$ kann wie folgt definiert werden:
\begin{equation}
  x=x(\xi,\eta),\quad y=y(\xi, \eta)
\end{equation}
\begin{equation}
  \pder[x]{\xi}=\xi_x=\frac{y_{\eta}}{J}
\end{equation}
Damit folgt für für die bei Konvektion und Diffusion auftretende
erste und zweite Ableitung:

\begin{align*}
  \pderf{x}&=\pder[\xi]{x}\pderf{\xi}+\pder[\eta]{x}\pderf{\eta}
  =\frac{y_{\eta}}{J} f_{\xi} - \frac{y_{\xi}}{J}f_{\eta}\\
  \pderfs{x^2} &= \pder{x}\left({\frac{y_{\eta}}{J} f_{\xi} - \frac{y_{\xi}}{J}f_{\eta}}\right)\\
               &= \cdots
\end{align*}


Im logischen Gebiet können nun die vorhandenen Ableitungen von $f$
($f_{\xi}$, $f_{\eta}$, $\dots$) diskretisiert werden.
Es werden hierbei Zentraldifferenzen zweiter Ordnung gewählt.

\begin{equation}
  f_{\xi}=\frac{f_{i+1,j}-f_{i-1,j}}{2\Delta i} - \frac{f_{\xi\xi\xi}}{6}\Delta i^2 + HOT
\end{equation}
\begin{equation}
  f_{\eta}=\frac{f_{i,j+1}-f_{i,j-1}}{2 \Delta i} - \frac{f_{\eta\eta\eta}}{6}\Delta i^2 + HOT
\end{equation}

Werden diese nun eingesetzt ergibt sich für $\Delta i = 1$ folgende Form:

\begin{equation}
  \pderf{x}=\frac{y_{\eta}}{J}\frac{f_{i+1,j}-f_{i-1,j}}{2}
  - \frac{y_{\xi}}{J}\frac{f_{i,j+1}-f_{i,j-1}}{2}
\end{equation}

sowie der entstehende Abbruchfehler:

\begin{equation}
  TE_x = -\frac{y_{\eta}}{J}\frac{f_{\xi\xi\xi}}{6}
  + \frac{y_{\xi}}{J}\frac{f_{\eta\eta\eta}}{6}+HOT
\end{equation}

Anschließend können die berechneten Werte für $f_{\xi},\dots$
eingesetzt werden.

\begin{align*}
  \fxi &= \xxi f_x + \yxi f_y\\
  \fxxi &= \xxxi f_x + \xxi^2f_{xx}+2\xxi \yxi f_{xy}
  +\yxi^2 f_{yy} + \yxxi f_y\\
  \fxxxi &= \xxxxi f_x + \yxxxi f_y + 3 \xxxi\xxi f_{xx}+
  3 \yxxi\yxi f_{yy} + \left({3\xxxi\yxi+3\yxxi\xxi}\right)f_{xy}\\
  &+ \xxi^3f_{xxx}+\yxi^3f_{yyy}+3\xxi^2\yxi f_{xxy}+3\xxi\yxi^2f_{xyy}\\
 \feeeta &= \xeeeta f_x + \yeeeta f_y + 3 \xeeta\xeta f_{xx}+
  3 \yeeta\yeta f_{yy} + \left({3\xeeta\yeta+3\yeeta\xeta}\right)f_{xy}\\
  &+ \xeta^3f_{xxx}+\yeta^3f_{yyy}+3\xeta^2\yeta f_{xxy}+3\xeta\yeta^2f_{xyy}
\end{align*}

Es ergibt sich:

\begin{align*}
  TE_x&= -\frac{y_{\eta}}{J} \frac{1}{6}\Big[\xxxxi f_x + \yxxxi f_y + 3 \xxxi\xxi f_{xx}+
  3 \yxxi\yxi f_{yy} + \left({3\xxxi\yxi+3\yxxi\xxi}\right)f_{xy}\\
&+ \xxi^3f_{xxx}+\yxi^3f_{yyy}+3\xxi^2\yxi f_{xxy}+3\xxi\yxi^2f_{xyy}\Big]\\
&+  \frac{y_{\xi}}{J}\frac{1}{6} \Big[\xeeeta f_x + \yeeeta f_y + 3 \xeeta\xeta f_{xx}+
  3 \yeeta\yeta f_{yy} + \left({3\xeeta\yeta+3\yeeta\xeta}\right)f_{xy}\\
  &+ \xeta^3f_{xxx}+\yeta^3f_{yyy}+3\xeta^2\yeta f_{xxy}+3\xeta\yeta^2f_{xyy}
\Big]\\
  &= TE_{x1} + TE_{x2} + TE_{x3}
\end{align*}

wobei $TE_{x1}$die ersten Ableitungen von $f$ enthält, usw.:

\begin{align*}
  TE_{x1} &= \frac{1}{6\ J}\left[{
  \left({-\yeta\xxxxi + \yxi\xeeeta}\right) f_x +
  \left({-\yeta\yxxxi + \yxi\yeeeta}\right) f_y
  }\right]\\
  TE_{x2} &= \frac{1}{2\ J} \Big[
  \left({-\yeta\xxxi\xxi + \yxi\xeeta\xeta}\right) f_{xx}+
  \left({-\yeta\yxxi\yxi + \yxi\yeeta\yeta}\right) f_{yy}\\&+
  \left({-\yeta \left({\xxxi\yxi+\yxxi\xxi}\right) +
  \yxi \left({\xeeta\yeta+\yeeta\xeta}\right)}\right) f_{xy}
  \Big]\\
  TE_{x3}&=\frac{1}{6\ J} \Big[
  \left({-\yeta\xxi^3+\yxi\xeta^3}\right) f_{xxx}+
  \yxi \yeta\left({-\yxi^2+\yeta^2}\right) f_{yyy}\\&+
  3 \xxi \xeta \left({-\xxi^2+\xeta^2}\right) f_{xxy}+
  3 \xxi \xeta \left({-\xxi\yxi+\xeta\yeta}\right) f_{xyy}
  \Big]
\end{align*}

\end{document}
