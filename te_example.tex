\documentclass[11pt, ngerman,colorback,accentcolor=tud2d]{tudreport}
\usepackage[utf8]{inputenc}
\usepackage[ngerman]{babel}
\usepackage{amsmath}

\usepackage{microtype}

\usepackage{pgfplots}
\pgfplotsset{compat=1.3}

\begin{document}

\chapter{Truncation Error}

Bei der Lösung von Problemen werden bei der Diskretisierung Vereinfachungen vorgenommen.
So können viele Approximationen durch Taylorreihenentwicklungen gewonnen werden. Dabei
werden jedoch zwangsläufig Terme weggelassen. Der so enstehende Fehler wird
Abbruchfehler (Truncation Error) genannt und kann zur Fehlerabschätzung bei FVM-Berechnungen
genutzt werden. Die grundlegenden Zusammenhänge sollen hier an einem einfachen
Beispiel gezeigt werden.

Wir nehmen an, die Funktion $\sin(x)$ soll zur einfacheren Berechenbarkeit durch
ein Polynom ausgedrückt werden. Anschließend soll ein Indikator für den dabei
entstehenden Fehler gefunden werden.

Zuerst wird die Taylorreihe von $\sin(x)$ um den Entwicklungspunkt $0$ berechnet.

\begin{equation}
  \sin(x) = x-\frac{x^3}{3!} +\frac{x^5}{5!} -\frac{x^7}{7!} +\frac{x^9}{9!}\cdots
\end{equation}

Nun wird die Approximation für $\sin(x)$ festgelegt. Diese soll in diesem Beispiel
$x-\frac{x^3}{6} $ sein. In einem Diagramm sieht man, wie sich die Güte der
Approximation mit zunehmenden Abstand von $0$ verschlechtert.

\begin{figure}[ht]
\begin{tikzpicture}
\begin{axis}[width=0.8*\textwidth, height=150pt, grid=major]
  \addplot[domain=-2*pi:2*pi, samples=100, color=tud2b, very thick]{sin(deg(x))};
  \addplot[domain=-3:3, samples=100, color=tud9b, very thick]{x - x*x*x/6};
  \legend{$\sin(x)$,$x-\frac{x^3}{6} $}
\end{axis}
\end{tikzpicture}
\centering
\caption{Vergleich von Funktion und Approximation}
\end{figure}


Der absolute Fehler kann nun einfach über die Formel $\sin(x) - x + \frac{x^3}{6}$
berechnet werden. Die Aufgabe besteht nun darin diesen geeignet über eine Formel abzuschätzen.

Die Taylorreihenglieder, die in der Approximation nicht beachtet wurden, sind für diese
Abweichung verantwortlich und können zur Abschätzund des Fehlers genutzt werden.
Die Formel für den Fehler ergibt sich bei Beachtung von zwei Gliedern zu
$\frac{x^5}{5!} -\frac{x^7}{7!} $.

\begin{figure}[ht]
\begin{tikzpicture}
\begin{axis}[width=0.8*\textwidth, height=150pt, grid=major, legend pos=north west]
  \addplot[domain=-6:6, samples=100, color=tud2b, very thick]{sin(deg(x))-x+x*x*x/6};
  \addplot[domain=-6:6, samples=100, color=tud9b, very thick]{x^5/120-x^7/(120*6*7)};
  \legend{Absoluter Fehler, Fehlerapproximation}
\end{axis}
\end{tikzpicture}
\centering
\caption{Vergleich des absoluten Fehlers mit der Fehlerschätzung}
\end{figure}

Man sieht die gute Übereinstimmung von absolutem Fehler und Fehlerapproximation.


\end{document}
