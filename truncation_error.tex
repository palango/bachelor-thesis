\documentclass[11pt, ngerman,colorback,accentcolor=tud2d]{tudreport}
\usepackage[utf8]{inputenc}
\usepackage[ngerman]{babel}
\usepackage{amsmath}

\usepackage{microtype}

\usepackage{pgfplots}
\pgfplotsset{compat=1.3}

\usepackage{listings}
\begin{document}

\newcommand{\pder}[2][]{\frac{\partial#1}{\partial#2}}
\newcommand{\pderf}[1]{\frac{\partial f}{\partial#1}}
\newcommand{\pderfs}[1]{\frac{\partial^2 f}{\partial#1}}



\chapter{Truncation Error}
\label{cha:Truncation_Error}


\section{Konvektionsterme}

\subsection{CDS-Verfahren}
Entwickelt man die Taylorreihe von $\phi$ um den Punkt $x_P$ und wertet sie anschließend
an den Punkten $x_e$ und $x_E$ aus, so erhält man folgende Gleichungen.

\begin{align}
  \phi_e &= \phi_P + \phi'_P(x_e-x_P)+\frac{1}{2}\phi''_P(x_e-x_P)^2
  +\frac{1}{6}\phi'''_P(x_e-x_P)^3+HOT
  \label{eq:taylor_konv_eP}\\
  \phi_E &= \phi_P + \phi'_P(x_E-x_P)+\frac{1}{2}\phi''_P(x_E-x_P)^2
  +\frac{1}{6}\phi'''_P(x_E-x_P)^3+HOT
  \label{eq:taylor_konv_eE}
\end{align}

Werden die Gleichungen~\eqref{eq:taylor_konv_eE} und \eqref{eq:taylor_konv_eP} nun
voneinander subtrahiert und nach $\phi_e$ umgestellt, so ergibt sich:

\begin{align*}
  \frac{\phi_e}{x_e-x_P} &= \frac{\phi_E}{x_E-x_P} + \frac{\phi_P}{x_e-x_P} -
  \frac{\phi_P}{x_E-x_P} + \frac{1}{2} \phi''_P \left({(x_e-x_P)-(x_E-x_P)}\right)
  + \frac{1}{6} \phi'''_P \left({(x_e-x_P)^2-(x_E-x_P)^2}\right)\\
  \phi_e &= \phi_E \frac{x_e-x_P}{x_E-x_P} + \phi_P \left({1-\frac{x_e-x_P}{x_E-x_P} }\right)
  + \frac{1}{2} \phi''_P (x_e-x_E)(x_e-x_P)
  + \frac{1}{6} \phi'''_P \left({(x_e-x_P)^2-(x_E-x_P)^2}\right)(x_e-x_P)\\
  \phi_e &= \phi_E \gamma_e + \phi_P (1-\gamma_e)+ \frac{1}{2} \phi''_P (x_e-x_E)(x_e-x_P)
  + \frac{1}{6} \phi'''_P \left({(x_e-x_P)^2-(x_E-x_P)^2}\right)(x_e-x_P)
\end{align*}

Der Truncation Error lässt sich damit ablesen zu:

\begin{equation*}
  TE_{e, CDS} =  \frac{1}{2} \phi''_P (x_e-x_E)(x_e-x_P)+ \frac{1}{6} \phi'''_P \left({(x_e-x_P)^2-(x_E-x_P)^2}\right)(x_e-x_P)
\end{equation*}


Hier müssen nun wiederum die auftretenden Ableitungen diskretisiert werden.

\begin{align}
  \phi'_P &= \frac{\phi_E-\phi_W}{x_E-x_W}\\
  \phi''_P &= \frac{1}{(x_e-x_w)} \left({\frac{\phi_E-\phi_P}{x_E-x_P}
  - \frac{\phi_P-\phi_W}{x_P-x_W} }\right)\\
  \phi'''_P &= \frac{1}{(x_e-x_w)} \left({
  \frac{1}{(x_E-x_P)} \left({\frac{\phi_{EE}-\phi_P}{x_{EE}-x_P}- \frac{\phi_E-\phi_W}{x_E-x_W} }\right)-
  \frac{1}{(x_P-x_W)} \left({\frac{\phi_E-\phi_W}{x_E-x_W} - \frac{\phi_P-\phi_{WW}}{x_P-x_{WW}} }\right)
  }\right)
\end{align}

Der diskretisierte Truncation Error ergibt sich damit zu:

\begin{align}
  TE_{e, CDS} &=  \frac{1}{2} \frac{1}{(x_e-x_w)} \left({\frac{\phi_E-\phi_P}{x_E-x_P}
  - \frac{\phi_P-\phi_W}{x_P-x_W} }\right) (x_e-x_E) \nonumber \\
  &+
 \frac{1}{(x_e-x_w)} \left({
  \frac{1}{(x_E-x_P)} \left({\frac{\phi_{EE}-\phi_P}{x_{EE}-x_P}- \frac{\phi_E-\phi_W}{x_E-x_W} }\right)-
  \frac{1}{(x_P-x_W)} \left({\frac{\phi_E-\phi_W}{x_E-x_W} - \frac{\phi_P-\phi_{WW}}{x_P-x_{WW}} }\right)
  }\right) \nonumber \\
  &\frac{1}{6} \left({(x_e-x_P)^2-(x_E-x_P)^2}\right)(x_e-x_P)
\end{align}




\subsection{UDS-Verfahren}

\paragraph{Osten}

Wertet man die Taylorreihe von $\phi$ mit dem Entwicklungspunkt $x_P$ im Punkte $\phi_e$
aus, so ergibt sich:

\begin{equation*}
  \phi_e = \phi_P +(x_e-x_P) \phi'_P + \frac{1}{2} (x_e-x_P)^2 \phi''_P+HOT
\end{equation*}

Da im Falle einer positiven Geschwindigkeit das UDS-Verfahren $\phi_e$ mit $\phi_P$
gleichsetzt, lässt sich der Truncation Error hier direkt ablesen.

\begin{equation*}
  TE_{e, UDS} = (x_e-x_P) \phi'_P + \frac{1}{2} (x_e-x_P)^2 \phi''_P
\end{equation*}

Mit den diskretisierten Ableitungen von oben ergibt sich damit:

\begin{equation}
  TE_{e, UDS} = (x_e-x_P) \frac{\phi_E-\phi_W}{x_E-x_W}+
  \frac{(x_e-x_P)^2}{2(x_e-x_w)} \left({\frac{\phi_E-\phi_P}{x_E-x_P}
  - \frac{\phi_P-\phi_W}{x_P-x_W} }\right)
\end{equation}

Für äquidistante Gitter ergibt sich damit:

\begin{equation}
  TE_{e, UDS} = \frac{3}{8} \phi_E-\frac{1}{4} \phi_P - \frac{1}{8} \phi_W
\end{equation}


\paragraph{Westen}

Wertet man die Taylorreihe von $\phi$ mit dem Entwicklungspunkt $x_P$ im Punkte $\phi_w$
aus, so ergibt sich:

\begin{equation*}
  \phi_w = \phi_W +(x_w-x_W) \phi'_W + \frac{1}{2} (x_w-x_W)^2 \phi''_W+HOT
\end{equation*}

Da im Falle einer positiven Geschwindigkeit das UDS-Verfahren $\phi_w$ mit $\phi_W$
gleichsetzt, lässt sich der Truncation Error hier direkt ablesen.

\begin{equation*}
  TE_{w, UDS} = (x_w-x_W) \phi'_W + \frac{1}{2} (x_w-x_W)^2 \phi''_W
\end{equation*}

%Mit den diskretisierten Ableitungen von oben ergibt sich damit:

%\begin{equation}
  %TE_{w, UDS} = (x_w-x_W) \frac{\phi_E-\phi_W}{x_E-x_W}+
  %\frac{(x_e-x_P)^2}{2(x_e-x_w)} \left({\frac{\phi_E-\phi_P}{x_E-x_P}
  %- \frac{\phi_P-\phi_W}{x_P-x_W} }\right)
%\end{equation}

%Für äquidistante Gitter ergibt sich damit:

%\begin{equation}
  %TE_{e, UDS} = \frac{3}{8} \phi_E-\frac{1}{4} \phi_P - \frac{1}{8} \phi_W
%\end{equation}



\subsection{``Flux-Blending''-Verfahren}

Das Flux-Blending-Verfahren setzt sich aus UDS- und CDS-Verfahren zusammen. Beide
Verfahren werden dabei über den Faktor $\beta$ gewichtet.

\begin{equation*}
\phi_e \approx (1-\beta)\phi_e^{UDS} + \beta \phi_e^{CDS} 
\end{equation*}

Aufgrund dessen ist es möglich den Truncation Error des Flux-Blending-Verfahrens aus
den vorangegenagen Betrachtungen zu UDS- und CDS-Verfahren herzuleiten.
Dabei werden Die Abbruchfehler der einzelnen Verfahren ebenso über $\beta$ gewichtet.

\begin{equation}
  TE_{e, Flux} = (1-\beta) TE_{e, UDS} + \beta TE_{e, CDS}
\end{equation}



\section{Truncation Error eines Kontrollvolumens}
\label{sec:Truncation Error eines Kontrollvolumens}

Der Truncation Error für ein Kontrollvolumen setzt sich nun aus den Fehlern von Quell-
und Diffusionstermen zusammen:

\begin{equation*}
  TE = TE_{source} - TE_e - TE_w
\end{equation*}

Für den äquidistanten Fall ergibt sich damit für zentrale Kontrollvolumen der folgende
Truncation Error. Wichtig ist es, hier die durch den Gauß'schen Integralsatz 
entstehenden Vorzeichen mit zu beachten.

\begin{align}
  TE &= \frac{\Delta x}{24} \left({f_E-2f_P+f_W}\right)
   +\frac{1}{24\Delta x}\left({
\phi_{EE}-3\phi_E+3\phi_P-\phi_W}\right)
  -\frac{1}{24 \Delta x}\left({
\phi_E-3\phi_P+3\phi_W-\phi_{WW}}\right)
\end{align}

\begin{figure}[h]
\begin{tikzpicture}
    \begin{axis}[
        height=7cm,
        width=0.9*\textwidth,
        grid=major,
        xlabel=$x$,
        ylabel=TE,
%scaled y ticks = false,
    ]
    \addplot[tud2d, mark=*] file {data/te_20_aequi.txt};
    \end{axis}
\end{tikzpicture}
\caption{Truncation Error}
\end{figure}

\section{Randwerte}
\label{sec:Randwerte}

Für die Auswertung der Ableitungen an Randpunkten ist es nicht möglich zentrale Differenzenquotienten
zu finden, da man dazu Punkte außerhalb des betrachteten Gebietes einbeziehen müsste.

Stattdessen erstellt man für die Randpunkte Interpolationspolynome und leitet diese ab.
Sie können dann einfach an den gewünschten Punkten ausgewertet werden.

Will man eine zweite Ableitung diskretisieren, so muss das Interpolationspolynom
ebensfalls die Ordnung 2 haben, um genügend Ableitungen zur Verfügung zu stellen.
Verallgemeinert bedeutet dies, dass für eine n-te Ableitung ein Polynom vom Grad n
aufgestellt werden muss.

Das Aufstellen der Polynome kann zum Beispiel über die Lagrange-Formel erfolgen.
Für Polynomgrade größer zwei ist es sinnvoll ein CAS-System zur Berechnung zu verwenden.
Ein möglicher Weg soll hier für das Programmpaket Maxima aufgezeigt werden.

\begin{lstlisting}
  load(interpol);
  data: [[a, fa], [b, fb], [c, fc]];
  poly: lagrange(data);
  d_poly: diff(poly, x);
  
\end{lstlisting}

\end{document}
