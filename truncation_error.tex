\documentclass[10pt, ngerman,colorback,accentcolor=tud2d]{tudreport}
\usepackage[utf8]{inputenc}
\usepackage[ngerman]{babel}
\usepackage{amsmath}

\usepackage{microtype}

\usepackage{pgfplots}
\pgfplotsset{compat=1.8}

\begin{document}

\chapter{Truncation Error}
\label{cha:Truncation Error}

\newcommand{\ab}{\frac{a+b}{2}}
\newcommand{\abk}{\left(\frac{a+b}{2}\right)}

\section{Quellterm}
\label{sec:Quellterm}


Die Taylorreihenentwicklung einer Funktion $f(x)$ um den Punkt $\ab$
liefert:

\begin{equation*}
  f(x) = f\abk +
  f'\abk\left(x-\ab\right) +
  \frac{1}{2}f''\abk{\left(x-\ab\right)}^2 + HOT
\end{equation*}

Nach Integration im Intervall $a$ bis $b$ ergibt sich für äquidistante Gitter:

\begin{align*}
  \int_a^b f(x) dx &= \int_a^b f\abk dx +
  \underbrace{\int_a^b f'\abk \left( x - \ab\right) dx}_{=0} +
  \int_a^b \frac{1}{2}f''\abk{\left(x-\ab\right)}^2 dx + HOT\\
  &= f\abk (b-a) + \frac{1}{6} f''\abk
  \left[\left( b-\ab \right)^3 - \left(a-\ab\right)^3 \right] + HOT
\end{align*}

Angewendet auf ein Kontrollvolumen ergibt sich daraus die folgende Formel:

\begin{equation*}
  \int_a^b f(x)dx = f_P(x_E-x_W) + \frac{1}{6} f''_P \left[{{(x_E-x_P)}^3-{(x_W-x_P)}^3}\right] + HOT
\end{equation*}

Hier muss nun die auftretende Ableitung durch einen Differenzenquotient beschriebn werden.

\begin{equation*}
  f''_P = \frac{1}{x_e-x_w}\left(\frac{f_E-f_P}{x_E-x_P}-\frac{f_P-f_W}{x_P-x_W}\right)
\end{equation*}

Der Truncation Error der Diskretisierung des Quelltermes lässt sich damit über die
folgende Gleichung beschrieben.

\begin{equation}
  TE_{source} = \frac{1}{6(x_e-x_w)}\left(\frac{f_E-f_P}{x_E-x_P}-\frac{f_P-f_W}{x_P-x_W}\right)
  \left[{{(x_E-x_P)}^3-{(x_W-x_P)}^3}\right] + HOT
\end{equation}

\subsection{Äquidistante Gitter}

Für äquidistante Gitter vereinfacht sich der Term und es ergibt sich:

\begin{equation*}
  TE_{source} = \frac{\Delta x}{3} \left({f_E-2f_P+f_W}\right)
\end{equation*}

\section{Diffusionsterme}
\label{sec:Diffusionsterme}


Um den Truncation Error der Diffusionsterme zu bestimmen, leiten wir die Differenzenquotienten
für die Ableitungen erster Ordnung aus den Taylorreihendarstellungen her. Beispielhaft
soll das anhand des östlichen Randes des Kontrollvolumens geschehen.

Zuerst entwickeln wir die Taylordarstellungen vom Punkt $x_e$ aus in Richtung der anliegenden
Kontrollvolumenmittelpunkte.

\begin{align}
  \phi_E &= \phi_e + \phi'_e(x_E-x_e)+\frac{1}{2}\phi''_e(x_E-x_e)^2
  +\frac{1}{6}\phi'''_e(x_E-x_e)^3+HOT
  \label{eq:taylor_eE}\\
  \phi_P &= \phi_e + \phi'_e(x_P-x_e)+\frac{1}{2}\phi''_e(x_P-x_e)^2
  +\frac{1}{6}\phi'''_e(x_P-x_e)^3+HOT
  \label{eq:taylor_eP}
\end{align}

Wird nun Gleichung~\ref{eq:taylor_eP} von Gleichung~\ref{eq:taylor_eE} subtrahiert, 
so ergibt sich:

\begin{equation*}
  \phi_E-\phi_P=\phi'_e(x_E-x_P)+
  \frac{1}{2}\phi''_e\left[{{(x_E-x_e)}^2-{(x_P-x_e)}^2}\right]+
  \frac{1}{6}\phi'''_e\left[{{(x_E-x_e)}^3-{(x_P-x_e)}^3}\right]+HOT
\end{equation*}

Nach Umstellen ergibt sich daraus für die Ableitung $\phi'_e$ der folgende Term:

\begin{equation}
  \phi'_e = \frac{\phi_E-\phi_P}{x_E-x_P}+\frac{1}{2}\phi''_e
\left({\frac{{(x_P-x_e)}^2-{(x_E-x_e)}^2}{x_E-x_P}}\right)+
\frac{1}{6} \phi'''_e \left({\frac{{(x_P-x_e)}^3-{(x_E-x_e)}^3}{(x_E-x_P)}}\right)+HOT
\end{equation}


Die westliche Seite des Kontrollvolumens kann auf die gleiche Weise hergeleitet werden.
Damit verbleibt es, die auftretenden Ableitungen zu diskretisieren. Werden dafür
möglichst lokale Differenzenquotienten genutzt ergeben sich die folgenden Ausdrücke.

\begin{align*}
  \phi''_e &= \frac{1}{(x_E-x_P)}\left({
\frac{\phi_{EE}-\phi_P}{x_{EE}-x_P}-\frac{\phi_E-\phi_W}{x_E-x_W}}\right)\\
\phi'''_e &= \frac{1}{(x_E-x_P)}\left({
\frac{1}{(x_{ee}-x_e)}
\left({\frac{\phi_{EE}-\phi_E}{x_{EE}-x_E}-\frac{\phi_E-\phi_P}{x_E-x_P} }\right)
-\frac{1}{(x_e-x_w)}
\left({\frac{\phi_E-\phi_P}{x_E-x_P} - \frac{\phi_P-\phi_W}{x_P-x_W}  }\right)
}\right)
\end{align*}


Der Truncation Error des Diffusionsterms an der östlichen Grenze des Kontrollvolumens
lässt sich damit über folgende Gleichung beschreiben.

\begin{align*}
  {TE}_e &= \frac{1}{2 (x_E-x_P)}\left({
\frac{\phi_{EE}-\phi_P}{x_{EE}-x_P}-\frac{\phi_E-\phi_W}{x_E-x_W}}\right) \left({\frac{{(x_P-x_e)}^2-{(x_E-x_e)}^2}{x_E-x_P}}\right)\\
&+
\left({
\frac{1}{(x_{ee}-x_e)}
\left({\frac{\phi_{EE}-\phi_E}{x_{EE}-x_E}-\frac{\phi_E-\phi_P}{x_E-x_P} }\right)
-\frac{1}{(x_e-x_w)}
\left({\frac{\phi_E-\phi_P}{x_E-x_P} - \frac{\phi_P-\phi_W}{x_P-x_W}  }\right)
}\right)\\
&\frac{1}{6(x_E-x_P)}\left({\frac{{(x_P-x_e)}^3-{(x_E-x_e)}^3}{(x_E-x_P)}}\right)
+HOT
\end{align*}

Der Truncation Error im Westen ergibt sich äquivalent zu:

\begin{align}
  \phi_P &= \phi_w + \phi'_w(x_P-x_w)+\frac{1}{2}\phi''_w(x_P-x_w)^2
  +\frac{1}{6}\phi'''_w(x_P-x_w)^3+HOT
  \label{eq:taylor_wP}\\
  \phi_W &= \phi_w + \phi'_w(x_W-x_w)+\frac{1}{2}\phi''_w(x_W-x_w)^2
  +\frac{1}{6}\phi'''_w(x_W-x_w)^3+HOT
  \label{eq:taylor_eP}
\end{align}

\begin{equation}
  \phi'_w = \frac{\phi_P-\phi_W}{x_P-x_W}+\frac{1}{2}\phi''_w
\left({\frac{{(x_W-x_w)}^2-{(x_P-x_w)}^2}{x_P-x_W}}\right)+
\frac{1}{6} \phi'''_w \left({\frac{{(x_W-x_w)}^3-{(x_P-x_w)}^3}{(x_P-x_W)}}\right)+HOT
\end{equation}

\begin{align*}
  \phi''_w &= \frac{1}{(x_P-x_W)}\left({
\frac{\phi_{E}-\phi_W}{x_{E}-x_W}-\frac{\phi_P-\phi_{WW}}{x_P-x_{WW}}}\right)\\
 \phi'''_w &= \frac{1}{(x_P-x_W)}\left({
\frac{1}{(x_e-x_w)}
\left({\frac{\phi_E-\phi_P}{x_E-x_P}-\frac{\phi_P-\phi_W}{x_P-x_W} }\right)
-\frac{1}{(x_w-x_{ww})}
\left({\frac{\phi_P-\phi_W}{x_P-x_W} - \frac{\phi_W-\phi_{WW}}{x_W-x_{WW}}  }\right)
}\right)
\end{align*}

\begin{align*}
  TE_w &= \frac{1}{2 (x_P-x_W)} \left({
\frac{\phi_{E}-\phi_W}{x_{E}-x_W}-\frac{\phi_P-\phi_{WW}}{x_P-x_{WW}}}\right)
  \left({\frac{{(x_W-x_w)}^2-{(x_P-x_w)}^2}{x_P-x_W}}\right)\\
&+
\left({
\frac{1}{(x_e-x_w)}
\left({\frac{\phi_E-\phi_P}{x_E-x_P}-\frac{\phi_P-\phi_W}{x_P-x_W} }\right)
-\frac{1}{(x_w-x_{ww})}
\left({\frac{\phi_P-\phi_W}{x_P-x_W} - \frac{\phi_W-\phi_{WW}}{x_W-x_{WW}}  }\right)
}\right)\\
&\frac{1}{6(x_P-x_W)}\left({\frac{{(x_W-x_w)}^3-{(x_P-x_w)}^3}{(x_P-x_W)}}\right)
  +HOT
\end{align*}

\subsection{Äquidistante Gitter}

Für äquidistante Gitter vereinfache sich die Terme des Truncation Error. So löschen
sich beispielsweise die quadratischen Terme gegenseitig aus. Es ergeben sich die
folgenden Fehler:

\begin{align*}
  {TE}_e &= \left({
\frac{1}{6\Delta x^2}
\left({\frac{\phi_{EE}-\phi_E}{\Delta x}-\frac{\phi_E-\phi_P}{\Delta x} }\right)
-\frac{1}{6\Delta x^2}
\left({\frac{\phi_E-\phi_P}{\Delta x} - \frac{\phi_P-\phi_W}{\Delta x}  }\right)
}\right)\left({-\frac{\Delta x^2}{4} }\right)+HOT\\
&= -\frac{1}{24\Delta x}\left({
\phi_{EE}-3\phi_E+3\phi_P-\phi_W}\right)+HOT
\end{align*}

\begin{align*}
  TE_w &=\left({
\frac{1}{6 \Delta x^2}
\left({\frac{\phi_E-\phi_P}{\Delta x}-\frac{\phi_P-\phi_W}{\Delta x} }\right)
-\frac{1}{6\Delta x^2}
\left({\frac{\phi_P-\phi_W}{\Delta x} - \frac{\phi_W-\phi_{WW}}{\Delta x}  }\right)
}\right)
\left({-\frac{\Delta x^2}{4} }\right)+HOT\\
&= -\frac{1}{24 \Delta x}\left({
\phi_E-3\phi_P+3\phi_W-\phi_{WW}}\right)+HOT
\end{align*}




\subsection{Kontrollvolumen am Rand}

Wie man sieht, werden für die Differenzenquotienten die Funktionswerte benachbarter
Kontrollvolumen benötigt. Deshalb müssen die Approximationen der Ableitung für
Kontrollvolumen am Rand angepasst werden. Es ergeben sich für gegebenes $\phi_w$:

\begin{equation*}
  \phi''_{w, Rand} = \frac{1}{(x_P-x_w)}\left({
\frac{\phi_{e}-\phi_w}{x_{e}-x_w}-\frac{\phi_P-\phi_w}{x_P-x_w}}\right)
\end{equation*}

Hier wird noch der Wert $\phi_e$ benutzt, der aber nicht bekannt ist. Er wird deshalb
durch eine lineare Interploation von $\phi_E$ und $\phi_P$ bestimmt.

\begin{equation}
  \phi_e = \phi_P \frac{x_E-x_e}{x_E-x_P} + \phi_E \frac{x_e-x_P}{x_E-x_P}
\end{equation}

Damit ergibt sich für $\phi''_{w,Rand}$:

\begin{equation}
  \phi''_{w,Rand}  = \frac{1}{(x_P-x_w)}\left({
\frac{
  \left({\phi_P \frac{x_E-x_e}{x_E-x_P} + \phi_E \frac{x_e-x_P}{x_E-x_P}
}\right)
-\phi_w}{x_{e}-x_w}-\frac{\phi_P-\phi_w}{x_P-x_w}}\right)
\end{equation}

Äquivalent ergibt sich $\phi''_{e, Rand}$ bei gegebenem $\phi_e$:

\begin{equation*}
  \phi''_{e, Rand} = \frac{1}{(x_e-x_P)}\left({
\frac{\phi_{e}-\phi_P}{x_{e}-x_P}-\frac{\phi_e-\phi_w}{x_e-x_w}}\right)
\end{equation*}

$\phi_w$ wird wie folgt linear interpoliert:

\begin{equation*}
  \phi_w = \phi_W \frac{x_P-x_w}{x_P-x_W} + \phi_P \frac{x_w-x_W}{x_P-x_W}
\end{equation*}

Damit ergibt sich für $\phi''_{e,Rand}$:

\begin{equation}
  \phi''_{e,Rand} = \frac{1}{(x_e-x_P)}\left({
\frac{\phi_{e}-\phi_P}{x_{e}-x_P}-\frac{\phi_e-
  \left({
   \phi_W \frac{x_P-x_w}{x_P-x_W} + \phi_P \frac{x_w-x_W}{x_P-x_W}
  }\right)
}{x_e-x_w}}\right)
\end{equation}

\section{Truncation Error eines Kontrollvolumens}
\label{sec:Truncation Error eines Kontrollvolumens}

Der Truncation Error für ein Kontrollvolumen setzt sich nun aus den Fehlern von Quell-
und Diffusionstermen zusammen:

\begin{equation*}
  TE = TE_{source} + TE_e + TE_w
\end{equation*}

Für den äquidistanten Fall ergibt sich damit für zentrale Kontrollvolumen der folgende
Truncation Error.

\begin{align}
  TE &= \frac{\Delta x}{3} \left({f_E-2f_P+f_W}\right)
   -\frac{1}{24\Delta x}\left({
\phi_{EE}-3\phi_E+3\phi_P-\phi_W}\right)
  -\frac{1}{24 \Delta x}\left({
\phi_E-3\phi_P+3\phi_W-\phi_{WW}}\right)\\
&=  \frac{\Delta x}{3} \left({f_E-2f_P+f_W}\right)
-\frac{1}{24\Delta x} \left({
\phi_{EE}-2\phi_E+2\phi_W-\phi_{WW}}\right)
\end{align}

\end{document}
